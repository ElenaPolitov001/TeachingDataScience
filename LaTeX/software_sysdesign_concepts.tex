%%%%%%%%%%%%%%%%%%%%%%%%%%%%%%%%%%%%%%%%%%%%%%%%%%%%%%%%%%%%%%%%%%%%%%
\begin{frame}[fragile]\frametitle{}
\begin{center}
{\Large System Design Concepts}
\end{center}

\end{frame}

%%%%%%%%%%%%%%%%%%%%%%%%%%%%%%%%%%%%%%%%%%%%%%%%%%%%%%%%%%%%%%%%%%%%%%
\begin{frame}
	\frametitle{CAP theorem}
		
			\begin{itemize}
				\item  `Consistency' : data being the same across all nodes in a distributed system—get the latest data or no data at all
				\item `Availability' : returning some data even if it’s not the latest
				\item `Partition tolerance' : can still return data when some nodes are down
				\item You can only have two of the three. 
				\item Realistically, you can only choose between C or A because if a system is partition intolerant and a server goes down, it all goes to hell. 
			\end{itemize}
\end{frame}

%%%%%%%%%%%%%%%%%%%%%%%%%%%%%%%%%%%%%%%%%%%%%%%%%%%%%%%%%%%%%%%%%%%%%%
\begin{frame}
	\frametitle{ACID and BASE}
		ACID
			\begin{itemize}
				\item `Atomicity' : transaction either fails or succeeds as a whole—no partial success
				\item `Consistency' : when a transaction completes the DB is structurally sound
				\item `Isolation' : transactions don’t conflict because they are monitored by the database, making it seem like sequential changes are made
				\item `Durability' : result of a transaction is permanent, even if it’s a failure. 
				\item ACID databases are write consistent (means something different to the CAP 
theorem here) but that requires sophisticated locking protocols under the hood, making is slower, 
especially at large scale. 
			\end{itemize}
		BASE
			\begin{itemize}
				\item basic availability (data is available most of the time)
				\item soft-state (system doesn’t have to be write consistent all the time)
				\item eventual consistency (changes are not immediately reflected, but 
eventually will be). 
			\end{itemize}		
\end{frame}

%%%%%%%%%%%%%%%%%%%%%%%%%%%%%%%%%%%%%%%%%%%%%%%%%%%%%%%%%%%%%%%%%%%%%%
\begin{frame}
	\frametitle{SQL vs NoSQL}
		SQL
			\begin{itemize}
				\item SQL is used for relational databases, characterized by their tables and rows, and facilitates 
complex read, write operations like JOIN and CASCADING DELETE. 
\item Typically vertically scalable (adding more memory/processing power). 
\item Use this any time you have complex queries or you 
need high reliability or fixed schema. 
			\end{itemize}
		NoSQL
			\begin{itemize}
				\item NoSQL stores data in JSON or XML (basically a version of HTML used for data storage), and 
schema are dynamic so categories (eq of ‘rows’ can be created on the fly, stored as graphs or 
key-value pairs), usually horizontally scaled (add more servers and get them to sync up with each 
other). 
\item Use these where schema are likely to change or you aren’t going to have complex queries 
or you need speed at scale. 
			\end{itemize}		
\end{frame}

%%%%%%%%%%%%%%%%%%%%%%%%%%%%%%%%%%%%%%%%%%%%%%%%%%%%%%%%%%%%%%%%%%%%%%
\begin{frame}
	\frametitle{Database Sharding}
			\begin{itemize}
				\item Basically horizontal partitioning a database table so that you can distribute it across multiple nodes 
in a distributed network. 
				\item Improves query times and makes database more resilient in case some 
servers are down, but requires more complex infrastructure with more moving parts to get the 
shards to communicate that often has to be built out at the application level (though some 
databases do have it as a feature). 
				\item Physical shards are the nodes in the distributed server system, 
and logical shards are the split up rows of the DB/table. 
				\item There are different strategies to distribute 
data across shards (structurally similar to load balancing)
			\end{itemize}		
\end{frame}

%%%%%%%%%%%%%%%%%%%%%%%%%%%%%%%%%%%%%%%%%%%%%%%%%%%%%%%%%%%%%%%%%%%%%%
\begin{frame}
	\frametitle{Database Locking (optimistic, pessimistic)}
		
			\begin{itemize}
				\item Essentially solves problem of concurrent access to database. 
				\item Pessimistic is easier to implement as 
you just lock out all other users when one is modifying the table (eg flight seats). 
				\item In optimistic, you 
still allow table modifications in parallel—just compare version of change submitted to current version in table; if they don’t match, someone else changed data before you committed code and 
so you have to go back and make your change to the DB again with the new data to avoid 
corrupting code. 
				\item Optimistic is default (and requires manual intervention), pessimistic is easier to 
implement. 
			\end{itemize}		
\end{frame}

%%%%%%%%%%%%%%%%%%%%%%%%%%%%%%%%%%%%%%%%%%%%%%%%%%%%%%%%%%%%%%%%%%%%%%
\begin{frame}
	\frametitle{Caching}
		
			\begin{itemize}
				\item Redis/Memcached are basically cache services on someone else’s hardware. They store your data 
that is read often in their RAM for fast read access, so you don’t have to check your disk on every 
operation. 
				\item Types of Cache include:
			\begin{itemize}
				\item Application Server Cache: Literally on back end of web server—can lead to cache mismatch in a system with multiple servers. 
				\item Distribute Cache: Cache divvied up between nodes in a server, where parts of cache are assigned to 
nodes based on consistent hashing function (you can easily trace where a certain part went since 
the caching function is deterministic).
				\item Global Cache: single cache for all servers to tap into . 
				\item CDN: distributing static media across several servers geographically spaced for ease of 
download. 
			\end{itemize}		

			\end{itemize}		
\end{frame}

%%%%%%%%%%%%%%%%%%%%%%%%%%%%%%%%%%%%%%%%%%%%%%%%%%%%%%%%%%%%%%%%%%%%%%
\begin{frame}
	\frametitle{Consistent Hashing}
		
			\begin{itemize}
				\item A method of uniformly distributing data to different servers for 
storage that is independent of the number of servers. 
				\item Thus, it doesn’t break apart in terms of uniformity when servers are added or taken away, a very useful real-world implication. 
				\item It thinks of hash tables as rings/circles, with servers and data points being nodes dotted along them, where you simply assign a data point to its (say) closest clockwise neighbor. 
				\item The advantage is that if one server is removed, its data points need to be assigned to new servers, sure, but all others remain untouched and uniformity maintained.

			\end{itemize}		
\end{frame}

%%%%%%%%%%%%%%%%%%%%%%%%%%%%%%%%%%%%%%%%%%%%%%%%%%%%%%%%%%%%%%%%%%%%%%
\begin{frame}
	\frametitle{Load Balancers}
		
			\begin{itemize}
				\item These can be hardware (custom build units, expensive but performant) or software (just a server 
re-purposed to load balance incoming requests). 
				\item And it can balance request loads at any point in the stack—to the front end, back end, or back-end making requests to the DB. 
				\item Most companies (startups) use cheap/open source software and re-purpose a server to act as the load balancer. 
				\item Popular load balancing algorithms include round robin (can be weighted), hashing the source IP/
destination URL, or just to the node that has the least response time.
			\end{itemize}		
\end{frame}