%%%%%%%%%%%%%%%%%%%%%%%%%%%%%%%%%%%%%%%%%%%%%%%%%%%%%%%%%%%%%%%%%%%%%%%%%%%%%%%%%%
 \begin{frame}[fragile]\frametitle{}
\begin{center}
{\Large Vector Multiplication}
\end{center}
\end{frame}

%%%%%%%%%%%%%%%%%%%%%%%%%%%%%%%%%%%%%%%%%%%%%%%%%%%%%%%%%%%%%%%%%%%%%%%%%%%%%%%%%%
\begin{frame}[fragile] \frametitle{Vector Multiplication}

Vector Multiplication is slightly complicated that plain Vector Addition. 

There are a few types of it.

\begin{itemize}
\item Scalar into Vector resulting in a vector: e.g. You have a list (a vector) of people's income. Tax rate is 15\%. How do you get a list of Tax amounts?
\item Vector into Vector resulting in a scalar: e.g. You have different amounts of foreign currencies. You know each ones conversion-to-INR rate. How do you compute total INRs you have?
\item Vector into Vector resulting in a vector: e.g. Area of a parallelogram with a right hand rule direction.
\end{itemize}

\end{frame}


%%%%%%%%%%%%%%%%%%%%%%%%%%%%%%%%%%%%%%%%%%%%%%%%%%%%%%%%%%%%%%%%%%%%%%%%%%%%%%%%%%
 \begin{frame}[fragile]\frametitle{}
\begin{center}
{\Large Scalar Vector Multiplication}
\end{center}
\end{frame}


%%%%%%%%%%%%%%%%%%%%%%%%%%%%%%%%%%%%%%%%%%%%%%%%%%%%%%%%%%%
 \begin{frame}[fragile] \frametitle{Scalar Vector Multiplication}

\begin{center}
\includegraphics[width=0.5\linewidth,keepaspectratio]{vec2}
\end{center}

Multiply each element of the vector by the scalar
 
\end{frame}

%%%%%%%%%%%%%%%%%%%%%%%%%%%%%%%%%%%%%%%%%%%%%%%%%%%%%%%%%%%
 \begin{frame}[fragile] \frametitle{Scalar Vector Multiplication}

 \begin{lstlisting}
import numpy as np
import matplotlib.pyplot as plt
import math

v = np.array([2,1])

w = 2 * v
print(w)

# Plot w
origin = [0], [0]
plt.grid()
plt.ticklabel_format(style='sci', axis='both', scilimits=(0,0))
plt.quiver(*origin, *w, scale=10)
plt.show()
 \end{lstlisting}
 

 
\end{frame}

%%%%%%%%%%%%%%%%%%%%%%%%%%%%%%%%%%%%%%%%%%%%%%%%%%%%%%%%%%%
 \begin{frame}[fragile] \frametitle{Scalar Vector Multiplication}


\begin{center}
\includegraphics[width=0.5\linewidth,keepaspectratio]{vec5}
\end{center}

 
\end{frame}

%%%%%%%%%%%%%%%%%%%%%%%%%%%%%%%%%%%%%%%%%%%%%%%%%%%%%%%%%%%
 \begin{frame}[fragile] \frametitle{Scalar Vector Multiplication}

 $\vec{b} = \frac{\vec{v}}{2}$
 
 \begin{lstlisting}
b = v / 2
print(b)

# Plot b
origin = [0], [0]
plt.axis('equal')
plt.grid()
plt.ticklabel_format(style='sci', axis='both', scilimits=(0,0))
plt.quiver(*origin, *b, scale=10)
plt.show()
 \end{lstlisting}
 
[1.  0.5]
 
\end{frame}

%%%%%%%%%%%%%%%%%%%%%%%%%%%%%%%%%%%%%%%%%%%%%%%%%%%%%%%%%%%
 \begin{frame}[fragile] \frametitle{Scalar Vector Multiplication}


\begin{center}
\includegraphics[width=0.5\linewidth,keepaspectratio]{vec6}
\end{center}

 
\end{frame}



%%%%%%%%%%%%%%%%%%%%%%%%%%%%%%%%%%%%%%%%%%%%%%%%%%%%%%%%%%%%%%%%%%%%%%%%%%%%%%%%%%
 \begin{frame}[fragile]\frametitle{}
\begin{center}
{\Large Dot Product}
\end{center}
\end{frame}



%%%%%%%%%%%%%%%%%%%%%%%%%%%%%%%%%%%%%%%%%%%%%%%%%%%%%%%%%%%
 \begin{frame}[fragile] \frametitle{Vector Vector Multiplication}
Dot Product
\begin{center}
\includegraphics[width=0.5\linewidth,keepaspectratio]{vec3}
\end{center}

Multiply the corresponding elements of the vectors and add the results In this
case, 3 times 2 is 6, and 1 times -4 is -4; and adding these together
gives us our scalar result of 2. 

\end{frame}


%%%%%%%%%%%%%%%%%%%%%%%%%%%%%%%%%%%%%%%%%%%%%%%%%%%%%%%%%%%
 \begin{frame}[fragile] \frametitle{Vector Vector Multiplication}

$\vec{v} \cdot \vec{s} = (v_{1} \cdot s_{1}) + (v_{2} \cdot s_{2}) ... + \; (v_{n} \cdot s_{n})$
 
 \begin{lstlisting}
import numpy as np

v = np.array([2,1])
s = np.array([-3,2])
d = np.dot(v,s)
print (d)
 \end{lstlisting}
 
-4
 
\end{frame}


%%%%%%%%%%%%%%%%%%%%%%%%%%%%%%%%%%%%%%%%%%%%%%%%%%%%%%%%%%%
 \begin{frame}[fragile] \frametitle{Vector Vector Multiplication}

 \begin{itemize}
 
\item Another form: $\vec{v} \cdot \vec{s} = \|\vec{v} \|\|\vec{s}\| \cos (\theta)$ 

\item So for our vectors v (2,1) and s (-3,2), our calculation looks like this:

\item $\cos(\theta) = \frac{(2 \cdot-3) + (-3 \cdot 2)}{\sqrt{2^{2} + 1^{2}} \times \sqrt{-3^{2} + 2^{2}}}$

\item So $\cos(\theta) = -0.496138938357$

\item $\theta \approx 119.74$
\end{itemize}
 
\end{frame}

%%%%%%%%%%%%%%%%%%%%%%%%%%%%%%%%%%%%%%%%%%%%%%%%%%%%%%%%%%%
 \begin{frame}[fragile] \frametitle{Angle Between Two Vectors}

 \begin{itemize}
 
\item Suppose we have two vectors $\vec{v} = (v, 0)$ lying on X axis and $\vec{w}= (w_1,w_2)$
\item $w_1 = ||\vec{w}||cos\theta$, so $\theta = cos^{-1}(\frac{w_1}{||\vec{w}||})$
\item Now, dot product is given as $\vec{v}\cdot\vec{w} = v_1.w_1 + 0.w_2 = v_1.w_1$
\item Putting value of $w_1$, eqn becomes $= v_1.||\vec{w}||cos\theta = ||\vec{v}||||\vec{w}||cos\theta$
\item Therefore: $cos\theta = \frac{\vec{v}\cdot\vec{w}}{||\vec{v}||||\vec{w}||}$
\item Applicable to Higher Dimensions also!!
\end{itemize}
 
\end{frame}

%%%%%%%%%%%%%%%%%%%%%%%%%%%%%%%%%%%%%%%%%%%%%%%%%%%%%%%%%%%%%%%%%%%%%%%%%%%%%%%%%%
\begin{frame}[fragile] \frametitle{Definition}

Suppose that $\vec{u}, \vec{v}\in \mathbb R^n$.  We define the
 \textbf{inner product} or \textbf{dot procuct} or $\vec{u}$ and $\vec{v}$ as
 $$
 u\cdot v = u^tv=\sum_{i=1}^n u_i v_i.
 $$


\textbf{Example}
$$
 \left[ \begin{array}{r} 1\\2\\3 \end{array}\right] \cdot   \left[\begin{array}{r} -1\\-2\\1 \end{array}\right] =
 (1)(-1) + (2)(-2) + (3)(1) = -2.
 $$

\end{frame}

% %%%%%%%%%%%%%%%%%%%%%%%%%%%%%%%%%%%%%%%%%%%%%%%%%%%%%%%%%%%%%%%%%%%%%%%%%%%%%%%%%%
% \begin{frame}[fragile] \frametitle{Theorem}

% Let $\vec{u}, \vec{v}, \vec{w} \in \mathbb R^n$ and let $c$  be a scalar.
% Then 
% \begin{itemize}
 % \item $\vec{u}\cdot \vec{v} = \vec{v} \cdot \vec{u}$.
% \item $(\vec{u} + \vec{v})\cdot \vec{w} = \vec{u}\cdot \vec{w}
% +\vec{v}\cdot \vec{w}$.
% \item $(c \vec{u})\cdot \vec{v}= c (\vec{u}\cdot \vec{v})$.
% \item $\vec{u} \cdot \vec{u} \geq 0$ and $\vec{u}\cdot \vec{u} = 0$ 
% if and only if $\vec{u} = \vec{0}$.
% \item $(c_1 \vec{u}_1 +\dots + c_p \vec{u}_p)\cdot \vec{w} = c_1\vec{u}_1\cdot \vec{w}
% +c_p\vec{u}_p\cdot \vec{w}$.
% \end{itemize}
 

% \textbf{Definition}[Length]
% The {\em length} (or {\em norm})  of $\vec{v}$ is the non-negative scalar $\|\vec{v}\|$ 
% defined by 
% \[
 % \| \vec{v} \| = \sqrt{\vec{v} \cdot \vec{v}} =
% \sqrt{v_1^2 + v_2^2 +\dots + v_n^2} \ \text{ and } \ \|\vec{v} \|^2 = 
% \vec{v}\cdot \vec{v}.
 % \]

% \end{frame}


% %%%%%%%%%%%%%%%%%%%%%%%%%%%%%%%%%%%%%%%%%%%%%%%%%%%%%%%%%%%%%%%%%%%%%%%%%%%%%%%%%%
% \begin{frame}[fragile] \frametitle{Note}
 % This definition is chosen so that the Pythagorean theorem holds (that is, in two 
% dimensions the length $c$ of the vector which is the hypotenuse of a right triangle 
% with horizontal length $a$ and vertical height $b$ satisfies $a^2 + b^2 = c^2$. 
 

% \textbf{Facts}
% \begin{itemize}
 % \item For any scalar $c$, $\| c\vec{v}\| = |c| \ \|\vec{v}\|$.  
 % \item A vector of length 1 is called a \textbf{unit vector}.  
 % \item  If $\vec{v}\neq \vec{0}$ then $\frac{1}{\| \vec{v} \|} \vec{v}$ 
% is a unit vector and is in the same direction as $\vec{v}$.   
 % \item  The above process is called normalizing. 
% \end{itemize}

% \end{frame}


% %%%%%%%%%%%%%%%%%%%%%%%%%%%%%%%%%%%%%%%%%%%%%%%%%%%%%%%%%%%%%%%%%%%%%%%%%%%%%%%%%%
% \begin{frame}[fragile] \frametitle{Example}
 % Find a unit vector which is in the same direction as $\vec{v} =
% \left[ \begin{array}{rrrr} 1 \\ 1 \\ 2 \end{array} \right]$.
 

% \textbf{Definition}
% For $\vec{u}$ and $\vec{v}$ in $\mathbb R^n$, the {\em distance between $\vec{u}$
% and $\vec{v}$}, written as $\text{dist}(\vec{u},\vec{v})$ is the length
% of the vector $\vec{u}-\vec{v}$.  That is,
% \[
 % \text{dist}(\vec{u},\vec{v})  = \| \vec{u} - \vec{v} \|.
% \]
  

% \textbf{Example}
 % Compute the distance 
% $\text{dist}\left[\left[ \begin{array}{rrrr}7 \\ 2 \end{array} \right],\left[ \begin{array}{rrrr} 4 \\ 3 \end{array} \right]\right]$.

% \end{frame}

% %%%%%%%%%%%%%%%%%%%%%%%%%%%%%%%%%%%%%%%%%%%%%%%%%%%%%%%%%%%%%%%%%%%%%%%%%%%%%%%%%%
% \begin{frame}[fragile] \frametitle{Orthogonal Vectors}
 % Two vectors in $\mathbb R^n$ are {\em orthogonal} if and only if $\vec{u}\cdot\vec{v}=0$.

% \begin{itemize}
% \item Angle between two vectors is 0, meaning $cos\theta = 0$ meaning $\theta = 90$ degrees.
% \item In 2D, there will be two such orthogonal (perpendicular) vectors to a given vector.
% \item if the given vector is 3D, there will be infinite such orthogonal vectors, but all will lie in a single plane.
% \item In higher dimensions, the orthogonal entity is called as a Hyperplane, all passing through base of the vector.
% \item Hyperplane separates space into 2  halves.
% \end{itemize}

% \end{frame}

% %%%%%%%%%%%%%%%%%%%%%%%%%%%%%%%%%%%%%%%%%%%%%%%%%%%%%%%%%%%%%%%%%%%%%%%%%%%%%%%%%%
% \begin{frame}[fragile] \frametitle{Orthogonal Vectors}
 % Two vectors in $\mathbb R^n$ are {\em orthogonal} if and only if $\vec{u}\cdot\vec{v}=0$.
  

% \textbf{Note}
% \begin{eqnarray*}
 % \| \vec{u} -\vec{v}\|^2 & =  & (\vec{u}-\vec{v})\cdot(\vec{u}- \vec{v})\\
% & = & \vec{u}\cdot \vec{u} - \vec{u} \cdot \vec{v} 
      % - \vec{v} \cdot \vec{u} + \vec{v}\vec{v} \\
% & = & \| \vec{u} \|^2 + \| \vec{v}\|^2 - 2 \vec{u}\cdot\vec{v}
% \end{eqnarray*}


% \textbf{Theorem}[Pythagorus]
 % $||\vec{u} - \vec{v}||^2 = ||\vec{u}||^2 + ||\vec{v}||^2$ if and only if $\vec{u}$ and $\vec{v}$ are orthogonal.


% \end{frame}

% %%%%%%%%%%%%%%%%%%%%%%%%%%%%%%%%%%%%%%%%%%%%%%%%%%%%%%%%%%%%%%%%%%%%%%%%%%%%%%%%%%
% \begin{frame}[fragile] \frametitle{Definition}

% Suppose that $W\le \mathbb R^n$.
% \begin{itemize}
  % \item  If for all $\vec{w}\in W$, $\vec{z}\cdot\vec{w}=0$,l then we say that $\vec{z}$ is orthogonal to $W$ and write $z\perp W$.
  % \item We define the orthogonal compliment of $W$ as $W^{\perp}=\{\vec{z}\in \mathbb R^n \vec{z}\perp W\}$.
% \end{itemize}
  

% \textbf{Example}
 % Suppose that $W=\mbox{Span} \left(\left(\begin{array}{r}1\\0\\0\end{array}\right), \left(\begin{array}{r} 0\\1\\0\end{array}\right)  \right)$.  \\
 % Describe $W^{\perp}$.

% \end{frame}


% %%%%%%%%%%%%%%%%%%%%%%%%%%%%%%%%%%%%%%%%%%%%%%%%%%%%%%%%%%%%%%%%%%%%%%%%%%%%%%%%%%
% \begin{frame}[fragile] \frametitle{Fact}
% \begin{itemize}
  % \item $\vec{x}\in W^{\perp}$ if and only if $\vec{x}\perp \vec{w}$ for all $\vec{w}$ in a spanning set for $W$.
  % \item  $W^{\perp}\le \mathbb R^n$.
% \end{itemize}
  

% \textbf{Theorem}
 % Suppose that $A$ is an $n\times n$ matrix.
% \begin{itemize}
  % \item $(\mbox{Row}(A))^{\perp} = \mbox{Nul}(A)$.
  % \item $(\mbox{Col}(A))^{\perp}=\mbox{Nul}(A^t)$.
% \end{itemize}

% \end{frame}


% %%%%%%%%%%%%%%%%%%%%%%%%%%%%%%%%%%%%%%%%%%%%%%%%%%%%%%%%%%%%%%%%%%%%%%%%%%%%%%%%%%
% \begin{frame}[fragile] \frametitle{Note}
 % In two or three dimensions, the projection of $\vec{v}$ onto $\vec{u}$
% has length $\| \vec{v} \| \cos \theta$, where $\theta$ is the angle between
% the vectors.  Hence
% \[
 % \| \vec{v} \| \cos \theta = c \| \vec{u} \|
% \]
% so that 
% \[
 % \vec{u}\cdot\vec{v} = \| \vec{u} \| \| \vec{v} \| \cos \theta.
% \]
% In higher dimensions than three we use this to {\em define} the angle between two vectors.

% \end{frame}

% %%%%%%%%%%%%%%%%%%%%%%%%%%%%%%%%%%%%%%%%%%%%%%%%%%%%%%%%%%%%%%%%%%%%%%%%%%%%%%%%%%
% \begin{frame}[fragile] \frametitle{Definition}
 % Two vectors in $\mathbb R^n$ are {\em orthogonal} if and only if $\vec{u}\cdot\vec{v}=0$.
  

% \textbf{Note}
% \begin{eqnarray*}
 % \| \vec{u} -\vec{v}\|^2 & =  & (\vec{u}-\vec{v})\cdot(\vec{u}- \vec{v})\\
% & = & \vec{u}\cdot \vec{u} - \vec{u} \cdot \vec{v} 
      % - \vec{v} \cdot \vec{u} + \vec{v}\vec{v} \\
% & = & \| \vec{u} \|^2 + \| \vec{v}\|^2 - 2 \vec{u}\cdot\vec{v}
% \end{eqnarray*}
 

% \textbf{Theorem}[Pythagorus]
 % $||\vec{u} - \vec{v}||^2 = ||\vec{u}||^2 + ||\vec{v}||^2$ if and only if $\vec{u}$ and $\vec{v}$ are orthogonal.


% \end{frame}



% %%%%%%%%%%%%%%%%%%%%%%%%%%%%%%%%%%%%%%%%%%%%%%%%%%%%%%%%%%%%%%%%%%%%%%%%%%%%%%%%%%
% \begin{frame}[fragile] \frametitle{Inner products}

% \textbf{Definition}: An {\em inner product} on a real vector space $V$ is an operation (function) that assigns to each pair of vectors $(\vec{u}, \vec{v})$ in $V$ a \textbf{scalar} $\avg{ \vec{u}, \vec{v} }$ satisfying the following axioms:

% \begin{itemize}

% \item  $\avg{ \vec{u}, \vec{v} } =  \avg{ \vec{v}, \vec{u} } $   \ (Symmetry)

% \item   $\avg{ \vec{u} + \vec{v}, \vec{w} }   =  \avg{ \vec{u}, \vec{w} } +  \avg{ \vec{v}, \vec{w} }$  \ (Additivity)

% \item  $\avg{ k \, \vec{u}, \vec{v} }  = k  \, \avg{ \vec{u}, \vec{v} }$ \ (Homogeneity)

% \item  $\avg{ \vec{v}, \vec{v} } \, \ge 0$  \ and  $\avg{ \vec{v}, \vec{v} } = 0$ iff $\vec{v} = \vec{0}$ \ (Positivity)


% \end{itemize}





% \textbf{Theorem} (basic properties): Given vectors $\vec{u}, \vec{v}, \vec{w}$ in an inner product space $V$, and a scalar $k$, the following properties hold:

% \begin{itemize}

% \item $\avg{ \vec{\rm{o}}, \vec{v} } = \avg{ \vec{v}, \vec{\rm{o}} }  = 0$

% \item $\avg{ \vec{u} - \vec{v}, \vec{w} }   =  \avg{ \vec{u}, \vec{w} } -  \avg{ \vec{v}, \vec{w} }$

% \item $\avg{ \vec{u}, \vec{v} + \vec{w} }   =  \avg{ \vec{u}, \vec{v} } +  \avg{ \vec{u}, \vec{w} }$

% \item $\avg{ \vec{u}, \vec{v} - \vec{w} }   =  \avg{ \vec{u}, \vec{v} }  -  \avg{ \vec{u}, \vec{w} }$

% \item  $\avg{ \vec{u}, k \vec{ v} }  = k  \, \avg{ \vec{u}, \vec{v} }$

% \end{itemize}

% \end{frame}


% %%%%%%%%%%%%%%%%%%%%%%%%%%%%%%%%%%%%%%%%%%%%%%%%%%%%%%%%%%%%%%%%%%%%%%%%%%%%%%%%%%
 % \begin{frame}[fragile]\frametitle{Norm and distance in an inner product space}

% \textbf{Definition}: If $V$ is a real inner product space then we define

% \begin{itemize}

% \item The norm (or length) of $\vec{v}$:
% $$\norm{\vec{v}} = \sqrt{ \avg{ \vec{v}, \vec{v} } } $$

% \item The distance between $\vec{u}$ and $\vec{v}$:
% $$ d (\vec{u}, \vec{v}) = \norm{ \vec{u} - \vec{v}} =  \sqrt{ \avg{ \vec{u} - \vec{v}, \vec{u} - \vec{v} } } $$

% \end{itemize}



% \textbf{Theorem} (basic properties): Given vectors $\vec{u}, \vec{v}$ in an inner product space $V$, and a scalar $k$, the following properties hold:

% \begin{itemize}

% \item $\norm{\vec{v}} \ge 0$ \ and \ $\norm{\vec{v}} = 0$ iff $\vec{v} = \vec{0}$.

% \item $\norm{ k \vec{v}} = |k| \, \norm{\vec{v}}$

% \item $d (\vec{u}, \vec{v}) = d (\vec{v}, \vec{u}) $

% \item $d (\vec{u}, \vec{v} ) \ge 0$ \ and \ $d (\vec{u}, \vec{v}) = 0$ iff $\vec{u} = \vec{v}$.


% \end{itemize}

% \end{frame}


% %%%%%%%%%%%%%%%%%%%%%%%%%%%%%%%%%%%%%%%%%%%%%%%%%%%%%%%%%%%%%%%%%%%%%%%%%%%%%%%%%%
 % \begin{frame}[fragile]\frametitle{Angle between vectors}

% \textbf{Theorem} (Cauchy-Schwarz): If $u$ and $v$ are vectors in an inner vector space, then
% $$\abs{ \avg{u, v} } \le \norm{u} \, \norm{v} $$




% \textbf{Definition}: The angle between two vectors  $u$ and $v$ in an inner vector space is defined as
% $$\theta = \cos^{-1} \, \frac{\avg{u,v}}{\norm{u} \, \norm{v}}  $$





% \textbf{Theorem} (the triangle inequality):  If $u, v$ and $w$ are vectors in an inner vector space, then
% \begin{itemize}
% \item $\norm{u + v} \le \norm{u} + \norm{v}$

% \item $d (u, v) \le d (u, w) + d (w, v)$
% \end{itemize}

% \end{frame}


% %%%%%%%%%%%%%%%%%%%%%%%%%%%%%%%%%%%%%%%%%%%%%%%%%%%%%%%%%%%
 % \begin{frame}[fragile] \frametitle{Orthogonality}
% \textbf{Definition}:  Two vectors  $u$ and $v$ in an inner vector space are called {\em orthogonal} if $\avg{u, v} = 0$.
% \smallskip
% Clearly $u \perp v$ iff the angle between them is $\theta = \frac{\pi}{2}$.
% \textbf{Theorem} (the Pythagorean theorem):  If $u$ and $v$ are \textbf{orthogonal} vectors in an inner vector space, then
% $$\norm{u + v}^2 = \norm{u}^2 + \norm{v}^2$$
% \end{frame}

% %%%%%%%%%%%%%%%%%%%%%%%%%%%%%%%%%%%%%%%%%%%%%%%%%%%%%%%%%%%%%%%%%%%%%%%%%%%%%%%%%%
% \begin{frame}[fragile]\frametitle{Orthogonality}
% \textbf{Definition}:  Let $W$ be a subspace of an inner product space $V$. 
% The set of vectors in $V$ which are orthogonal to \textbf{every} vector in $W$ is called the {\em orthogonal complement} of $W$ and it is denoted by $W^\perp$.
% \textbf{Theorem}: The orthogonal complement has the following properties:
% \begin{itemize}
% \item $W^\perp$ is a subspace of $V$.
% \item $W \cap W^\perp = \{ \vec{\rm{o}} \}$.
% \item If $V$ has finite dimension then $(W^\perp)^\perp = W$.
% \end{itemize}
% \end{frame}

% %%%%%%%%%%%%%%%%%%%%%%%%%%%%%%%%%%%%%%%%%%%%%%%%%%%%%%%%%%%%%%%%%%%%%%%%%%%%%%%%%%
 % \begin{frame}[fragile]\frametitle{Orthogonal sets, orthonormal sets}
% Let $(V, \avg{ \ })$ be an inner product space and let $S$ be a set of vectors in $V$.
% \textbf{Definition}: The set $S$  is called {\em orthogonal} if any two vectors in $S$  are orthogonal.
% The set $S$ is called {\em orthonormal} if it is orthogonal and any vector in $S$ has norm $1$.
% \textbf{Theorem}: Every orthogonal set of nonzero vectors is linearly independent.

% \end{frame}

% %%%%%%%%%%%%%%%%%%%%%%%%%%%%%%%%%%%%%%%%%%%%%%%%%%%%%%%%%%%%%%%%%%%%%%%%%%%%%%%%%%
 % \begin{frame}[fragile]
% \frametitle{Orthogonal sets, orthonormal sets}
% \textbf{Definition}: A set of vectors $S$  is called an {\em orthogonal} basis (OGB) for $V$ if $S$ is a basis and an orthogonal set (that is, $S$ is a basis where all vectors are perpendicular).

% A set of vectors $S$  is called an {\em orthonormal} basis (ONB) for $V$ if $S$ is a basis and an orthonormal set (that is, $S$ is a basis where all vectors are perpendicular and have norm $1$).
% \end{frame}

% %%%%%%%%%%%%%%%%%%%%%%%%%%%%%%%%%%%%%%%%%%%%%%%%%%%%%%%%%%%%%%%%%%%%%%%%%%%%%%%%%%
 % \begin{frame}[fragile]\frametitle{Orthogonal sets, orthonormal sets}

% Let $(V, \avg{ \ })$ be an inner product space.

% \textbf{Theorem}:  If $S = \{ v_1, v_2, \ldots, v_n \}$ is an orthogonal basis in $V$ and $u$ is any vector in $V$, then
% $$u = \frac{\avg{u, v_1}}{\norm{v_1}^2} \, v_1 + \frac{\avg{u, v_2}}{\norm{v_2}^2} \, v_2 + \ldots + \frac{\avg{u, v_n}}{\norm{v_n}^2} \, v_n$$

% If $S = \{ v_1, v_2, \ldots, v_n \}$ is an orthonormal basis in $V$ and $u$ is any vector in $V$, then
% $$u = \avg{u, v_1} \, v_1 + \avg{u, v_2} \, v_2  + \ldots + \avg{u, v_n} \, v_n$$ 

% \end{frame}


%%%%%%%%%%%%%%%%%%%%%%%%%%%%%%%%%%%%%%%%%%%%%%%%%%%%%%%%%%%%%%%%%%%%%%%%%%%%%%%%%%
 \begin{frame}[fragile]\frametitle{}
\begin{center}
{\Large Cross Product}
\end{center}
\end{frame}


%%%%%%%%%%%%%%%%%%%%%%%%%%%%%%%%%%%%%%%%%%%%%%%%%%%%%%%%%%%
 \begin{frame}[fragile] \frametitle{Vector Vector Multiplication}
Cross Product (for 3D vectors)
\begin{center}
\includegraphics[width=0.5\linewidth,keepaspectratio]{vec4}
\end{center}
Skipping the current row and column, calculate determinant value of remaining sub matrix for that position.
\end{frame}


%%%%%%%%%%%%%%%%%%%%%%%%%%%%%%%%%%%%%%%%%%%%%%%%%%%%%%%%%%%
 \begin{frame}[fragile] \frametitle{Vector Vector Multiplication}
Cross Product
 \begin{itemize}
 
\item $\vec{p} = \begin{bmatrix}2 \\ 3 \\ 1 \end{bmatrix}\;\; \vec{q} = \begin{bmatrix}1 \\ 2 \\ -2 \end{bmatrix}$
\item \begin{align}r_{1} = p_{2}q_{3} - p_{3}q_{2} \\ r_{2} = p_{3}q_{1} - p_{1}q_{3} \\ r_{3} = p_{1}q_{2} - p_{2}q_{1} \end{align}
\item $\vec{r} = \vec{p} \times \vec{q} = \begin{bmatrix}(3 \cdot -2) - (1 \cdot 2) \\ (1 \cdot 1) - (2 \cdot -2) \\ (2 \cdot 2) - (3 \cdot 1) \end{bmatrix} = \begin{bmatrix}-6 - 2 \\ 1 - -4 \\ 4 - 3 \end{bmatrix} = \begin{bmatrix}-8 \\ 5 \\ 1 \end{bmatrix}$
\end{itemize}
 
\end{frame}


%%%%%%%%%%%%%%%%%%%%%%%%%%%%%%%%%%%%%%%%%%%%%%%%%%%%%%%%%%%
 \begin{frame}[fragile] \frametitle{Vector Vector Multiplication}

Cross Product
 
 \begin{lstlisting}
import numpy as np

p = np.array([2,3,1])
q = np.array([1,2,-2])
r = np.cross(p,q)
print (r)
 \end{lstlisting}
 
[-8  5  1]
\end{frame}

% %%%%%%%%%%%%%%%%%%%%%%%%%%%%%%%%%%%%%%%%%%%%%%%%%%%%%%%%%%%%%%%%%%%%%%%%%%%%%%%%%%
 % \begin{frame}[fragile]\frametitle{}
% \begin{center}
% {\Large Linear Combinations}
% \end{center}
% \end{frame}




% %%%%%%%%%%%%%%%%%%%%%%%%%%%%%%%%%%%%%%%%%%%%%%%%%%%%%%%%%%%
  % \begin{frame}[fragile]\frametitle{Linear Combinations}
% \textbf{Definition}
% Let $p$ be a positive integer. Given vectors $\vec{u}_1, \vec{u}_2, \dots \vec{u}_p$
% in $\mathbb R^n$, and $c_1, c_2, \dots c_p$ in $\mathbb R$, the vector
% \[
 % \vec{u} = c_1 \vec{u}_1 + c_2 \vec{u}_2 + \dots c_p\vec{u}_p
% \]
% is called a linear combination of the vectors $\vec{u}_1, \dots \vec{u}_p$ with
% weights $c_1, \dots c_p$.

% \end{frame}

% %%%%%%%%%%%%%%%%%%%%%%%%%%%%%%%%%%%%%%%%%%%%%%%%%%%%%%%%%%%
  % \begin{frame}[fragile]
% \textbf{Example}
% \[
 % 2\left[\begin{array}{r}1 \\-2 \\ 7  \end{array}\right]\ 
 % + \ 3\left[\begin{array}{r} 1 \\ 1 \\ 1 \end{array}\right] \ 
 % + \ -2\left[\begin{array}{r}2 \\ 3 \\ -8  \end{array}\right]
 % = \left[\begin{array}{r}\qquad  \\ \qquad \\ \qquad \end{array}\right] \]     
% is a linear combination of $ \left[\begin{array}{r}1 \\-2 \\ 7  \end{array}\right],$ $\left[\begin{array}{r} 1 \\ 1 \\ 1 \end{array}\right]$ and $\left[\begin{array}{r}2 \\ 3 \\ -8  \end{array}\right]$ with weights $2,3 ,-2$.


% \textbf{Geometry} 
% Let $\vec{u} = \left[\begin{array}{r} 1 \\ 1 \end{array}\right]$ and 
% $\vec{v} = \left[\begin{array}{r} 1 \\ -1 \end{array}\right]$.  Show all
% linear combinations of $\vec{u}$ and $\vec{v}$ on a graph.

% \end{frame}

% %%%%%%%%%%%%%%%%%%%%%%%%%%%%%%%%%%%%%%%%%%%%%%%%%%%%%%%%%%%
  % \begin{frame}[fragile]
% \textbf{Exercise}
% let 
% $\vec{u}_1 = \left[\begin{array}{r} 1 \\ 2 \\ 5 \end{array}\right]$
% $\vec{u}_2 = \left[\begin{array}{r} 2 \\ 1 \\ 3 \end{array}\right]$ and 
% $\vec{b} = \left[\begin{array}{r} -4 \\ 1 \\ 1 \end{array}\right]$
% Is $\vec{b}$ a linear combination of $\vec{u}_1$ and $\vec{u}_2$

% \end{frame}

% %%%%%%%%%%%%%%%%%%%%%%%%%%%%%%%%%%%%%%%%%%%%%%%%%%%%%%%%%%%
  % \begin{frame}[fragile]\frametitle{Vector Equations}
% \textbf{Facts}
% A vector equation 
% \[
 % x_1 \vec{v}_1  + x_2 \vec{v}_2 + \dots + x_n \vec{v}_n =\vec{b}
% \]
% has the same solution set as the system of equations whose augmented matrix is 
% \[
 % \left[\begin{array}{ccccc}
  % | & | & \dots & | & | \\
  % \vec{v}_1 & \vec{v}_2 & \dots & \vec{v}_n & \vec{b} \\
  % | & | & \dots & | & | \\
 % \end{array}\right]
% \]
% In particular, $\vec{b}$ is a linear combination of $\vec{v}_1, \dots \vec{v}_n$
% if and only if the system of linear equations is consistent.

% \end{frame}

% %%%%%%%%%%%%%%%%%%%%%%%%%%%%%%%%%%%%%%%%%%%%%%%%%%%%%%%%%%%
  % \begin{frame}[fragile]
% \textbf{Definition}
 % Suppose $\vec{v}_1, \vec{v}_2, \dots \vec{v}_p \in \mathbb R^n$.
 % We define 
 % \[
  % \mbox{Span}(\vec{v}_1, \vec{v}_2, \dots ,\vec{v}_p) =
 % \left\{
 % c_1 \vec{v}_1 + c_2 \vec{v}_2 + \dots c_p \vec{v}_p: c_1, c_2, \dots , c_p \in \mathbb R 
 % \right\}
 % \]    
 % That is, Span$(\vec{v}_1, \dots , \vec{v}_p)$ is the set of all linear combinations
% of $\vec{v}_1, \dots , \vec{v}_p$.

% \end{frame}

% %%%%%%%%%%%%%%%%%%%%%%%%%%%%%%%%%%%%%%%%%%%%%%%%%%%%%%%%%%%
  % \begin{frame}[fragile]\frametitle{Geometry}
% \textbf{Note}
% \begin{itemize}
% \item The span of $\vec{0}$ in $\mathbb R^2$ or $\mathbb R^3$ is the single point $\vec{0}$.
% \item The span of a single non-zero vector in $\mathbb R^2$ or $\mathbb R^3$ is a line through $\vec{0}$.
% \item The span of two non-zero vectors in $\mathbb R^3$ is either a plane through $\vec{0}$ or, if one
% vector is a scalar multiple of the other, a line through $\vec{0}$.
% \end{itemize}


% \textbf{Exercise}
% Let $\vec{v}_1 = \left[\begin{array}{r}1 \\ 1 \\ 2 \end{array}\right]$, 
% $\vec{v}_2 = \left[\begin{array}{r} 2 \\ 5 \\ -3 \end{array}\right]$ and 
% $\vec{b} = \left[\begin{array}{r} -9 \\ -30 \\ 31 \end{array}\right]$.
% Span$(\vec{v}_1,\vec{v}_2)$ is a plane in $\mathbb R^3$  Is $\vec{b}$ in that plane?

% \end{frame}
