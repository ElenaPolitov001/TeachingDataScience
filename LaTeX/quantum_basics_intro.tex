%%%%%%%%%%%%%%%%%%%%%%%%%%%%%%%%%%%%%%%%%%%%%%%%%%%%%%%%%%%%%%%%%%%%%%%%%%%%%%%%%%
\begin{frame}[fragile]\frametitle{}
\begin{center}
{\Large Land of Quantum Computing}
\end{center}
\end{frame}

%%%%%%%%%%%%%%%%%%%%%%%%%%%%%%%%%%%%%%%%%%%%%%%%%%%%%%%%%%%
 \begin{frame}[fragile]\frametitle{What's wrong in computing today}
\begin{itemize}
	\item Not enough resolution on displays
	\item Not enough processing power and memory
	\item Not enough parallelism
	\item Software tools are ``flat'' and sequential \newline
		rather than hierarchical
\end{itemize}
	
\tiny{(Ref: A brief overview of quantum computing - Tom Carter)}

\end{frame}



%%%%%%%%%%%%%%%%%%%%%%%%%%%%%%%%%%%%%%%%%%%%%%%%%%%%%%%%%%%
 \begin{frame}[fragile]\frametitle{Qubit zoo: Quantum vocabulary and terminology}
\begin{itemize}
\item  Qubits not bits. Quantum computers do calculations with quantum bits, or qubits, rather than the digital bits in traditional computers. Qubits allow quantum computers to consider previously unimaginable amounts of information.

\item  Superposition. Quantum objects can be in more than one state at the same time, a situation depicted by Schrödinger’s cat, a fictional feline that is simultaneously alive and dead. For example, a qubit can represent the values 0 and 1 simultaneously, whereas classical bits can only be either a 0 or a 1.

\item  Entanglement. When qubits are entangled, they form a connection to each other that survives no matter the distance between them. A change to one qubit will alter its entangled twin, a finding that baffled even Einstein, who called entanglement “spooky action at a distance.”

\item  Types of qubits. At the core of the quantum computer is the qubit, a quantum bit of information typically made from a particle so small that it exhibits quantum properties rather than obeying the classical laws of physics that govern our everyday lives. Several types of qubits are in development
\end{itemize}

	
\tiny{(Ref: The Emerging Paths Of Quantum Computing - Chuck Brooks)}

\end{frame}

%%%%%%%%%%%%%%%%%%%%%%%%%%%%%%%%%%%%%%%%%%%%%%%%%%%%%%%%%%%
 \begin{frame}[fragile]\frametitle{Types of Qubits}
\begin{itemize}
\item  Superconducting qubits, or transmons. Already in use in prototype computers made by Google, IBM and others, these qubits are made from superconducting electrical circuits.
\item  Trapped atoms. Atoms trapped in place by lasers can behave as qubits. Trapped ions (charged atoms) can also act as qubits.
\item  Silicon spin qubits. An up-and-coming technology involves trapping electrons in silicon chambers to manipulate a quantum property known as spin.
\item  Topological qubits. Still quite early in development, quasi-particles called Majorana fermions, which exist in certain materials, have the potential for use as qubits. Quantum computing: Opening new realms of possibilities (princeton.edu)
\end{itemize}

	
\tiny{(Ref: The Emerging Paths Of Quantum Computing - Chuck Brooks)}

\end{frame}


%%%%%%%%%%%%%%%%%%%%%%%%%%%%%%%%%%%%%%%%%%%%%%%%%%%%%%%%%%%%%%%%%%%%%%%%%%%%%%%%%%
\begin{frame}[fragile]\frametitle{Quantum AI Google}
\begin{itemize}
\item Classical computers and quantum computers are used to solve different types of problems. 
\item For classical computing, represent the world as bits of information in sequences of zeros and ones, and use logic to make decisions. 
\item NISQ computers can perform tasks with imperfect reliability, but beyond the capability of classical computers. Even with imperfect reliability, we can advance our knowledge of science in the NISQ era.
\item Chemical reactions are most accurately described by quantum mechanics.
\item Once we have high-quality physical qubits organized into a logical qubit, the next challenge is scale. With 1000 qubits, it should be possible to store quantum data for nearly a year. Scaling up requires making many of our components smaller: including our wiring, amplifiers, filters, and electronics.
\item The next task will be to stitch several logical qubits together so that they can all work together on increasingly complex algorithms. For this, we’ll need coherence between chips so that the processor can communicate across qubits. 
\end{itemize}
\end{frame}

%%%%%%%%%%%%%%%%%%%%%%%%%%%%%%%%%%%%%%%%%%%%%%%%%%%%%%%%%%%%%%%%%%%%%%%%%%%%%%%%%%
\begin{frame}[fragile]\frametitle{QnA with Himanshu Vaidya}


\begin{itemize}
\item Q - Why do a ML Engineering has to bother about underlying hardware, it can be CPU, GPU, TPU or QPU?
\item A - QPU is not an improved and similar replaceable processor. Quantum computing paradigm itself is very different than normal computing. The classical computing is based on bits (0s, 1s) where is Quantum is on Qubits. This data type definitions, the data storage methodologies are very different. The abstractions are not yet developed to make QPU replaceable.
\end{itemize}

\begin{itemize}
\item Q - How practical is it to use Quantum Computing for ML?
\item A - Not all ML problems can be solved using Quantum. Similar to parallel processing paradigm, where problems themselves are split to be parallel, some problems themselves have to be quantumiz-able. Even with 70 qubits, with overlapping/superimpose storage, huge simulations can be solved.
\end{itemize}
\end{frame}