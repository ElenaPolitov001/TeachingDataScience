%%%%%%%%%%%%%%%%%%%%%%%%%%%%%%%%%%%%%%%%%%%%%%%%%%%%%%%%%%%%%%%%%%%%%%%%%%%%%%%%%%
\begin{frame}[fragile]\frametitle{}
\begin{center}
{\Large Describing Data}
\end{center}
\end{frame}

%%%%%%%%%%%%%%%%%%%%%%%%%%%%%%%%%%%%%%%%%%%%%%%%%%%%%%%%%%
\begin{frame}[fragile]\frametitle{Describing Data}
\begin{itemize}
\item Univariate Analysis: Statistical moments (based on degree)
\begin{itemize}
\item 1st degree: Central tendency: mean, median, mode
\item 2nd degree: Standard Deviation: how wide data is around mean
\item 3rd degree: Skewness: Asymmetric around mean
\item 4th degree: Kurtosis: shape of skewness.
\end{itemize}
\item BiVariate Analysis: covariance, correlation
\item MultiVariate Analysis
\end{itemize}
\end{frame}


%%%%%%%%%%%%%%%%%%%%%%%%%%%%%%%%%%%%%%%%%%%%%%%%%%%%%%%%%%%
\begin{frame}[fragile]\frametitle{Univariate Analysis}
\begin{itemize}
\item Measure of Central Tendency
\item Measure of Spread
\item Measure of Asymmetry
\item Measure of Skewness
\end{itemize}
\end{frame}


%%%%%%%%%%%%%%%%%%%%%%%%%%%%%%%%%%%%%%%%%%%%%%%%%%%%%%%%%%%
%\begin{frame}[fragile]\frametitle{Frequency}	
%$ frequency(v_i) = \frac{Objects\_with\_Attribute\_v_i}{n}$
%\begin{center}
%\includegraphics[width=0.8\linewidth,keepaspectratio]{frequency}
%\end{center}
%Often used with categorical values.
%\end{frame}
%

%%%%%%%%%%%%%%%%%%%%%%%%%%%%%%%%%%%%%%%%%%%%%%%%%%%%%%%%%%%%%%%%%%%%%%%%%%%%%%%%%%
\begin{frame}[fragile]\frametitle{}
\begin{center}
{\Large Measure of Central Tendency}
\end{center}
\end{frame}

%%%%%%%%%%%%%%%%%%%%%%%%%%%%%%%%%%%%%%%%%%%%%%%%%%%%%%%%%%
\begin{frame}[fragile]\frametitle{Mean}	
\begin{itemize}
\item Measure the ``location'' of a set of values
\item Mean is a very, very common measurement
\item But is sensitive to outliers
\end{itemize}
$mean(x) = \bar{x} = 1/n \sum x_i$
%\code{Use `for' loop to calculate}
\end{frame}

%%%%%%%%%%%%%%%%%%%%%%%%%%%%%%%%%%%%%%%%%%%%%%%%%%%%%%%%%%
\begin{frame}[fragile]\frametitle{Mean}	
\begin{center}
\includegraphics[width=\linewidth,keepaspectratio]{da5}
\end{center}
Sum = 153

Mean = 153/25 = 6.12
\end{frame}

%%%%%%%%%%%%%%%%%%%%%%%%%%%%%%%%%%%%%%%%%%%%%%%%%%%%%%%%%%
\begin{frame}[fragile]\frametitle{Outliers}	
\begin{itemize}
\item Extreme data point
\item May affect calculations
\item  Can occur in any given data set and in any distribution
\item  May indicate an experimental error or incorrect recording of data
\end{itemize}
\begin{center}
\includegraphics[width=0.6\linewidth,keepaspectratio]{da15}
\end{center}
\end{frame}


%%%%%%%%%%%%%%%%%%%%%%%%%%%%%%%%%%%%%%%%%%%%%%%%%%%%%%%%%%%%%%%%%%%%%%%%
\begin{frame}[fragile]\frametitle{Mean}
Implement mean
\begin{lstlisting}
def mean(datalist):
	:
	return m

lst = [9,3,7,2,7,10,23,44,12,42,19,11,22,5,3,4,3,21,3]
result = mean(lst)
print("Mean : {}".format(result))
\end{lstlisting}
\end{frame}

%%%%%%%%%%%%%%%%%%%%%%%%%%%%%%%%%%%%%%%%%%%%%%%%%%%%%%%%%%
\begin{frame}[fragile]\frametitle{Mean}
\begin{lstlisting}
def mean(datalist):
	total = 0
	m = 0
	for item in datalist:
		total += item
	m = total / float(len(datalist))
	return m

Mean : 13.157894736842104
\end{lstlisting}
\end{frame}

%%%%%%%%%%%%%%%%%%%%%%%%%%%%%%%%%%%%%%%%%%%%%%%%%%%%%%%%%%
\begin{frame}[fragile]\frametitle{Median}	
\begin{itemize}
\item Commonly used instead of mean if outliers are present
\item Median is the middle value 
\item if odd number of values are present; average of the two middle values if even number of values
\item Not easily affected by outliers (extreme values). 
\item Always exists and unique.
\end{itemize}
$median(x) = x_{r=1} for \quad odd, 1/2(x_r + x_{r+1}) for \quad even$
%\code{Cater to both odd/even case}
\end{frame}


%%%%%%%%%%%%%%%%%%%%%%%%%%%%%%%%%%%%%%%%%%%%%%%%%%%%%%%%%%
\begin{frame}[fragile]\frametitle{Median}	
\begin{center}
\includegraphics[width=0.5\linewidth,keepaspectratio]{da6}
\end{center}
Median = 6

Medians are less reliable: medians of samples drawn from same population vary more widely than sample means.
%\code{Calculate and verify the answer}
\end{frame}


%%%%%%%%%%%%%%%%%%%%%%%%%%%%%%%%%%%%%%%%%%%%%%%%%%%%%%%%%%%%%%%%%%%%%%%%
\begin{frame}[fragile]\frametitle{Median}
Implement median
\begin{lstlisting}
def median(datalist):
	:
	return m

lst = [9,3,7,2,7,10,23,44,12,42,19,11,22,5,3,4,3,21,3]
result = median(lst)
print("Median : {}".format(result))
\end{lstlisting}
\end{frame}

%%%%%%%%%%%%%%%%%%%%%%%%%%%%%%%%%%%%%%%%%%%%%%%%%%%%%%%%%%
\begin{frame}[fragile]\frametitle{Median}
\begin{lstlisting}
def median(datalist):
    n = len(datalist)
    numsort = sorted(datalist)
    mid = n // 2
    m = 1
    if n % 2 == 0:
        lo = mid - 1
        hi = mid
        m = (numsort[lo] + numsort[hi])/2
    else:
        m = numsort[mid]
    return m

Median : 9
\end{lstlisting}
\end{frame}



%%%%%%%%%%%%%%%%%%%%%%%%%%%%%%%%%%%%%%%%%%%%%%%%%%%%%%%%%%
\begin{frame}[fragile]\frametitle{Mode}	

\begin{itemize}
\item The value that has the highest frequency.
\item Requires no calculation, only counting
\item Often used with categorical values.
\item The mode (especially with discrete / continuous data) may reveal value that symbolizes a missing value.
\item Not a stable measure : it depends only a few values
\item May not exist
\item May not be unique
\end{itemize}

%\begin{center}
%\includegraphics[width=0.8\linewidth,keepaspectratio]{mode}
%\end{center}
%\code{Calculate and verify the answer}
\end{frame}

%%%%%%%%%%%%%%%%%%%%%%%%%%%%%%%%%%%%%%%%%%%%%%%%%%%%%%%%%%
\begin{frame}[fragile]\frametitle{Mode}	
\begin{center}
\includegraphics[width=0.6\linewidth,keepaspectratio]{da7}
\end{center}
Median = 7
\end{frame}


%%%%%%%%%%%%%%%%%%%%%%%%%%%%%%%%%%%%%%%%%%%%%%%%%%%%%%%%%%%%%%%%%%%%%%%%
\begin{frame}[fragile]\frametitle{Mode}
Implement mode
\begin{lstlisting}
def mode(datalist):
	:
	return m

lst = [9,3,7,2,7,10,23,44,12,42,19,11,22,5,3,4,3,21,3]
result = mode(lst)
print("Mode : {}".format(result))
\end{lstlisting}
\end{frame}

%%%%%%%%%%%%%%%%%%%%%%%%%%%%%%%%%%%%%%%%%%%%%%%%%%%%%%%%%%
\begin{frame}[fragile]\frametitle{Mode}
\begin{lstlisting}
def frequency_distribution(datalist):
	freqs = dict()
	for item in datalist:
		if item not in frees.keys():
			freqs[item] = 1
		else:
			freqs[item] += 1
	return freqs

def mode(datalist):
    d = frequency_distribution(datalist)
    print(d)
    most_often = 0
    m = 0
    for item in d.keys():
        if d[item] > most_often:
            most_often = d[item]
            m = item
    return m

Mode : 3
\end{lstlisting}
\end{frame}

%%%%%%%%%%%%%%%%%%%%%%%%%%%%%%%%%%%%%%%%%%%%%%%%%%%%%%%%%%
\begin{frame}[fragile]\frametitle{Mode}
Another implementation. Counter returns dictionary of frquencies and values.
\begin{lstlisting}
from collections import Counter

def mode2(x):
    counts = Counter(x)
    max_count = max(counts.values())
    return [x_i for x_i, count in counts.items() if count == max_count]  # multiple modes are possible

Mode : [3]
\end{lstlisting}
\end{frame}



%%%%%%%%%%%%%%%%%%%%%%%%%%%%%%%%%%%%%%%%%%%%%%%%%%%%%%%%%%
\begin{frame}[fragile]\frametitle{Central Tendency}	

\begin{itemize}
\item Mean: Summarizes all the information in the data set
\item Median: Splits the data sets into two halves: there are an equal number of values above and below it.
\item Mode: The most common value in the data set.
\end{itemize}

%\begin{center}
%\includegraphics[width=0.8\linewidth,keepaspectratio]{mode}
%\end{center}

\end{frame}

%%%%%%%%%%%%%%%%%%%%%%%%%%%%%%%%%%%%%%%%%%%%%%%%%%%%%%%%%%
\begin{frame}[fragile]\frametitle{Locating Central Tendency}	
\begin{center}
\includegraphics[width=\linewidth,keepaspectratio]{da8}
\end{center}
Mean = 6.12		

Median = 6		

Mode = 7

\end{frame}

%%%%%%%%%%%%%%%%%%%%%%%%%%%%%%%%%%%%%%%%%%%%%%%%%%%%%%%%%%%%%%%%%%%%%%%%%%%%%%%%%%
\begin{frame}[fragile]\frametitle{}
\begin{center}
{\Large Descriptive Statistics Exercise}
\end{center}
\end{frame}

%%%%%%%%%%%%%%%%%%%%%%%%%%%%%%%%%%%%%%%%%%%%%%%%%%%%%%%%%%
\begin{frame}[fragile]\frametitle{Exercise}	
Our Data: Store as list of integers and calculate Mean, Median and Mode

\begin{lstlisting}
lst = [9,3,7,2,7,10,23,44,12,42,19,11,22,5,3,4,3,21,3]
\end{lstlisting}

\end{frame}

% %%%%%%%%%%%%%%%%%%%%%%%%%%%%%%%%%%%%%%%%%%%%%%%%%%%%%%%%%%
% \begin{frame}[fragile]\frametitle{Calculate Mean}	
% % \begin{lstlisting}
% % def mean(datalist):
	% % total = 0
	% % mean = 0
	% % for item in datalist:
		% % total += item
	% % mean = total / float(len(datalist))
	% % return mean
% % \end{lstlisting}

% \code{Result?}
% \end{frame}

% %%%%%%%%%%%%%%%%%%%%%%%%%%%%%%%%%%%%%%%%%%%%%%%%%%%%%%%%%%
% \begin{frame}[fragile]\frametitle{Calculate Median}	
% % \begin{lstlisting}
% % def median(datalist):
	% % numsort = sorted(datalist)
	% % mid = len(numsort) / 2
	% % median = 1
	% % if len(numsort) % 2 == 0:
		% % median = (numsort[mid - 1] +
	% % else:
		% % median = numsort[(len(numsort) 
	% % return median
% % \end{lstlisting}

% \code{Result?}
% \end{frame}

% %%%%%%%%%%%%%%%%%%%%%%%%%%%%%%%%%%%%%%%%%%%%%%%%%%%%%%%%%%
% \begin{frame}[fragile]\frametitle{Calculate Mode}	
% % \begin{lstlisting}
% % def frequency_distribution(datalist):
	% % freqs = dict()
	% % for item in datalist:
		% % if item not in frees.keys():
			% % freqs[item] = 1
		% % else:
			% % freqs[item] += 1
	% % return freqs

% % def mode(datalist):
	% % d = frequency_distribution(datalist)
	% % most_often = 0
	% % mode = 0
	% % for item in d.keys():
		% % if d[item] > most_often:
			% % most_often = d[item]
		% % mode = item
	% % return mode
% % \end{lstlisting}

% \code{Result?}
% \end{frame}

%%%%%%%%%%%%%%%%%%%%%%%%%%%%%%%%%%%%%%%%%%%%%%%%%%%%%%%%%%%%%%%%%%%%%%%%%%%%%%%%%%
\begin{frame}[fragile]\frametitle{}
\begin{center}
{\Large Descriptive Statistics Exercise}
\end{center}
\end{frame}

%%%%%%%%%%%%%%%%%%%%%%%%%%%%%%%%%%%%%%%%%%%%%%%%%%%%%%%%%%
\begin{frame}[fragile]\frametitle{Exercise}	
\begin{lstlisting}
crater_diameter = [46, 51, 49, 82, 74, 63, 49, 70, 48, 47, 79, 48, 52, 55, 49, 51, 58, 82, 72, 45]
print mean(crater_diameter)
print median(crater_diameter)
print mode(crater_diameter)
\end{lstlisting}

\code{Result? 58.5,51.5,49}
\end{frame}
