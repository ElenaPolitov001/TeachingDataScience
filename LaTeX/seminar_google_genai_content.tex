%%%%%%%%%%%%%%%%%%%%%%%%%%%%%%%%%%%%%%%%%%%%%%%%%%%%%%%%%%%%%%%%%%%%%%%%%%%%%%%%%%
\begin{frame}[fragile]\frametitle{}
\begin{center}
{\Large Generative AI}
\end{center}
\end{frame}

%%%%%%%%%%%%%%%%%%%%%%%%%%%%%%%%%%%%%%%%%%%%%%%%%%%%%%%%%%%%%%%%%%%%%%%%%%%%%%%%%%
\begin{frame}[fragile]{Introduction}
\begin{itemize}
\item Generative AI is a rapidly growing technology that uses machine learning algorithms to generate new data.
\item Google is one of the leading companies in the field of generative AI, with a range of tools and frameworks for developers and businesses.
\item This presentation will provide an overview of the concepts, framework, applications, and conclusion of generative AI by Google.
\end{itemize}
\end{frame}

%%%%%%%%%%%%%%%%%%%%%%%%%%%%%%%%%%%%%%%%%%%%%%%%%%%%%%%%%%%%%%%%%%%%%%%%%%%%%%%%%%
\begin{frame}[fragile]{Concepts}
\begin{itemize}
\item Generative AI is a type of machine learning that involves training algorithms to generate new data that is similar to existing data.
\item Google's generative AI tools are based on deep learning algorithms, which use neural networks to learn patterns in data and generate new data.
\item Some of the key concepts in generative AI include autoencoders, generative adversarial networks (GANs), and variational autoencoders (VAEs).
\end{itemize}
\end{frame}

%%%%%%%%%%%%%%%%%%%%%%%%%%%%%%%%%%%%%%%%%%%%%%%%%%%%%%%%%%%%%%%%%%%%%%%%%%%%%%%%%%
\begin{frame}[fragile]{Framework}
\begin{itemize}
\item Google offers a range of frameworks and tools for developers and businesses to build generative AI models.
\item Some of the key frameworks include TensorFlow, Keras, and PyTorch.
\item Google also offers pre-trained models and APIs for developers to use in their applications, such as Cloud AutoML and Vertex AI.
\end{itemize}
\end{frame}

%%%%%%%%%%%%%%%%%%%%%%%%%%%%%%%%%%%%%%%%%%%%%%%%%%%%%%%%%%%%%%%%%%%%%%%%%%%%%%%%%%
\begin{frame}[fragile]{Applications}
\begin{itemize}
\item Generative AI has a wide range of applications, from image and video generation to natural language processing and music composition.
\item Google's generative AI tools are used in a variety of industries, including healthcare, finance, and entertainment.
\item Some of the specific applications of generative AI by Google include Gmail, Docs, Slides, Sheets, and more.
\end{itemize}
\end{frame}

%%%%%%%%%%%%%%%%%%%%%%%%%%%%%%%%%%%%%%%%%%%%%%%%%%%%%%%%%%%%%%%%%%%%%%%%%%%%%%%%%%
\begin{frame}[fragile]{Conclusion}
\begin{itemize}
\item Generative AI by Google is a powerful technology that has the potential to revolutionize many industries.
\item With its range of frameworks, tools, and pre-trained models, Google is well-positioned to be a leader in the field of generative AI.
\item As the technology continues to evolve, we can expect to see even more innovative applications of generative AI by Google and other companies.
\end{itemize}
\end{frame}