%%%%%%%%%%%%%%%%%%%%%%%%%%%%%%%%%%%%%%%%%%%%%%%%%%%%%%%%%%%%%%%%%%%%%%%%%%%%%%%%%%
\begin{frame}[fragile]\frametitle{}
\begin{center}
{\Large Not just a Chatbot}
\end{center}
\end{frame}

%%%%%%%%%%%%%%%%%%%%%%%%%%%%%%%%%%%%%%%%%%%%%%%%%%%%%%%%%%%
\begin{frame}[fragile]\frametitle{In Other Industries?}


\begin{itemize}
\item Generating automated responses to frequently asked questions 
\item Can help improve customer service as well as
train technicians
\item Helping write professional and effective emails, quotes, and proposals;
\item Generating marketing and advertising content
\item Providing technical assistance, such as troubleshooting
\item Helping generate training material for employees.
\end{itemize}	 

\tiny{(Ref:ChatGPT Could Be the ‘Write’ Choice for HVACR Contractors - Joanna R. Turpin)}
\end{frame}


%%%%%%%%%%%%%%%%%%%%%%%%%%%%%%%%%%%%%%%%%%%%%%%%%%%%%%%%%%%%%%%%%%%%%%%%%%%%%%%%%%
\begin{frame}[fragile]\frametitle{HVAC QnA}
Prompt: ``Diagnose a frosted evaporator coil''

Result: 
``There are several reasons why an evaporator coil may become frosted:
\begin{itemize}
\item Low refrigerant levels: If the refrigerant levels in the system are too low, it can cause the
evaporator coil to freeze. This is because the refrigerant is responsible for absorbing heat
from the air, and without enough refrigerant, the coil will not be able to absorb enough heat
to prevent freezing.
\item Dirty air filter: A dirty air filter can restrict airflow to the evaporator coil, causing the coil
to freeze.
\item Blocked or restricted airflow: If the airflow to the evaporator coil is blocked or
restricted, the coil may freeze. This can be caused by a number of things such as dirty ducts or
registers, closed or blocked vents, or a malfunctioning blower fan.
\item Obstruction in the drain line: A blockage in the drain line can cause water to back up and freeze on the coil.
\end{itemize}	 
It's important to fix the issue as soon as possible, as a frozen evaporator coil can lead to decreased
efficiency and can cause damage to the unit.
''

\tiny{(Ref:ChatGPT Could Be the ‘Write’ Choice for HVACR Contractors - Joanna R. Turpin)}
\end{frame}

%%%%%%%%%%%%%%%%%%%%%%%%%%%%%%%%%%%%%%%%%%%%%%%%%%%%%%%%%%%%%%%%%%%%%%%%%%%%%%%%%%
\begin{frame}[fragile]\frametitle{HVAC QnA}
Prompt: ``How to respond to a customer who thinks an air conditioner is
too expensive''

Result: ``
\begin{itemize}
\item Understand their concerns: Find out why they think the air conditioner is too expensive.
Are they comparing it to a lower-quality unit? Are they on a tight budget? Understanding
their concerns will help you address them more effectively.
\item Provide a detailed explanation of the costs: Explain the costs associated with the air
conditioner, including the cost of the unit, installation, and any additional materials or labor
that may be required.
\item Compare costs with the benefits: Explain the benefits of the air conditioner, such as
energy efficiency, durability, and long-term cost savings. Explain how the cost of the unit may
be offset by energy savings in the long run.
\item Offer financing options: If the customer is unable to afford the air conditioner upfront,
offer financing options that can help them spread the cost over time.

\end{itemize}	 

\ldots

\tiny{(Ref:ChatGPT Could Be the ‘Write’ Choice for HVACR Contractors - Joanna R. Turpin)}
\end{frame}

%%%%%%%%%%%%%%%%%%%%%%%%%%%%%%%%%%%%%%%%%%%%%%%%%%%%%%%%%%%%%%%%%%%%%%%%%%%%%%%%%%
\begin{frame}[fragile]\frametitle{HVAC QnA}

\ldots 

\begin{itemize}
\item Provide a written estimate: Provide the customer with a detailed, written estimate that
outlines the costs associated with the air conditioner and the benefits it will provide. This will
help them make an informed decision.
\item Provide a guarantee or warranty on the unit: Provide the customer with a guarantee
or warranty on the unit to give them peace of mind in their investment.
\item Provide an alternative solution: If the customer cannot afford the recommended
solution, offer an alternative solution that fits their budget while still meeting their needs.
\end{itemize}	 
It's important to be patient, understanding and provide all the information customers need to
make a decision that best fits their needs and budget.
''

\tiny{(Ref:ChatGPT Could Be the ‘Write’ Choice for HVACR Contractors - Joanna R. Turpin)}
\end{frame}

%%%%%%%%%%%%%%%%%%%%%%%%%%%%%%%%%%%%%%%%%%%%%%%%%%%%%%%%%%%%%%%%%%%%%%%%%%%%%%%%%%
\begin{frame}[fragile]\frametitle{If This Then That}


\begin{itemize}
\item With well designed prompts, one can format output in json (read, key-value pairs)
\item Parsing via python program, one can trigger Actions. (GPT3 APIs for now)
\item Whole pipeline can be put in place programmatically, doing any trigger based work (IFTTT - If This Then That)
\item Possible applications:
\begin{itemize}
\item Voice-Controlled Devices: ChatGPT’s ability to understand natural language can be combined with Arduino to create voice-controlled devices. These devices can be programmed to respond to specific commands and perform specific actions. For example, you can create a voice-controlled light switch or a voice-controlled robot. 
\item Text-Based User Interfaces: ChatGPT can create text-based user interfaces for Arduino-powered devices. This makes it easier for users to interact with the device and access its features. For example, you can create a weather station that provides information about the current weather conditions, and users can interact with it using text commands.
\item Multi-Language Support: ChatGPT’s ability to translate languages can be combined with Arduino to create multi-language support for devices. This feature makes it possible for users from different countries to interact with the device in their language. For example, you can create a device that provides information about local tourist attractions and supports multiple languages.
\item Home Automation: One unique application of this combination is in the field of home automation. With ChatGPT’s ability to understand and respond to natural language and the versatility of the Arduino platform, it’s possible to create a fully voice-controlled smart home system. 
\end{itemize}	 

\end{itemize}	 


\tiny{(Ref: ChatGPT for Arduino - MoonPreneur)}
\end{frame}
