%%%%%%%%%%%%%%%%%%%%%%%%%%%%%%%%%%%%%%%%%%%%%%%%%%%%%%%%%%%%%%%%%%%%%%%%%%%%%%%%%%
\begin{frame}[fragile]\frametitle{}
\begin{center}
{\Large Numbers}
\end{center}
\end{frame}

%%%%%%%%%%%%%%%%%%%%%%%%%%%%%%%%%%%%%%%%%%%%%%%%%%%%%%%%%%%
 \begin{frame}[fragile]\frametitle{Numbers}
\begin{itemize}
\item $ x = 3 \Rightarrow \mathbb{N}$: Natural numbers: 1,2,3,\ldots
\item $ x + 5 = 3 \Rightarrow \mathbb{Z}$: Integers: -1,0,1,2,3,\ldots
\item $ 2x = 3 \Rightarrow \mathbb{Q}$: Rational numbers, ratios: $\frac{3}{2},\frac{1}{3}$,\ldots
\item $ x^2 = 2 \Rightarrow \mathbb{P}$: Irrational numbers, cannot be expressed as ratios: $\sqrt{2},\sqrt[3]{2}$,\ldots.
\item $\pi, e \Rightarrow \mathbb{R}$: Real numbers, these cannot be represented by polynomials, cannot be roots, etc.
\item $ x^2 +1 = 0 \Rightarrow \mathbb{C}$: Complex numbers: $i,2+3i$,\ldots.All polynomials have roots in $\mathbb{C}$
\end{itemize}

$\mathbb{N} \subseteq \mathbb{Z} \subseteq \mathbb{Q} \subseteq \mathbb{P} \subseteq \mathbb{R} \subseteq \mathbb{C} $

Note: all the coefficients in the equation are $\mathbb{N}$ but the resultant $x$ is of different types.
\end{frame}

%%%%%%%%%%%%%%%%%%%%%%%%%%%%%%%%%%%%%%%%%%%%%%%%%%%%%%%%%%%
 \begin{frame}[fragile]\frametitle{Natural Numbers}
\begin{itemize}
\item Natural numbers are the ``natural" objects to count things around
us with. The first thing we learn is to add natural numbers, then
later on we start to multiply.
\item Axioms for $\mathbb{N}$:
\begin{description}
\item[Axiom N1] $1\in\mathbb{N}$.
\item[Axiom N2] If $n\in\mathbb{N}$, then $n+1\in\mathbb{N}$.
\item[Axiom N3] There is no natural number $n\in\mathbb{N}$, such that $n+1=1$.
\item[Axiom N4] For every natural number $n\in\mathbb{N}$, there is a
real number $r\in\mathbb{R}$ such that $n\leq r<n+1$.
\end{description}
\end{itemize}

\tiny{(Ref: Helmut Knaust Class 200510 3341)}

\end{frame}

%%%%%%%%%%%%%%%%%%%%%%%%%%%%%%%%%%%%%%%%%%%%%%%%%%%%%%%%%%%
 \begin{frame}[fragile]\frametitle{Integers, Rational and Irrational Numbers}
\begin{itemize}
\item Deficiencies of the system of natural numbers start to appear
when we want to divide, the quotient of two natural numbers is
not necessarily a natural number, or when we want to
subtract, the difference of two natural numbers is not
necessarily a natural number. 
\item This leads quite naturally to two
extensions of the concept of number.
\item The set of \textsc{integers}, denoted by $\mathbb{Z}$, is the set
\[{\mathbb{Z}}=\{0,1,-1,2,-2,3,-3,\ldots\}.\]
The set of \textsc{rational numbers} $\mathbb{Q}$ is defined as
\[\mathbb{Q}=\left\{\frac{p}{q} \ \left|\ p,q\in\mathbb{Z}\mbox{ and }
q\not=0\right.\right\}.\]
\item Real numbers that are not rational are called \textsc{irrational
numbers}. 
\end{itemize}

\tiny{(Ref: Helmut Knaust Class 200510 3341)}

\end{frame}

%%%%%%%%%%%%%%%%%%%%%%%%%%%%%%%%%%%%%%%%%%%%%%%%%%%%%%%%%%%
 \begin{frame}[fragile]\frametitle{Integers, Rational and Irrational Numbers}
\begin{itemize}
\item The existence of irrational numbers, first discovered by
the Pythagoreans in about 520~B.C.,  must have come as a major
surprise to Greek Mathematicians:
\item The
square root of $2$ is irrational. ($\sqrt{2}$ is the positive
number whose square is $2$.)

\end{itemize}

\tiny{(Ref: Helmut Knaust Class 200510 3341)}

\end{frame}


%%%%%%%%%%%%%%%%%%%%%%%%%%%%%%%%%%%%%%%%%%%%%%%%%%%%%%%%%%%
 \begin{frame}[fragile]\frametitle{Operations}
 
 Addition and multiplication of rational and real numbers interact
in a reasonable manner---the following \textsc{distributive law}
holds:

For all $a,b,c \in \mathbb{R}$
\[(a+b)\cdot c= (a\cdot c) + (b\cdot c)\]


\tiny{(Ref: Helmut Knaust Class 200510 3341)}

\end{frame}