%%%%%%%%%%%%%%%%%%%%%%%%%%%%%%%%%%%%%%%%%%%%%%%%%%%%%%%%%%%%%%%%%%%%%%%%%%%%%%%%%%
\begin{frame}[fragile]\frametitle{}
\begin{center}
{\Large Installation}

% {\tiny (Ref: https://rasa.com/docs/core/installation/)}
\end{center}
\end{frame}

% %%%%%%%%%%%%%%%%%%%%%%%%%%%%%%%%%%%%%%%%%%%%%%%%%%%%%%%%%%%
 % \begin{frame}[fragile]\frametitle{Prerequisites}
% \begin{itemize}
% \item Python
% \item Spacy
% \item Rasa Starter Pack
% \item Rasa NLU
% \item Rasa Core
% \item Slack credentials (for Slack as front-end)
% \end{itemize}
% \end{frame}


%%%%%%%%%%%%%%%%%%%%%%%%%%%%%%%%%%%%%%%%%%%%%%%%%%%%%%%%%%%
 \begin{frame}[fragile]\frametitle{Python}
\begin{itemize}
\item Install Anaconda for Python 3.7 %4.2.0 for Python 3.5 or Ananconda 5.2.0 for Python 3.6 
\item Else from Python.org and then additionally all requisite libraries
\item Ubuntu : 
\begin{lstlisting}
sudo apt-get install build-essential python-dev git
\end{lstlisting}
\item Windows Build tools: Make sure the Microsoft $VC++$ Compiler Visual Studio 2015 is installed, so python can compile any dependencies or https://visualstudio.microsoft.com/visual-cpp-build-tools/ Download the installer and select VC++ Build tools in the list.
\end{itemize}

% (Note: On main page of Anaconda is https://www.anaconda.com/distribution/ you see Python 3.7, which is problematic, gives HTTP error while creating env. I had to delete 3.7 and get 3.6. Btw, this site does not mention which Anaconda installer has which Python version. WHY? thats the most crucial info. Anyway, I have given them here.)

\end{frame}

%%%%%%%%%%%%%%%%%%%%%%%%%%%%%%%%%%%%%%%%%%%%%%%%%%%%%%%%%%%
 \begin{frame}[fragile]\frametitle{Conda Installation}
\begin{itemize}
\item Install the Conda(miniconda) from https://docs.conda.io/en/latest/miniconda.html as per the OS
\item Check the Conda version \lstinline|conda --version| conda 4.7.10
\item In case need to upgrade, run below command \lstinline|conda update conda|
\end{itemize}
\end{frame}

%%%%%%%%%%%%%%%%%%%%%%%%%%%%%%%%%%%%%%%%%%%%%%%%%%%%%%%%%%%
 \begin{frame}[fragile]\frametitle{Setup in Virtual Environment}
\begin{itemize}
\item By installing conda, you get base or the root environment, which is the default.
\item Practical tip: DO NOT install any packages in the root. ALWAYS create and env and install inside the new env.
\item Env is needed especially for fragile packages like Python (its treated as a package) and rasa.
\item So, \lstinline|conda create -n rasa_env python=3.7| 
\item Python 3.6 as different asyncio format, so better to do it in 3.7
\item Activate the new environment to use it
\begin{lstlisting}
LINUX, macOS: conda activate rasa
WINDOWS: activate rasa
\end{lstlisting}
\end{itemize}

(Note: Env management is again a sour point. It creates complete copy (deeeep) of python 3.7 and all other packages inside "envs" folder. Goes to 1.5 GB!! Can someone optimize it?)
\end{frame}

%%%%%%%%%%%%%%%%%%%%%%%%%%%%%%%%%%%%%%%%%%%%%%%%%%%%%%%%%%%
 \begin{frame}[fragile]\frametitle{Rasa Installation}
\begin{itemize}
\item Install latest Rasa stack.
\item Rasa NLU $+$ Core is now in the single package. Do not install them separately like in past. Your mileage may vary.
\item \lstinline|pip install rasa|
\end{itemize}

If you get Microsoft Build tool error:
\begin{itemize}
\item Go to https://visualstudio.microsoft.com/downloads/\#build-tools-for-visual-studio-2017
\item Build tools for Visual Studio 19 are fine too. Select only $C++$ Build tools. Install.
\item Re-run rasa install command
\end{itemize}
 
\end{frame}

%%%%%%%%%%%%%%%%%%%%%%%%%%%%%%%%%%%%%%%%%%%%%%%%%%%%%%%%%%%
 \begin{frame}[fragile]\frametitle{Spacy Installation}
 Additionally, spaCy:
\begin{lstlisting}
pip install rasa[spacy]
python -m spacy download en
python -m spacy download en_core_web_md
python -m spacy link en_core_web_md en
\end{lstlisting}

If you get linking permission error:
\begin{itemize}
\item Run cmd as administrator, 
\item Activate rasa\_env
\item Do all the above spacy commands.
\end{itemize}
 
\end{frame}


%%%%%%%%%%%%%%%%%%%%%%%%%%%%%%%%%%%%%%%%%%%%%%%%%%%%%%%%%%%
 \begin{frame}[fragile]\frametitle{Other Installations}
To show conda envs in notebook \lstinline|conda install nb_conda_kernels|

Apart from this, may need to
\begin{itemize}
\item \lstinline|pip install nest_asyncio|
\item Write following code to test, in ipynb put this in the first cell
\end{itemize}


\begin{lstlisting}
import nest_asyncio

nest_asyncio.apply()
print("Event loop ready.")
\end{lstlisting}

Some more (optional):
\begin{itemize}
\item Download and install ngrok from https://ngrok.com/download
\item Need to have API keys for Slack, CRICINFO, ZOMATO, etc
\end{itemize}
 
\end{frame}

% %%%%%%%%%%%%%%%%%%%%%%%%%%%%%%%%%%%%%%%%%%%%%%%%%%%%%%%%%%%
 % \begin{frame}[fragile]\frametitle{Spacy}
% If not done already, Spacy Installation: https://spacy.io/usage/
% \begin{itemize}
% \item Pip
% \begin{lstlisting}
% pip install -U spacy
% \end{lstlisting}
% \item Conda
% \begin{lstlisting}
% conda install -c conda-forge spacy
% \end{lstlisting}
% \item (Optional) Download English model (in CMD terminal with Administrator permission)
% \begin{lstlisting}
% python -m spacy download en
% \end{lstlisting}
% \item Validate
% \begin{lstlisting}
% python -m spacy validate
% \end{lstlisting}
% \end{itemize}
% \end{frame}



%%%%%%%%%%%%%%%%%%%%%%%%%%%%%%%%%%%%%%%%%%%%%%%%%%%%%%%%%%%
 \begin{frame}[fragile]\frametitle{Exercise: Check Installation}
RASA Starter Pack
\begin{itemize}
\item Clone the starter pack provided by Rasa from https://github.com/RasaHQ/starter-pack-rasa-stack
\item It gives latest running application
\item Let's take a look at the folder structure and the files that were created 
\item Install git, can git clone above package.
\item \lstinline|cd starter-pack-rasa-stack;activate rasa; pip install -r requirements.txt|

\end{itemize}
\end{frame}

%%%%%%%%%%%%%%%%%%%%%%%%%%%%%%%%%%%%%%%%%%%%%%%%%%%%%%%%%%%
 \begin{frame}[fragile]\frametitle{Exercise: Check Installation}
RASA Starter Pack
\begin{itemize}
\item The ``domain.yml'' file describes the travel assistant's domain. It specifies the list of intents, entities, slots, and response templates that the assistant understands and operates with.
\item The ``data/nlu\_data.md'' file describes each intent with a set of examples that are then fed to Rasa NLU for training.
\item The ``data/stories.md'' file provides Rasa with sample conversations between users and the travel assistant that it can use to train its dialog management model.
\item Rasa provides a lot flexibility in terms of configuring the NLU and core components. For now, we'll use the default ``nlu\_config.yml'' for NLU and ``policies.yml'' for the core model.
\end{itemize}


\end{frame}

%%%%%%%%%%%%%%%%%%%%%%%%%%%%%%%%%%%%%%%%%%%%%%%%%%%%%%%%%%%
 \begin{frame}[fragile]\frametitle{Exercise: Check Installation}
RASA Starter Pack
\begin{itemize}
\item May need to rename data files to have standard names, else use command line flags in the following commands
\item Train
\item Use Rasa Shell to execute.
\item Look at data files to see what interaction has been coded.
\item Accordingly run the chatbot
\end{itemize}


\end{frame}



%%%%%%%%%%%%%%%%%%%%%%%%%%%%%%%%%%%%%%%%%%%%%%%%%%%%%%%%%%%
 \begin{frame}[fragile]\frametitle{Exercise: Check Installation}

\begin{lstlisting}
mkdir firstbot
cd firstbot
activate rasa_env
rasa init  --no-prompt
\end{lstlisting}

\begin{itemize}
\item Creates project structure and dummy files. 
\item Good to run and execute
\item You can run as is or add your data to nlu.md, stories.md and domain.yml; retrain and run.
\item If you get any warnings replace ``rasa'' with ``python -W ignore -m rasa ''
\end{itemize}

{\tiny (Ref: How to build awesome Rasa chatbot for a web - Martin Novak)}

\end{frame}

%%%%%%%%%%%%%%%%%%%%%%%%%%%%%%%%%%%%%%%%%%%%%%%%%%%%%%%%%%%
 \begin{frame}[fragile]\frametitle{Exercise: Check Installation}

To run the bot:

\begin{lstlisting}
python -W ignore -m rasa shell --quiet --cors * -m models 


# Sample Chat
Bot loaded. Type a message and press enter (use '/stop' to exit):
Your input ->  hi
Hey! How are you?
Your input ->  perfect
Great, carry on!
Your input ->  are you a bot?
I am a bot, powered by Rasa.
Your input ->  /stop
\end{lstlisting}

Now, sky is the limit \ldots
\end{frame}



% %%%%%%%%%%%%%%%%%%%%%%%%%%%%%%%%%%%%%%%%%%%%%%%%%%%%%%%%%%%
 % \begin{frame}[fragile]\frametitle{Pipelines}
% \begin{itemize}
% \item RASA-NLU is made up of a few components, each doing some specific work (intent detection, entity extraction, etc.). 
% \item Each component may have some specific dependencies and installations. 
% \item Options like MITIE (NLP + ML), Spacy and Sklearn are available to choose from. 
% \item We will be using Spacy-Sklearn here.
% \end{itemize}

% \end{frame}
% %%%%%%%%%%%%%%%%%%%%%%%%%%%%%%%%%%%%%%%%%%%%%%%%%%%%%%%%%%%
 % \begin{frame}[fragile]\frametitle{Rasa NLU and Core}
% \begin{itemize}
% \item Rasa NLU  with spacy\_sklearn pipeline (Most recent github, https://rasa.com/docs/nlu/installation/)
% \begin{lstlisting}
% pip install rasa_nlu[spacy]
% \end{lstlisting}
% \item Rasa Core (https://rasa.com/docs/core/installation/)
% \begin{lstlisting}
% pip install -U rasa_core==0.9.6
% \end{lstlisting}
% Note: we are not using the latest. Trying latest as well.
% % \item Language Model
% % \begin{lstlisting}
% % python -m spacy download en_core_web_md
% % python -m spacy link en_core_web_md en --force;
% % \end{lstlisting}
% \end{itemize}
% \end{frame}

% %%%%%%%%%%%%%%%%%%%%%%%%%%%%%%%%%%%%%%%%%%%%%%%%%%%%%%%%%%%
 % \begin{frame}[fragile]\frametitle{Alternatives}
% \begin{itemize}
% \item To use the tensorflow\_embedding pipeline you will need to install tensorflow as well as the scikit-learn and sklearn-crfsuite libraries. 
% \begin{lstlisting}
% pip install rasa_nlu[tensorflow]
% \end{lstlisting}
% \item MITIE : The MITIE backend performs well for small datasets, but training can take very long if you have more than a couple of hundred examples.
% \begin{lstlisting}
% pip install git+https://github.com/mit-nlp/MITIE.git
% pip install rasa_nlu[mitie]
% \end{lstlisting}
% \end{itemize}

% \end{frame}

% %%%%%%%%%%%%%%%%%%%%%%%%%%%%%%%%%%%%%%%%%%%%%%%%%%%%%%%%%%%
 % \begin{frame}[fragile]\frametitle{Rasa Core}
% \begin{itemize}
% \item Latest (Most recent github)
% \begin{lstlisting}
% git clone https://github.com/RasaHQ/rasa_nlu.git
% cd rasa_nlu
% pip install -r requirements.txt
% pip install -e .
% \end{lstlisting}
% \item Ready installation
% \begin{lstlisting}
% pip install rasa_core
% \end{lstlisting}
% \end{itemize}
% \end{frame}



% %%%%%%%%%%%%%%%%%%%%%%%%%%%%%%%%%%%%%%%%%%%%%%%%%%%%%%%%%%%
 % \begin{frame}[fragile]\frametitle{UI/Front-End}
% \begin{itemize}
% \item Client UI can be a web page (using frameworks like Flask in Python) or a mobile app. 
% \item Flask is simple to code and runs locally. 
% \begin{lstlisting}
% pip install flask
% \end{lstlisting}
% \item More info at https://www.tutorialspoint.com/flask/
% \item Many UI front-ends are possible, such as Facebook, Slack, etc. They need API based credentials for fetching what user types and posting the answers generated by Chatbot.
% \end{itemize}

% \end{frame}

