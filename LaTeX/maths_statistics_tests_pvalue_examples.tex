%%%%%%%%%%%%%%%%%%%%%%%%%%%%%%%%%%%%%%%%%%%%%%%%%%%%%%%%%%%
\begin{frame}
\begin{center}
{\Large P-Value : Another Example on How to Calculate P Value}
\end{center}
\end{frame}

%%%%%%%%%%%%%%%%%%%%%%%%%%%%%%%%%%%%%%%%%%%%%%%%%%%
\begin{frame}
\frametitle{Introduction (recap)}

\begin{itemize}
\item P value is a statistical measure that helps scientists determine whether their hypotheses are correct.
\item Usually, if the P value of a data set is below a certain pre-determined amount (like, for instance, 0.05), scientists will reject the "null hypothesis" of their experiment - in other words, they'll rule out the hypothesis that the variables of their experiment had no meaningful effect on the results. 
\item Today, p values are usually found on a reference table by first calculating a chi square value.

\end{itemize}
\end{frame}

%%%%%%%%%%%%%%%%%%%%%%%%%%%%%%%%%%%%%%%%%%%%%%%%%%%
\begin{frame}
\frametitle{Expected Results (recap)}

\begin{itemize}
\item Usually, when scientists conduct an experiment and observe the results, they have an idea of what "normal" or "typical" results will look like beforehand. This can be based on past experimental results, trusted sets of observational data, etc.
\item For your experiment, determine your expected results and express them as a number.
\end{itemize}
\end{frame}

%%%%%%%%%%%%%%%%%%%%%%%%%%%%%%%%%%%%%%%%%%%%%%%%%%%
\begin{frame}
\frametitle{Toy Experiment}

\begin{itemize}
\item Let's say prior studies have shown that, nationally, speeding tickets are given more often to red cars than they are to blue cars
\item Let's say the average results nationally show a 2:1 preference for red cars. We want to find out whether or not the police in our town also demonstrate this bias by analyzing speeding tickets given by our town's police. 
\item If we take a random pool of 150 speeding tickets given to either red or blue cars in our town, we would expect 100 to be for red cars and 50 to be for blue cars if our town's police force gives tickets according to the national bias.
\end{itemize}

\begin{center}
\includegraphics[width=0.5\linewidth,keepaspectratio]{pvalue1}
\end{center}
\end{frame}


%%%%%%%%%%%%%%%%%%%%%%%%%%%%%%%%%%%%%%%%%%%%%%%%%%%
\begin{frame}
\frametitle{Determine your experiment's observed results}
You can conduct your experiment and find your actual (or "observed") values. Again, express these results as numbers. Following were numbers from LOCAL (not national) observations. See if chaning this data source (National to local) had any significant effect.

\begin{center}
\includegraphics[width=0.5\linewidth,keepaspectratio]{pvalue2}
\end{center}
\end{frame}

%%%%%%%%%%%%%%%%%%%%%%%%%%%%%%%%%%%%%%%%%%%%%%%%%%%
\begin{frame}
\frametitle{Experiment's observed results}
\begin{itemize}
\item The observed results differ from this expected results, two possibilities are possible: either this happened by chance, or our experimental variables caused the difference. 
\item The purpose of finding a p-value is to determine whether the observed results differ from the expected results to such a degree that the "null hypothesis" - the hypothesis that there is no relationship between the experimental variable(s) and the observed results - is unlikely enough to reject. It did not happen by chance.
\item Are our local police as biased as the national average suggests, and we're just observing a chance variation? A p value will help us determine this.

\end{itemize}

\end{frame}

%%%%%%%%%%%%%%%%%%%%%%%%%%%%%%%%%%%%%%%%%%%%%%%%%%%
\begin{frame}
\frametitle{Determine your experiment's degrees of freedom}
\begin{itemize}
\item Degrees of freedom are a measure the amount of variability involved in the research, which is determined by the number of categories you are examining. The equation for degrees of freedom is Degrees of freedom = n-1, where "n" is the number of categories or variables being analyzed in your experiment.
\item Example: Our experiment has two categories of results: one for red cars and one for blue cars. Thus, in our experiment, we have 2-1 = 1 degree of freedom. If we had compared red, blue, and green cars, we would have 2 degrees of freedom, and so on.
\end{itemize}

\begin{center}
\includegraphics[width=0.5\linewidth,keepaspectratio]{pvalue3}
\end{center}

\end{frame}

%%%%%%%%%%%%%%%%%%%%%%%%%%%%%%%%%%%%%%%%%%%%%%%%%%%
\begin{frame}
\frametitle{Compare expected results to observed results with chi square}
\begin{itemize}
\item Chi square(written "x2") is a numerical value that measures the difference between an experiment's expected and observed values. 
\item The equation for chi square is: $x2 = \sum((o-e)2/e)$, where ``o'' is the observed value and ``e'' is the expected value
\begin{itemize}
\item $x2 = ((90-100)2/100) + (60-50)2/50)$
\item $x2 = ((-10)2/100) + (10)2/50)$
\item $x2 = (100/100) + (100/50) = 1 + 2 = 3$
\end{itemize}

\end{itemize}

\begin{center}
\includegraphics[width=0.5\linewidth,keepaspectratio]{pvalue4}
\end{center}

\end{frame}

%%%%%%%%%%%%%%%%%%%%%%%%%%%%%%%%%%%%%%%%%%%%%%%%%%%
\begin{frame}
\frametitle{Choose a significance level}
\begin{itemize}
\item Basically, the significance level is a measure of how certain we want to be about our results - low significance values correspond to a low probability that the experimental results happened by chance, and vice versa.
\item By convention, scientists usually set the significance value for their experiments at 0.05, or 5 percent.
\item This means that experimental results that meet this significance level have, at most, a 5\% chance of being reproduced in a random sampling process

\end{itemize}

\end{frame}

%%%%%%%%%%%%%%%%%%%%%%%%%%%%%%%%%%%%%%%%%%%%%%%%%%%
\begin{frame}
\frametitle{Chi square table}

\begin{center}
\includegraphics[width=0.9\linewidth,keepaspectratio]{pvalue5}
\end{center}

\end{frame}

%%%%%%%%%%%%%%%%%%%%%%%%%%%%%%%%%%%%%%%%%%%%%%%%%%%
\begin{frame}
\frametitle{Use a chi square distribution table to approximate your p-value}
\begin{itemize}
\item Use these tables by first finding your degrees of freedom, then reading that row across from the left to the right until you find the first value bigger than your chi square value. 
\item Look at the corresponding p value at the top of the column - your p value is between this value and the next-largest value (the one immediately to the left of it.)
\item We'll go from left to right along this row until we find a value higher than 3 - our chi square value. The first one we encounter is 3.84. Looking to the top of this column, we see that the corresponding p value is 0.05. 
\item This means that our p value is between 0.05 and 0.1 (the next-biggest p value on the table).
\end{itemize}

\end{frame}

%%%%%%%%%%%%%%%%%%%%%%%%%%%%%%%%%%%%%%%%%%%%%%%%%%%
\begin{frame}
\frametitle{Decide whether to reject or keep your null hypothesis}
\begin{itemize}
\item Our p value is between 0.05 and 0.1 . It is not smaller than 0.05, so, unfortunately, we can't reject our null hypothesis. 
\item This means that we didn't reach the criterion we decided upon to be able to say that our town's police give tickets to red and blue cars at a rate that's significantly different than the national average.
\item In other words, random sampling from the national data would produce a result 10 tickets off from the national average 5-10\% of the time.

\item Since we were looking for this percentage to be less than 5\%, we can't say that we're sure our town's police are less biased towards red cars.
\end{itemize}

\end{frame}
