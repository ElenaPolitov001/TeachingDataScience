%%%%%%%%%%%%%%%%%%%%%%%%%%%%%%%%%%%%%%%%%%%%%%%%%%%%%%%%%%%%%%%%%%%%%%%%%%%%%%%%%%
\begin{frame}[fragile]\frametitle{}
\begin{center}
{\Large Statistical Distributions Examples}
\end{center}
\end{frame}

%%%%%%%%%%%%%%%%%%%%%%%%%%%%%%%%%%%%%%%%%%%%%%%%%%%%%%%%%%%%%%%%%%%%%%%%%%%%%%%%%%
\begin{frame}[fragile]\frametitle{}
\begin{center}
{\Large Normal Distribution in Practice}
\end{center}
\end{frame}


%%%%%%%%%%%%%%%%%%%%%%%%%%%%%%%%%%%%%%%%%%%%%%%%%%%%%%%%%%%
\begin{frame}
\frametitle{Normal Distribution in Practice}

\begin{itemize}
\item Applied to single variable continuous data e.g. heights of plants, weights of lambs, lengths of time
\item Used to calculate the probability of occurrences less than, more than, between given values e.g. 
\begin{itemize}
\item The probability that the plants will be less than 70mm
\item The probability that the lambs will be heavier than 70kg
\item The probability that the time taken will be between 10 and 12 minutes
\end{itemize}
\item Standard Normal tables give probabilities
\end{itemize}


{\tiny (Ref: Normal, Binomial, Poisson Distributions -  Lincoln University)}
\end{frame}

%%%%%%%%%%%%%%%%%%%%%%%%%%%%%%%%%%%%%%%%%%%%%%%%%%%%%%%%%%%
\begin{frame}
\frametitle{How to use Normal Distribution table?}

\begin{itemize}
\item First need to calculate how many standard deviations above (or below) the mean a
particular value is, i.e., calculate the value of the ``standard score'' or ``Z-score''.
\item Use the following formula to convert a raw data value, $X$, to a standard score:
$Z = \frac{(X - \mu)}{\sigma}$
\item eg. Suppose a particular population has $\mu= 4$ and $\sigma = 2$. Find the probability of a
randomly selected value being greater than 6
\item The Z score corresponding to $X = 6$ is $Z = \frac{(6 - 4)}{2} = 1$
\item $Z=1$ means that the value $X = 6$ is 1 standard deviation away from the mean
\item Use  standard normal tables to find $P(Z>1) = 0.6587$
\end{itemize}


{\tiny (Ref: Normal, Binomial, Poisson Distributions -  Lincoln University)}
\end{frame}

%%%%%%%%%%%%%%%%%%%%%%%%%%%%%%%%%%%%%%%%%%%%%%%%%%%%%%%%%%%
\begin{frame}
\frametitle{Example}
Wool fibre breaking strengths are normally distributed with mean $\mu = 23.56$ Newtons
and standard deviation, $\sigma = 4.55$.
What proportion of fibres would have a breaking strength of 14.45 or less? 
\begin{itemize}
\item Draw a diagram, label and shade area required

\begin{center}
\includegraphics[width=0.3\linewidth,keepaspectratio]{stats1}
\end{center}

\item Convert raw score $X$ to standard score $Z$: $Z = \frac{(14.45 - 23.56)}{4.55} = - 2.0$
\item That is, the raw score of 14.45 is equivalent to a standard score of -2.0.
\item It is negative because it is on the left hand side of the curve.
\item Use tables to find probability and adjust this result to required probability: 
\begin{align*}
p(X < 14.45) = p(Z ,-2.0) &= 0.5 - p(0< Z < 2)\\
&= 0.5 - 0.4772\\
&=0.0228
\end{align*}
\end{itemize}


{\tiny (Ref: Normal, Binomial, Poisson Distributions -  Lincoln University)}
\end{frame}


%%%%%%%%%%%%%%%%%%%%%%%%%%%%%%%%%%%%%%%%%%%%%%%%%%%%%%%%%%%
\begin{frame}
\frametitle{Inverse Process}
To find a value for X, corresponding to a given probability
\begin{itemize}
\item Draw a diagram and label
\item Shade area given as per question
\item Use probability tables to find Z -score
\item Convert standard score $Z$ to raw score $X$ using inverse formula
\end{itemize}


{\tiny (Ref: Normal, Binomial, Poisson Distributions -  Lincoln University)}
\end{frame}

%%%%%%%%%%%%%%%%%%%%%%%%%%%%%%%%%%%%%%%%%%%%%%%%%%%%%%%%%%%
\begin{frame}
\frametitle{Example}
Carrots entering a processing factory have an average length of 15.3 cm, and
standard deviation of 5.4 cm. If the lengths are approximately normally distributed,
what is the maximum length of the lowest 5\% of the load?
(i.e., what value cuts off the lowest 5 \%?)
\begin{itemize}
\item Draw a diagram, label and shade area required

\begin{center}
\includegraphics[width=0.3\linewidth,keepaspectratio]{stats2}
\end{center}

\item Use standard Normal tables to find the Z -score corresponding to this area of probability. 
\item Convert the standard score $Z$ to a raw score $X$ using the inverse formula: $X = Z \times \sigma + \mu$
\item For $p(Z < z) = 0.05$ the Normal tables give the corresponding z-score as -1.645.
(Negative because it is left of the mean.)
\item Hence the raw score is 
\begin{align*}
X &= Z \times \sigma + \mu\\
&= -1.645 \times 5.44 + 15.3\\
&=6.4
\end{align*}
\item ie the lowest maximum length is 6.4cm
\end{itemize}


{\tiny (Ref: Normal, Binomial, Poisson Distributions -  Lincoln University)}
\end{frame}

%%%%%%%%%%%%%%%%%%%%%%%%%%%%%%%%%%%%%%%%%%%%%%%%%%%%%%%%%%%%%%%%%%%%%%%%%%%%%%%%%%
\begin{frame}[fragile]\frametitle{}
\begin{center}
{\Large Binomial Distribution}
\end{center}
\end{frame}

%%%%%%%%%%%%%%%%%%%%%%%%%%%%%%%%%%%%%%%%%%%%%%%%%%%%%%%%%%%%%%%%%%%%%%%%
\begin{frame}[fragile]\frametitle{Binomial Distribution}
Example: Which drink people prefer: Tea or Coffee?


	\begin{itemize}
	\item If 4 out of 7 people prefer Tea, do we say that general population prefers Tea?
	\item Or it could be that people in general do not have any preference over each other and this observation is just due to random chance and a small sample size.
	\item If we had surveyed another set of 7 people, we have have had 4 people preferring Coffee.
	\item In such Yes/No outcomes, we use Binomial distribution as the model.
	\item We will see how data fits the model.
	\item If the model is a poor fit, we will reject the idea that both tea and coffee are loved equally (ie there is no preference within them).
	\end{itemize}



  
\tiny{(Ref: StatQuest: The Binomial Distribution and Test, Clearly Explained!!! - Josh Starmer )}
\end{frame}

%%%%%%%%%%%%%%%%%%%%%%%%%%%%%%%%%%%%%%%%%%%%%%%%%%%%%%%%%%%%%%%%%%%%%%%%
\begin{frame}[fragile]\frametitle{Binomial Distribution}
Example: Which drink people prefer: Tea or Coffee?


	\begin{itemize}
	\item If people really did not prefer Tea over Coffee (and vice versa), then we will assume that there is 50\% chance they will pick Tea and 50\% chance that they will pick Coffee.
	\item Out of 3 people, lets calculate probability of first 2 people choosing Tea and the last one choosing Coffee.
	\item 1st Tea: 0.5
	\item 2nd Tea: 0.5
	\item 3rd Coffee: 0.5
	\item Total: $0.5 \times 0.5 \times 0.5 = 0.125$ 
	\item This is true for 3 cases: TTC, TCT, CTT
	\item So, Probability that any 2 out of 3 people preferring Ta is sum of these = 0.125 + 0.125 + 0.125 = 0.375
	\item This is readily given by $p(x|n,p) = (\frac{n!}{x!(n-x)!})p^x(1-p)^{n-x}$
	\end{itemize}

 
\tiny{(Ref: StatQuest: The Binomial Distribution and Test, Clearly Explained!!! - Josh Starmer )}
\end{frame}

%%%%%%%%%%%%%%%%%%%%%%%%%%%%%%%%%%%%%%%%%%%%%%%%%%%%%%%%%%%%%%%%%%%%%%%%
\begin{frame}[fragile]\frametitle{Binomial Distribution}
$p(x|n,p) = (\frac{n!}{x!(n-x)!})p^x(1-p)^{n-x}$


	\begin{itemize}
	\item x: number of people preferring Tea (2)
	\item n : total number of people we asked (3)
	\item p: probability that someone will pick Tea (0.5)
	\item The first part of the formula is just combinations formula, how many ways 2 things can be arranged out of 3. Its $= (\frac{3!}{2!(3-2)!})=3$.
	\item $p^x$ shows Tea probability x times, ie $p^x = 0.5^2 = 0.25$
	\item $(1-p)^{n-x}$ shows the remaining, ie coffee's probability $(1-0.5)^{3-2} = 0.5$
	\item So, the probability of x( the number of people who prefer Tea), given n (the total number of people asked) and p (probability of picking Tea) is $= (\frac{3!}{2!(3-2)!})(0.5)^x(1-0.5)^{3-2}$
	\end{itemize}

 
\tiny{(Ref: StatQuest: The Binomial Distribution and Test, Clearly Explained!!! - Josh Starmer )}
\end{frame}

%%%%%%%%%%%%%%%%%%%%%%%%%%%%%%%%%%%%%%%%%%%%%%%%%%%%%%%%%%%%%%%%%%%%%%%%
\begin{frame}[fragile]\frametitle{Binomial Distribution}
Going back to the original example: out of 7 , 4 prefer Tea. Can we say population prefers Tea?


	\begin{itemize}
	\item x: 4
	\item n : 7
	\item p: 0.5
	\item So, the probability someone randomly would pick Tea is $= (\frac{7!}{4!(7-4)!})(0.5)^4(1-0.5)^{7-4} = 35 \times 0.5^4(1-0.5)^3 = 0.273$
	\item Whats the p value of 4 of 7 people preferring Tea? It is the probability of 4 of 7 people preferring Tea over Coffee, plus the probabilities of all other possibilities that are equally likely or rarer.
	\item Means, we need to calculate, TTTTCCC,TTTTTCC,TTTTTTC,TTTTTTT. First is observed, reaming three are rare possibilities.
	\item We will also need probabilities CCCCTTT, CCCCCTT, CCCCCCT, CCCCCCC. With this we are calculating two sided (tailed) p value.
	\item Tea heavy arrangements give $= 0.273 + 0.164 + 0.055 + 0.008$ similarly for Coffee heavy arrangements. Total probability $0.5 + 0.5 = 1$
	\item Thus this is a good fit, meaning both Tea and Coffee are equally preferred.
	\item 
	\end{itemize}

 
\tiny{(Ref: StatQuest: The Binomial Distribution and Test, Clearly Explained!!! - Josh Starmer )}
\end{frame}


%%%%%%%%%%%%%%%%%%%%%%%%%%%%%%%%%%%%%%%%%%%%%%%%%%%%%%%%%%%
\begin{frame}
\frametitle{Binomial Distribution in Practice}

\begin{itemize}
\item Applied to single variable discrete data where results are the numbers of
``successful outcomes'' in a given scenario. 
\begin{itemize}
\item Number of times the lights are red in 20 sets of traffic lights
\item Number of students with green eyes in a class of 40
\item Number of plants with diseased leaves from a sample of 50 plants
\end{itemize}
\item Used to calculate the probability of occurrences exactly, less than, more than,
between given values
\begin{itemize}
\item The probability that the number of red lights will be exactly 5
\item The probability that the number of green eyed students will be less than 7
\item The probability that the number of diseased plants will be more than 10
\end{itemize}
\end{itemize}


{\tiny (Ref: Normal, Binomial, Poisson Distributions -  Lincoln University)}
\end{frame}


%%%%%%%%%%%%%%%%%%%%%%%%%%%%%%%%%%%%%%%%%%%%%%%%%%%%%%%%%%%
\begin{frame}
\frametitle{Binomial Distribution in Practice}

\begin{center}
\includegraphics[width=0.8\linewidth,keepaspectratio]{stats3}
\end{center}

Read as “the probability of getting $x$ successes is equal to the number of ways of choosing ``$x$  successes from n trials'' times ``the probability of success to
the power of the number of successes required'' times ''the probability of failure to
the power of the number of resulting failures.''


{\tiny (Ref: Normal, Binomial, Poisson Distributions -  Lincoln University)}
\end{frame}


%%%%%%%%%%%%%%%%%%%%%%%%%%%%%%%%%%%%%%%%%%%%%%%%%%%%%%%%%%%
\begin{frame}
\frametitle{Example }

An automatic camera records the number of cars running a red light at an
intersection (that is, the cars were going through when the red light was against the
car). Analysis of the data shows that on average 15\% of light changes record a car
running a red light. Assume that the data has a binomial distribution. What is the
probability that in 20 light changes there will be exactly three (3) cars running a red
light?

\begin{itemize}
\item Write out the key statistics from the information given $p =0.15, n = 20, X = 3$
\item Apply the formula, substituting these values: $P(X=3) = {20 \choose 3} 0.15^3 0.85^{17} = 0.243$
\item That is, the probability that in 20 light changes there will be three (3) cars running a
red light is 0.24 (24\%)
\end{itemize}

{\tiny (Ref: Normal, Binomial, Poisson Distributions -  Lincoln University)}
\end{frame}


%%%%%%%%%%%%%%%%%%%%%%%%%%%%%%%%%%%%%%%%%%%%%%%%%%%%%%%%%%%%%%%%%%%%%%%%%%%%%%%%%%
\begin{frame}[fragile]\frametitle{}
\begin{center}
{\Large Poisson Distribution}
\end{center}
\end{frame}

%%%%%%%%%%%%%%%%%%%%%%%%%%%%%%%%%%%%%%%%%%%%%%%%%%%%%%%%%%%
\begin{frame}
\frametitle{Poisson Distribution in Practice}

This is often known as the distribution of rare events. Firstly, a Poisson process is
where DISCRETE events occur in a CONTINUOUS, but finite interval of time or
space. The following conditions must apply:


\begin{itemize}
\item For a small interval the probability of the event occurring is proportional to the size of the interval.
\item The probability of more than one occurrence in the small interval is negligible (i.e. they are rare events). Events must not occur simultaneously
\item Each occurrence must be independent of others and must be at random.
\item  The events are often defects, accidents or unusual natural happenings, such as
earthquakes, where in theory there is no upper limit on the number of events.
The interval is on some continuous measurement such as time, length or area
\end{itemize}


{\tiny (Ref: Normal, Binomial, Poisson Distributions -  Lincoln University)}
\end{frame}



%%%%%%%%%%%%%%%%%%%%%%%%%%%%%%%%%%%%%%%%%%%%%%%%%%%%%%%%%%%
\begin{frame}
\frametitle{Poisson Distribution in Practice}
\begin{itemize}
\item The parameter for the Poisson distribution is $\lambda$(lambda). 
\item It is the average or mean
number of occurrences over a given interval.
\item The probability function is $p(x) = \frac{e^{-\lambda}\lambda^x}{x!}$ for $x=0,1,2,\ldots$
\item Example: The average number of accidents at a level-crossing every year
is 5. Calculate the probability that there are exactly 3 accidents
there this year.
\item Here, $\lambda = 5$, and $x = 3$.
\item $p(X=3) = \frac{e^{-5}5^3}{3!} = 0.1404$
\item That is, there is a 14\% chance that there will be exactly 3 accidents there this year
\end{itemize}


{\tiny (Ref: Normal, Binomial, Poisson Distributions -  Lincoln University)}
\end{frame}