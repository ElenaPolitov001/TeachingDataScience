%%%%%%%%%%%%%%%%%%%%%%%%%%%%%%%%%%%%%%%%%%%%%%%%%%%%%%%%%%%%%%%%%%%%%%%%%%%%%%%%%%
\begin{frame}[fragile]\frametitle{}
\begin{center}
{\Large Sets}
\end{center}
\end{frame}

%%%%%%%%%%%%%%%%%%%%%%%%%%%%%%%%%%%%%%%%%%%%%%%%%%%%%%%%%%
\begin{frame}{Sets}
\begin{example}
The set consisting of all positive integers less than $10$ can be denoted by $\{1,2,3,4,5,6,7,8,9\}$.  $\{1,2,3,\ldots,9\}$ is also used since the general pattern is obvious.
\end{example}
\end{frame}


%%%%%%%%%%%%%%%%%%%%%%%%%%%%%%%%%%%%%%%%%%%%%%%%%%%%%%%%%%
\begin{frame}{Sets}
\begin{example}
Members of a set need not be numeric.  The set of all possible outcomes when tossing a coin is  $\{H,T\}$.
\end{example}
\end{frame}

%%%%%%%%%%%%%%%%%%%%%%%%%%%%%%%%%%%%%%%%%%%%%%%%%%%%%%%%%%
\begin{frame}{Set Builder Notation}
For example, $O=\{1,3,5,7,9\}$ can also be written as $$O=\{x: x \text{ is a positive odd integer less than } 10\}.$$

\begin{example} 
\begin{itemize}
\item $A=\{x: x \text{ is a positive even number less than }15\}=\{2,4,6,8,10,12,14\}$.
\item $B=\{n: n \text{ is a positive integer}\}=\{1,2,3,4,5,\ldots\}$.
\item $C=\{2n: n \text{ is a whole number and }1\leq n\leq 4\}=\{2,4,6,8\}$.
\end{itemize}
\end{example}
\end{frame}

%%%%%%%%%%%%%%%%%%%%%%%%%%%%%%%%%%%%%%%%%%%%%%%%%%%%%%%%%%
\begin{frame}{Some Basic Notations}

\begin{itemize}
\item $\emptyset$, The \underline{empty set}\index{empty set} (a set with \textit{no} elements). $\{\quad\}$ is also used. 
\item $\mathbb{N}=\{1,2,3,\ldots\}$, the set of natural numbers.\index{natural number}\index{$\mathbb{N}$}
\item $\mathbb{Z}=\{\ldots,-2,-1,0,1,2,\ldots\}$, the set of integers. \index{integer}\index{$\mathbb{Z}$}
\item $\mathbb{R}$ or $(-\infty,\infty)$, the set of real numbers. \index{real number}\index{$\mathbb{R}$}
\item $[a,b]$, the set of all real numbers $x$ such that $a\leq x \leq b$.
\item $(a,b)$, the set of all real numbers $x$ such that $a< x < b$. 
\item $(a,b]$, the set of all real numbers $x$ such that $a< x \leq  b$. 
\item $[a,b)$, the set of all real numbers $x$ such that $a\leq x< b$. 
\end{itemize}

\tiny{(Ref: MAT 1348 B: Discrete Mathematics for Computing - Supartha Podder)}

\end{frame}

%%%%%%%%%%%%%%%%%%%%%%%%%%%%%%%%%%%%%%%%%%%%%%%%%%%%%%%%%%
\begin{frame}{Subsets and Equality}
\begin{definition} A set $A$ is said to be a \underline{subset} of another set $B$  if every element of $A$ is also an element of $B$.  We use the notation $A\subseteq B$.  Two sets $A$ and $B$ are said to be \underline{equal} (notation $A=B$)  if they have the same elements.  In other words, $A$ and $B$ are equal if $A\subseteq B$ and $B\subseteq A$.
\end{definition}
\end{frame}

%%%%%%%%%%%%%%%%%%%%%%%%%%%%%%%%%%%%%%%%%%%%%%%%%%%%%%%%%%
\begin{frame}{Subsets and Equality}
\begin{example} $\{2,3,5\}\subseteq \{-1,2,3,5,7\}$.  However, $\{1,2,3,4\} \nsubseteq \{1,3,4,5,6,7\}$,  because $2\notin \{1,3,4,5,6,7\}$.
\end{example}

\begin{example} By definition, $\mathbb{N}\subseteq \mathbb{Z}\subseteq \mathbb{R}$.
\end{example}
\end{frame}

%%%%%%%%%%%%%%%%%%%%%%%%%%%%%%%%%%%%%%%%%%%%%%%%%%%%%%%%%%
\begin{frame}{Set Operations}
 Let $A$ and $B$ be sets. 
\begin{itemize}
\item The \underline{union} of $A$ and $B$, denoted by $A\cup B$, is the set that contains those elements that are either in $A$ or $B$, or in both.  In other words, $A\cup B=\{x: x\in A \text{ or } x\in B\}$. 
\item The \underline{intersection} of $A$ and $B$, denoted by $A\cap B$, is the set that contains those elements in both $A$ and $B$.  In other words, $A\cap B=\{x: x\in A \text{ and } x\in B\}$. 
\item Two sets are said to be \underline{disjoint} if their intersection is the empty set. 
\item Let $S$ be the universal set (the set of all elements under consideration. It depends on the context).  Then the \underline{complement} $A^c$ of $A$ is defined to be the set of all elements in $S$ that are not in $A$.
\end{itemize}

\end{frame}

%%%%%%%%%%%%%%%%%%%%%%%%%%%%%%%%%%%%%%%%%%%%%%%%%%%%%%%%%%
\begin{frame}{Set Operations, continued}
\begin{example} Let $A=\{1,3,4,5\}$ and $B=\{1,2,3\}$, then $A\cup B=\{1,2,3,4,5\}$ and $A\cap B=\{1,3\}$.
\end{example}

\begin{center}
\includegraphics[width=0.4\linewidth,keepaspectratio]{setun}
\includegraphics[width=0.4\linewidth,keepaspectratio]{setint}
\end{center}
\end{frame}

%%%%%%%%%%%%%%%%%%%%%%%%%%%%%%%%%%%%%%%%%%%%%%%%%%%%%%%%%%
\begin{frame}{Set Operations, continued}
\begin{example}
If $P=(-\infty, 2)$ and $Q=[-1,\infty)$, then $P\cap Q=[-1,2)$ and $P\cup Q=(-\infty,\infty)=\mathbb{R}$.
\end{example}
\end{frame}

%%%%%%%%%%%%%%%%%%%%%%%%%%%%%%%%%%%%%%%%%%%%%%%%%%%%%%%%%%
\begin{frame}{Set Operations, continued}
\begin{example} Let $O=\{2k+1: k\in \mathbb{Z}\}$ and $E=\{2k: k\in \mathbb{Z}\}$, then $O$ and $E$ are disjoint.  If one takes $\mathbb{Z}$ as the universal set, then $O^c=E$ and $E^c=O$.
\end{example}

\end{frame}

% %%%%%%%%%%%%%%%%%%%%%%%%%%%%%%%%%%%%%%%%%%%%%%%%%%%%%%%%%%
% \begin{frame}{Set algebra/Operations}
% Equality

% \begin{lstlisting}
% S1 = {1,2}
% S2 = {2,2,1,1,2}
% print ("S1 and S2 are equal because order or repetition of elements do not matter for sets\nS1==S2:", S1==S2)

% S1 and S2 are equal because order or repetition of elements do not matter for sets
% S1==S2: True

% S1 = {1,2,3,4,5,6}
% S2 = {1,2,3,4,0,6}
% print ("S1 and S2 are NOT equal because at least one element is different\nS1==S2:", S1==S2)

% S1 and S2 are NOT equal because at least one element is different
% S1==S2: False
% \end{lstlisting}

% \end{frame}

% %%%%%%%%%%%%%%%%%%%%%%%%%%%%%%%%%%%%%%%%%%%%%%%%%%%%%%%%%%
% \begin{frame}{Set algebra/Operations}
% Intersection

% In mathematics, the intersection of two sets A and B is the set that contains all elements of A that also belong to B (or equivalently, all elements of B that also belong to A), but no other elements. Formally, $A\cap B=\{x:x\in A{\text{ and }}x\in B\}$

% \begin{lstlisting}
% # Define a set using list comprehension
% S1 = set([x for x in range(1,11) if x%3==0])
% print("S1:", S1)
% >>S1: {9, 3, 6}
% S2 = set([x for x in range(1,7)])
% print("S2:", S2)
% >>S2: {1, 2, 3, 4, 5, 6}
% # Both intersection method or & can be used
% S_intersection = S1.intersection(S2)
% print("Intersection of S1 and S2:", S_intersection)

% S_intersection = S1 & S2
% print("Intersection of S1 and S2:", S_intersection)
% >>Intersection of S1 and S2: {3, 6}
% >>Intersection of S1 and S2: {3, 6}
% S3 = set([x for x in range(6,10)])
% print("S3:", S3)
% S1_S2_S3 = S1.intersection(S2).intersection(S3)
% print("Intersection of S1, S2, and S3:", S1_S2_S3)
% >>S3: {8, 9, 6, 7}
% >>Intersection of S1, S2, and S3: {6}
% \end{lstlisting}

% \end{frame}

% %%%%%%%%%%%%%%%%%%%%%%%%%%%%%%%%%%%%%%%%%%%%%%%%%%%%%%%%%%
% \begin{frame}{Set algebra/Operations}
% Union

% In set theory, the union of a collection of sets is the set of all elements in the collection. It is one of the fundamental operations through which sets can be combined and related to each other. Formally, $A\cup B=\{x:x\in A{\text{ or }}x\in B\}$

% \begin{lstlisting}
% # Both union method or | can be used
% S1 = set([x for x in range(1,11) if x%3==0])
% print("S1:", S1)
% S2 = set([x for x in range(1,5)])
% print("S2:", S2)

% S_union = S1.union(S2)
% print("Union of S1 and S2:", S_union)
% S_union = S1 | S2
% print("Union of S1 and S2:", S_union)

% >>S1: {9, 3, 6}
% >>S2: {1, 2, 3, 4}
% >>Union of S1 and S2: {1, 2, 3, 4, 6, 9}
% >>Union of S1 and S2: {1, 2, 3, 4, 6, 9}
% \end{lstlisting}

% \end{frame}


%%%%%%%%%%%%%%%%%%%%%%%%%%%%%%%%%%%%%%%%%%%%%%%%%%%%%%%%%%
\begin{frame}{Set algebra laws}
Commutative law:

$A\cap B=B\cap A$

$A\cup (B\cup C)=(A\cup B)\cup C$

Associative law:

$(A\cap B)\cap C=A\cap (B\cap C)$

$A\cap (B\cup C)=(A\cap B)\cup (A\cap C)$

Distributive law:

$A\cap (B\cup C)=(A\cap B)\cup (A\cap C)$

$A\cup (B\cap C)=(A\cup B)\cap (A\cup C)$

\end{frame}


%%%%%%%%%%%%%%%%%%%%%%%%%%%%%%%%%%%%%%%%%%%%%%%%%%%%%%%%%%
\begin{frame}{Cardinality of the Union of Sets}
Let $|S|$ denote the number of elements of a set $S$.  Obviously, $A\cup B$ contains both $A$ and $B$, so $|A\cup B|\geq |A|$ and $|A\cup B|\geq |B|$.  But exactly how many elements are there in $A\cup B$?

\begin{theorem} Let $A,B$ be sets with finitely many elements, then
$$|A\cup B|=|A|+|B|-|A\cap B|.$$  In particular, if $A$ and $B$ are disjoint, then $|A\cup B|=|A|+|B|$.
\end{theorem}

\end{frame}


%%%%%%%%%%%%%%%%%%%%%%%%%%%%%%%%%%%%%%%%%%%%%%%%%%%%%%%%%%
\begin{frame}{Cardinality of the Union of Sets}

\begin{theorem} Let $A,B,C$ be sets with finitely many elements, then
$$|A\cup B \cup C|=|A|+|B|+|C|-|A\cap B|-|B\cap C|-|A\cap C|+|A\cap B \cap C|.$$ 
\end{theorem}
\end{frame}

%%%%%%%%%%%%%%%%%%%%%%%%%%%%%%%%%%%%%%%%%%%%%%%%%%%%%%%%%%%%%%%%%%%%%%%%%%%%%%%%%%
\begin{frame}[fragile]\frametitle{}
\begin{center}
{\Large Advanced Set: Algebra}
\end{center}
\end{frame}


%%%%%%%%%%%%%%%%%%%%%%%%%%%%%%%%%%%%%%%%%%%%%%%%%%%%%%%%%%
\begin{frame}{Sets Summary}
\begin{itemize}
\item  A set is a collection of unique elements.
  The definition of a specific set determines which elements
  are members of the set.
  Elements not specifically defined as members of a set are not
  in the set.

\item The cardinality of a set is the count of the number of elements
  in a set based on the sets definition. The cardinality may be
  finite or infinite.

\item Cardinality example: The cardinality of the set of dwarfs in
  the Snow White story is 7.

\item The cardinality of the set of integers is NOT the same as the
  cardinality of the set of real numbers. Both are infinite.
\end{itemize}

\tiny{(ref: Sets, Groups, Rings and Algebras - CSEE UMBC)}

\end{frame}

%%%%%%%%%%%%%%%%%%%%%%%%%%%%%%%%%%%%%%%%%%%%%%%%%%%%%%%%%%
\begin{frame}{Algebra}

\begin{itemize}
\item  An algebra is a set of elements and a set of laws that apply to the elements.

\item One way to define various types of algebras such as rings, fields, Galois
Fields and the like, is to list the possible laws (axioms, postulates, rules)
that might apply, then define each algebra in terms of which laws apply.

\item Note: The particular symbols and the particular operations are not important.
      Addition and multiplication are commonly used for convenience, yet the
      logical operations "and" and "or could be used, the set operations
      union and intersection could be used, as well as many other pairs of
      operations.
\end{itemize}


\tiny{(ref: Sets, Groups, Rings and Algebras - CSEE UMBC)}

\end{frame}

%%%%%%%%%%%%%%%%%%%%%%%%%%%%%%%%%%%%%%%%%%%%%%%%%%%%%%%%%%
\begin{frame}{Algebra Laws}

Closure Laws:
\begin{description}
\item[A0]. Addition is well defined, for every ordered pair
                      $a, b \in S, c = a+b$ implies c is a unique element in S.
\item[M0] Multiplication is well defined, for every ordered pair
                      $a, b \in S, c = a*b$ implies c is a unique element in S.
\end{description}

Associative Laws: 
\begin{description}
\item[A1] $(a+b)+c = a+(b+c)$  {we do not repeat the obvious  
                                          for all a, b, c in S, after this}
\item[M1]. $(a*b)+c = a*(b*c)$
\end{description}
					
									
\tiny{(ref: Sets, Groups, Rings and Algebras - CSEE UMBC)}

\end{frame}


%%%%%%%%%%%%%%%%%%%%%%%%%%%%%%%%%%%%%%%%%%%%%%%%%%%%%%%%%%
\begin{frame}{Algebra Laws}

Commutative  Laws:
\begin{description}
\item[A2] Addition $a+b = b+a$
\item[M2] Multiplication $a*b = b*a$
\end{description}

Identity  Laws: 
\begin{description}
\item[A3] 0 is in S and  $0+a = a+0 = a $
\item[M3] 1 is in S and  $1*a = a*1 = a$
\end{description}
				
Inverse   Laws: 
\begin{description}
\item[A4] for every a in S, -a is a unique element in S and
                      $(-a)+a = a+(-a) = 0$
\item[M4] 1 is in S and  $1*a = a*1 = a$
\end{description}				
									
\tiny{(ref: Sets, Groups, Rings and Algebras - CSEE UMBC)}

\end{frame}

%%%%%%%%%%%%%%%%%%%%%%%%%%%%%%%%%%%%%%%%%%%%%%%%%%%%%%%%%%
\begin{frame}{Algebra Laws}

Commutative  Laws:
\begin{description}
\item[A2]. Addition $a+b = b+a$
\item[M2] Multiplication $a*b = b*a$
\end{description}

Identity  Laws: 
\begin{description}
\item[A3] 0 is in S and  $0+a = a+0 = a $
\item[M3] 1 is in S and  $1*a = a*1 = a$
\end{description}
				
Inverse   Laws: 
\begin{description}
\item[A4]for every a in S, -a is a unique element in S and
                      $(-a)+a = a+(-a) = 0$
\item[M4] for every a in S except 0, $a**-1$ is a unique element in S
                      and $(a**-1)*a = a*(a**-1) = 1$
\end{description}				
									
Distributive    Laws: 
\begin{description}
\item[D1] $a(b+c) = ab + ac$
\item[D2] $(b+c)a = ba + ca$
\end{description}				
																		
\tiny{(ref: Sets, Groups, Rings and Algebras - CSEE UMBC)}

\end{frame}

%%%%%%%%%%%%%%%%%%%%%%%%%%%%%%%%%%%%%%%%%%%%%%%%%%%%%%%%%%
\begin{frame}{Sets Objects}
\begin{itemize}
\item Groups, Rings, and Fields are  all sets of elements with additional structure
\item Meaning, various ways of combining elements to produce an element of the set. 
\item A GROUP is a set in which you can perform one operation (usually
addition or multiplication mod n for us) with some nice properties. 
\item A RING is a set equipped with two operations, called addition and multiplication. A RING is a GROUP under addition
and satisfies some of the properties of a group for multiplication. 
\item A FIELD is a GROUP under both addition and multiplication.

\item ${\mathbb Z} / n {\mathbb Z}$, fancy notation for the integers mod $n$ under addition.
\item $({\mathbb Z} / n {\mathbb Z})^\times$, more fancy notation for the integers mod $n$ under multiplication.
\end{itemize}

\tiny{(ref: The Very Basics of Groups, Rings, and Fields)}

\end{frame}

%%%%%%%%%%%%%%%%%%%%%%%%%%%%%%%%%%%%%%%%%%%%%%%%%%%%%%%%%%
\begin{frame}{Group}
\begin{itemize}
\item  A group is an algebraic system consisting of a set, an identity element,
  one operation and its inverse operation.
\item A example group, $G = ( S, O, I )$

	\begin{itemize}
	\item  S is set of integers
	\item O is the operation of addition, the inverse operation is subtraction
	\item I is the identity element zero (0)
	\end{itemize}
	
\item Another example group, G = ( S, O, I )

	\begin{itemize}
	\item  S is set of real numbers excluding zero
	\item O is the operation of multiplication, the inverse operation is division
	\item I is the identity element one (1)
	\end{itemize}	
	
\item   The operation does not have to be addition or multiplication.
\item   The set does not have to be numeric.	
\end{itemize}

\tiny{(ref: Sets, Groups, Rings and Algebras - CSEE UMBC)}

\end{frame}


%%%%%%%%%%%%%%%%%%%%%%%%%%%%%%%%%%%%%%%%%%%%%%%%%%%%%%%%%%
\begin{frame}{Group Axioms}

let a, b and c be elements of a group

\begin{description}
\item[G1: Closure] The operation can be applied to any two elements of the group
        and the result is an element of the group.
        $\forall a, b and c  O(a,b)=c$
	
\item[G2: Associative].
        $\forall a, b and c  (a+b)+c = a+(b+c)$ if operation is addition
                            $(ab)c = a(bc)$ if operation is multiplication 

\item[G3: Identity element]
        The identity element must be a member of the group and is 
        its own inverse. The identity element is provably unique,
        there is exactly one identity element.
        $0+a=a+0=a$ if operation is addition
        $1a=a1=a$ if operation is multiplication

\item[G4: Inverse]
        Every element of the group has an inverse element in the group.
        $a+(-a) = (-a)+a = 0$ if operation is addition
        $aa  = a  a = 1$ if operation is multiplication	
\end{description}

\tiny{(ref: Sets, Groups, Rings and Algebras - CSEE UMBC)}

\end{frame}

%%%%%%%%%%%%%%%%%%%%%%%%%%%%%%%%%%%%%%%%%%%%%%%%%%%%%%%%%%
\begin{frame}{Groups}
\begin{Definition} A {\bf GROUP} is a set $G$ which is CLOSED under an operation $\ast$ (that is, for any $x,y \in G$, $x \ast y \in G$) and satisfies the following properties:
\begin{enumerate}
\item Identity -- There is an element $e$ in $G$, such that for every $x \in G$, $e \ast x = x \ast e = x$.
\item Inverse -- For every $x$ in $G$ there is an element $y \in G$ such that $x \ast y = y \ast x = e$, where again $e$ is the identity.
\item Associativity -- The following identity holds for every $x,y,z \in G$:
$$ x \ast (y \ast z) = (x \ast y) \ast z $$
\end{enumerate}
\end{Definition}


\tiny{(ref: The Very Basics of Groups, Rings, and Fields)}

\end{frame}


%%%%%%%%%%%%%%%%%%%%%%%%%%%%%%%%%%%%%%%%%%%%%%%%%%%%%%%%%%
\begin{frame}{Groups}

\begin{itemize}
\item A group is said to be ``abelian" if $x \ast y = y \ast x$ for every $x,y \in G$ 
\item An Abelian Group or commutative group has an additional axiom
\begin{itemize}
\item $a+b = b+a$  if the operation is addition
\item $ab = ba$  if the operation is multiplication
\end{itemize}
\item A Lie Group is a topological group with only countably many connected
  components and an identity element that is open.
	\item A Cyclic Group is a group that has elements that are all powers of one
  of its elements.
\end{itemize}


\tiny{(ref: Sets, Groups, Rings and Algebras - CSEE UMBC)}

\end{frame}


%%%%%%%%%%%%%%%%%%%%%%%%%%%%%%%%%%%%%%%%%%%%%%%%%%%%%%%%%%
\begin{frame}{Abelian Groups}

A set $G$ with a binary operation $*$ is called an
\textsc{abelian group}, if $(G,*)$ satisfies the following axioms:

\begin{description}
\item[G1] $*$ is a well-defined map from $G\times G$ to
$G$.
\item[G2 (Associativity)] For all $a,b,c\in G$
\[(a*b)*c=a*(b*c)\]
\item[G3 (Existence of a neutral element)] There is an element
$n\in G$ such that for all $a\in G$
\[a*n=a\]
\item[G4 (Existence of inverse elements)] For every $a\in G$ there exists $b\in
G$ such that
\[a*b=n\]
\item[G5 (Commutativity)] For all $a,b\in G$
\[a*b=b*a\]
\end{description}


\tiny{(Ref: Helmut Knaust Class 200510 3341)}

\end{frame}

%%%%%%%%%%%%%%%%%%%%%%%%%%%%%%%%%%%%%%%%%%%%%%%%%%%%%%%%%%
\begin{frame}{Abelian Groups Examples}


\begin{itemize}
\item The sets $\mathbb{Z},\mathbb{Q}$ and $\mathbb{R}$ are examples of abelian groups
when endowed with the usual addition $+$\ . The natural element
in these cases is $0$; it is customary to denote the inverse
element of $a$ by $-a$.

\item The sets $\mathbb{Q}\setminus\{0\}$ and $\mathbb{R}\setminus\{0\}$ also form
abelian groups under the usual multiplication $\cdot$\ . In these
cases we denote the natural element by $1$; the inverse element
of $a$ is customarily denoted by $1/a$ or by $a^{-1}$.
\end{itemize}

\tiny{(Ref: Helmut Knaust Class 200510 3341)}

\end{frame}


%%%%%%%%%%%%%%%%%%%%%%%%%%%%%%%%%%%%%%%%%%%%%%%%%%%%%%%%%%
\begin{frame}{Groups Axioms}


\begin{enumerate} 
\item CLOSURE: Given any two integers mod $n$, their sum (via addition modulo $n$) is an integer mod $n$ by definition. (Again, to be clear, the operation $\ast$ described above is addition modulo $n$.)
\item IDENTITY: 0 mod $n$ is the identity element, since $a \ast 0$ means $a + 0$ mod $n$, which is clearly $a$ mod $n$.
\item INVERSE: Given any $a$ (mod $n$), we must find an inverse $b$ so that $a \ast b = e$ in the group, i.e. $a + b \equiv 0$ (mod $n$). The inverse to any $a$ in this case is $n-a$.
\item ASSOCIATIVITY: The integers are associative, by basic rules of addition, so the integers mod $n$ are also associative. That is, since
$$ a + (b+c) = (a+b) + c, \quad \text{then it follows that} \quad a + (b+c) \equiv (a+b)+c \; (n) $$ 
\end{enumerate}

\tiny{(ref: The Very Basics of Groups, Rings, and Fields)}

\end{frame}

%%%%%%%%%%%%%%%%%%%%%%%%%%%%%%%%%%%%%%%%%%%%%%%%%%%%%%%%%%
\begin{frame}{Ring}

\begin{itemize}
\item A ring is an algebraic system consisting of a set, an identity element,
  two operations and the inverse operation of the first operation.
\item An example ring, $R = ( S, O1, O2, I )$
\begin{itemize}
\item S is set of real numbers
\item  O1 is the operation of addition, the inverse operation is subtraction
\item O2 is the operation of multiplication
\item I is the identity element zero (0)
\end{itemize}
\end{itemize}


\tiny{(ref: Sets, Groups, Rings and Algebras - CSEE UMBC)}

\end{frame}

%%%%%%%%%%%%%%%%%%%%%%%%%%%%%%%%%%%%%%%%%%%%%%%%%%%%%%%%%%
\begin{frame}{Rings}

\begin{Definition} A {\bf RING} is a set $R$ which is CLOSED under two operations $+$ and $\times$ and satisfying the following properties:
\begin{enumerate}
\item $R$ is an abelian group under $+$.
\item Associativity of $\times$ -- For every $a,b,c \in R$,
$$ a \times (b \times c) = (a \times b) \times c $$
\item Distributive Properties -- For every $a,b,c \in R$ the following identities hold:
$$ a \times (b + c) = (a \times b) + (a \times c) $$
and
$$ (b + c) \times a = b \times a + c \times a. $$
\end{enumerate}
\end{Definition}




\tiny{(ref: The Very Basics of Groups, Rings, and Fields)}

\end{frame}

%%%%%%%%%%%%%%%%%%%%%%%%%%%%%%%%%%%%%%%%%%%%%%%%%%%%%%%%%%
\begin{frame}{Rings Examples}

{\bf Examples:} \begin{enumerate} \item Both the examples ${\mathbb Z} / n {\mathbb Z}$ and ${\mathbb Z}$ from before are also RINGS. Note that we don't require multiplicative inverses.
\item ${\mathbb Z} [ x ]$, fancy notation for all polynomials with integer coefficients. Multiplication and addition is the usual multiplication and addition of polynomials.
\end{enumerate}


\tiny{(ref: The Very Basics of Groups, Rings, and Fields)}

\end{frame}

%%%%%%%%%%%%%%%%%%%%%%%%%%%%%%%%%%%%%%%%%%%%%%%%%%%%%%%%%%
\begin{frame}{Field}

\begin{itemize}
\item  A field is an algebraic system consisting of a set, an identity
  element for each operation, two operations and their respective inverse
  operations.
\item An example field, $F = ( S, O1, O2, I1, I2 )$
\begin{itemize}
\item S is set of real numbers
\item  O1 is the operation of addition, the inverse operation is subtraction
\item O2 is the operation of multiplication
\item I1 is the identity element zero (0)
\item I2 is the identity element one (1)
\end{itemize}
\end{itemize}


\tiny{(ref: Sets, Groups, Rings and Algebras - CSEE UMBC)}

\end{frame}



%%%%%%%%%%%%%%%%%%%%%%%%%%%%%%%%%%%%%%%%%%%%%%%%%%%%%%%%%%
\begin{frame}{Fields}

\begin{Definition} A {\bf FIELD} is a set $F$ which is closed under two operations $+$ and $\times$ such that
\begin{enumerate} \item $F$ is an abelian group under $+$
and
\item $F - \{ 0 \}$ (the set $F$ without the additive identity $0$) is an abelian group under $\times$.
\end{enumerate}
\end{Definition}

In short, a set $F$ together with an addition $+$ and a
multiplication $\cdot$ is called a \textsc{field}, if
\begin{description}
\item[F1] $(F,+)$ is an abelian group (with neutral element $0$).
\item[F2] $(F\setminus\{0\},\cdot)$ is an abelian group (with neutral element $1$).
\item[F3] For all $a,b,c \in F:\ (a+b)\cdot c= (a\cdot c) + (b\cdot c).$
\end{description}



\tiny{(ref: The Very Basics of Groups, Rings, and Fields)}

\end{frame}

%%%%%%%%%%%%%%%%%%%%%%%%%%%%%%%%%%%%%%%%%%%%%%%%%%%%%%%%%%
\begin{frame}{Fields Examples}

\begin{itemize}

\item The rational numbers and the real numbers are examples of fields.
\item Another example of a field is the set of complex numbers ${\mathbb C}$.
\[{\mathbb C}=\{a+bi \ |\ a,b\in {\mathbb R}\}\]

\item {\bf Examples:} ${\mathbb Z} / p {\mathbb Z}$ is a field, since ${\mathbb Z} / p {\mathbb Z}$ is an additive group and $({\mathbb Z} / p {\mathbb Z}) - \{ 0 \} = ({\mathbb Z} / p {\mathbb Z})^\times$ is a group under multiplication. Sometimes when we (or Cox) want to emphasize that ${\mathbb Z} / p {\mathbb Z}$ is a field, we use the notation ${\mathbb F}_p$. Other examples: ${\mathbb R}$, the set of real numbers, and ${\mathbb C}$, the set of complex numbers are both infinite fields. So is ${\mathbb Q}$, the set of rational numbers, but not ${\mathbb Z}$, the integers. (What fails?)
\item {\bf Another NON-Example:} If $n$ is not a prime, then ${\mathbb Z} / n {\mathbb Z}$ is not a field, since $({\mathbb Z} / n {\mathbb Z}) - \{ 0 \} \ne ({\mathbb Z} / n {\mathbb Z})^\times$. There are, in general, lots of other elements than 0 which are not relatively prime to $n$ and hence have no inverse under multiplication.
\end{itemize}


\tiny{(ref: The Very Basics of Groups, Rings, and Fields)}

\end{frame}

%%%%%%%%%%%%%%%%%%%%%%%%%%%%%%%%%%%%%%%%%%%%%%%%%%%%%%%%%%%
 \begin{frame}[fragile]\frametitle{Ordered Fields}


A field $F$ endowed with a relation $<$ is called an
\textsc{ordered field} if
\begin{description}
\item[O1] For all $x,y,z\in F$
\[x<y \mbox{ implies } x+z<y+z\]
\item[O2] For all $x,y\in F$ and all $z>0$
\[x<y \mbox{ implies } x\cdot z<y\cdot z\]
\item[O3] For all $x,y,z\in F$
\[x<y \mbox{ and } y<z \mbox{ implies }x<z\]

\item[O4] For all $x,y\in F$
\[x<y,\quad y<x,\quad \mbox{or } x=y\]
\end{description}

Both the rational numbers ${\mathbb R}$ and the real numbers ${\mathbb R}$
form ordered fields. The complex numbers ${\mathbb C}$ cannot be ordered
in such a way.

\tiny{(Ref: Helmut Knaust Class 200510 3341)}

\end{frame}

%%%%%%%%%%%%%%%%%%%%%%%%%%%%%%%%%%%%%%%%%%%%%%%%%%%%%%%%%%
\begin{frame}{Ideal}

\begin{itemize}
\item  An Ideal, I, is a subset of a Ring, R, with the properties:
\begin{itemize}
\item I is a subgroup of the additive group of R and
\item for every i in I and every r in R, ir and ri are in I.
\end{itemize}
\item   Example: The set of all multiples of any integer is an Ideal.

\end{itemize}


\tiny{(ref: Sets, Groups, Rings and Algebras - CSEE UMBC)}

\end{frame}


%%%%%%%%%%%%%%%%%%%%%%%%%%%%%%%%%%%%%%%%%%%%%%%%%%%%%%%%%%%
 \begin{frame}[fragile]\frametitle{The Completeness Axiom}

You probably have seen books entitled ``Real Analysis" and
``Complex Analysis" in the library. There are no books on
``Rational Analysis".

Why? What is the main difference between the two ordered fields of
${\mathbb Q}$ and ${\mathbb R}$?---The ordered field ${\mathbb R}$ of real numbers
is \textsc{complete}: sequences of real numbers have the following
property.

\begin{description}
\item[C] Let $(a_n)$ be an  increasing sequence of real numbers.
If $(a_n)$ is bounded from above, then $(a_n)$ converges.
\end{description}

The ordered field ${\mathbb R}$ of rational numbers, on the other hand,
is \textbf{not} complete.

The complex numbers ${\mathbb C}$ also form a complete field.

\tiny{(Ref: Helmut Knaust Class 200510 3341)}

\end{frame}