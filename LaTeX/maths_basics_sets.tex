%%%%%%%%%%%%%%%%%%%%%%%%%%%%%%%%%%%%%%%%%%%%%%%%%%%%%%%%%%%%%%%%%%%%%%%%%%%%%%%%%%
\begin{frame}[fragile]\frametitle{}
\begin{center}
{\Large Sets}
\end{center}
\end{frame}

%%%%%%%%%%%%%%%%%%%%%%%%%%%%%%%%%%%%%%%%%%%%%%%%%%%%%%%%%%
\begin{frame}{Sets}
\begin{example}
The set consisting of all positive integers less than $10$ can be denoted by $\{1,2,3,4,5,6,7,8,9\}$.  $\{1,2,3,\ldots,9\}$ is also used since the general pattern is obvious.
\end{example}
\end{frame}


%%%%%%%%%%%%%%%%%%%%%%%%%%%%%%%%%%%%%%%%%%%%%%%%%%%%%%%%%%
\begin{frame}{Sets}
\begin{example}
Members of a set need not be numeric.  The set of all possible outcomes when tossing a coin is  $\{H,T\}$.
\end{example}
\end{frame}

%%%%%%%%%%%%%%%%%%%%%%%%%%%%%%%%%%%%%%%%%%%%%%%%%%%%%%%%%%
\begin{frame}{Set Builder Notation}
For example, $O=\{1,3,5,7,9\}$ can also be written as $$O=\{x: x \text{ is a positive odd integer less than } 10\}.$$

\begin{example} 
\begin{itemize}
\item $A=\{x: x \text{ is a positive even number less than }15\}=\{2,4,6,8,10,12,14\}$.
\item $B=\{n: n \text{ is a positive integer}\}=\{1,2,3,4,5,\ldots\}$.
\item $C=\{2n: n \text{ is a whole number and }1\leq n\leq 4\}=\{2,4,6,8\}$.
\end{itemize}
\end{example}
\end{frame}

%%%%%%%%%%%%%%%%%%%%%%%%%%%%%%%%%%%%%%%%%%%%%%%%%%%%%%%%%%
\begin{frame}{Some Basic Notations}

\begin{itemize}
\item $\emptyset$, The \underline{empty set}\index{empty set} (a set with \textit{no} elements). $\{\quad\}$ is also used. 
\item $\mathbb{N}=\{1,2,3,\ldots\}$, the set of natural numbers.\index{natural number}\index{$\mathbb{N}$}
\item $\mathbb{Z}=\{\ldots,-2,-1,0,1,2,\ldots\}$, the set of integers. \index{integer}\index{$\mathbb{Z}$}
\item $\mathbb{R}$ or $(-\infty,\infty)$, the set of real numbers. \index{real number}\index{$\mathbb{R}$}
\item $[a,b]$, the set of all real numbers $x$ such that $a\leq x \leq b$.
\item $(a,b)$, the set of all real numbers $x$ such that $a< x < b$. 
\item $(a,b]$, the set of all real numbers $x$ such that $a< x \leq  b$. 
\item $[a,b)$, the set of all real numbers $x$ such that $a\leq x< b$. 
\end{itemize}

\end{frame}

%%%%%%%%%%%%%%%%%%%%%%%%%%%%%%%%%%%%%%%%%%%%%%%%%%%%%%%%%%
\begin{frame}{Subsets and Equality}
\begin{definition} A set $A$ is said to be a \underline{subset} of another set $B$  if every element of $A$ is also an element of $B$.  We use the notation $A\subseteq B$.  Two sets $A$ and $B$ are said to be \underline{equal} (notation $A=B$)  if they have the same elements.  In other words, $A$ and $B$ are equal if $A\subseteq B$ and $B\subseteq A$.
\end{definition}
\end{frame}

%%%%%%%%%%%%%%%%%%%%%%%%%%%%%%%%%%%%%%%%%%%%%%%%%%%%%%%%%%
\begin{frame}{Subsets and Equality}
\begin{example} $\{2,3,5\}\subseteq \{-1,2,3,5,7\}$.  However, $\{1,2,3,4\} \nsubseteq \{1,3,4,5,6,7\}$,  because $2\notin \{1,3,4,5,6,7\}$.
\end{example}

\begin{example} By definition, $\mathbb{N}\subseteq \mathbb{Z}\subseteq \mathbb{R}$.
\end{example}
\end{frame}

%%%%%%%%%%%%%%%%%%%%%%%%%%%%%%%%%%%%%%%%%%%%%%%%%%%%%%%%%%
\begin{frame}{Set Operations}
 Let $A$ and $B$ be sets. 
\begin{itemize}
\item The \underline{union} of $A$ and $B$, denoted by $A\cup B$, is the set that contains those elements that are either in $A$ or $B$, or in both.  In other words, $A\cup B=\{x: x\in A \text{ or } x\in B\}$. 
\item The \underline{intersection} of $A$ and $B$, denoted by $A\cap B$, is the set that contains those elements in both $A$ and $B$.  In other words, $A\cap B=\{x: x\in A \text{ and } x\in B\}$. 
\item Two sets are said to be \underline{disjoint} if their intersection is the empty set. 
\item Let $S$ be the universal set (the set of all elements under consideration. It depends on the context).  Then the \underline{complement} $A^c$ of $A$ is defined to be the set of all elements in $S$ that are not in $A$.
\end{itemize}

\end{frame}

%%%%%%%%%%%%%%%%%%%%%%%%%%%%%%%%%%%%%%%%%%%%%%%%%%%%%%%%%%
\begin{frame}{Set Operations, continued}
\begin{example} Let $A=\{1,3,4,5\}$ and $B=\{1,2,3\}$, then $A\cup B=\{1,2,3,4,5\}$ and $A\cap B=\{1,3\}$.
\end{example}

\begin{center}
\includegraphics[width=0.4\linewidth,keepaspectratio]{setun}
\includegraphics[width=0.4\linewidth,keepaspectratio]{setint}
\end{center}
\end{frame}

%%%%%%%%%%%%%%%%%%%%%%%%%%%%%%%%%%%%%%%%%%%%%%%%%%%%%%%%%%
\begin{frame}{Set Operations, continued}
\begin{example}
If $P=(-\infty, 2)$ and $Q=[-1,\infty)$, then $P\cap Q=[-1,2)$ and $P\cup Q=(-\infty,\infty)=\mathbb{R}$.
\end{example}
\end{frame}

%%%%%%%%%%%%%%%%%%%%%%%%%%%%%%%%%%%%%%%%%%%%%%%%%%%%%%%%%%
\begin{frame}{Set Operations, continued}
\begin{example} Let $O=\{2k+1: k\in \mathbb{Z}\}$ and $E=\{2k: k\in \mathbb{Z}\}$, then $O$ and $E$ are disjoint.  If one takes $\mathbb{Z}$ as the universal set, then $O^c=E$ and $E^c=O$.
\end{example}

\end{frame}

% %%%%%%%%%%%%%%%%%%%%%%%%%%%%%%%%%%%%%%%%%%%%%%%%%%%%%%%%%%
% \begin{frame}{Set algebra/Operations}
% Equality

% \begin{lstlisting}
% S1 = {1,2}
% S2 = {2,2,1,1,2}
% print ("S1 and S2 are equal because order or repetition of elements do not matter for sets\nS1==S2:", S1==S2)

% S1 and S2 are equal because order or repetition of elements do not matter for sets
% S1==S2: True

% S1 = {1,2,3,4,5,6}
% S2 = {1,2,3,4,0,6}
% print ("S1 and S2 are NOT equal because at least one element is different\nS1==S2:", S1==S2)

% S1 and S2 are NOT equal because at least one element is different
% S1==S2: False
% \end{lstlisting}

% \end{frame}

% %%%%%%%%%%%%%%%%%%%%%%%%%%%%%%%%%%%%%%%%%%%%%%%%%%%%%%%%%%
% \begin{frame}{Set algebra/Operations}
% Intersection

% In mathematics, the intersection of two sets A and B is the set that contains all elements of A that also belong to B (or equivalently, all elements of B that also belong to A), but no other elements. Formally, $A\cap B=\{x:x\in A{\text{ and }}x\in B\}$

% \begin{lstlisting}
% # Define a set using list comprehension
% S1 = set([x for x in range(1,11) if x%3==0])
% print("S1:", S1)
% >>S1: {9, 3, 6}
% S2 = set([x for x in range(1,7)])
% print("S2:", S2)
% >>S2: {1, 2, 3, 4, 5, 6}
% # Both intersection method or & can be used
% S_intersection = S1.intersection(S2)
% print("Intersection of S1 and S2:", S_intersection)

% S_intersection = S1 & S2
% print("Intersection of S1 and S2:", S_intersection)
% >>Intersection of S1 and S2: {3, 6}
% >>Intersection of S1 and S2: {3, 6}
% S3 = set([x for x in range(6,10)])
% print("S3:", S3)
% S1_S2_S3 = S1.intersection(S2).intersection(S3)
% print("Intersection of S1, S2, and S3:", S1_S2_S3)
% >>S3: {8, 9, 6, 7}
% >>Intersection of S1, S2, and S3: {6}
% \end{lstlisting}

% \end{frame}

% %%%%%%%%%%%%%%%%%%%%%%%%%%%%%%%%%%%%%%%%%%%%%%%%%%%%%%%%%%
% \begin{frame}{Set algebra/Operations}
% Union

% In set theory, the union of a collection of sets is the set of all elements in the collection. It is one of the fundamental operations through which sets can be combined and related to each other. Formally, $A\cup B=\{x:x\in A{\text{ or }}x\in B\}$

% \begin{lstlisting}
% # Both union method or | can be used
% S1 = set([x for x in range(1,11) if x%3==0])
% print("S1:", S1)
% S2 = set([x for x in range(1,5)])
% print("S2:", S2)

% S_union = S1.union(S2)
% print("Union of S1 and S2:", S_union)
% S_union = S1 | S2
% print("Union of S1 and S2:", S_union)

% >>S1: {9, 3, 6}
% >>S2: {1, 2, 3, 4}
% >>Union of S1 and S2: {1, 2, 3, 4, 6, 9}
% >>Union of S1 and S2: {1, 2, 3, 4, 6, 9}
% \end{lstlisting}

% \end{frame}


%%%%%%%%%%%%%%%%%%%%%%%%%%%%%%%%%%%%%%%%%%%%%%%%%%%%%%%%%%
\begin{frame}{Set algebra laws}
Commutative law:

$A\cap B=B\cap A$

$A\cup (B\cup C)=(A\cup B)\cup C$

Associative law:

$(A\cap B)\cap C=A\cap (B\cap C)$

$A\cap (B\cup C)=(A\cap B)\cup (A\cap C)$

Distributive law:

$A\cap (B\cup C)=(A\cap B)\cup (A\cap C)$

$A\cup (B\cap C)=(A\cup B)\cap (A\cup C)$

\end{frame}


%%%%%%%%%%%%%%%%%%%%%%%%%%%%%%%%%%%%%%%%%%%%%%%%%%%%%%%%%%
\begin{frame}{Cardinality of the Union of Sets}
Let $|S|$ denote the number of elements of a set $S$.  Obviously, $A\cup B$ contains both $A$ and $B$, so $|A\cup B|\geq |A|$ and $|A\cup B|\geq |B|$.  But exactly how many elements are there in $A\cup B$?

\begin{theorem} Let $A,B$ be sets with finitely many elements, then
$$|A\cup B|=|A|+|B|-|A\cap B|.$$  In particular, if $A$ and $B$ are disjoint, then $|A\cup B|=|A|+|B|$.
\end{theorem}

\end{frame}


%%%%%%%%%%%%%%%%%%%%%%%%%%%%%%%%%%%%%%%%%%%%%%%%%%%%%%%%%%
\begin{frame}{Cardinality of the Union of Sets}

\begin{theorem} Let $A,B,C$ be sets with finitely many elements, then
$$|A\cup B \cup C|=|A|+|B|+|C|-|A\cap B|-|B\cap C|-|A\cap C|+|A\cap B \cap C|.$$ 
\end{theorem}
\end{frame}

