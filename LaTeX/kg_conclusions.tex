%%%%%%%%%%%%%%%%%%%%%%%%%%%%%%%%%%%%%%%%%%%%%%%%%%%%%%%%%%%
\begin{frame}[fragile]\frametitle{Your Schema}

\begin{itemize}
\item  your.schema.org: is an internal version of schema.org that can be used to define shared ontologies that map to downstream BI and AI tools, providing a business-friendly understanding of the data across the organisation. 
\item Linked Data: Linked data is a way of publishing data on the web and linking it to other data sources. By publishing internal data as virtual or materialised Linked Data Sets you can ensure that your data is linked to other data sources, providing a universal view. 
\item Data Catalog: a Connected Data Catalog organises the Linked Datasets by the concepts in your.schema.org, providing a simple way to find data. 
\item By employing these standards, any organisation can construct a Shared Semantic Layer, which gives a uniform and consistent understanding of data. 
\end{itemize}
	  
{\tiny (Ref: LinkedIn posts by Tony Seale)}
	  
\end{frame}

%%%%%%%%%%%%%%%%%%%%%%%%%%%%%%%%%%%%%%%%%%%%%%%%%%%%%%%%%%%
\begin{frame}[fragile]\frametitle{Conclusion}

\begin{itemize}
\item How to leverage transformers based NER and spacy’s relation extraction models to create knowledge graph with Neo4j.
\item In addition to information extraction, the graph topology can be used as an input to another machine learning model. 
\end{itemize}
	  
\end{frame}

%%%%%%%%%%%%%%%%%%%%%%%%%%%%%%%%%%%%%%%%%%%%%%%%%%%%%%%%%%%
\begin{frame}[fragile]\frametitle{References}

\begin{itemize}
\item DAT278x - From Graph and Knowledge Graph - EdX course
\item A Universe of Knowledge Graphs -  Dr. Maya Natarajan, Dr. Jesus Barrasa
\item How to Build a Knowledge Graph with Neo4J and Transformers - Walid Amamou
\end{itemize}
	  
\end{frame}
