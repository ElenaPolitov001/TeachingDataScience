%%%%%%%%%%%%%%%%%%%%%%%%%%%%%%%%%%%%%%%%%%%%%%%%%%%%%%%%%%%
\begin{frame}
\begin{center}
{\Large Hypothesis Testing}
\end{center}
\end{frame}

%%%%%%%%%%%%%%%%%%%%%%%%%%%%%%%%%%%%%%%%%%%%%%%%%%%%%%%%%%%
\begin{frame}[fragile]\frametitle{Statistical Hypothesis Testing}
\begin{itemize}
\item t-test = compares 1/2 numerical groups
\item ANOVA = compares $> 2$ numerical groups
\item Chi-square test  = compares categorical variables
\end{itemize}
\end{frame}

%%%%%%%%%%%%%%%%%%%%%%%%%%%%%%%%%%%%%%%%%%%%%%%%%%%%%%%%%%%
\begin{frame}
\begin{center}
{\Large t-Test}
\end{center}
\end{frame}

%%%%%%%%%%%%%%%%%%%%%%%%%%%%%%%%%%%%%%%%%%%%%%%%%%%%%%%%%%%%%%%%%%%%%%%%
\begin{frame}[fragile]\frametitle{t-Test}

	\begin{itemize}
	
	\item There are two types of t-tests: paired and unpaired.
	\item Paired t-tests are used when you have `before' and `after' measurements from the same subjects. E.g body temperature before and after medicine.
	\item Unpaired: you are comparing height data for two groups. 
	\item Subcategories of Unpaired t-tests:
	\begin{itemize}
	\item Assumes that variations with both groups are same.
	\item Does NOT Assume that variations with both groups are same.
	\end{itemize}
	\item Another classification: 1-tail or 2-tails t-tests
	\item In case of comparing heights of two groups, the 2-tail t-test will evaluate of group A is Higher than B as well as Shorter than B, ie at both ends.
	\item It is used when you already know which direction has to be compared, higher or lower.
	\end{itemize}
  
 
\tiny{(Ref: StatQuest: Which t test to use? - Josh Starmer )}
\end{frame}

%%%%%%%%%%%%%%%%%%%%%%%%%%%%%%%%%%%%%%%%%%%%%%%%%%%%%%%%%%%%%%%%%%%%%%%%
\begin{frame}[fragile]\frametitle{1 vs 2 tailed t-Tests}
Example: a clinical trial is done on 6 patients to find effect of a new drug.


	\begin{itemize}
	\item Data shows that with new drug, results are much better (more red dots on left)
	\item The red dot in the middle is a bit problematic.
	\item After doing stats, you get p-value of 0.03 for 1 tail t test. 
	\item 0.03 is smaller than threshold of 0.05 (CI). Meaning this new drug is significantly different. Its mean result is far away from other means found during existing treatments.
	\item 2-tailed gives p-value of 0.06. Not good.
	\item Which p-value to use?
	\end{itemize}

      \begin{center}
      \includegraphics[width=\linewidth,keepaspectratio]{statq41}
	  	\end{center}

\tiny{(Ref: StatQuest: One or Two Tailed P-Values - Josh Starmer )}
\end{frame}

%%%%%%%%%%%%%%%%%%%%%%%%%%%%%%%%%%%%%%%%%%%%%%%%%%%%%%%%%%%%%%%%%%%%%%%%
\begin{frame}[fragile]\frametitle{1 vs 2 tailed t-Tests}
Example: a clinical trial is done on 6 patients to find effect of a new drug.


	\begin{itemize}
	\item 1 tailed p value tests the hypothesis that your new drug is better than the existing drug. Whereas,
	\item 2 tailed p-value tests whether the new drug is better , worse or not significantly different.
	\item 1 tailed p value is smaller because it does not distinguish between ``worse'' or ``not significantly different''.
	\item AS we wish to know if the new drug was worse than the existing treatment, we should use 2 tailed p value test.
	\end{itemize}

      \begin{center}
      \includegraphics[width=\linewidth,keepaspectratio]{statq41}
	  	\end{center}

\tiny{(Ref: StatQuest: One or Two Tailed P-Values - Josh Starmer )}
\end{frame}


%%%%%%%%%%%%%%%%%%%%%%%%%%%%%%%%%%%%%%%%%%%%%%%%%%%%%%%%%%%
\begin{frame}[fragile]\frametitle{t-tests}
\begin{itemize}
\item One sample hypothesis testing
\begin{itemize}
\item We are given population $\sigma$: use z distribution, z-test
\item We are Not given population $\sigma$, need to estimate it: t-distribution, t-test
\end{itemize}
\item Two sample t-test 
%\item Paired t-test 
\end{itemize}
\end{frame}

%%%%%%%%%%%%%%%%%%%%%%%%%%%%%%%%%%%%%%%%%%%%%%%%%%%
\begin{frame}
\frametitle{Z distribution}
Same as Sampling distribution
\begin{center}
\includegraphics[width=\linewidth,keepaspectratio]{zdist}
\end{center}
\end{frame}

%%%%%%%%%%%%%%%%%%%%%%%%%%%%%%%%%%%%%%%%%%%%%%%%%%%
\begin{frame}
\frametitle{t distribution}
Every t distribution is unique and is specific to sample size. For n=20, look at t table with degrees of freedom 20-1=19
\begin{center}
\includegraphics[width=0.8\linewidth,keepaspectratio]{tdist}
\end{center}
Range is a bit more than Z dist. Like pushing it down from top!!
\end{frame}

%%%%%%%%%%%%%%%%%%%%%%%%%%%%%%%%%%%%%%%%%%%%%%%%%%%%%%%%%%%
\begin{frame}[fragile]\frametitle{One Sample t-test}
\begin{itemize}
\item Objective: Test if hypothesized population mean matches with the actual population mean.
\item $\mu_0$ is hypothesized population mean
\item $\mu$ is the true/actual population mean
\item Method: Test using sample means and confidence interval
\end{itemize}
\end{frame}

%%%%%%%%%%%%%%%%%%%%%%%%%%%%%%%%%%%%%%%%%%%%%%%%%%%%%%%%%%%
\begin{frame}[fragile]\frametitle{One Sample t-test}
\begin{itemize}
\item When?: To test whether a given sample is coming from a population whose mean is specified value
$$
t = \frac{\bar{x} - \mu_0}{s /\sqrt{n}}
$$
\item $\bar{x}$: sample mean of n observations
\item $s$: sample std deviation of n observations
\item $\mu_0$: hypothesized population mean
\item $H_0: \bar{x} = \mu_0$
\item $H_1: \bar{x} \neq \mu_0$
\end{itemize}
\end{frame}

%%%%%%%%%%%%%%%%%%%%%%%%%%%%%%%%%%%%%%%%%%%%%%%%%%%%%%%%%%%
\begin{frame}[fragile]\frametitle{One Sample t-test}
\begin{itemize}
\item t value computed is similar to z score.
\item Shows how many std deviations you are away from mean.
\item With the given t score, in the t table find the location on graph
\item Looking at cut off boundaries, are you landing in the non rejection region or the rejection region?
\end{itemize}
\end{frame}

%%%%%%%%%%%%%%%%%%%%%%%%%%%%%%%%%%%%%%%%%%%%%%%%%%%%%%%%%%%
\begin{frame}[fragile]\frametitle{Example: One Sample t-test}
\begin{itemize}
\item New tyres launched are claim to have life of 40000 km.
\item A retailer wants to test this claim.
\item He has taken random sample of 8 tyres.
\item He is testing life of the tyres under normal condition.
\end{itemize}
\end{frame}

%%%%%%%%%%%%%%%%%%%%%%%%%%%%%%%%%%%%%%%%%%%%%%%%%%%%%%%%%%%
\begin{frame}[fragile]\frametitle{Example: One Sample t-test}
\begin{itemize}
\item We don't know population std dev. So t test.
\item $H_0; \mu_0 = 40000$ 
\item $H_1; \mu_0 \ne 40000$ 
\item Lets take $\alpha = 0.05$
\item $df = n -1 = 8 - 1 = 7$
\end{itemize}
\end{frame}

%%%%%%%%%%%%%%%%%%%%%%%%%%%%%%%%%%%%%%%%%%%%%%%%%%%%%%%%%%%
\begin{frame}[fragile]\frametitle{Example: One Sample t-test}
\begin{center}
\includegraphics[width=0.5\linewidth,keepaspectratio]{ttble}
\end{center}
\begin{itemize}
\item Critical boundary-value = 2.3646
\item If t false below -2.3646 or above 2.3646, then reject $H_0$.
\end{itemize}
\end{frame}



%%%%%%%%%%%%%%%%%%%%%%%%%%%%%%%%%%%%%%%%%%%%%%%%%%%%%%%%%%%
\begin{frame}[fragile]\frametitle{Example: One Sample t-test}

\begin{itemize}
\item  His findings below:
 
\begin{tabular}{l|l}
Sr	&Km\\ \hline
1 & 35000\\
2 & 38000\\
3 & 42000\\
4 & 41000\\
5 & 39000\\
6 & 41500\\
7 & 43000\\
8 & 38500\\
\end{tabular}
\item Given goal mean $\mu_0 = 40000$
\item $n = 8$
\item Calculate sample mean $\bar{x}$, it comes to $39750$
\item Calculate $s$, it comes to $2618.615$
\item $df = n -1 = 7$
\end{itemize}
\end{frame}

%%%%%%%%%%%%%%%%%%%%%%%%%%%%%%%%%%%%%%%%%%%%%%%%%%%%%%%%%%%
\begin{frame}[fragile]\frametitle{Example: One Sample t-test}

\begin{itemize}
\item $t = \frac{\bar{x} - \mu_0}{s /\sqrt{n}} = \frac{39750 - 40000}{2618.615 /\sqrt{8}} = \frac{-250}{925.82} = - 0.27$
\item It is right of negative boundary and left of positive boundary, 
\item So, in valid region.
\item So, cannot reject $H_0$. 
\item So, no difference. 
\item Sample mean is same as given mean.
\end{itemize}
\end{frame}


%%%%%%%%%%%%%%%%%%%%%%%%%%%%%%%%%%%%%%%%%%%%%%%%%%%%%%%%%%%
\begin{frame}[fragile]\frametitle{Two Samples t-test}
\begin{itemize}
\item When?: To test whether given two samples are coming from a population
$$
t = \frac{\bar{x_1} - \bar{x_2}}{s_p /\sqrt{\frac{1}{n_1} + \frac{1}{n_2}}}
$$
\item $\bar{x_1}$: sample mean of $n_1$ observations
\item $\bar{x_2}$: sample mean of $n_2$ observations
\item $s_p$: Pooled std deviation of two samples
\item $\mu_0$: specified population mean
\item $H_0: \bar{x_1} = \bar{x_2}$
\item $H_1: \bar{x_1} \neq \bar{x_2}$
\end{itemize}
\end{frame}

%%%%%%%%%%%%%%%%%%%%%%%%%%%%%%%%%%%%%%%%%%%%%%%%%%%%%%%%%%%
\begin{frame}[fragile]\frametitle{Example: Two Sample t-test}
\begin{itemize}
\item Is there any appreciable difference of IQs of Males and Females?
\item $H_0: \bar{x_{male}} = \bar{x_{female}}$
\item $H_1: \bar{x_{male}} \neq \bar{x_{female}}$
\item Lets take $\alpha = 0.05$
\item $n_m = n_f = 18; n= 36$
\item $ \bar{x_{male}} = 98.9$
\item $ \bar{x_{female}} = 102.4 $
\item $s = 12$; same for both
\end{itemize}
\end{frame}

%%%%%%%%%%%%%%%%%%%%%%%%%%%%%%%%%%%%%%%%%%%%%%%%%%%%%%%%%%%
\begin{frame}[fragile]\frametitle{Example: Two Sample t-test}
\begin{itemize}
\item Finding critical values value for t at $1 - \alpha/2$ for two tail test and  $df = n -1 = 35$ = 2.030
\item $t = \frac{\bar{x_{female}} - \bar{x_{male}}}{s_p /\sqrt{\frac{1}{n_f} + \frac{1}{n_m}}}$
\item $ = \frac{102.4- 98.9}{12/ \sqrt{1/18 + 1/18}} = 0.097$
\item Lies in acceptable region.
\item Can not reject $H_0$
\item No diff in IQs.
\end{itemize}
\end{frame}

%%%%%%%%%%%%%%%%%%%%%%%%%%%%%%%%%%%%%%%%%%%%%%%%%%%
\begin{frame}[fragile]\frametitle{Two Samples t-test}
What is t-test?
\begin{itemize}
\item Compares two averages (means) and tells you if they are different from each other. 
\item Tells you how significant the differences are
\item It lets you know if those differences could have happened by chance.
\end{itemize}
\end{frame}

%%%%%%%%%%%%%%%%%%%%%%%%%%%%%%%%%%%%%%%%%%%%%%%%%%%
\begin{frame}[fragile]\frametitle{Two Samples t-test}
What is t-test?
\begin{itemize}
\item The t score is a ratio between the difference between two groups and the difference within the groups.
\item A large t-score tells you that the groups are different.
\item A small t-score tells you that the groups are similar.
\item  The bigger the t-value, the more likely it is that the results are repeatable.
\end{itemize}
\end{frame}

%%%%%%%%%%%%%%%%%%%%%%%%%%%%%%%%%%%%%%%%%%%%%%%%%%%
\begin{frame}[fragile]\frametitle{Two Samples t-test}
\begin{itemize}
\item Every t-value has a p-value to go with it. A p-value is the probability that the results from your sample data occurred by chance. 
\item a p-value of .01 means there is only a 1\% probability that the results from an experiment happened by chance.
\end{itemize}
\end{frame}

%%%%%%%%%%%%%%%%%%%%%%%%%%%%%%%%%%%%%%%%%%%%%%%%%%%
\begin{frame}[fragile]\frametitle{Two Samples t-test}
Types of t-tests?
\begin{itemize}
\item  An Independent Samples t-test compares the means for two groups.
\item  A Paired sample t-test compares means from the same group at different times (say, one year apart).
\item  A One sample t-test tests the mean of a single group against a known mean.
\end{itemize}
\end{frame}

%%%%%%%%%%%%%%%%%%%%%%%%%%%%%%%%%%%%%%%%%%%%%%%%%%%
\begin{frame}[fragile]\frametitle{Two Samples t-test}
Example: to test whether the height of men in the population is different from height of women in general.

Steps:
\begin{itemize}
\item  Determine a null and alternate hypothesis.
\begin{itemize}
\item  Null-Hypothesis: no statistically significant difference, height of men and women are the same
\item  Alternate-Hypothesis: statistically significant difference, height of men and women are different
\end{itemize}
\item Collect sample data. Twos sets: men and women. Generally sizes should be same, but can be diferent, $n_x$ and $n_y$
\end{itemize}
\end{frame} 
%%%%%%%%%%%%%%%%%%%%%%%%%%%%%%%%%%%%%%%%%%%%%%%%%%%
\begin{frame}[fragile]\frametitle{Two Samples t-test}
Steps (cont.):
\begin{itemize}
\item  Determine a confidence interval and degrees of freedom
\item $\alpha$ of 0.005 is 95\% confidence that the conclusion of this test will be valid
\item The degree of freedom $df = n_x + n_y - 2$
\item Calculate the t-statistic

\begin{center}
\includegraphics[width=0.8\linewidth,keepaspectratio]{ttestform}
\end{center}
\end{itemize}
\end{frame} 


%%%%%%%%%%%%%%%%%%%%%%%%%%%%%%%%%%%%%%%%%%%%%%%%%%%
\begin{frame}[fragile]\frametitle{Two Samples t-test}
Steps (cont.):
\begin{itemize}
\item  Calculate the critical t-value from the t distribution
\begin{center}
\includegraphics[width=0.6\linewidth,keepaspectratio]{ttesttab}
\end{center}
\item or call ready Python function
\end{itemize}
\end{frame} 


%%%%%%%%%%%%%%%%%%%%%%%%%%%%%%%%%%%%%%%%%%%%%%%%%%%
\begin{frame}[fragile]\frametitle{Two Samples t-test}
Generate data
\begin{lstlisting}
#Sample Size
N = 10
#Gaussian distributed data with mean = 2 and var = 1
a = np.random.randn(N) + 2
#Gaussian distributed data with with mean = 0 and var = 1
b = np.random.randn(N)
\end{lstlisting}
\end{frame}


%%%%%%%%%%%%%%%%%%%%%%%%%%%%%%%%%%%%%%%%%%%%%%%%%%%
\begin{frame}[fragile]\frametitle{Two Samples t-test}
\begin{lstlisting}
#Calculate the variance to get the standard deviation
#For unbiased max likelihood estimate we have to divide the var by N-1, and therefore the parameter ddof = 1

var_a = a.var(ddof=1)
var_b = b.var(ddof=1)

#std deviation
s = np.sqrt((var_a + var_b)/2)
\end{lstlisting}
\end{frame}


%%%%%%%%%%%%%%%%%%%%%%%%%%%%%%%%%%%%%%%%%%%%%%%%%%%
\begin{frame}[fragile]\frametitle{Two Samples t-test}
\begin{lstlisting}
## Calculate the t-statistics
t = (a.mean() - b.mean())/(s*np.sqrt(2/N))
## Compare with the critical t-value
#Degrees of freedom
df = 2*N - 2
#p-value after comparison with the t 
p = 1 - stats.t.cdf(t,df=df)
#Note that we multiply the p value by 2 because its a twp tail t-test

## Cross Checking with the internal scipy function
t2, p2 = stats.ttest_ind(a,b)
\end{lstlisting}
\end{frame} 


%%%%%%%%%%%%%%%%%%%%%%%%%%%%%%%%%%%%%%%%%%%%%%%%%%%%%%%%%%%
\begin{frame}[fragile]\frametitle{Paired t-test}
\begin{itemize}
\item When?: To check, effectiveness of a new treatment, a new method 
employed 
\item  observations on  every  unit  are  made  before  and  after  applying  the  treatment  or 
method.  
\item Hence  if  treatment  or  method  is  effective,  there  will  be 
significant  difference  in  observations  before  applying  it  and  after 
applying it. 
$$
t = \frac{\bar{x_D} - \mu_0}{\sigma_D /\sqrt{n}}
$$
\item $\bar{x_D}$: mean difference of n paired obsrevations
\item $\sigma_D$:  std deviation of difference of n paired obsrevations obsrevations
\item $\mu_0$: mean difference of n paired observations under $H_0$
\item $H_0: $ Mean of samples before and after treatment is same. 
\item $H_1:$ Mean of samples before and after treatment is not same.
\end{itemize}
\end{frame}


%%%%%%%%%%%%%%%%%%%%%%%%%%%%%%%%%%%%%%%%%%%%%%%%%%%%%%%%%%%
\begin{frame}[fragile]\frametitle{Chi-Square test}
\begin{itemize}
\item When?: To test whether any 
two  categorical variables  are  associated  with  each  other  or  they  are 
independent  of  each  other,
$$
x^2= \sum \frac{(O_i - E_i)^2}{E_i}
$$
\item $O_i:$ Observed frequency of ith variable
\item $E_i:$ Expected frequency of ith variable
\item $H_0: $ Two variable are independent of each other. 
\item $H_1:$ Two variable are not independent of each other. 
\end{itemize}
\end{frame}

%%%%%%%%%%%%%%%%%%%%%%%%%%%%%%%%%%%%%%%%%%%%%%%%%%%%%%%%%%%
\begin{frame}[fragile]\frametitle{F test}
\begin{itemize}
\item When?: to  know 
whether  these  two  sample  have  same  population  variance
$$
F = \frac{s_1^2}{s_2^2}
$$
\item $S_1^2$ Sample variance of first sample of size $n_1$
\item $S_2^2$ Sample variance of second sample of size $n_2$
\item $H_0: $ Variance ratio is one
\item $H_1:$ Variance ratio not equal to one
\end{itemize}
\end{frame}
