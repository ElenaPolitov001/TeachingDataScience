\newcommand{\R}{\mathbb{R}}
\newcommand{\norm}[1]{\lVert#1\rVert} % norm
\newcommand{\avg}[1]{\left< #1 \right>} % average
\newcommand{\abs}[1]{\bigl| #1 \bigr|} % absolute values

%%%%%%%%%%%%%%%%%%%%%%%%%%%%%%%%%%%%%%%%%%%%%%%%%%%%%%%%%%%%%%%%%%%%%%%%%%%%%%%%%%
\begin{frame}[fragile]\frametitle{}
\begin{center}
{\Large Vectors}
\end{center}
\end{frame}


%%%%%%%%%%%%%%%%%%%%%%%%%%%%%%%%%%%%%%%%%%%%%%%%%%%%%%%%%%%
 \begin{frame}[fragile] \frametitle{Vectors}

\begin{itemize}

\item At its simplest, a vector is an entity that has both magnitude and direction. 
\item The magnitude represents a distance (for example, ``2 miles'') and the direction indicates which way the vector is headed (for example,``East''). 
\item One more way is $\bar{v} = 2\hat{i} + 3\hat{j}$; meaning?
\item Is Magnitude-Direction form equivalent to i-j form?
\item Inter-convertible? How?
\item Can it have just two components?
\end{itemize}

\end{frame}




%%%%%%%%%%%%%%%%%%%%%%%%%%%%%%%%%%%%%%%%%%%%%%%%%%%%%%%%%%%
 \begin{frame}[fragile] \frametitle{Vectors}
Two-dimensional example:
\begin{itemize}

\item A vector that is defined by a point in a two-dimensional plane
\item A two dimensional coordinate consists of an x and a y value, and in this case we'll use 2 for x and 1 for y
\item Its is written in matrix form as : $\vec{v} = \begin{bmatrix}2 \\ 1 \end{bmatrix}$
\item Describes the movements required to get to the end point (of head) of the vector 
\item So, it is not a point in space. It gives Direction, like a movement recipe. 
\item When added to a point, results into a transformed point.
\item In this case, we need to move 2 units in the x dimension, and 1 unit in the y dimension
\end{itemize}

\end{frame}

%%%%%%%%%%%%%%%%%%%%%%%%%%%%%%%%%%%%%%%%%%%%%%%%%%%%%%%%%%%
 \begin{frame}[fragile] \frametitle{Vectors}
Two-dimensional example:
\begin{itemize}

\item Note that we don't specify a starting point for the vector
\item We're simply describing a destination coordinate that encapsulate the magnitude and direction of the vector. 
\item Think about it as the directions you need to follow to get to {\bf there} from {\bf here}, without specifying where {\bf here} actually is!
\item Generally using the point 0,0 as the starting point (or origin). Also called as Position Vector.
\item Our vector of (2,1) is shown as an arrow that starts at 0,0 and moves 2 units along the x axis (to the right) and 1 unit along the y axis (up).
\end{itemize}

\end{frame}

%%%%%%%%%%%%%%%%%%%%%%%%%%%%%%%%%%%%%%%%%%%%%%%%%%%%%%%%%%%
 \begin{frame}[fragile] \frametitle{Vectors}
Calculating Magnitude
\begin{itemize}

\item $\|\vec{v}\| = \sqrt{v_{1}\;^{2} + v_{2}\;^{2}}$
\item Double-bars are often used to avoid confusion with absolute values. 
\item Note that the components of the vector are indicated by subscript indices $(v_1, v_2,\ldots v_n)$
\item In this case, the vector v has two components with values 2 and 1, so our magnitude calculation is:
\item $\|\vec{v}\| = \sqrt{2^{2} + 1^{2}} = \sqrt{5} \approx 2.24$
\end{itemize}

\end{frame}


%%%%%%%%%%%%%%%%%%%%%%%%%%%%%%%%%%%%%%%%%%%%%%%%%%%%%%%%%%%
 \begin{frame}[fragile] \frametitle{Vectors}
Calculating Direction
\begin{itemize}

\item We can get the angle of the vector by calculating the inverse tangent; sometimes known as the arctan
\item For our v vector (2,1):$tan(\theta) = \frac{1}{2}$
\item $\theta = tan^{-1} (0.5) \approx 26.57^{o}$
\item use the following rules:
\begin{itemize}

\item Both x and y are positive: Use the tan-1 value.
\item x is negative, y is positive: Add 180 to the tan-1 value.
\item Both x and y are negative: Add 180 to the tan-1 value.
\item x is positive, y is negative: Add 360 to the tan-1 value.
\end{itemize}
\end{itemize}


\end{frame}




%%%%%%%%%%%%%%%%%%%%%%%%%%%%%%%%%%%%%%%%%%%%%%%%%%%%%%%%%%%
 \begin{frame}[fragile] \frametitle{Vectors}

\begin{itemize}

\item Vectors are defined by an n-dimensional coordinate that describe a point in space that can be connected by a line from an arbitrary origin.
\item Are n-dimensional Points and Vectors equivalent? How?
\item $\|\vec{v}\| = \sqrt{v_{1}\;^{2} + v_{2}\;^{2} ... + v_{n}\;^{2}}$
\end{itemize}

\end{frame}



%%%%%%%%%%%%%%%%%%%%%%%%%%%%%%%%%%%%%%%%%%%%%%%%%%%%%%%%%%%
  \begin{frame}[fragile]
\textbf{Definition}
  A {\em vector} is a matrix with one column.


 \textbf{Example}
  \[
  \left[\begin{array}{r}
   1 \\2 \\-5\\ 9  
  \end{array}\right]
 \]
  
  
\textbf{Note}
Two vectors are equal precisely when they have the same number
of rows and all their corresponding entries are  equal.

\end{frame}

%%%%%%%%%%%%%%%%%%%%%%%%%%%%%%%%%%%%%%%%%%%%%%%%%%%%%%%%%%%%%%%%%%%%%%%%
\begin{frame}[fragile]\frametitle{Vectors (Recap)}
\begin{itemize}
\item A vector has magnitude (how long it is) and direction
\item A point can be a vector (position vector, from Origin)
\item A data row is a point in n-dimensions, thus a vector as well.
\end{itemize}
\begin{center}
\includegraphics[width=0.5\linewidth]{vec}
\end{center}
\end{frame}

%%%%%%%%%%%%%%%%%%%%%%%%%%%%%%%%%%%%%%%%%%%%%%%%%%%%%%%%%%%
 \begin{frame}[fragile] \frametitle{Vector Addition}

\begin{center}
\includegraphics[width=0.5\linewidth,keepaspectratio]{vec1}
\end{center}

\begin{itemize}
\item To add these vectors: We just add the individual components, so 3 plus 2
is 5; and 1 plus -4 is -3.
\item It is simply adding another leg to the journey; so if
we follow vector V along 3 and up 1, and then follow vector W along 2 and down 4,
we end up at the head of the vector we calculated by adding V and
W together.
\end{itemize}

\end{frame}


%%%%%%%%%%%%%%%%%%%%%%%%%%%%%%%%%%%%%%%%%%%%%%%%%%%%%%%%%%%
  \begin{frame}[fragile]{Vector Addition}
 \textbf{Definition}
    We define the sum and  of two vectors by 
     \[
      \left[\begin{array}{r}
       u_1 \\ u_2 \\ \vdots \\ u_n
      \end{array}\right]
    +
    \left[\begin{array}{r}
       v_1 \\ v_2 \\ \vdots \\ v_n
      \end{array}\right]
     = 
     \left[\begin{array}{r}
       u_1 + v_1  \\ u_2 + v_2 \\ \vdots \\ u_n + v_n
      \end{array}\right]
    \]     
    and the product of a scalar and a vector by
    \[
     \alpha \left[\begin{array}{r}
       u_1 \\ u_2 \\ \vdots \\ u_n
      \end{array}\right]
    =  
    \left[\begin{array}{r}
       \alpha u_1 \\ \alpha u_2 \\ \vdots \\ \alpha u_n
      \end{array}\right]
    \]
 
\end{frame}

%%%%%%%%%%%%%%%%%%%%%%%%%%%%%%%%%%%%%%%%%%%%%%%%%%%%%%%%%%%
\begin{frame}[fragile]\frametitle{Example}

  \[
  \left[\begin{array}{r}
   1 \\ 3 \\ -5
  \end{array}\right]
+
\left[\begin{array}{r}
   2 \\ 2 \\ 7
  \end{array}\right]
 =   
 \left[\begin{array}{r}
    3 \\ 5 \\ 2 
  \end{array}\right]   
  \qquad 
  \mbox{ and } \qquad  
   3 \left[\begin{array}{r}
   5 \\ 2 \\ 1
  \end{array}\right]
=  
\left[\begin{array}{r} 15 \\ 6 \\ 3  \end{array}\right]
\]
  
\end{frame}


%%%%%%%%%%%%%%%%%%%%%%%%%%%%%%%%%%%%%%%%%%%%%%%%%%%%%%%%%%%
\begin{frame}[fragile]\frametitle{Exercise}

Let $\vec{u}$ and $\vec{v}$ be given by 
\[
 \vec{u} = \left[\begin{array}{r} 1 \\ 1 \end{array}\right]
 \qquad \mbox{and} \qquad
 \vec{v} = \left[\begin{array}{r} 1 \\ -1 \end{array}\right]
\]
Plot  $\vec{u}$, $\vec{v}$, $\vec{2u}$ and 
$\vec{u}+\vec{v}$.


\textbf{Parallelogram rule for vector addition}
Suppose $\vec{u}$ and $\vec{v}\in \mathbb R^2$.  Then 
$\vec{u}+\vec{v}$ corresponds to the fourth vertex
of the parallelogram whose opposite vertex is $\vec{0}$
and whose other two vertices are $\vec{u}$ and $\vec{v}$.

\end{frame}

%%%%%%%%%%%%%%%%%%%%%%%%%%%%%%%%%%%%%%%%%%%%%%%%%%%%%%%%%%%
\begin{frame}[fragile]\frametitle{Exercise}

     Let $\vec{u} = \left[\begin{array}{r} 6 \\3 \end{array}\right]$
    and $\vec{v} = \left[\begin{array}{r} 5 \\2 \end{array}\right]$.  Display 
    $\vec{u}$, $-2/3\vec{u}$, $\vec{v}$ and $-2/3\vec{u} + \vec{v}$
    on a graph.
  
\end{frame}

%%%%%%%%%%%%%%%%%%%%%%%%%%%%%%%%%%%%%%%%%%%%%%%%%%%%%%%%%%%
  \begin{frame}[fragile]\frametitle{ $\mathbb R^n$}
In general we will consider vectors in $\mathbb R^n$, that is, having $n$ real entries.
$
 \vec{u} = \left[\begin{array}{r} u_1 \\ u_2\\ \vdots \\ u_n \end{array}\right] \in \mathbb R^n
$

The zero vector is $\vec{0} = \left[\begin{array}{r} 0 \\ 0 \\ \vdots \\ 0 \end{array}\right]$
having $n$ entries, each equal to $0$.
\end{frame}

%%%%%%%%%%%%%%%%%%%%%%%%%%%%%%%%%%%%%%%%%%%%%%%%%%%%%%%%%%%
  \begin{frame}[fragile]\frametitle{Properties of $R^n$}
\textbf{Theorem}  Suppose that $\vec{u}, \vec{v},\vec{w}\in \mathbb R^n$ and $c,d\in \mathbb R$.  Then,
\begin{itemize}
 \item $\vec{u} + \vec{v} = \vec{v} + \vec{u}$.
 \item $(\vec{u} + \vec{v} ) + \vec{w} = 
        \vec{u} + ( \vec{v} + \vec{w})$
 \item $\vec{u} + \vec{0} = \vec{0} +\vec{u} = \vec{u}$
 \item $\vec{u} + -\vec{u} = -\vec{u}+ \vec{u} = \vec{0}$
 $\qquad \qquad$ ($\ -\vec{u} = (-1)\vec{u}\ $)
 \item $c(\vec{u} + \vec{v} ) = c\vec{u} + c\vec{v}$
 \item $(c+d)\vec{u} = c\vec{u} + d\vec{u}$
 \item $c(d\vec{u}) = (cd)\vec{u}$
 \item $1 \cdot \vec{u} =\vec{u}$
\end{itemize}


\end{frame}
