\newcommand{\ignore}[1]{}
\newcommand{\arcosh}{\qopname \relax o{arcosh}}
\newcommand{\divop}{\qopname \relax o{div}}
\newcommand{\dist}{\qopname \relax o{dist}}
\newcommand{\D}{\qopname \relax o{D}\!} 	% Jacobi-Matrix
\newcommand{\dop}{\qopname \relax o{d}\!} 	% Differentialoperator
\newcommand{\ds}{\dop s}
\newcommand{\dt}{\dop t}
\newcommand{\dx}{\dop x}
\newcommand{\dy}{\dop y}
\newcommand{\Epi}{\qopname \relax o{Epi}}
\newcommand{\hot}{\qopname \relax o{hot}}
\newcommand{\id}{\qopname \relax o{id}}
\newcommand{\im}{\qopname \relax o{im}}
\newcommand{\osc}{\qopname \relax o{osc}}
%\newcommand{\rg}{\qopname \relax o{rg}}
\newcommand{\rot}{\qopname \relax o{rot}}
%\newcommand{\Satz}{Proposition }
\newcommand{\SatzEnde}{r}
\newcommand{\sgn}{\qopname \relax o{sgn}}
\newcommand{\tr}{\qopname \relax o{tr}}
\hyphenation{Hil-bert-raum}
%adrian
\newcommand{\Norm}[1]{\left\| #1 \right\|}
\newcommand{\SKP}[1]{\left\langle #1 \right\rangle}

%\newcommand\mat[2][c]{\left(\hspace{-2mm}\begin{array}{#1}#2\end{array}\hspace{-2mm}\right)}

% praktische Extras:
% Liste mit a), b) etc. vor den Items (alphanumerisch)
% zuerst der Anfang
\newcommand{\bal}[1]{\newcounter{#1}\stepcounter{#1}\begin{list}{\alph{#1})}{}}
  % jetzt die einzelnen items
  \newcommand{\aitem}[1]{\item\stepcounter{#1}}
  % zuletzt das Ende
  \newcommand{\eal}{\end{list}}
% Liste mit i), ii) etc. vor den Items (roman)
% Anfang
\newcommand{\brl}[1]{\newcounter{#1}\stepcounter{#1}\begin{list}{\roman{#1})}{}}
  % item
  \newcommand{\ritem}[1]{\item\stepcounter{#1}}
  % Ende
  \newcommand{\erl}{\end{list}}
% Matrix-Anfang und -Ende
% Anfang
\newcommand{\matb}[1]{\left(\begin{array}{#1}}
    % Ende
    \newcommand{\mate}{\end{array}\right)}
% Vektor-Anfang und -Ende
% Anfang
\newcommand{\vekb}{\left(\begin{array}{c}}
    % Ende
    \newcommand{\veke}{\end{array}\right)}
% Die Standardk�rper Q,R,C,F,N,Z
% Q
\newcommand{\Ku}{\mathbb{Q}}
% R
\newcommand{\Er}{\mathbb{R}}
% C
\newcommand{\Ze}{\mathbb{C}}
% F
\newcommand{\Ef}{\mathbb{F}}
% N
\newcommand{\En}{\mathbb{N}}
% Z
\newcommand{\Zett}{\mathbb{Z}}
% eqnarray
% Anfang
\newcommand{\bea}{\begin{eqnarray*}}
% Ende
\newcommand{\eea}{\end{eqnarray*}}
% Theoreme, S�tze, Beispiele etc.
\theoremstyle{plain}
%\newtheorem{theorem}{Theorem}[section]
%\newtheorem{korollar}[theorem]{Korollar}
%\newtheorem{lemma}[theorem]{Lemma}
\newtheorem{proposition}[theorem]{\Satz}
%\newtheorem{conjecture}[theorem]{Vermutung}
%\newtheorem{satz}[theorem]{\Satz}
%\theoremstyle{definition}
%\newtheorem{definition}[theorem]{Definition}
%%\theoremstyle{remark}
%\newtheorem{notation}[theorem]{Notation}
%\newtheorem{remark}[theorem]{Remark}
%\newtheorem{corollary}[theorem]{Corollary}
%\newtheorem{application}[theorem]{Application}
%\newtheorem{example}[theorem]{Example}
%\newtheorem*{remark*}{Bemerkung}
%\newtheorem*{example*}{Example}
%\newtheorem*{notation*}{Bezeichnung}
%\newtheorem*{application*}{Anwendung}

%\newenvironment{proof}[1][Beweis]{\begin{trivlist}
%  \item[\hskip \labelsep {\bfseries #1}]}{\end{trivlist}}
% \newenvironment{definition}[1][Definition]{\begin{trivlist}
% \item[\hskip \labelsep {\bfseries #1}]}{\end{trivlist}}
%\newenvironment{example}[1][Example]{\begin{trivlist}
%  \item[\hskip \labelsep {\bfseries #1}]}{\end{trivlist}}
%\newenvironment{remark}[1][Remark]{\begin{trivlist}
%  \item[\hskip \labelsep {\bfseries #1}]}{\end{trivlist}}

% Beginn Symboldef Mitterlwertintegral-----------------------------------
\def \Xint #1{\mathchoice						
  {\XXint \displaystyle \textstyle {#1}} %
  {\XXint \textstyle \scriptstyle {#1}} %
  {\XXint \scriptstyle \scriptscriptstyle {#1}} %
  {\XXint \scriptscriptstyle \scriptscriptstyle {#1}} %
  \!\int }

\def \XXint #1#2#3{{\setbox 0=\hbox {$#1{#2#3}{\int }$}\vcenter {\hbox {$#2#3$}}\kern -0.5\wd 0}}

\def \dashint {\Xint -}							

% Fallunterscheidung
\newcommand {\Fall}[5]
    { #1 = \left \{
        \begin{array}
         {c@ {\quad \mbox{ if } \quad} l}
          #2 & #3 \\
           #4 & #5           \end{array}
            \right .    }


\newcommand {\Falldrei}[7]
       { #1 = \left \{
          \begin{array}
             {c@ {\quad \mbox{ if } \quad} l}
               #2 & #3 \\
            #4 & #5 \\ 														             \right .    }


