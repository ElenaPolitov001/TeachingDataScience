%%%%%%%%%%%%%%%%%%%%%%%%%%%%%%%%%%%%%%%%%%%%%%%%%%%%%%%%%%%%%%%%%%%%%%%%%%%%%%%%%%
  \begin{frame}[fragile]\frametitle{}
\begin{center}
{\Large Measure Theory }
\end{center}
\end{frame}


%%%%%%%%%%%%%%%%%%%%%%%%%%%%%%%%%%%%%%%%%%%%%%%%%%%%%%%%%%%
  \begin{frame}[fragile]{Background}
\begin{itemize} 
\item Real Analysis and, in particular, measure theory, is very important in probability and statistics. 
\item Measure theory itself can be very abstract and difficult. (I am not even skin deep into it)
\item A \emph{measure} is a generalization of the concept of length, area, volume, etc. 
\end{itemize}

\end{frame}


%%%%%%%%%%%%%%%%%%%%%%%%%%%%%%%%%%%%%%%%%%%%%%%%%%%%%%%%%%%
  \begin{frame}[fragile]{Introduction}
\begin{itemize} 
\item  Measure Theoretic Probability offers a very generalized view of probability. \item Using sets rather than distributions represented by either discrete or continuous functions
\item It allows for complex problems to be understood more simply (if you can get past the rigorous math!)

\end{itemize}

\tiny{Measure Theory for Probability: A Very Brief Introduction - Count Bayesie}
\end{frame}

%%%%%%%%%%%%%%%%%%%%%%%%%%%%%%%%%%%%%%%%%%%%%%%%%%%%%%%%%%%
  \begin{frame}[fragile]{Measure for Measure}
\begin{itemize} 
\item  Measure Theory is the formal theory of things that are measurable
\item Examples: ``How tall are you?'', ``What shoe size do you wear?'', ``How far to the gas station?'' 
\item All of these are questions about measuring something in one dimension.
\item ``How many square feet is that house?'', ``How many acres is the farm?'' This is measurement in two dimensions.
\item ``How many gallons of milk do you need?'', ``How many cubic yards of rocks to fill that hole?''  This of course just extends the idea of measurement into three dimensions.
\end{itemize}

\tiny{Measure Theory for Probability: A Very Brief Introduction - Count Bayesie}
\end{frame}

%%%%%%%%%%%%%%%%%%%%%%%%%%%%%%%%%%%%%%%%%%%%%%%%%%%%%%%%%%%
  \begin{frame}[fragile]{Measure for Measure}
\begin{itemize} 
\item We also have weight, time, velocity, income, age, etc.
\item All of these can be measured.
\item Sometimes we can come up with things that are NOT measurable! 
\item Imagine that you were to build a wall with Lego and then took this wall apart and were somehow able to build two identical walls from only the bricks in the first! 
\item If we were trying to take the probability of something, but it turned out to be non-measurable than we would clearly end up in some very strange territory.
\end{itemize}

\tiny{Measure Theory for Probability: A Very Brief Introduction - Count Bayesie}
\end{frame}

%%%%%%%%%%%%%%%%%%%%%%%%%%%%%%%%%%%%%%%%%%%%%%%%%%%%%%%%%%%
  \begin{frame}[fragile]{Measure for Measure}
\begin{itemize} 
\item The most basic point of probability is that you are measuring the likelihood of events on a scale from 0 to 1. 
\item When we imagine all the things that could happen we're really imagining a 'set' of events.
\item A measurable space is a collection of events
$B$, and the set of all outcomes $\Omega$, which is sometimes called the sample space
\item A measure $\mu$ takes a set $A$ (from a measurable collection of sets $B$), and returns ``the measure of $A$,'' which is some positive real number. 
$\mu : B \rightarrow [0,\infty)$. 
\item Intuition: when $\mu$ is applied to any set, then it gives \emph{Size} of it.
\item Measure of rectangle is $a.b$.
\item Measure of circle is $\pi r^2$
\item Measure of entire real line is ?? $\infty$
\item An example measure is a volume, which goes by the name \emph{Lebesgue
measure}
\end{itemize}

\tiny{A Measure Theory Tutorial (Measure Theory for Dummies) - Maya Gupta}
\end{frame}

%%%%%%%%%%%%%%%%%%%%%%%%%%%%%%%%%%%%%%%%%%%%%%%%%%%%%%%%%%%
  \begin{frame}[fragile]{Properties of Measure}
\begin{itemize} 
\item Nonnegativity: $\mu(A) \geq 0 \forall A \in B$
\item $\mu(\emptyset) = 0$
\item If $A$ is subset of $B$ then $\mu(A) \leq B$
\item Countable Additivity: If $A_i \in B$ are disjoint sets for $i = 1, 2, \ldots$, then the measure of the union of the $A_i$ is equal to the sum of the measures of the $A_i$.
\item You can see how our ordinary notion of volume satisfies these two properties. Volume cannot be negative and addition of volume is sum (no deduction due to intersection like in sets addition)
\item A probability measure $P$ has
the two above properties of a measure but it’s also normalized, such that $P(\Omega) = 1$.

\end{itemize}

\tiny{A Measure Theory Tutorial (Measure Theory for Dummies) - Maya Gupta}
\end{frame}