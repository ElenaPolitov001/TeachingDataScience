%%%%%%%%%%%%%%%%%%%%%%%%%%%%%%%%%%%%%%%%%%%%%%%%%%%%%%%%%%%%%%%%%%%%%%%%%%%%%%%%%%
\begin{frame}[fragile]\frametitle{}

\begin{center}
{\Large Classification in spaCy}
\end{center}
\end{frame}

%%%%%%%%%%%%%%%%%%%%%%%%%%%%%%%%%%%%%%%%%%%%%%%%%%%%%%%%%%%%%%%%%%%%%%%%%%%%%%%%%%
\begin{frame}[fragile]\frametitle{Text Classification}


\begin{itemize}
\item As a pipeline component \lstinline|textcat (TextCategorizer)|, 
\item Can assign categories (or labels) to text data and use that as training data for a neural network.
\item General steps:
\begin{itemize}
\item Add the textcat component to the existing pipeline.
\item Add valid labels to the textcat component.
\item Load, shuffle, and split your data.
\item Train the model, evaluating on each training loop.
\item Use the trained model to predict the sentiment of non-training data.
\item Optionally, save the trained model.
  \end{itemize}

  \end{itemize}

\end{frame}

%%%%%%%%%%%%%%%%%%%%%%%%%%%%%%%%%%%%%%%%%%%%%%%%%%%%%%%%%%%%%%%%%%%%%%%%%%%%%%%%%%
\begin{frame}[fragile]\frametitle{Example: Sentiment Analyzer}


General steps:
\begin{itemize}
\item Download data (https://ai.stanford.edu/~amaas/data/sentiment/)
\item Loading data
\item Preprocessing
\item Training the classifier
\item Classifying data
\end{itemize}


\end{frame}


%%%%%%%%%%%%%%%%%%%%%%%%%%%%%%%%%%%%%%%%%%%%%%%%%%%%%%%%%%%%%%%%%%%%%%%%%%%%%%%%%%
\begin{frame}[fragile]\frametitle{Load Data}

\begin{itemize}
\item Load text and labels from the file and directory structures.
\item Shuffle the data.
\item Split the data into training and test sets.
\item Return the two sets of data.
\end{itemize}


\begin{lstlisting}
def load_training_data(
    data_directory: str = "aclImdb/train",
    split: float = 0.8,
    limit: int = 0
) -> tuple:
\end{lstlisting}

* Note the use of Python 3’s type annotations to make it absolutely clear which types your function expects and what it will return.

\end{frame}

%%%%%%%%%%%%%%%%%%%%%%%%%%%%%%%%%%%%%%%%%%%%%%%%%%%%%%%%%%%%%%%%%%%%%%%%%%%%%%%%%%
\begin{frame}[fragile]\frametitle{Load Data and Preprocessing}

\begin{lstlisting}
def load_training_data(data_directory: str = "aclImdb/train",
    split: float = 0.8,limit: int = 0) -> tuple:
    reviews = []
    for label in ["pos", "neg"]:
        labeled_directory = f"{data_directory}/{label}"
        for review in os.listdir(labeled_directory):
            if review.endswith(".txt"):
                with open(f"{labeled_directory}/{review}") as f:
                    text = f.read()
                    text = text.replace("<br />", "\n\n")
                    if text.strip():
                        spacy_label = {
                            "cats": {
                                "pos": "pos" == label,
                                "neg": "neg" == label
                            }
                        }
                        reviews.append((text, spacy_label))
    random.shuffle(reviews)
    if limit:
        reviews = reviews[:limit]
    split = int(len(reviews) * split)
    return reviews[:split], reviews[split:]												
\end{lstlisting}

\end{frame}

%%%%%%%%%%%%%%%%%%%%%%%%%%%%%%%%%%%%%%%%%%%%%%%%%%%%%%%%%%%%%%%%%%%%%%%%%%%%%%%%%%
\begin{frame}[fragile]\frametitle{Training Classifier}

\begin{itemize}
\item Modifying the base spaCy pipeline to include the textcat component
\item Building a training loop to train the textcat component
\item Evaluating the progress of your model training after a given number of training loops
\end{itemize}

Uses convolutional neural network (CNN) for classifying text data.


\end{frame}

%%%%%%%%%%%%%%%%%%%%%%%%%%%%%%%%%%%%%%%%%%%%%%%%%%%%%%%%%%%%%%%%%%%%%%%%%%%%%%%%%%
\begin{frame}[fragile]\frametitle{Modifying Pipeline }

\begin{lstlisting}
import os
import random
import spacy

def train_model(
    training_data: list,
    test_data: list,
    iterations: int = 20
) -> None:
    # Build pipeline
    nlp = spacy.load("en_core_web_sm")
    if "textcat" not in nlp.pipe_names:
        textcat = nlp.create_pipe(
            "textcat", config={"architecture": "simple_cnn"}
        )
        nlp.add_pipe(textcat, last=True)	
    else:
        textcat = nlp.get_pipe("textcat")

    textcat.add_label("pos")
    textcat.add_label("neg")				
\end{lstlisting}

\end{frame}

%%%%%%%%%%%%%%%%%%%%%%%%%%%%%%%%%%%%%%%%%%%%%%%%%%%%%%%%%%%%%%%%%%%%%%%%%%%%%%%%%%
\begin{frame}[fragile]\frametitle{Building Training Loop }

\begin{itemize}
\item Set pipeline to train only the textcat component, 
\item Generate batches of data for it with spaCy’s minibatch() and compounding() utilities, 
\item Go through them and update your model.
\end{itemize}

\end{frame}

%%%%%%%%%%%%%%%%%%%%%%%%%%%%%%%%%%%%%%%%%%%%%%%%%%%%%%%%%%%%%%%%%%%%%%%%%%%%%%%%%%
\begin{frame}[fragile]\frametitle{Building Training Loop }

\begin{lstlisting}
def train_model(
    training_data: list,
    test_data: list,
    iterations: int = 20
) -> None:

		:
		:

    textcat.add_label("pos")
    textcat.add_label("neg")			
		
    # Train only textcat
    training_excluded_pipes = [
        pipe for pipe in nlp.pipe_names if pipe != "textcat"
    ]		
    with nlp.disable_pipes(training_excluded_pipes):
        optimizer = nlp.begin_training()
        # Training loop
        print("Beginning training")
        batch_sizes = compounding(
            4.0, 32.0, 1.001
        )  # A generator that yields infinite series of input numbers		
\end{lstlisting}

\end{frame}

%%%%%%%%%%%%%%%%%%%%%%%%%%%%%%%%%%%%%%%%%%%%%%%%%%%%%%%%%%%%%%%%%%%%%%%%%%%%%%%%%%
\begin{frame}[fragile]\frametitle{Building Training Loop }

\begin{lstlisting}
def train_model(training_data: list,test_data: list,iterations: int = 20
) -> None:
		:
		:
    with nlp.disable_pipes(training_excluded_pipes):
        optimizer = nlp.begin_training()
        # Training loop
        print("Beginning training")
        batch_sizes = compounding(
            4.0, 32.0, 1.001
        )  # A generator that yields infinite series of input numbers		
        for i in range(iterations):
            loss = {}
            random.shuffle(training_data)
            batches = minibatch(training_data, size=batch_sizes)
            for batch in batches:
                text, labels = zip(*batch)
                nlp.update(
                    text,
                    labels,
                    drop=0.2,
                    sgd=optimizer,
                    losses=loss
                )				
\end{lstlisting}

\end{frame}


%%%%%%%%%%%%%%%%%%%%%%%%%%%%%%%%%%%%%%%%%%%%%%%%%%%%%%%%%%%%%%%%%%%%%%%%%%%%%%%%%%
\begin{frame}[fragile]\frametitle{Evaluating Model Training }

\begin{itemize}
\item A separate evaluate\_model() function.
\item Run the documents in your test set against the unfinished model to get your model’s predictions 
\item Compare them to the correct labels of that data.
\item For evaluate\_model(), pass in the pipeline’s tokenizer component, the textcat component, and your test dataset:
\end{itemize}

\end{frame}

%%%%%%%%%%%%%%%%%%%%%%%%%%%%%%%%%%%%%%%%%%%%%%%%%%%%%%%%%%%%%%%%%%%%%%%%%%%%%%%%%%
\begin{frame}[fragile]\frametitle{Evaluating Model Training }

\begin{lstlisting}
def evaluate_model(tokenizer, textcat, test_data: list) -> dict:
    reviews, labels = zip(*test_data)
    reviews = (tokenizer(review) for review in reviews)
    true_positives = 0
    false_positives = 1e-8  # Can't be 0 because of presence in denominator
    true_negatives = 0
    false_negatives = 1e-8
    for i, review in enumerate(textcat.pipe(reviews)):
        true_label = labels[i]
        for predicted_label, score in review.cats.items():
            # Every cats dictionary includes both labels. You can get all
            # the info you need with just the pos label.
            if (predicted_label == "neg"):
                continue
            if score >= 0.5 and true_label["pos"]:
                true_positives += 1
            elif score >= 0.5 and true_label["neg"]:
                false_positives += 1
            elif score < 0.5 and true_label["neg"]:
                true_negatives += 1
            elif score < 0.5 and true_label["pos"]:
                false_negatives += 1
\end{lstlisting}

\end{frame}

%%%%%%%%%%%%%%%%%%%%%%%%%%%%%%%%%%%%%%%%%%%%%%%%%%%%%%%%%%%%%%%%%%%%%%%%%%%%%%%%%%
\begin{frame}[fragile]\frametitle{Evaluating Model Training }

\begin{lstlisting}
def evaluate_model(
    tokenizer, textcat, test_data: list
) -> dict:
		:
		:
    precision = true_positives / (true_positives + false_positives)
    recall = true_positives / (true_positives + false_negatives)

    if precision + recall == 0:
        f_score = 0
    else:
        f_score = 2 * (precision * recall) / (precision + recall)
    return {"precision": precision, "recall": recall, "f-score": f_score}								
\end{lstlisting}

\end{frame}

%%%%%%%%%%%%%%%%%%%%%%%%%%%%%%%%%%%%%%%%%%%%%%%%%%%%%%%%%%%%%%%%%%%%%%%%%%%%%%%%%%
\begin{frame}[fragile]\frametitle{Evaluating Model Training }

\begin{lstlisting}
def train_model(training_data: list, test_data: list, iterations: int = 20):
    # Previously seen code omitted for brevity.
        # Training loop
        print("Beginning training")
        print("Loss\tPrecision\tRecall\tF-score")
        batch_sizes = compounding(
            4.0, 32.0, 1.001
        )  # A generator that yields infinite series of input numbers
        for i in range(iterations):
						:
						:
            with textcat.model.use_params(optimizer.averages):
                evaluation_results = evaluate_model(
                    tokenizer=nlp.tokenizer,
                    textcat=textcat,
                    test_data=test_data
                )
                print(
                    f"{loss['textcat']}\t{evaluation_results['precision']}"
                    f"\t{evaluation_results['recall']}"
                    f"\t{evaluation_results['f-score']}"
                )		
    # Save model
    with nlp.use_params(optimizer.averages):
        nlp.to_disk("model_artifacts")								
\end{lstlisting}

\end{frame}

%%%%%%%%%%%%%%%%%%%%%%%%%%%%%%%%%%%%%%%%%%%%%%%%%%%%%%%%%%%%%%%%%%%%%%%%%%%%%%%%%%
\begin{frame}[fragile]\frametitle{Test Model}

\begin{itemize}
\item Time to test it against a real review.
\item Load the previously saved model.
\end{itemize}

\begin{lstlisting}
TEST_REVIEW = """..."""				
def test_model(input_data: str = TEST_REVIEW):
    #  Load saved trained model
    loaded_model = spacy.load("model_artifacts")
    # Generate prediction
    parsed_text = loaded_model(input_data)
    # Determine prediction to return
    if parsed_text.cats["pos"] > parsed_text.cats["neg"]:
        prediction = "Positive"
        score = parsed_text.cats["pos"]
    else:
        prediction = "Negative"
        score = parsed_text.cats["neg"]
    print(
        f"Review text: {input_data}\nPredicted sentiment: {prediction}"
        f"\tScore: {score}"
    )			
\end{lstlisting}

\end{frame}

%%%%%%%%%%%%%%%%%%%%%%%%%%%%%%%%%%%%%%%%%%%%%%%%%%%%%%%%%%%%%%%%%%%%%%%%%%%%%%%%%%
\begin{frame}[fragile]\frametitle{The Main}

\begin{lstlisting}
if __name__ == "__main__":
    train, test = load_training_data(limit=2500)
    train_model(train, test)
    print("Testing model")
    test_model()
\end{lstlisting}

\begin{itemize}
\item What did the model predict? 
\item Do you agree with the result? 
\item What happens if you increase or decrease the limit parameter when loading the data? 
\end{itemize}

\end{frame}


%%%%%%%%%%%%%%%%%%%%%%%%%%%%%%%%%%%%%%%%%%%%%%%%%%%%%%%%%%%%%%%%%%%%%%%%%%%%%%%%%%
\begin{frame}[fragile]\frametitle{What Next?}

\begin{itemize}
\item Rewrite your code to remove stop words during preprocessing or data loading. How does the mode performance change? Can you incorporate this preprocessing into a pipeline component instead?
\item Explore the configuration parameters for the textcat pipeline component and experiment with different configurations.
\end{itemize}

\end{frame}






