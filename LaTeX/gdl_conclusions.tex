%%%%%%%%%%%%%%%%%%%%%%%%%%%%%%%%%%%%%%%%%%%%%%%%%%%%%%%%%%%%%%%%%%%%%%%%%%%%%%%%%%
\begin{frame}[fragile]\frametitle{}
\begin{center}
{\Large Conclusions}
\end{center}
\end{frame}

%%%%%%%%%%%%%%%%%%%%%%%%%%%%%%%%%%%%%%%%%%%%%%%%%%%%%%%%%%%
\begin{frame}[fragile]\frametitle{Contributions}

Geometric Deep Learning has 3 main contributions:

\begin{itemize}
\item Use of non-euclidean data
\item Maximize on the information from the data we collect
\item Use this data to teach machine learning algorithms
\end{itemize}
	  
{\tiny (Ref: What is Geometric Deep Learning? - Flawnson Tong)}

\end{frame}

%%%%%%%%%%%%%%%%%%%%%%%%%%%%%%%%%%%%%%%%%%%%%%%%%%%%%%%%%%%
\begin{frame}[fragile]\frametitle{In Essence}


\begin{itemize}
\item The notion of relationships, connections, and shared properties is a concept that is naturally occurring in humans and nature. 
\item Understanding and learning from these connections is something we take for granted. 
\item Geometric Deep Learning is significant because it allows us to take advantage of data with inherent relationships, connections, and shared properties.
\end{itemize}
	  
{\tiny (Ref: What is Geometric Deep Learning? - Flawnson Tong)}

\end{frame}


%%%%%%%%%%%%%%%%%%%%%%%%%%%%%%%%%%%%%%%%%%%%%%%%%%%%%%%%%%%
\begin{frame}[fragile]\frametitle{Key Takeaways
}


\begin{itemize}
\item  The Euclidean domain and non-euclidean domain have different rules that are followed; data in each domain specializes in certain formats (image, text vs graphs, manifolds) and convey differing amounts of information
\item  Geometric Deep Learning is the class of Deep Learning that can operate on the non-euclidean domain with the goal of teaching models how to perform predictions and classifications on relational datatypes
\item  The difference between traditional Deep Learning and Geometric Deep Learning can be illustrated by imagining the accuracy between scanning an image of a person versus scanning the surface of the person themselves.
\item  In traditional Deep Learning, dimensionality is directly correlated with the number of features in the data whereas in Geometric Deep Learning, it refers to the type of the data itself, not the number of features it has.
\end{itemize}
	  
{\tiny (Ref: What is Geometric Deep Learning? - Flawnson Tong)}

\end{frame}

%%%%%%%%%%%%%%%%%%%%%%%%%%%%%%%%%%%%%%%%%%%%%%%%%%%%%%%%%%%
\begin{frame}[fragile]\frametitle{References}

\begin{itemize}
\item What is Geometric Deep Learning? - Flawnson Tong
\end{itemize}
	  
\end{frame}
