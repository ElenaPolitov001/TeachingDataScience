%%%%%%%%%%%%%%%%%%%%%%%%%%%%%%%%%%%%%%%%%%%%%%%%%%%%%%%%%%%%%%%%%%%%%%%%%%%%%%%%%%
\begin{frame}[fragile]\frametitle{}
\begin{center}
{\Large Descriptive Statistics}
\end{center}
\end{frame}


%%%%%%%%%%%%%%%%%%%%%%%%%%%%%%%%%%%%%%%%%%%%%%%%%%%%%%%%%%%
\begin{frame}[fragile]\frametitle{Descriptive Statistics}
\begin{itemize}
\item Describes the data characteristics
\item To make sense of the data
\item To make rational decisions
\item E.g. Demographics, clinical data.
\item Measures of Central Tendencies
\item Measures of Variability
\item Measures of Shape
\end{itemize}
\end{frame}

%%%%%%%%%%%%%%%%%%%%%%%%%%%%%%%%%%%%%%%%%%%%%%%%%%%%%%%%%%%
\begin{frame}[fragile]\frametitle{Why Descriptive Statistics?}
\begin{itemize}
\item Population: the whole
\item Sample: small subset of the population
\item Gauging Population by examining traits of Sample.
\item Example Question: Finding height of Americans?
\item Not going to measure everyone height, but that in a {\bf representative} sample.
\item Example: Election sampling?
\end{itemize}
\end{frame}



%%%%%%%%%%%%%%%%%%%%%%%%%%%%%%%%%%%%%%%%%%%%%%%%%%%%%%%%%%%
\begin{frame}[fragile]\frametitle{Why Descriptive Statistics?}
\begin{itemize}
\item To check the accuracy and precision of the process
\item To reduce variability and improve process capability
\item To know the truth about the real world
\end{itemize}
\end{frame}


%%%%%%%%%%%%%%%%%%%%%%%%%%%%%%%%%%%%%%%%%%%%%%%%%%%%%%%%%%%%%%%%%%%%%%%%%%%%%%%%%%
\begin{frame}[fragile]\frametitle{}
\begin{center}
{\Large Basic Terms}
\end{center}
\end{frame}

%%%%%%%%%%%%%%%%%%%%%%%%%%%%%%%%%%%%%%%%%%%%%%%%%%%%%%%%%%%%%%%%%%%%%%%%
\begin{frame}[fragile]\frametitle{Histogram}

\begin{columns}
    \begin{column}[T]{0.6\linewidth}
Example
	\begin{itemize}
	\item Say, we are measuring height of people.
	\item Plotting them on X axis.
	\item The dots would look very crowded where there are many close or repetitive observations.
	\item Some dots get hidden.
	\item We can improve the visualization, by plotting frequency (number of occurrences) on Y axis.
	\item But in case of contiguous variable, like, exact measurements are rare. So we `bin' them and measure occurrences.
	\item Thats Histogram.
	\end{itemize}

    \end{column}
    \begin{column}[T]{0.4\linewidth}
      \begin{center}
      \includegraphics[width=\linewidth,keepaspectratio]{statq1}
	  
	  \includegraphics[width=\linewidth,keepaspectratio]{statq2}
	  
	  \includegraphics[width=\linewidth,keepaspectratio]{statq3}	  
	  	\end{center}
    \end{column}

  \end{columns}
  

\tiny{(Ref: StatQuest: What is a Histogram? - Josh Starmer )}
\end{frame}

%%%%%%%%%%%%%%%%%%%%%%%%%%%%%%%%%%%%%%%%%%%%%%%%%%%%%%%%%%%%%%%%%%%%%%%%
\begin{frame}[fragile]\frametitle{Histogram}

\begin{columns}
    \begin{column}[T]{0.6\linewidth}

	\begin{itemize}
	\item Histogram can be used to predict probability of getting (future) measurements.
	\item Getting measurement (as shown in the box) in the middle region is more likely.
	\item Measurements at both the ends are rare.
	\item We can approximate this histogram of observations by a `distribution'.
	\item Looks like `Normal' distribution, or a bell-curve
	\end{itemize}

    \end{column}
    \begin{column}[T]{0.4\linewidth}
      \begin{center}
      \includegraphics[width=\linewidth,keepaspectratio]{statq4}
	  
	  \includegraphics[width=\linewidth,keepaspectratio]{statq5}
	  
	  \includegraphics[width=\linewidth,keepaspectratio]{statq6}	  
	  	\end{center}
    \end{column}

  \end{columns}
  

\tiny{(Ref: StatQuest: What is a Histogram? - Josh Starmer )}
\end{frame}

%%%%%%%%%%%%%%%%%%%%%%%%%%%%%%%%%%%%%%%%%%%%%%%%%%%%%%%%%%%%%%%%%%%%%%%%
\begin{frame}[fragile]\frametitle{Histogram}

\begin{columns}
    \begin{column}[T]{0.6\linewidth}

	\begin{itemize}
	\item If the frequency of measurements seem decreasing, it may be an exponential distribution.
	\item Binning criterion is critical. They can not be too narrow or too wide.
	\item Try different bin widths/formulas to plot a histogram.
	\end{itemize}

    \end{column}
    \begin{column}[T]{0.4\linewidth}
      \begin{center}
      \includegraphics[width=\linewidth,keepaspectratio]{statq7}
	  
	  \includegraphics[width=\linewidth,keepaspectratio]{statq8}
	  
	  \includegraphics[width=\linewidth,keepaspectratio]{statq9}	  
	  	\end{center}
    \end{column}

  \end{columns}
  

\tiny{(Ref: StatQuest: What is a Histogram? - Josh Starmer )}
\end{frame}


%%%%%%%%%%%%%%%%%%%%%%%%%%%%%%%%%%%%%%%%%%%%%%%%%%%%%%%%%%%%%%%%%%%%%%%%%%%%%%%%%%
\begin{frame}[fragile]\frametitle{}
\begin{center}
{\Large Descriptive Statistics Example}
\end{center}
\end{frame}


%%%%%%%%%%%%%%%%%%%%%%%%%%%%%%%%%%%%%%%%%%%%%%%%%%%%%%%%%%%
\begin{frame}[fragile]\frametitle{Descriptive Statistics}
\begin{itemize}
\item Describes features of data sets using numbers
\item Individual row: Data
\item Full table: Dataset
\item Purpose: Answer questions.
\end{itemize}
\begin{center}
\includegraphics[width=0.35\linewidth,keepaspectratio]{da1}
\end{center}
\end{frame}

%%%%%%%%%%%%%%%%%%%%%%%%%%%%%%%%%%%%%%%%%%%%%%%%%%%%%%%%%%%
\begin{frame}[fragile]\frametitle{Questions}
\begin{center}
\includegraphics[width=0.35\linewidth,keepaspectratio]{da1}
\end{center}
\begin{itemize}
\item What is Bobby's score?
\item Out of? (Total \# entries)
\item Highest/Lowest scores?
\end{itemize}
\end{frame}

%%%%%%%%%%%%%%%%%%%%%%%%%%%%%%%%%%%%%%%%%%%%%%%%%%%%%%%%%%%
\begin{frame}[fragile]\frametitle{Questions}
\begin{center}
\includegraphics[width=0.35\linewidth,keepaspectratio]{da1}
\end{center}
\begin{itemize}
\item Class average?
\item Most Common/frequent Score?
\item Any other questions?
\end{itemize}
\end{frame}

%%%%%%%%%%%%%%%%%%%%%%%%%%%%%%%%%%%%%%%%%%%%%%%%%%%%%%%%%%%
\begin{frame}[fragile]\frametitle{Numerical Measures}
\begin{itemize}
\item Highest to Lowest Score: RANGE
\item Most Common Score: MODE
\item Average Score: MEAN
\item Any other measures?
\end{itemize}
\end{frame}

%%%%%%%%%%%%%%%%%%%%%%%%%%%%%%%%%%%%%%%%%%%%%%%%%%%%%%%%%%%
\begin{frame}[fragile]\frametitle{Descriptive Statistics}
\begin{itemize}
\item Examines ALL data (not sample)
\item Cannot generalize to other datasets
\end{itemize}
\end{frame}

%%%%%%%%%%%%%%%%%%%%%%%%%%%%%%%%%%%%%%%%%%%%%%%%%%%%%%%%%%%%%%%%%%%%%%%%%%%%%%%%%%
\begin{frame}[fragile]\frametitle{}
\begin{center}
{\Large Descriptive Statistics Example}
\end{center}
\end{frame}


%%%%%%%%%%%%%%%%%%%%%%%%%%%%%%%%%%%%%%%%%%%%%%%%%%%
\begin{frame}[fragile] \frametitle{Descriptive Tasks}

\adjustbox{valign=t}{
\begin{minipage}{0.45\linewidth}
\begin{center}
\includegraphics[width=\linewidth,keepaspectratio]{descriptivetable}
\end{center}
\end{minipage}
}
\hfill
\adjustbox{valign=t}{
\begin{minipage}{0.45\linewidth}
\begin{itemize}
\item Objective: Derive patterns, summarize underlying relationships
\item More exploratory of current state
\end{itemize}
\end{minipage}
}

\end{frame}



%%%%%%%%%%%%%%%%%%%%%%%%%%%%%%%%%%%%%%%%%%%%%%%%%%%%%%%%%%
\begin{frame}[fragile]\frametitle{Data Example}	
\begin{center}
\includegraphics[width=0.8\linewidth,keepaspectratio]{da2}
\end{center}
What sense it makes?
Any pattern?


%\code{Store as list of integers}
\end{frame}

%%%%%%%%%%%%%%%%%%%%%%%%%%%%%%%%%%%%%%%%%%%%%%%%%%%%%%%%%%
\begin{frame}[fragile]\frametitle{Visualize}	
\begin{center}
\includegraphics[width=0.8\linewidth,keepaspectratio]{da3}
\end{center}
Makes sense?


%\code{Plot the data points, not the curve}
\end{frame}

%%%%%%%%%%%%%%%%%%%%%%%%%%%%%%%%%%%%%%%%%%%%%%%%%%%%%%%%%%
\begin{frame}[fragile]\frametitle{The Shape of The Distribution}	
Better to see
\begin{center}
\includegraphics[width=0.8\linewidth,keepaspectratio]{da4}
\end{center}
Symmetric? Skewed right/left?
%\code{Plot the curve}
\end{frame}

%%%%%%%%%%%%%%%%%%%%%%%%%%%%%%%%%%%%%%%%%%%%%%%%%%%%%%%%%%%%%%%%%%%%%%%%%%%%%%%%%%
\begin{frame}[fragile]\frametitle{}
\begin{center}
{\Large Describing Data}
\end{center}
\end{frame}

%%%%%%%%%%%%%%%%%%%%%%%%%%%%%%%%%%%%%%%%%%%%%%%%%%%%%%%%%%
\begin{frame}[fragile]\frametitle{Describing Data}
\begin{itemize}
\item Univariate Analysis: Statistical moments (based on degree)
\begin{itemize}
\item 1st degree: Central tendency: mean, median, mode
\item 2nd degree: Standard Deviation: how wide data is around mean
\item 3rd degree: Skewness: Asymmetric around mean
\item 4th degree: Kurtosis: shape of skewness.
\end{itemize}
\item BiVariate Analysis: covariance, correlation
\item MultiVariate Analysis
\end{itemize}
\end{frame}


%%%%%%%%%%%%%%%%%%%%%%%%%%%%%%%%%%%%%%%%%%%%%%%%%%%%%%%%%%%
\begin{frame}[fragile]\frametitle{Univariate Analysis}
\begin{itemize}
\item Measure of Central Tendency
\item Measure of Spread
\item Measure of Asymmetry
\item Measure of Skewness
\end{itemize}
\end{frame}


%%%%%%%%%%%%%%%%%%%%%%%%%%%%%%%%%%%%%%%%%%%%%%%%%%%%%%%%%%%
%\begin{frame}[fragile]\frametitle{Frequency}	
%$ frequency(v_i) = \frac{Objects\_with\_Attribute\_v_i}{n}$
%\begin{center}
%\includegraphics[width=0.8\linewidth,keepaspectratio]{frequency}
%\end{center}
%Often used with categorical values.
%\end{frame}
%

%%%%%%%%%%%%%%%%%%%%%%%%%%%%%%%%%%%%%%%%%%%%%%%%%%%%%%%%%%%%%%%%%%%%%%%%%%%%%%%%%%
\begin{frame}[fragile]\frametitle{}
\begin{center}
{\Large Measure of Central Tendency}
\end{center}
\end{frame}

%%%%%%%%%%%%%%%%%%%%%%%%%%%%%%%%%%%%%%%%%%%%%%%%%%%%%%%%%%
\begin{frame}[fragile]\frametitle{Mean}	
\begin{itemize}
\item Measure the ``location'' of a set of values
\item Mean is a very, very common measurement
\item But is sensitive to outliers
\end{itemize}
$mean(x) = \bar{x} = 1/n \sum x_i$
%\code{Use `for' loop to calculate}
\end{frame}

%%%%%%%%%%%%%%%%%%%%%%%%%%%%%%%%%%%%%%%%%%%%%%%%%%%%%%%%%%
\begin{frame}[fragile]\frametitle{Mean}	
\begin{center}
\includegraphics[width=\linewidth,keepaspectratio]{da5}
\end{center}
Sum = 153

Mean = 153/25 = 6.12
\end{frame}

%%%%%%%%%%%%%%%%%%%%%%%%%%%%%%%%%%%%%%%%%%%%%%%%%%%%%%%%%%
\begin{frame}[fragile]\frametitle{Outliers}	
\begin{itemize}
\item Extreme data point
\item May affect calculations
\item  Can occur in any given data set and in any distribution
\item  May indicate an experimental error or incorrect recording of data
\end{itemize}
\begin{center}
\includegraphics[width=0.6\linewidth,keepaspectratio]{da15}
\end{center}
\end{frame}


%%%%%%%%%%%%%%%%%%%%%%%%%%%%%%%%%%%%%%%%%%%%%%%%%%%%%%%%%%%%%%%%%%%%%%%%
\begin{frame}[fragile]\frametitle{Mean}
Implement mean
\begin{lstlisting}
def mean(datalist):
	:
	return m

lst = [9,3,7,2,7,10,23,44,12,42,19,11,22,5,3,4,3,21,3]
result = mean(lst)
print("Mean : {}".format(result))
\end{lstlisting}
Result?
\end{frame}

%%%%%%%%%%%%%%%%%%%%%%%%%%%%%%%%%%%%%%%%%%%%%%%%%%%%%%%%%%
\begin{frame}[fragile]\frametitle{Mean}
\begin{lstlisting}
def mean(datalist):
	total = 0
	m = 0
	for item in datalist:
		total += item
	m = total / float(len(datalist))
	return m
\end{lstlisting}
Mean : 13.157894736842104
\end{frame}

%%%%%%%%%%%%%%%%%%%%%%%%%%%%%%%%%%%%%%%%%%%%%%%%%%%%%%%%%%
\begin{frame}[fragile]\frametitle{Median}	
\begin{itemize}
\item Commonly used instead of mean if outliers are present
\item Median is the middle value 
\item if odd number of values are present; average of the two middle values if even number of values
\item Not easily affected by outliers (extreme values). 
\item Always exists and unique.
\end{itemize}
$median(x) = x_{r=1} for \quad odd, 1/2(x_r + x_{r+1}) for \quad even$
%\code{Cater to both odd/even case}
\end{frame}


%%%%%%%%%%%%%%%%%%%%%%%%%%%%%%%%%%%%%%%%%%%%%%%%%%%%%%%%%%
\begin{frame}[fragile]\frametitle{Median}	
\begin{center}
\includegraphics[width=0.5\linewidth,keepaspectratio]{da6}
\end{center}
Median = 6

Medians are less reliable: medians of samples drawn from same population vary more widely than sample means.
%\code{Calculate and verify the answer}
\end{frame}


%%%%%%%%%%%%%%%%%%%%%%%%%%%%%%%%%%%%%%%%%%%%%%%%%%%%%%%%%%%%%%%%%%%%%%%%
\begin{frame}[fragile]\frametitle{Median}
Implement median
\begin{lstlisting}
def median(datalist):
	:
	return m

lst = [9,3,7,2,7,10,23,44,12,42,19,11,22,5,3,4,3,21,3]
result = median(lst)
print("Median : {}".format(result))
\end{lstlisting}
Result?
\end{frame}

%%%%%%%%%%%%%%%%%%%%%%%%%%%%%%%%%%%%%%%%%%%%%%%%%%%%%%%%%%
\begin{frame}[fragile]\frametitle{Median}
\begin{lstlisting}
def median(datalist):
    n = len(datalist)
    numsort = sorted(datalist)
    mid = n // 2
    m = 1
    if n % 2 == 0:
        lo = mid - 1
        hi = mid
        m = (numsort[lo] + numsort[hi])/2
    else:
        m = numsort[mid]
    return m
\end{lstlisting}
Median : 9
\end{frame}



%%%%%%%%%%%%%%%%%%%%%%%%%%%%%%%%%%%%%%%%%%%%%%%%%%%%%%%%%%
\begin{frame}[fragile]\frametitle{Mode}	

\begin{itemize}
\item The value that has the highest frequency.
\item Requires no calculation, only counting
\item Often used with categorical values.
\item The mode (especially with discrete / continuous data) may reveal value that symbolizes a missing value.
\item Not a stable measure : it depends only a few values
\item May not exist
\item May not be unique
\end{itemize}

%\begin{center}
%\includegraphics[width=0.8\linewidth,keepaspectratio]{mode}
%\end{center}
%\code{Calculate and verify the answer}
\end{frame}

%%%%%%%%%%%%%%%%%%%%%%%%%%%%%%%%%%%%%%%%%%%%%%%%%%%%%%%%%%
\begin{frame}[fragile]\frametitle{Mode}	
\begin{center}
\includegraphics[width=0.6\linewidth,keepaspectratio]{da7}
\end{center}
Median = 7
\end{frame}


%%%%%%%%%%%%%%%%%%%%%%%%%%%%%%%%%%%%%%%%%%%%%%%%%%%%%%%%%%%%%%%%%%%%%%%%
\begin{frame}[fragile]\frametitle{Mode}
Implement mode
\begin{lstlisting}
def mode(datalist):
	:
	return m

lst = [9,3,7,2,7,10,23,44,12,42,19,11,22,5,3,4,3,21,3]
result = mode(lst)
print("Mode : {}".format(result))
\end{lstlisting}
Result?
\end{frame}

%%%%%%%%%%%%%%%%%%%%%%%%%%%%%%%%%%%%%%%%%%%%%%%%%%%%%%%%%%
\begin{frame}[fragile]\frametitle{Mode}
\begin{lstlisting}
def frequency_distribution(datalist):
	freqs = dict()
	for item in datalist:
		if item not in frees.keys():
			freqs[item] = 1
		else:
			freqs[item] += 1
	return freqs

def mode(datalist):
    d = frequency_distribution(datalist)
    print(d)
    most_often = 0
    m = 0
    for item in d.keys():
        if d[item] > most_often:
            most_often = d[item]
            m = item
    return m
\end{lstlisting}
Mode : 3
\end{frame}

%%%%%%%%%%%%%%%%%%%%%%%%%%%%%%%%%%%%%%%%%%%%%%%%%%%%%%%%%%
\begin{frame}[fragile]\frametitle{Mode}
Another implementation. Counter returns dictionary of frquencies and values.
\begin{lstlisting}
from collections import Counter

def mode2(x):
    counts = Counter(x)
    max_count = max(counts.values())
    return [x_i for x_i, count in counts.items() if count == max_count]  # multiple modes are possible
\end{lstlisting}
Mode : [3]
\end{frame}



%%%%%%%%%%%%%%%%%%%%%%%%%%%%%%%%%%%%%%%%%%%%%%%%%%%%%%%%%%
\begin{frame}[fragile]\frametitle{Central Tendency}	

\begin{itemize}
\item Mean: Summarizes all the information in the data set
\item Median: Splits the data sets into two halves: there are an equal number of values above and below it.
\item Mode: The most common value in the data set.
\end{itemize}

%\begin{center}
%\includegraphics[width=0.8\linewidth,keepaspectratio]{mode}
%\end{center}

\end{frame}

%%%%%%%%%%%%%%%%%%%%%%%%%%%%%%%%%%%%%%%%%%%%%%%%%%%%%%%%%%
\begin{frame}[fragile]\frametitle{Locating Central Tendency}	
\begin{center}
\includegraphics[width=\linewidth,keepaspectratio]{da8}
\end{center}
Mean = 6.12		

Median = 6		

Mode = 7

\end{frame}

%%%%%%%%%%%%%%%%%%%%%%%%%%%%%%%%%%%%%%%%%%%%%%%%%%%%%%%%%%%%%%%%%%%%%%%%%%%%%%%%%%
\begin{frame}[fragile]\frametitle{}
\begin{center}
{\Large Descriptive Statistics Exercise}
\end{center}
\end{frame}

%%%%%%%%%%%%%%%%%%%%%%%%%%%%%%%%%%%%%%%%%%%%%%%%%%%%%%%%%%
\begin{frame}[fragile]\frametitle{Exercise}	
Our Data
\begin{lstlisting}
lst = [9,3,7,2,7,10,23,44,12,42,19,11,22,5,3,4,3,21,3]
\end{lstlisting}

Store as list of integers and calculate Mean, Median and Mode
\end{frame}

% %%%%%%%%%%%%%%%%%%%%%%%%%%%%%%%%%%%%%%%%%%%%%%%%%%%%%%%%%%
% \begin{frame}[fragile]\frametitle{Calculate Mean}	
% % \begin{lstlisting}
% % def mean(datalist):
	% % total = 0
	% % mean = 0
	% % for item in datalist:
		% % total += item
	% % mean = total / float(len(datalist))
	% % return mean
% % \end{lstlisting}

% \code{Result?}
% \end{frame}

% %%%%%%%%%%%%%%%%%%%%%%%%%%%%%%%%%%%%%%%%%%%%%%%%%%%%%%%%%%
% \begin{frame}[fragile]\frametitle{Calculate Median}	
% % \begin{lstlisting}
% % def median(datalist):
	% % numsort = sorted(datalist)
	% % mid = len(numsort) / 2
	% % median = 1
	% % if len(numsort) % 2 == 0:
		% % median = (numsort[mid - 1] +
	% % else:
		% % median = numsort[(len(numsort) 
	% % return median
% % \end{lstlisting}

% \code{Result?}
% \end{frame}

% %%%%%%%%%%%%%%%%%%%%%%%%%%%%%%%%%%%%%%%%%%%%%%%%%%%%%%%%%%
% \begin{frame}[fragile]\frametitle{Calculate Mode}	
% % \begin{lstlisting}
% % def frequency_distribution(datalist):
	% % freqs = dict()
	% % for item in datalist:
		% % if item not in frees.keys():
			% % freqs[item] = 1
		% % else:
			% % freqs[item] += 1
	% % return freqs

% % def mode(datalist):
	% % d = frequency_distribution(datalist)
	% % most_often = 0
	% % mode = 0
	% % for item in d.keys():
		% % if d[item] > most_often:
			% % most_often = d[item]
		% % mode = item
	% % return mode
% % \end{lstlisting}

% \code{Result?}
% \end{frame}

%%%%%%%%%%%%%%%%%%%%%%%%%%%%%%%%%%%%%%%%%%%%%%%%%%%%%%%%%%%%%%%%%%%%%%%%%%%%%%%%%%
\begin{frame}[fragile]\frametitle{}
\begin{center}
{\Large Descriptive Statistics Exercise}
\end{center}
\end{frame}

%%%%%%%%%%%%%%%%%%%%%%%%%%%%%%%%%%%%%%%%%%%%%%%%%%%%%%%%%%
\begin{frame}[fragile]\frametitle{Exercise}	
\begin{lstlisting}
crater_diameter = [46, 51, 49, 82, 74, 63, 49, 70, 48, 47, 79, 48, 52, 55, 49, 51, 58, 82, 72, 45]
print mean(crater_diameter)
print median(crater_diameter)
print mode(crater_diameter)
\end{lstlisting}

\code{Result? 58.5,51.5,49}
\end{frame}


%%%%%%%%%%%%%%%%%%%%%%%%%%%%%%%%%%%%%%%%%%%%%%%%%%%%%%%%%%%%%%%%%%%%%%%%%%%%%%%%%%
\begin{frame}[fragile]\frametitle{}
\begin{center}
{\Large Measure of Spread}
\end{center}
\end{frame}

%%%%%%%%%%%%%%%%%%%%%%%%%%%%%%%%%%%%%%%%%%%%%%%%%%%%%%%%%%
\begin{frame}[fragile]\frametitle{Measure of Spread}	

In which example (below), the data is spread?

\begin{center}
\includegraphics[width=0.8\linewidth,keepaspectratio]{spread}
\end{center}

How do you quantify the spread?

\end{frame}


%%%%%%%%%%%%%%%%%%%%%%%%%%%%%%%%%%%%%%%%%%%%%%%%%%%%%%%%%%
\begin{frame}[fragile]\frametitle{Measure of Spread}	

\begin{itemize}
\item Range: The largest value minus the smallest value. Suffers from Outliers.
\item Semi-Interquartile range: One half of the difference between the 75th percentile and the 25th percentile. Not affected by Outliers.
\item Standard Deviation:	The square root of the average of the squared deviations from the mean
\end{itemize}

%\begin{center}
%\includegraphics[width=0.8\linewidth,keepaspectratio]{mode}
%\end{center}

\end{frame}



%%%%%%%%%%%%%%%%%%%%%%%%%%%%%%%%%%%%%%%%%%%%%%%%%%%%%%%%%%
\begin{frame}[fragile]\frametitle{Range}	
\begin{itemize}
\item Variation between the smallest and the largest values
\item Can be misleading if most values are concentrated, but a few values are extreme
\end{itemize}
$range(x) = max(x) - min(x)$

\end{frame}

%%%%%%%%%%%%%%%%%%%%%%%%%%%%%%%%%%%%%%%%%%%%%%%%%%%%%%%%%%
\begin{frame}[fragile]\frametitle{Range}	
\begin{center}
\includegraphics[width=0.6\linewidth,keepaspectratio]{da9}
\end{center}
Range = 11 -1 = 10

\begin{itemize}
\item A `quick and easy' indication of variability
\item No indication of dispersion within
\item Unstable, as depends ONLY on Outliers/Extremes
\end{itemize}
\code{Calculate and verify the answer}
\end{frame}

%%%%%%%%%%%%%%%%%%%%%%%%%%%%%%%%%%%%%%%%%%%%%%%%%%%%%%%%%%%%%%%%%%%%%%%%
\begin{frame}[fragile]\frametitle{Range}
Implement my\_range. It cannot be called as ``range'' is already there in Python, so a different name
\begin{lstlisting}
def my_range(datalist):
	:
	return min, max, diff

lst = [9,3,7,2,7,10,23,44,12,42,19,11,22,5,3,4,3,21,3]
min,max,diff = my_range(lst)
print("Range: Min {}, Max {}, Diff {}".format(min,max,diff))
\end{lstlisting}
\end{frame}

%%%%%%%%%%%%%%%%%%%%%%%%%%%%%%%%%%%%%%%%%%%%%%%%%%%%%%%%%%
\begin{frame}[fragile]\frametitle{Range}
\begin{lstlisting}
def my_range(abclist):
    smallest = abclist[0]
    largest = abclist[0]
    range_of_values = 0
    for item in abclist[1:]:
        if item < smallest:
            smallest = item
        elif item > largest:
            largest = item
    range_of_values = largest - smallest
    return smallest, largest, range_of_values
\end{lstlisting}
Range: Min 2, Max 44, Diff 42
\end{frame}

%%%%%%%%%%%%%%%%%%%%%%%%%%%%%%%%%%%%%%%%%%%%%%%%%%%%%%%%%%
\begin{frame}[fragile]\frametitle{Range}
min max functions are available on list
\begin{lstlisting}
def my_range2(x):
	return max(x) - min(x)

diff = my_range2(lst)
print("Range: {}".format(diff))	
\end{lstlisting}
Range: 42
\end{frame}



% %%%%%%%%%%%%%%%%%%%%%%%%%%%%%%%%%%%%%%%%%%%%%%%%%%%%%%%%%%
% \begin{frame}[fragile]\frametitle{Range}	
% % \begin{lstlisting}
% % def range_min_max(abclist):
    % % smallest = abclist[0]
    % % largest = abclist[0]
    % % range_of_values = 0
    % % for item in abclist[1:]:
        % % if item < smallest:
            % % smallest = item
        % % elif item > largest:
            % % largest = item
    % % range_of_values = largest - smallest
    % % return smallest, largest, range_of_values
% % \end{lstlisting}

% %\code{Result? 58.5,51.5,49}
% \end{frame}






%%%%%%%%%%%%%%%%%%%%%%%%%%%%%%%%%%%%%%%%%%%%%%%%%%%%%%%%%%
\begin{frame}[fragile]\frametitle{Percentiles}	
\begin{itemize}
\item For ordered data, percentile is useful.
\item Given an ordinal or continuous attribute x and a number p between 0 and 100, the pth percentile $x_p$ is a value of x such that p\% of the observed values are less than $x_p$.
\item Example: the 75th percentile is the value such that 75\% of all values are less than it.
\end{itemize}

\end{frame}


%%%%%%%%%%%%%%%%%%%%%%%%%%%%%%%%%%%%%%%%%%%%%%%%%%%%%%%%%%
\begin{frame}[fragile]\frametitle{Quantile}	
Quantile can be of any number between 0 to 1. Quartiles are about quarters so they are quantiles of 0.25, 0.5,0.75

Quantiles are cut points in set of data. They can represent the bottom
ten percent of the data or the top 75\% or any \% from 0 to 100.
% \begin{lstlisting}
% def quantile(datalist,num):
	% index = int(num * len(datalist)) # slicing parameter
	% if num > .5:
		% return sorted(datalist)[index:]
	% else:
		% return sorted(datalist)[:index]
% \end{lstlisting}
% quantile(lst,0.10)
% quantile(lst,0.25)
% quantile(lst,0.75)
% quantile(lst,0.90)
% \code{Result?[2],[2,3,3,3],[21,22,23,42,44],[42,44] }
\end{frame}


%%%%%%%%%%%%%%%%%%%%%%%%%%%%%%%%%%%%%%%%%%%%%%%%%%%%%%%%%%%%%%%%%%%%%%%%
\begin{frame}[fragile]\frametitle{Quantile}
Implement quantile
\begin{lstlisting}
def quantile(datalist):
	:
	return q

def interquartile_range(x):
	:
	return iqr
	
lst = [9,3,7,2,7,10,23,44,12,42,19,11,22,5,3,4,3,21,3]
result1 = quantile(lst,0.10)
result2 = quantile(lst,0.25)
result3 = quantile(lst,0.75)
result4 = quantile(lst,0.90)
result5 = interquartile_range(lst)
print("Q10 {}, Q25 {}, Q50 {} Q90 {} IQR {}".format(result1,result2,result3,result4,result5))
\end{lstlisting}
\end{frame}


%%%%%%%%%%%%%%%%%%%%%%%%%%%%%%%%%%%%%%%%%%%%%%%%%%%%%%%%%%
\begin{frame}[fragile]\frametitle{Quantile}
\begin{lstlisting}
def quantile(datalist,num):
    index = int(num * len(datalist)) # slicing parameter
    return sorted(datalist)[index]
    # For values :
    # if num > .5:
    #     return sorted(datalist)[index:]
    # else:
    #     return sorted(datalist)[:index]

def interquartile_range(x):
    return	quantile(x, 0.75) - quantile(x, 0.25)
\end{lstlisting}
\# Q10 [2], Q25 [2, 3, 3, 3], Q50 [21, 22, 23, 42, 44] Q90 [42, 44]
Q10 3, Q25 3, Q50 21 Q90 42 IQR 18
\end{frame}

%%%%%%%%%%%%%%%%%%%%%%%%%%%%%%%%%%%%%%%%%%%%%%%%%%%%%%%%%%
\begin{frame}[fragile]\frametitle{Semi-Interquartile Range}	
\begin{itemize}
\item Quartiles are Quantiles at 25\% and 75\%.
\item Inter Quartile Range (IQR) is between 25\% and 75\%.
\item More resistant to extreme values than the range
\item Does not utilize all the values in the data or set for its computation
\item If small, the values are concentrated near the median
\end{itemize}
\end{frame}

%%%%%%%%%%%%%%%%%%%%%%%%%%%%%%%%%%%%%%%%%%%%%%%%%%%%%%%%%%
\begin{frame}[fragile]\frametitle{Semi-Interquartile Range}	
\begin{itemize}
\item 75th percentile: the value in the date set which is exceeded by 75\% of the total number of items in the set
\item 25 x (0.75) = 18.75
\item 18.75 : rank of the 75th percentile
\item 18th  and 19th items, both  8
\item 75th percentile = 8
\end{itemize}
\begin{center}
\includegraphics[width=0.4\linewidth,keepaspectratio]{da13}
\end{center}
\end{frame}


%%%%%%%%%%%%%%%%%%%%%%%%%%%%%%%%%%%%%%%%%%%%%%%%%%%%%%%%%%
\begin{frame}[fragile]\frametitle{Semi-Interquartile Range}	
\begin{itemize}
\item 25th percentile: the value in the date set which is exceeded by 25\% of the total number of items in the set
\item 25 x (0.25) = 6.25
\item 6.25 : rank of the 25th percentile
\item 6th item = 3 and 7th item = 4
\item 25th percentile = 3 + (0.25)(4-3)
\item 25th percentile = 3.25
\end{itemize}
\begin{center}
\includegraphics[width=0.4\linewidth,keepaspectratio]{da14}
\end{center}
\code{Calculate and verify the answer}
\end{frame}


%%%%%%%%%%%%%%%%%%%%%%%%%%%%%%%%%%%%%%%%%%%%%%%%%%%%%%%%%%
\begin{frame}[fragile]\frametitle{Semi-Interquartile Range}	
\begin{itemize}
\item 75th percentile = 8
\item 25th percentile = 3.25
\item SIQR =1/2 (8 - 3.25)
\item SIQR = 2.375
\item Semi-interquartile range = 2.375
\end{itemize}

\begin{center}
\includegraphics[width=0.8\linewidth,keepaspectratio]{da10}
\end{center}
%\code{Calculate and verify the answer}
\end{frame}



%%%%%%%%%%%%%%%%%%%%%%%%%%%%%%%%%%%%%%%%%%%%%%%%%%%%%%%%%%
\begin{frame}[fragile]\frametitle{Standard Deviation}	
\begin{itemize}
\item How far each value if from the mean
\item Uses all the values in the data for its computation
\item If small, the values are concentrated near the mean.
\item If LARGE, the values are scattered widely about the mean
\item z score: how many std deviations from the mean.
\end{itemize}
\begin{center}
\includegraphics[width=\linewidth,keepaspectratio]{da16}
\end{center}
\end{frame}


%%%%%%%%%%%%%%%%%%%%%%%%%%%%%%%%%%%%%%%%%%%%%%%%%%%%%%%%%%%%%%%%%%%%%%%%
\begin{frame}[fragile]\frametitle{Variance, Standard Deviation}
Implement variance and standard deviation.
\begin{center}
\includegraphics[width=0.5\linewidth,keepaspectratio]{da16}
\end{center}

\begin{lstlisting}
def variance(datalist):
	:
	return v

def std_dev(datalist):
	:
	return s

lst = [9,3,7,2,7,10,23,44,12,42,19,11,22,5,3,4,3,21,3]
result1 = variance(lst)
result2 = std_dev(lst)
print("Variance {}, Std Dev {}".format(result1,result2))
\end{lstlisting}
\end{frame}


% %%%%%%%%%%%%%%%%%%%%%%%%%%%%%%%%%%%%%%%%%%%%%%%%%%%%%%%%%%
% \begin{frame}[fragile]\frametitle{Calculate Deviation}	
% \begin{lstlisting}
% def mean(datalist):
    % total = 0
    % mean = 0
    % for item in datalist:
        % total += item
    % mean = total / float(len(datalist))
    % return mean
 
% def avg_dev(thislist):
    % average = mean(thislist)
    % sum_of_dev = 0
    % avg_dev = 0
    % for item in thislist:
        % sum_of_dev += abs((average - item))
    % avg_dev = sum_of_dev / len(thislist)
    % return avg_dev
% \end{lstlisting}
% \end{frame}


%%%%%%%%%%%%%%%%%%%%%%%%%%%%%%%%%%%%%%%%%%%%%%%%%%%%%%%%%%
\begin{frame}[fragile]\frametitle{Standard Deviation}	
To reparametrize Covariance value between -1 and 1, need to divide by std devs. Empirical Rule for symmetric bell-shaped distributions
\begin{itemize}
\item About 68\% of the values will lie within 1 standard deviation of the mean
\item About 95\% of the values will lie Within 2 standard deviation 
\item About 99.7\% of the values will lie within 3 standard deviation of the mean
\end{itemize}
$variance(x) = s_x^2 = 1/(n-1)\sum (x_i - \bar{x})^2$
$sd(x) = s_x = \sqrt{1/(n-1)\sum (x_i - \bar{x})^2}$
\end{frame}

% %%%%%%%%%%%%%%%%%%%%%%%%%%%%%%%%%%%%%%%%%%%%%%%%%%%%%%%%%%
% \begin{frame}[fragile]\frametitle{Calculate Deviation}	
% \begin{lstlisting}
% def variance(thatlist):
    % average = mean(thatlist)
    % sum_of_sqrt_dev = 0
    % variance = 0
    % for item in thatlist:
        % sum_of_sqrt_dev += (average - item) ** 2
    % variance = sum_of_sqrt_dev / len(thatlist)
    % return variance
 
% def std_dev(anotherlist):
    % std_dev = variance(anotherlist) ** 0.5
    % return std_dev
% \end{lstlisting}
% \end{frame}


%%%%%%%%%%%%%%%%%%%%%%%%%%%%%%%%%%%%%%%%%%%%%%%%%%%%%%%%%%
\begin{frame}[fragile]\frametitle{Variance, Standard Deviation}
\begin{lstlisting}
def de_mean(x):
	"""translate x by subtracting its mean"""
	x_bar = mean(x)
	return	[x_i - x_bar for x_i in x]

def sum_of_squares(diffs):
	sum_of_squares = 0
	for df in diffs:
          sum_of_squares += (df) ** 2
     return sum_of_squares

def variance(x):
	"""assumes x has at	least two elements"""
	n = len(x)
	deviations = de_mean(x)
	return	sum_of_squares(deviations) / (n - 1)
 
def std_dev(anotherlist):
	std_dev = variance(anotherlist) ** 0.5
	return std_dev
\end{lstlisting}
Variance 158.36257309941527, Std Dev 12.584219208970229
\end{frame}


%%%%%%%%%%%%%%%%%%%%%%%%%%%%%%%%%%%%%%%%%%%%%%%%%%%%%%%%%%%
%\begin{frame}[fragile]\frametitle{Standard Deviation}	
%Mean = 6.12
%Standard Deviation = 2.934
%\begin{center}
%\includegraphics[width=0.6\linewidth,keepaspectratio]{da11}
%\end{center}
%\code{Calculate and verify the answer}
%\end{frame}

%%%%%%%%%%%%%%%%%%%%%%%%%%%%%%%%%%%%%%%%%%%%%%%%%%%%%%%%%%
\begin{frame}[fragile]\frametitle{Standard Deviation}	
Standard Deviation = 2.934
\begin{center}
\includegraphics[width=\linewidth,keepaspectratio]{da12}
\end{center}
\end{frame}

%%%%%%%%%%%%%%%%%%%%%%%%%%%%%%%%%%%%%%%%%%%%%%%%%%%%%%%%%%%%%%%%%%%%%%%%%%%%%%%%%%
\begin{frame}[fragile]\frametitle{}
\begin{center}
{\Large Descriptive Statistics Exercise}
\end{center}
\end{frame}

%%%%%%%%%%%%%%%%%%%%%%%%%%%%%%%%%%%%%%%%%%%%%%%%%%%%%%%%%%
\begin{frame}[fragile]\frametitle{Exercise}	
\begin{lstlisting}
crater_diameter = [46, 51, 49, 82, 74, 63, 49, 70, 48, 47, 79, 48, 52, 55, 49, 51, 58, 82, 72, 45]
 
print range_min_max(crater_diameter)
print avg_dev(crater_diameter)
print variance(crater_diameter)
print std_dev(crater_diameter)
\end{lstlisting}
\code{Result?(45,82,37),11.25,161.45,12.7062 }
\end{frame}


%%%%%%%%%%%%%%%%%%%%%%%%%%%%%%%%%%%%%%%%%%%%%%%%%%%%%%%%%%
\begin{frame}[fragile]\frametitle{Exercise}	
Find the mean, median, range and standard deviation for the following set of data:

\lstinline|2.8, 8.7, 0.7, 4.9, 3.4, 2.1 & 4.0|
\end{frame}

%%%%%%%%%%%%%%%%%%%%%%%%%%%%%%%%%%%%%%%%%%%%%%%%%%%%%%%%%%
\begin{frame}[fragile]\frametitle{Exercise}	
Find the mean, median, range and standard deviation for the following set of data:

\begin{center}
\includegraphics[width=0.6\linewidth,keepaspectratio]{da17}
\end{center}
\end{frame}

%
%%%%%%%%%%%%%%%%%%%%%%%%%%%%%%%%%%%%%%%%%%%%%%%%%%%%%%%%%%%%%%%%%%%%%%%%
\begin{frame}[fragile]\frametitle{Difference between Standard Deviation and Standard Error}
\begin{columns}
    \begin{column}[T]{0.6\linewidth}
	\begin{itemize}
	\item For a set of normally distributed observations you have mean and standard deviation.
	\item If you do this for different samples, you get their own respective means and standard deviations.
	\end{itemize}

    \end{column}
    \begin{column}[T]{0.4\linewidth}
      \begin{center}
      \includegraphics[width=\linewidth,keepaspectratio]{statq28}
	  
	  \includegraphics[width=\linewidth,keepaspectratio]{statq29}
	   
	  	\end{center}
    \end{column}

  \end{columns}
  
\tiny{(Ref: StatQuest: Difference between Standard Deviation and Standard Error - Josh Starmer )}
\end{frame}

%%%%%%%%%%%%%%%%%%%%%%%%%%%%%%%%%%%%%%%%%%%%%%%%%%%%%%%%%%%%%%%%%%%%%%%%
\begin{frame}[fragile]\frametitle{Difference between Standard Deviation and Standard Error}

	\begin{itemize}
	\item Plotting those sample means, and sample standard deviations, can form another (meta?) distribution
	\item Standard deviation of this meta distribution is called Standard Error
	\end{itemize}

      \begin{center}
      \includegraphics[width=0.8\linewidth,keepaspectratio]{statq30}
	\end{center}

  
\tiny{(Ref: StatQuest: Difference between Standard Deviation and Standard Error - Josh Starmer )}
\end{frame}

%%%%%%%%%%%%%%%%%%%%%%%%%%%%%%%%%%%%%%%%%%%%%%%%%%%%%%%%%%%%%%%%%%%%%%%%%%%%%%%%%%
\begin{frame}[fragile]\frametitle{}
\begin{center}
{\Large Measure of Asymmetry}
\end{center}
\end{frame}

%%%%%%%%%%%%%%%%%%%%%%%%%%%%%%%%%%%%%%%%%%%%%%%%%%%%%%%%%%
\begin{frame}[fragile]\frametitle{Measures of Shape}	
\begin{itemize}
\item  To have a general idea of its shape, or distribution
\item Helps identifying which descriptive statistic to use
\item  Symmetrical or nonsymmetrical
\item Skewness.
\item Kurtosis.
\end{itemize}
\begin{center}
\includegraphics[width=0.6\linewidth,keepaspectratio]{da18}
\end{center}
\end{frame}

%%%%%%%%%%%%%%%%%%%%%%%%%%%%%%%%%%%%%%%%%%%%%%%%%%%%%%%%%%
\begin{frame}[fragile]\frametitle{Symmetric}	
\begin{itemize}

\item  Uniform.
\item  Normal.
\item  Camel-back.
\item  Bow-tie shaped.
\end{itemize}
\begin{center}
\includegraphics[width=0.6\linewidth,keepaspectratio]{da19}
\end{center}
\end{frame}

%%%%%%%%%%%%%%%%%%%%%%%%%%%%%%%%%%%%%%%%%%%%%%%%%%%%%%%%%%
\begin{frame}[fragile]\frametitle{Skewness}	
Measures the degree to which the values are symmetrically distributed about the center
\begin{center}
\includegraphics[width=0.8\linewidth,keepaspectratio]{skewness}
\end{center}
If the distribution of values is skewed, then the median is a better indicator of the middle, compare to the mean.
\end{frame}

%%%%%%%%%%%%%%%%%%%%%%%%%%%%%%%%%%%%%%%%%%%%%%%%%%%%%%%%%%
\begin{frame}[fragile]\frametitle{Skewness}	
For perfectly symmetrical distribution, like Normal Distribution (middle figure):
\begin{center}
\includegraphics[width=0.8\linewidth,keepaspectratio]{skewness}
\end{center}
\begin{itemize}
\item  Whats the mean?: the middle axis point
\item Whats the mode?: Highest frequency, top most point
\item Whats the median?: Half split of the curve is at the middle.
\end{itemize}
All Points/Axes are same.
\end{frame}

%%%%%%%%%%%%%%%%%%%%%%%%%%%%%%%%%%%%%%%%%%%%%%%%%%%%%%%%%%
\begin{frame}[fragile]\frametitle{Skewness}	
For skewed distribution (left and right figures):
\begin{center}
\includegraphics[width=0.8\linewidth,keepaspectratio]{skewness}
\end{center}
\begin{itemize}
\item Whats the mean?: towards tail, as most of the heavy (+ve or -ve) points are there
\item Whats the mode?: Highest frequency, top most point
\item Whats the median?: somewhere between these two
\end{itemize}
All Points/Axes are different. Sides of Mean and Mode can decide right/left skewness.
\end{frame}

%%%%%%%%%%%%%%%%%%%%%%%%%%%%%%%%%%%%%%%%%%%%%%%%%%%%%%%%%%
\begin{frame}[fragile]\frametitle{Pearson's Skewness Coefficient}	
Karl Pearson coefficient of Skewness 
$sk_p = \frac{3(\mu - median)}{\sigma}$

\begin{itemize}
\item  The direction of skewness is given by the sign.
\item The coefficient compares the sample distribution with a normal distribution. The larger the value, the larger the difference.
\item A value of zero means no skewness at all.
\item A large negative value means the distribution is negatively skewed.
\item A large positive value means the distribution is positively skewed.
\end{itemize}

\end{frame}

%%%%%%%%%%%%%%%%%%%%%%%%%%%%%%%%%%%%%%%%%%%%%%%%%%%%%%%%%%
\begin{frame}[fragile]\frametitle{3rd Moment Skewness Coefficient}	
$sk_t = \frac{\sum (x_i - \mu)^3}{\sigma^3}$

\begin{itemize}
\item If the power would have been 1 (instead of 3) then $\sum (x_i -\mu)$ would have been 0. +ve and -ve will cancel each other.
\item  Odd moments are increased when there is a long tail to the right and decreased when there is a long tail to the left. 
\end{itemize}

\end{frame}

%%%%%%%%%%%%%%%%%%%%%%%%%%%%%%%%%%%%%%%%%%%%%%%%%%%%%%%%%%
\begin{frame}[fragile]\frametitle{Skewness}	
\begin{itemize}
\item  Zero indicates perfect symmetry
\item  Negative value implies left-skewed data
\item Positive value implies right-skewed data.
\end{itemize}
\begin{center}
\includegraphics[width=\linewidth,keepaspectratio]{da20}
\end{center}
\end{frame}

%%%%%%%%%%%%%%%%%%%%%%%%%%%%%%%%%%%%%%%%%%%%%%%%%%%%%%%%%%%%%%%%%%%%%%%%%%%%%%%%%%
\begin{frame}[fragile]\frametitle{}
\begin{center}
{\Large Measure of Skewness}
\end{center}
\end{frame}


%%%%%%%%%%%%%%%%%%%%%%%%%%%%%%%%%%%%%%%%%%%%%%%%%%%%%%%%%%
\begin{frame}[fragile]\frametitle{Kurtosis}	
\begin{itemize}
\item  Measures the degree of flatness (or peakness)
\item Clustered around middle? More peak, more kurtosis value
\item If values spread evenly, flatted, less kurtosis value
\end{itemize}
\begin{center}
\includegraphics[width=\linewidth,keepaspectratio]{da21}
\end{center}
\end{frame}

%%%%%%%%%%%%%%%%%%%%%%%%%%%%%%%%%%%%%%%%%%%%%%%%%%%%%%%%%%
\begin{frame}[fragile]\frametitle{Kurtosis Skewness Coefficient}	
$sk_k = \frac{\sum (x_i - \mu)^4}{\sigma^4}$

\begin{itemize}
\item Since the exponent in the above is 4, the term in the summation will always be positive 
\item Moments of even order are increased when either tail is long. 
\item Kurtosis is a measure of outlier content. High if longer the tails so more the outliers.
\item The third and fourth moments are the smallest examples of these so are used for skewness and kurtosis measures.
\end{itemize}
\end{frame}

% %%%%%%%%%%%%%%%%%%%%%%%%%%%%%%%%%%%%%%%%%%%%%%%%%%%%%%%%%%
% \begin{frame}[fragile]\frametitle{Weighted Average}	
% \begin{center}
% \includegraphics[width=0.7\linewidth,keepaspectratio]{wtavrg}
% \end{center}
% \end{frame}
%%%%%%%%%%%%%%%%%%%%%%%%%%%%%%%%%%%%%%%%%%%%%%%%%%%%%%%%%%%%%%%%%%%%%%%%%%%%%%%%%%
\begin{frame}[fragile]\frametitle{}
\begin{center}
{\Large Bi-variate Analysis}
\end{center}
\end{frame}

%%%%%%%%%%%%%%%%%%%%%%%%%%%%%%%%%%%%%%%%%%%%%%%%%%%%%%%%%%
\begin{frame}[fragile]\frametitle{Bi-variate Analysis}	
\begin{center}
\includegraphics[width=0.7\linewidth,keepaspectratio]{bivar}
\end{center}
\end{frame}

%%%%%%%%%%%%%%%%%%%%%%%%%%%%%%%%%%%%%%%%%%%%%%%%%%%%%%%%%%%%%%%%%%%%%%%%%%%%%%%%%%
\begin{frame}[fragile]\frametitle{}

\begin{center}
{\large Correlations and Covariance}
\end{center}
\end{frame}


%%%%%%%%%%%%%%%%%%%%%%%%%%%%%%%%%%%%%%%%%%%%%%%%%%%%%%%%%%%
\begin{frame}[fragile]\frametitle{Covariance and Correlation}
Both show association between two variables
\begin{itemize}
\item Positive: If one goes up, the other does too and vice versa.
\item Example: Height and weight
\item Not always, but tendency
\item Another example: Temperature and Ice-creame sales
\item Negative: Temperature and sale of woolen clothes
\end{itemize}
\end{frame}

%%%%%%%%%%%%%%%%%%%%%%%%%%%%%%%%%%%%%%%%%%%%%%%%%%%%%%%%%%%
\begin{frame}[fragile]\frametitle{Correlation}
\begin{itemize}
\item Correlation is a value standardized between -1 to 1
\item Relation between two variables is linear, 
\item Directly proportional in case of Positive Corr
\item Inversely proportional in case of Negative Corr
\item The value of corr is the factor of proportionality
\item No correlation, ie no dependence so value = 0
\end{itemize}
\end{frame}

%%%%%%%%%%%%%%%%%%%%%%%%%%%%%%%%%%%%%%%%%%%%%%%%%%%%%%%%%%%
\begin{frame}[fragile]\frametitle{Covariance and Correlation}
\begin{center}
\includegraphics[width=0.55\linewidth,keepaspectratio]{corrplot}
\end{center}
\end{frame}

%%%%%%%%%%%%%%%%%%%%%%%%%%%%%%%%%%%%%%%%%%%%%%%%%%%%%%%%%%%%%%%%%%%%%%%%
\begin{frame}[fragile]\frametitle{Covariance}
Implement covariance, the paired analogue of variance.
The variance measures how a single variable deviates from its mean, covariance measures how two variables vary in tandem from their means.
\begin{center}
\includegraphics[width=0.5\linewidth,keepaspectratio]{corrcov}
\end{center}
\begin{lstlisting}
x = [2, 3, 0, 1, 3]
y = [ 2, 1, 0, 1, 2]
result1 = covariance(x,y)
result2 = correlation(x,y)print("CoVariance {}, Correlation {}".format(result1,result2))
\end{lstlisting}
\end{frame}

%%%%%%%%%%%%%%%%%%%%%%%%%%%%%%%%%%%%%%%%%%%%%%%%%%%%%%%%%%
\begin{frame}[fragile]\frametitle{Covariance}
Covariance is like a dot product and tell how two quantities (centered, meaning subtracted by Mean) are together/similar.
\begin{lstlisting}
def elemwise_multi(v, w):
    """v_1 * w_1 + ... + v_n * w_n"""
    return sum(v_i * w_i for v_i, w_i in zip(v, w))

def covariance(x, y):
	n = len(x)
	return	elemwise_multi(de_mean(x), de_mean(y)) / (n - 1)

\end{lstlisting}
CoVariance 0.8
\end{frame}

%%%%%%%%%%%%%%%%%%%%%%%%%%%%%%%%%%%%%%%%%%%%%%%%%%%%%%%%%%
\begin{frame}[fragile]\frametitle{Correlation}
Covariance is like a dot product normalized by standard deviation.
\begin{lstlisting}
def correlation(x, y):
	stdev_x = std_dev(x)
  	stdev_y = std_dev(y)
	if stdev_x > 0 and stdev_y > 0:
		return	covariance(x, y) / stdev_x / stdev_y
	else:
		return	0 # if	no variation, correlation is zero
\end{lstlisting}
Correlation 0.7333587976225691
\end{frame}

%%%%%%%%%%%%%%%%%%%%%%%%%%%%%%%%%%%%%%%%%%%%%%%%%%%%%%%%%%%%%%%%%%%%%%%%
\begin{frame}[fragile]\frametitle{$R^2$}

	\begin{itemize}
	\item Correlation, the regular `R' has values from -1 to 1 and is good enough to tell you that the two quantitative variables are strongly related.
	\item Why do you need $R^2$ then?
	\item Plain $R$ is not easier to interpret. 
	\item Example: $R=0.7$ is twice as good as $R=0.5$
	\item But its more clear when $R^2 = 0.7$ is 1.4 times as good as $R^2=.5$
	\end{itemize}
  
\tiny{(Ref: StatQuest: R-squared explained - Josh Starmer )}
\end{frame}

%%%%%%%%%%%%%%%%%%%%%%%%%%%%%%%%%%%%%%%%%%%%%%%%%%%%%%%%%%%%%%%%%%%%%%%%
\begin{frame}[fragile]\frametitle{$R^2$}

\begin{columns}
    \begin{column}[T]{0.6\linewidth}
	\begin{itemize}
	\item $R^2$ is used to decide the quality of the linear fitting.
	\item $Var(mean)$ represents the variation of just the mean line, ie black line.
	\item $Var(line)$ represents the variation calculated using he fitted line, ie blue line.
	\item Taking just relative ratio to make $R^2$ in range 0 t 1 and as a percentage.
	\item If the value is 0.81, it means there is 81\% less variation around fitted line than the benchmark black line.
	\item So, if one variable is input (size) and one is output (weight), then we say that 81\% of weight variation is explained by size.
	\end{itemize}

    \end{column}
    \begin{column}[T]{0.4\linewidth}
      \begin{center}
      \includegraphics[width=\linewidth,keepaspectratio]{statq31}
	  
	   
	  	\end{center}
    \end{column}

  \end{columns}
  
	
\tiny{(Ref: StatQuest: R-squared explained - Josh Starmer )}
\end{frame}