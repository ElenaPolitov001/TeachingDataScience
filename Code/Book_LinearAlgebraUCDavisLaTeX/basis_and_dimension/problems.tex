


\begin{enumerate}
\item \begin{enumerate}
\item Draw the collection of all unit vectors in $\Re^2$. 
\item  Let $S_x=\left\{ \colvec{1\\0}, x \right\}$, where $x$ is a unit vector in $\Re^2$.  For which $x$ is $S_x$ a basis of $\Re^2$?
\item Sketch all unit vectors in $\Re^3$.
\item For which $x\in \Re^3$ is $S_x=\left\{ \colvec{1\\0\\0},\colvec{0\\1\\0}, x \right\}$ a basis for $\Re^3$.  
\item Discuss the generalization of the above to $\Re^n$.
\end{enumerate}

\phantomnewpage

\item \label{prob_bit_matrices_basis} Let $B^n$ be the vector space of column vectors with bit entries $0, 1$.  Write down every basis for $B^1$ and $B^2$.  How many bases are there for $B^3$? $B^4$?  Can you make a conjecture for the number of bases for $B^n$?

(Hint: You can build up a basis for $B^n$ by choosing one vector at a time, such that the vector you choose is not in the span of the previous vectors you've chosen.  How many vectors are in the span of any one vector?  Any two vectors?  How many vectors are in the span of any $k$ vectors, 
for $k\leq n$?)

\Videoscriptlink{basis_and_dimension_hint.mp4}{Hint}{scripts_basis_and_dimension_hint}

\phantomnewpage

\item \label{lotsofbases} Suppose that \(V\) is an \(n\)-dimensional vector space.
\begin{enumerate}
\item Show that any \(n\) linearly independent vectors in \(V\) form a basis.

(Hint: Let \(\{w_1, \ldots, w_m\}\) be a collection of \(n\) linearly independent vectors in \(V\), and let \(\{v_1, \ldots, v_n\}\) be a basis for \(V\). Apply the method of Lemma~\ref{mlessn} to these two sets of vectors.)
\item Show that any set of \(n\) vectors in \(V\) which span \(V\) forms a basis for \(V\).

(Hint: Suppose that you have a set of \(n\) vectors which span \(V\) but do not form a basis. What must be true about them? How could you get a basis from this set? Use Corollary~\ref{corsame} to derive a contradiction.)
\end{enumerate}

\phantomnewpage

\item Let $S=\{v_1, \ldots, v_n\}$ be a subset of a vector space $V$.  Show that if every vector $w$ in $V$ can be expressed uniquely as a linear combination of vectors in $S$, then $S$ is a basis of $V$. In other words: suppose that for every vector \(w\) in \(V\), there is exactly one set of constants \(c^1, \ldots, c^n\) so that \(c^1v_1+\cdots+c^nv_n=w\). Show that this means that the set \(S\) is linearly independent and spans \(V\). (This is the converse to theorem~\ref{uniqvec}.)

\phantomnewpage

\item Vectors are objects that you can add together; show that the set of all linear transformations mapping 
$\Re^3\rightarrow \Re$ is itself a vector space.  Find a basis for this vector space.  Do you think your proof could be modified to work for linear transformations $\Re^n\rightarrow \Re$? For $\Re^{\mathbb{N}}\rightarrow \Re^m$? For $\Re^\Re$?

\emph{Hint: Represent $\Re^3$ as column vectors, and argue that a linear transformation $T \colon \Re^3\rightarrow \Re$ is just a row vector. 
%If you are stuck or just curious, see \hyperref[dualspaces]{dual space}\index{Dual vector space}.
}

\phantomnewpage

\item Let $S_n$ denote the vector space of all $n \times n$ symmetric matrices;  $$S_n:=\{M:\mathbb{R}^n\to \mathbb{R}^n ~ |~ M = M^T\}.$$ Let $A_n$ denote the vector space of all $n \times n$ anti-symmetric matrices; 
$$A_n=\{M:\mathbb{R}^n\to \mathbb{R}^n ~ |~ M = -M^T\}.$$
\begin{enumerate}
\item Find a basis for $S_3$.

\item Find a basis for $A_3$.

\item %(Extra Credit) 
Can you find a basis for $S_n$? For $A_n$?

\emph{Hint: Describe it in terms of combinations of the matrices $F^i_j$ which have a~1 in the $i$-th row and the $j$-th column and~0 everywhere else. Note that $\{F^i_j \mid 1 \leq i \leq r, 1 \leq j \leq k\}$ is a basis for $M^r_k$.}
\end{enumerate}

\item Give the matrix of the linear transformation $L$ with respect to the input and output bases $B$ and $B'$ listed below:
\begin{enumerate}
\item $L:V\rightarrow W$ where $B=(v_1,\ldots, v_n)$ is a basis for $V$ and $B'=(L(v_1),\ldots, L(v_n))$ is a basis for $W$.
\item $L:V\rightarrow V$ where $B=B'=(v_1,\ldots,v_n)$ and $L(v_i)=\lambda_i v_i$ for all~$1\leq i\leq n$.
\end{enumerate}


\end{enumerate}

\phantomnewpage