
\subsection*{Hint for Review Problem~\ref{ddd}}

%%%Insert this to get the typewriter font so it looks like a real movie script
{\ttfamily
\fontdimen2\font=0.4em
\fontdimen3\font=0.2em
\fontdimen4\font=0.1em
\fontdimen7\font=0.1em
\hyphenchar\font=`\-


\hypertarget{scripts_eigenvalues_and_eigenvectors_ii_hint}{ We are looking at} the matrix $M$, and a sequence of vectors starting with $v(0) = \colvec{x(0) \\y(0)}$ and defined recursively so that 
\[v(1) = \colvec{x(1) \\y(1)} = M\colvec{x(0) \\y(0)}.\] We first examine the eigenvectors and eigenvalues of 
\[M=
\begin{pmatrix}
3 & 2 \\
2 &3 \\
\end{pmatrix}.\]
We can find the eigenvalues and vectors by solving \[\det (M - \lambda I) = 0\] for $\lambda$.
\[
\det \begin{pmatrix}
3 -\lambda & \mc2 \\
\mc2 & 3 -\lambda \\
\end{pmatrix}
= 0
\]
By computing the determinant and solving for $\lambda$ we can find the eigenvalues $\lambda = 1$ and $5$, and the corresponding eigenvectors. You should do the computations to find these for yourself. 

When we think about the question in part (b) which asks  to find a vector $v(0)$ such that $ v(0) = v(1) =v(2) \ldots$, we must look for a vector that satisfies $v = M v$. What eigenvalue does this correspond to? If you found a $v(0)$ with this property would $c v(0)$ for a scalar $c$ also work? Remember that eigenvectors have to be nonzero, so what if $c=0$?

For part (c) if we tried an eigenvector would we have restrictions on what the eigenvalue should be? Think about what it means to be pointed in the same direction.
} % Closing bracket for font

%\newpage
