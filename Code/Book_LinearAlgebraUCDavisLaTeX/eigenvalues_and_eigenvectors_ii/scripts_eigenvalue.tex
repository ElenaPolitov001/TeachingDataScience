
\subsection*{Eigenvalues}

%%%Insert this to get the typewriter font so it looks like a real movie script
{\ttfamily
\fontdimen2\font=0.4em
\fontdimen3\font=0.2em
\fontdimen4\font=0.1em
\fontdimen7\font=0.1em
\hyphenchar\font=`\-


\hypertarget{scripts_eigenvalues_and_eigenvectors_ii_lecture}{Eigenvalues and eigenvectors}
are extremely important. In this video we review the theory of eigenvalues.
Consider a linear transformation 
$$L:V\longrightarrow V$$
where  $\dim V=n<\infty$. Since $V$ is finite dimensional, we can represent $L$ by a square matrix $M$
by choosing a basis for $V$.

So the eigenvalue equation
\begin{center}
\shabox{$Lv=\lambda v$}
\end{center}
becomes
\begin{center}
\shabox{$M v = \lambda v$,}
\end{center}
where $v$ is a column vector and $M$ is an $n\times n$ matrix (both expressed in whatever basis we chose for $V$).
The scalar $\lambda$ is called an eigenvalue of $M$ and the job of this video is to show you how to find all the eigenvalues of $M$.


The first step is to put all terms on the left hand side of the equation, this gives
$$
(M-\lambda I) v = 0\, .
$$
Notice how we used the identity matrix $I$ in order to get a matrix times $v$ equaling zero. Now here comes a VERY important fact
\begin{center}
\shabox{$N u = 0$ and $u\neq 0$ $\Longleftrightarrow$ $\det N=0$.}
\end{center}
{\it I.e., a square matrix can  have an eigenvector with vanishing eigenvalue if and only if its determinant vanishes!}
Hence
\begin{center}\shabox{$
\det(M-\lambda I) = 0$.}
\end{center}
The quantity on the left (up to a possible minus sign) equals the so-called characteristic polynomial
$$
P_M(\lambda):=\det(\lambda I - M)\, .
$$
It is a polynomial of degree~$n$ in the variable~$\lambda$. To see why, try a simple $2\times 2$ example
$$
\det\left(\begin{pmatrix}a & b\\c & d\end{pmatrix}-\begin{pmatrix}\lambda & 0\\0 & \lambda\end{pmatrix}\right)=
\det\begin{pmatrix}a-\lambda & b\\c & d-\lambda\end{pmatrix}=(a-\lambda)(d-\lambda)-bc\, ,
$$
which is clearly a polynomial of order $2$ in $\lambda$. For the $n\times n$ case, the order $n$ term comes from the product of diagonal matrix elements also.

There is an amazing fact about polynomials called the {\it fundamental theorem of algebra}: they can always be factored over complex numbers. This means that degree $n$ polynomials have $n$ complex roots (counted with multiplicity).
The word can does not mean that explicit formulas for this are known (in fact explicit formulas can only be give for degree four or less). The necessity for complex numbers 
is easily seems from a polynomial like
$$
z^2+1
$$
whose roots would require us to solve $z^2=-1$ which is impossible for real number $z$. However, introducing the imaginary unit $i$ with 
$$
i^2=-1\, ,
$$
we have 
$$
z^2+1=(z-i)(z+i)\, .
$$
Returning to our characteristic polynomial, we call on the fundamental theorem of algebra to write
$$
P_M(\lambda)=(\lambda-\lambda_1)(\lambda-\lambda_2)\cdots (\lambda-\lambda_n)\, .
$$
The roots $\lambda_1$, $\lambda_2$,...,$\lambda_n$ are the eigenvalues of $M$ (or its underlying linear transformation $L$).


} % Closing bracket for font

%\newpage
