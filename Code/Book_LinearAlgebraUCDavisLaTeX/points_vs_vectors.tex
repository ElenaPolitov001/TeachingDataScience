\section{Points Vs. Vectors}
\label{points_vs_vectors}

This is an expanded explanation of \hyperlink{note_points_versus_vectors}{this remark}. People might interchangeably use the term \emph{point} and \emph{vector} in $\mathbb{R}^n$, however these are not quite the same concept. There is a notion of a point in $\mathbb{R}^n$ representing a vector, and while we can do this in a purely formal (mathematical) sense, we really cannot add two points together (there is the related notion of this using \href{http://en.wikipedia.org/wiki/Convex_combination}{convex combinations}, but that is for a different course) or scale a point. We can ``subtract'' two points which gives us the vector between as done which describing \hyperlink{choosing_the_origin}{choosing the origin}, thus if we take any point $P$, we can represent it as a vector (based at the origin $O$) by taking $v = P - O$. Naturally (as we should be able to) we can add vectors to points and get a point back.

To make all of this mathematically (and computationally) rigorous, we ``lift'' $\mathbb{R}^n$ up to $\mathbb{R}^{n+1}$ (sometimes written as $\widetilde{\mathbb{R}}^n$) by stating that all tuples $\widetilde{p} = (p^1, p^2, \dotsc, p^n, 1) \in \mathbb{R}^{n+1}$ correspond to a point $p \in \mathbb{R}^n$ and $\widetilde{v} = (v^1, \dotsc, v^n, 0) \in \mathbb{R}^{n+1}$ correspond to a vector $v \in \mathbb{R}^n$. Note that if the last coordinate $w$ is not 0 or 1, then it does not carry meaning in terms of $\mathbb{R}^n$ but just exists in a formal sense. However we can project it down to a point by scaling by $\frac{1}{w}$, and this concept is highly used in rendering computer graphics.

We also do a similar procedure for all matrices acting on $\mathbb{R}^n$ by the following. Let $A$ be a $k \times n$ matrix, then when we lift, we get the following $(k+1) \times (n+1)$ matrix
\[
\widetilde{A} = \left( \begin{array}{c|c} A & 0 \\ \hline 0 & 1 \end{array} \right).
\]
Note that we keep the last coordinate fixed, so we move points to points and vectors to vectors. We can also act on $\mathbb{R}^n$ in a somewhat non-linear fashion by taking matrices of the form
\[
\left( \begin{array}{c|c} \ast & \ast \\ \hline 0 & 1 \end{array} \right)
\]
and this still fixes the last coordinate. For example we can also represent a translation, which is \hyperlink{non_linear_example}{non-linear} since it moves the origin, in the direction $v = (v^1, v^2, \dotsc, v^n)$ by the following matrix
\[
T_v = \left( \begin{array}{c|c} I_n & v \\ \hline 0 & 1 \end{array} \right).
\]
where $I_n$ is the $n \times n$ identity matrix. Note that this is an \hyperref[inverse_matrix]{invertible matrix} with \hyperref[elementarydeterminants]{determinant} 1, and it is stronger, a translation is what is known an \emph{isometry} on $\mathbb{R}^n$ (note it is not an isometry on $\mathbb{R}^{n+1}$), an operator where $\norm{T_v x} = \norm{x}$ for all vectors $x \in \mathbb{R}^n$.

A good exercises to try are to check that lifting $\mathbb{R}^2$ to $\mathbb{R}^3$ allows us to add, subtract, and scale points and vectors as described and generates nonsense when we can't (i.e. adding two points gives us a 2 in the last coordinate, so it is neither a point nor a vector). Another good exercise is to describe all isometries of $\mathbb{R}^2$. As hint, you can get all of them by rotation about the origin, reflection about a single line, and translation.

\newpage
