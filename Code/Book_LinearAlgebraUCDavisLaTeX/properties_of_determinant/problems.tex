


\begin{enumerate}

\item Find the determinant via expanding by minors.
$$
\begin{pmatrix}
2 & 1&3&7 \\
6& 1&4&4 \\
2 & 1&8&0 \\
1 & 0&2&0 
\end{pmatrix}
$$


\item Even if $M$ is not a square matrix, both $MM^{T}$ and $M^{T}M$ are square. Is it true that $\det (MM^T) =\det (M^T M) $ for all matrices $M$? How about $\tr(MM^T)= \tr (M^TM)$? 


%\item Let $M=
%\begin{pmatrix}
%a & b \\
%c & d
%\end{pmatrix}$.    Show:
%\[
%\det M = \frac{1}{2}(\tr M)^2 - \frac{1}{2}\tr (M^2)
%\]
%
%
%Suppose $M$ is a $3\times 3$ matrix.  Find and verify a similar formula for $\det M$ in terms of $\tr(M^3)$, $\tr(M^2)$, and $\tr M$.
%{\it Hint: make an ansatz for your formula and derive a system of linear equations for any unknowns you introduce by testing it on
%explicit matrices.}
%

\item \label{invsum} Let $\sigma^{-1}$ denote the inverse permutation of $\sigma$. Suppose the function $f:\{1,2,3,4\}\to {\mathbb R}$. Write out explicitly the following two sums:
$$
\sum_\sigma f\big(\sigma(s)\big)\mbox{ and } \sum_\sigma f\big(\sigma^{-1}(s)\big)\, .
$$
What do you observe? Now write a brief explanation why the following equality holds
$$
\sum_\sigma F(\sigma) =\sum_\sigma F(\sigma^{-1})\, ,
$$
where the domain of the function $F$ is the set of all permutations of $n$ objects.

\phantomnewpage

\item Suppose $M=LU$ is an LU decomposition.  Explain how you would efficiently compute $\det M$ in this case. How does this decomposition allow you to easily see if $M$ is invertible? 

\phantomnewpage


\item \label{problem_complexity} In computer science, the \emph{complexity} of an algorithm is (roughly) computed by counting the number of times a given operation is performed.  Suppose adding or subtracting any two numbers takes $a$ seconds, and multiplying two numbers takes $m$ seconds.  Then, for example, computing $2\cdot6-5$ would take $a+m$ seconds.



\begin{enumerate}
\item  How many additions and multiplications does it take to compute the determinant of a general $2\times 2$ matrix?

\item  Write a formula for the number of additions and multiplications it takes to compute the determinant of a general $n\times n$ matrix using the definition of the determinant as a sum over permutations.  Assume that finding and multiplying by the sign of a permutation is free.

\item How many additions and multiplications does it take to compute the determinant of a general $3\times 3$ matrix using expansion by minors?  Assuming $m=2a$, is this faster than computing the determinant from the definition?
\end{enumerate}

\Videoscriptlink{properties_of_matrices_hint.mp4}{Hint}{scripts_properties_of_determinant_hint}

\end{enumerate}

\phantomnewpage