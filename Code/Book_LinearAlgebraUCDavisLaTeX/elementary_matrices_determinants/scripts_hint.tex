
\subsection{\elemMatDetTitle: Hints for \hyperref[prob_inversion_number]{Problem~\ref*{prob_inversion_number}}}

%%%Insert this to get the typewriter font so it looks like a real movie script
{\ttfamily
\fontdimen2\font=0.4em
\fontdimen3\font=0.2em
\fontdimen4\font=0.1em
\fontdimen7\font=0.1em
\hyphenchar\font=`\-


\hypertarget{scripts_elementary_matrices_determinants_hint}{Here we will examine} the \hyperlink{inversion_number}{inversion number} and the effect of the transposition $\tau_{1,2}$ and $\tau_{2,4}$ on the permutation $\nu = [3, 4, 1, 2]$. Recall that the inversion number is basically the number of items out of order. So the inversion number of $\nu$ is $4$ since $3 > 1$ and $4 > 1$ and $3 > 2$ and $4 > 2$. Now we have $\tau_{1,2} \nu = [4, 3, 1, 2]$ by interchanging the first and second entries, and the inversion number is now $5$ since we now also have $4 > 3$. Next we have $\tau_{2,4} \nu = [3, 2, 1, 4]$ whose inversion number is $3$ since $3 > 2 > 1$. Finally we have $\tau_{1,2} \tau_{2,4} \nu = [2, 3, 1, 4]$ and the resulting inversion number is $2$ since $2 > 1$ and $3 > 1$. Notice how when we are applying $\tau_{i,j}$ the parity of the inversion number changes.


} % Closing bracket for font

\newpage
