
\subsection{\linTransTitle: A Linear and A Non-Linear Example}

%%%Insert this to get the typewriter font so it looks like a real movie script
{\ttfamily
\fontdimen2\font=0.4em
\fontdimen3\font=0.2em
\fontdimen4\font=0.1em
\fontdimen7\font=0.1em
\hyphenchar\font=`\-


%%%%put a hypertarget around the opening bit of text
\hypertarget{scripts_linear_transformations_example}{This video} gives
an example of  a linear transformation as well as a transformation that
is not linear. In what happens below remember the properties that 
make a transformation linear:
$$
L(u+v)=L(u)+L(v)\, \qquad\mbox{and} \qquad L(cu)=cL(u)\, .
$$


The first example is the map
$$
L:{\mathbb R}^2\longrightarrow {\mathbb R}^2\, ,
$$
via
$$
\begin{pmatrix}
x\\y
\end{pmatrix}
\mapsto 
\begin{pmatrix}
2&-3\\1&1
\end{pmatrix}
\begin{pmatrix}
x\\y
\end{pmatrix}\, .
$$
Here we focus on the scalar multiplication property $L(cu)=cL(u)$ which needs to hold for {\it any} scalar $c\in {\mathbb R}$
and {\it any} vector $u$. The additive property $L(u+v)=L(u)+L(v)$ is left as a fun exercise.
The calculation looks like this:
$$
L(cu)=L\left(c\begin{pmatrix}
x\\y
\end{pmatrix}\right)
=L\left(\begin{pmatrix}
cx\\cy
\end{pmatrix}\right)
=
\begin{pmatrix}
2cx-3cy\\cx+cy
\end{pmatrix}\qquad\qquad\qquad
$$
$$
\qquad\qquad\qquad\qquad\qquad
=
c\begin{pmatrix}
2x-3y\\x+y
\end{pmatrix}
=cL\left(\begin{pmatrix}
x\\y
\end{pmatrix}\right)=cL(u)\, .
$$
The first equality uses the fact that $u$ is a vector in ${\mathbb R}^2$, next comes the
rule for multiplying a vector by a number, then the rule for the given linear transformation $L$ is used.
The $c$ is then factored out and we recognize that the vector next to $c$ is just our linear transformation again.
This verifies the scalar multiplication property $L(cu)=cL(u)$.

\hypertarget{non_linear_example}{For a non-linear example} lets take the vector space ${\mathbb R}^1= {\mathbb R}$
with
$$L: {\mathbb R}\longrightarrow  {\mathbb R}$$
via
$$ x\mapsto x+1\, .$$
This looks linear because the variable $x$ appears once, but the constant term will be our downfall!
Computing $L(cx)$ we get:
$$
L(cx) = cx +1\, ,
$$
but on the other hand
$$
cL(x) = c(x+1) = cx +c\, .
$$
Now we see the problem, unless we are lucky and $c=1$ the two expressions above are not linear. Since we
need $L(cu)=cL(u)$ for {\it any}~$c$, the game is up! $x\mapsto x+1$ is {\it not} a linear transformation.

%%%%don't forget to close the bracket so the stuff after your file doesn't look like a movie!
}

\newpage
