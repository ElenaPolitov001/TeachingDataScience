
\chapter{\diagSymMatTitle}\label{symmetricmatrices}

Symmetric matrices have many applications.  For example, if we consider the shortest distance between pairs of important cities, we might get a table like the following.
\[
\begin{array}{c|ccc}
 & \text{Davis} & \text{Seattle} 
& \text{San Francisco} \\ \hline
\text{Davis} & 0 & 2000 & 80 \\
\text{Seattle} & 2000 & 0 & 2010 \\
\text{San Francisco} & 80 & 2010 & 0
\end{array}
\]
Encoded as a matrix, we obtain
\[
M=\begin{pmatrix}
\mc{0} & \mc{2000} & \mc{80} \\
\mc{2000} & \mc{0} & \mc{2010} \\
\mc{80} & \mc{2010} & \mc{0}
\end{pmatrix}=M^T.
\]

\begin{definition}
A matrix $M$ is {\bf symmetric}\index{Symmetric matrix} if  $M^T=M.$
\end{definition}

One very nice property of symmetric matrices is that they always have real eigenvalues.  Review exercise~\ref{prob_real_eigenvalues} guides you through the general proof, but below is an example for $2\times 2$ matrices.

\begin{example}
For a general symmetric $2\times 2$ matrix, we have:
\begin{eqnarray*}
P_\lambda \begin{pmatrix} a & b \\ b& d \end{pmatrix}
 &=&
\det\begin{pmatrix}\lambda-a&\mc{-b}\\\mc{-b}&\lambda-d \end{pmatrix}\\[1mm]
&=& (\lambda-a)(\lambda-d)-b^2 \\[2mm]
&=& \lambda^2-(a+d)\lambda-b^2+ad\\[1mm]
\Rightarrow \lambda &=& \frac{a+d}{2}\pm \sqrt{b^2+\left(\frac{a-d}{2}\right)^2}.
\end{eqnarray*}
Notice that the discriminant $4b^2+(a-d)^2$ is always positive, so that the eigenvalues must be real.
\end{example}

Now, suppose a symmetric matrix $M$ has two distinct eigenvalues $\lambda \neq \mu$ and eigenvectors $x$ and $y$;
\[
Mx=\lambda x, \qquad My=\mu y.
\] 
Consider the dot product $x\dotprod y = x^Ty = y^Tx$ and calculate
\begin{eqnarray*}
x^TM y &=& x^T\mu y = \mu x\dotprod y, \text{ and }\\[3mm]
x^TM y &=& (y^TMx)^T \text{ (by transposing a $1\times 1$ matrix)}\\[1mm]
       &=& (y^T\lambda x)^T \\
       &=& (\lambda x\dotprod y)^T \\
             &=& \lambda x\dotprod y.
\end{eqnarray*}
Subtracting these two results tells us that:
\begin{eqnarray*}
0 &=& x^TMy-x^TMy=(\mu-\lambda)\,x\dotprod y.
\end{eqnarray*}
Since $\mu$ and $\lambda$ were assumed to be distinct eigenvalues, $\lambda-\mu$ is non-zero, and so $x\dotprod y=0$.  We have proved the following theorem.

\begin{theorem}
Eigenvectors of a symmetric matrix with distinct eigenvalues are orthogonal.
\end{theorem}

%\begin{center}\href{\webworkurl ReadingHomework23/1/}{Reading homework: problem \ref{symmetricmatrices}.1}\end{center}
\Reading{DiagonalizingSymmetricMatrices}{1}

\begin{example}
The matrix $M=\begin{pmatrix}2&1\\1&2\end{pmatrix}$
has eigenvalues determined by
\[
\det(M-\lambda I)=(2-\lambda)^2-1=0.
\] 
So the eigenvalues of $M$ are $3$ and $1$, and the associated eigenvectors turn out to be 
$\colvec{1\\1}$ and $\colvec{1\\-1}$.  It is easily seen that these eigenvectors are \hyperref[orthogonal]{orthogonal}; 
\[
\colvec{1\\1} \dotprod \colvec{1\\-1}=0.
\]
\end{example}

In \hyperlink{basisorthog}{chapter~\ref{orthonormalbases}} we saw that the matrix $P$ built from any orthonormal basis  $(v_1,\ldots, v_n )$
for ${\mathbb R}^n$ as its columns,
\[
P=\rowvec{v_1 & \cdots & v_n}\, ,
\]
was an orthogonal matrix. This means that 
\[
P^{-1}=P^T, \text{ or } PP^T=I=P^TP.
\]
Moreover, given any (unit) vector $x_1$, one can always find vectors $x_2, \ldots, x_n$ such that $(x_1,\ldots, x_n)$ is an orthonormal basis.  (Such a basis can be obtained using the~\hyperref[GramSchmidt]{Gram-Schmidt procedure}.)

Now suppose $M$ is a symmetric $n\times n$ matrix and $\lambda_1$ is an eigenvalue with eigenvector $x_1$ (this is always the case because every matrix has at least one eigenvalue--see Review Problem~\ref{atleastone}).  
Let $P$ be the square matrix of orthonormal column vectors 
\[
P=\rowvec{x_1 & x_2 & \cdots & x_n},
\]
While $x_1$ is an eigenvector for $M$, the others are not necessarily eigenvectors for $M$.  
Then
\[
MP=\rowvec{\lambda_1 x_1 & Mx_2 & \cdots & Mx_n}.
\]
But $P$ is an orthogonal matrix, so $P^{-1}=P^T$.  Then:
\begin{eqnarray*}
P^{-1}=P^T &=& \ccolvec{x_1^T\\ \vdots \\ x_n^T} \\[1mm]
\Rightarrow P^TMP &=& \begin{pmatrix}
  x_1^T\lambda_1x_1  & * & \cdots & *\\
  x_2^T\lambda_1x_1  & * & \cdots & *\\
  \mc\vdots             &   & & \mc\vdots\\
   x_n^T\lambda_1x_1 & * & \cdots & *\\
  \end{pmatrix}\\[2mm]
&=& \begin{pmatrix}
  \lambda_1  & * & \cdots & *\\
  \mc 0  & * & \cdots & *\\
 \mc \vdots             & *  & & \mc\vdots\\
  \mc 0 & * & \cdots & *\\
  \end{pmatrix}\\[2mm]
&=& \begin{pmatrix}
  \lambda_1  & 0 & \cdots & 0\\
  \mc 0          & & & \\
  \mc\vdots     & & \hat{M} & \\
  \mc0          & & & \\
  \end{pmatrix}\, .\\
\end{eqnarray*}
The last equality follows since $P^TMP$ is symmetric.  The asterisks in the matrix are where ``stuff'' happens; this extra information is denoted by $\hat{M}$ in the final expression.  We know nothing about $\hat{M}$ except that it is an $(n-1)\times (n-1)$ matrix and that it is symmetric.  But then, by finding an (unit) eigenvector for $\hat{M}$, we could repeat this procedure successively.  The end result would be a diagonal matrix with eigenvalues of $M$ on the diagonal. Again, we have proved a theorem: %we also need that every matrix has an eigenvector.

\begin{theorem}
Every symmetric matrix is similar to a diagonal matrix of its eigenvalues.  In other words,
\[
M=M^T \Leftrightarrow M=PDP^T
\]
where $P$ is an orthogonal matrix and $D$ is a diagonal matrix whose entries are the eigenvalues of $M$.
\end{theorem}

%\begin{center}\href{\webworkurl ReadingHomework23/2/}{Reading homework: problem \ref{symmetricmatrices}.2}
\Reading{DiagonalizingSymmetricMatrices}{2}
%\end{center}

To diagonalize a real symmetric matrix, begin by building an orthogonal matrix from an orthonormal basis of eigenvectors, as in the example below. 

\begin{example}
The symmetric matrix 
$$M=\begin{pmatrix}2&1\\1&2\end{pmatrix}\,  ,$$ has eigenvalues $3$ and $1$ with eigenvectors $\colvec{1\\1}$ and $\colvec{1\\-1}$ respectively.  After normalizing these eigenvectors, we  build the orthogonal matrix:
\[
P = \begin{pmatrix}
\frac{1}{\sqrt{2}} & \frac{1}{\sqrt{2}} \\[2mm]
\frac{1}{\sqrt{2}} & \frac{-1}{\sqrt{2}}
\end{pmatrix}\, .
\]
Notice that $P^TP=I$.  Then:
\[
MP = \begin{pmatrix}
\frac{3}{\sqrt{2}} & \frac{1}{\sqrt{2}} \\[2mm]
\frac{3}{\sqrt{2}} & \frac{-1}{\sqrt{2}}
\end{pmatrix} = 
\begin{pmatrix}
\frac{1}{\sqrt{2}} & \frac{1}{\sqrt{2}} \\[2mm]
\frac{1}{\sqrt{2}} & \frac{-1}{\sqrt{2}}
\end{pmatrix} \begin{pmatrix}
3 & 0 \\[2mm]
0 & 1
\end{pmatrix}.
\]
In short, $MP=PD$, so $D=P^TMP$.  Then $D$ is the diagonalized form of $M$ and $P$ the associated change-of-basis matrix from the standard basis to the basis of eigenvectors.
\end{example}

\Videoscriptlink{diagonalizing_symmetric_matrices_3by3_example.mp4}{ $3\times 3$ Example}{scripts_diagonalizing_symmetric_matrices_3by3_example}












%\section*{References}
%Hefferon, Chapter Three, Section V: Change of Basis
%\\
%Beezer, Chapter E, Section PEE, Subsection EHM
%\\
%Beezer, Chapter E, Section SD, Subsection D
%\\
%Wikipedia:
%\begin{itemize}
%\item \href{http://en.wikipedia.org/wiki/Symmetric_matrix}{Symmetric Matrix}
%\item \href{http://en.wikipedia.org/wiki/Diagonalizable_matrix}{Diagonalizable Matrix}
%\item \href{http://en.wikipedia.org/wiki/Similar_matrix}{Similar Matrix}
%\end{itemize}

\section{Review Problems}

{\bf Webwork:} 
\begin{tabular}{|c|c|}
\hline
Reading Problems & 
 \hwrref{DiagonalizingSymmetricMatrices}{1}, 
 \hwrref{DiagonalizingSymmetricMatrices}{2}, 
 \\
Diagonalizing a symmetric matrix &  \hwref{DiagonalizingSymmetricMatrices}{3}, \hwref{DiagonalizingSymmetricMatrices}{4}\\
   \hline
\end{tabular}







\begin{enumerate}

\item Let $D=\begin{pmatrix}
\lambda_1 & \mc0 \\
\mc0 & \lambda_2 \\
\end{pmatrix}$.
\begin{enumerate}
\item Write $D$ in terms of the vectors $e_1$ and $e_2$, and their transposes.
\item Suppose $P=\begin{pmatrix}
a & b \\
c & d \\
\end{pmatrix}$ is invertible.  Show that $D$ is similar to
\[
M=\frac{1}{ad-bc}\begin{pmatrix}
\lambda_1ad-\lambda_2bc & -(\lambda_1-\lambda_2)ab \\[1mm]
(\lambda_1-\lambda_2)cd & -\lambda_1bc + \lambda_2ad
\end{pmatrix}.
\]
\item Suppose the vectors $\rowvec{a,b}$ and $\rowvec{c,d}$ are orthogonal.  What can you say about $M$ in this case? (Hint: think about what \(M^T\) is equal to.)
\end{enumerate}

\phantomnewpage

\item \label{orthogprob} Suppose $S=\{v_1, \ldots, v_n \}$ is an \emph{orthogonal} (not orthonormal) basis for~$\Re^n$.  Then we can write any vector $v$ as $v=\sum_ic^iv_i$ for some constants $c^i$.  Find a formula for the constants $c^i$ in terms of $v$ and the vectors in~$S$.

\Videoscriptlink{orthonormal_bases_hint.mp4}{Hint}{scripts_orthonormal_bases_hint}
\phantomnewpage

\item \label{orthogprojprob} Let $u,v$ be linearly independent vectors in $\Re^3$, and $P=\spa \{ u,v\}$ be the plane spanned by $u$ and $v$.  
\begin{enumerate}
\item Is the vector $v^\bot := v-\frac{u\cdot v}{u\cdot u}u$ in the plane $P$?
\item  What is the (cosine of the) angle between $v^\bot$ and $u$?
\item %Given your solution to the above, 
How can you find a third vector perpendicular to both $u$ and $v^\bot$?
\item  Construct an orthonormal basis for $\Re^3$ from $u$ and $v$.
\item  Test your abstract formul\ae\ starting with 
\[
u=\rowvec{1 , 2 , 0} \text{ and } v=\rowvec{0 , 1 , 1}.
\]
\end{enumerate}

\Videoscriptlink{orthonormal_bases_hint3.mp4}{Hint}{scripts_orthonormal_bases_hint3}

\phantomnewpage



\item Find an orthonormal  basis for $\Re^4$ which includes $(1,1,1,1)$ using the following procedure:\\
\begin{enumerate} 
\item Pick a vector perpendicular to the vector 
$$v_1 =\colvec{1\\1\\1\\1}$$ from the solution set of the matrix equation $$v_1^Tx=0\, .$$ Pick the vector $v_2$ obtained from the standard Gaussian elimination procedure which is the coefficient of $x_2$.
\item Pick a vector perpendicular to both $v_1$ and $v_2$ from the solutions set of the matrix equation $$\colvec{v_1^T\\[1mm]v_2^T}x=0\, .$$ Pick the vector $v_3$ obtained from the standard Gaussian elimination procedure with $x_3$ as the coefficient. 
\item Pick a vector perpendicular to $v_1,v_2,$ and $v_3$ from the solution set of the matrix equation $$\colvec{v_1^T\\[1mm]v_2^T\\[1mm]v_3^T}x=0\, .$$  Pick the vector $v_4$ obtained from the standard Gaussian elimination procedure with $x_3$ as the coefficient. 
\item Normalize the four vectors obtained   above.
\end{enumerate}


\item Use the inner product $$f\cdot g := \int_0^1 f(x)g(x)dx$$  on the vector space $V={\rm span} \{1,x,x^2,x^3\}$ to perform the Gram-Schmidt procedure on the set of vectors $\{1,x,x^2,x^3\}$. 

\item Use the inner product $$f\cdot g := \int_0^{2\pi} f(x)g(x)dx$$  on the vector space $V={\rm span} \{\sin(x),\sin(2x),\sin(3x) \}$ to perform the Gram-Schmidt procedure on the set of vectors $\{\sin(x),\sin(2x),\sin(3x) \}$. \\
Try to build an orthonormal basis for the vector space $$\spa \{ \sin(nx)~| ~n\in \N \}\, .$$
%What do you suspect about the vector space $\spa \{ \sin(nx)~| ~n\in \N \}$?\\
%What do you suspect about the vector space $\spa \{ \sin(ax)~|~ a \in \Re \}$?
\item 
\begin{enumerate}
\item
Show that if $Q$ is an orthogonal $n\times n$ matrix, then $$u\dotprod v = (Qu)\dotprod (Qv)\, ,$$ for any $u,v\in \Re^n$. That is, $Q$ preserves the inner product. 
\item Does $Q$ preserve the outer product? 
\item  If the set of vectors $\{ u_1,\dots,u_n\}$ is orthonormal and $\{ \lambda_1,\cdots,\lambda_n\}$ is a set of numbers, 
then what are the eigenvalues and eigenvectors of the matrix
$M=\sum_{i=1}^n \lambda_i u_i u_i^T$? 
\item How would the eigenvectors and eigenvalues of this matrix change if we replaced  $\{ u_1,\dots,u_n\}$ by $\{ Qu_1,\dots,Q u_n\}$?
\end{enumerate}


\item Carefully write out the Gram-Schmidt procedure for the set of vectors 
$$\left\{ \colvec{1\\1\\1}, \colvec{1\\-1\\1}, \colvec{1\\1\\-1} \right\} \, .$$ Is it possible to rescale the second vector obtained in the procedure to a vector with integer components? 


\item 
\label{basisortho}
\begin{enumerate}
\item Suppose $u$ and $v$ are linearly independent.  Show that $u$ and $v^\perp$ are also linearly independent.  Explain why $\{u, v^\perp\}$ is a basis for $\spa \{u,v\}$.



\Videoscriptlink{gram_schmidt_and_orthogonal_complements_hint.mp4}{Hint}{gram_schmidt_and_orthogonal_complements_hint}

\item Repeat the previous problem, but with three independent vectors $u,v,w$
 where $v^\perp$ and $w^\perp$ are as defined by the Gram-Schmidt procedure. 
\end{enumerate}

\phantomnewpage


\item \label{QRprob} Find the $QR$ factorization of
$$
M=\begin{pmatrix}1&0&\phantom{\!-}2\\-1&2&0\\-1&-2&2
\end{pmatrix}\, .
$$

\phantomnewpage

\item Given any three vectors $u,v,w$, when do $v^\perp$ or $w^\perp$ of the Gram--Schmidt procedure vanish?

\phantomnewpage

\item For $U$ a subspace of $W$, use the subspace theorem to check that $U^\perp$ is a subspace of $W$.

\phantomnewpage


\phantomnewpage

\item %(Extra Credit) 
Let $S_n$ and $A_n$ define the space of $n \times n$ symmetric and anti-symmetric matrices, respectively. These are subspaces of the vector space $M^n_n$ of all $n\times n$ matrices. What is $\dim M^n_n$, $\dim S_n$, and $\dim A_n$? Show that $M^n_n = S_n + A_n$. Define an inner product on square matrices
$$
M\cdot N ={\rm tr} MN\, .
$$
Is $A_n^{\perp}=S_n$? Is $M^n_n = S_n \oplus A_n$?

%\emph{Hint: Note that $\dim S_n = \dim U_n$ where $U_n$ is the vector space of all $n \times n$ upper triangular matrices, and also note that $\dim A_n = \dim \widetilde{U}_n$ where $\widetilde{U}_n$ is the vector space of all strictly $n \times n$ upper triangular matrices (\emph{i.e.} the diagonal entries are all 0).}

\item The vector space $V={\rm span} \{ \sin(t),\sin(2t), \sin(3t) , \sin(3t)\}$ has an inner product: 
$$f\cdot g:=\int _0^{2\pi}f(t)g(t) dt\, .$$ Find the orthogonal compliment to $U={\rm span} \{ \sin(t)+\sin(2t) \}$ in $V$. Express $\sin(t)-\sin(2t)$ as  the sum of vectors from $U$ and $U^\perp$.

\end{enumerate}

\phantomnewpage

%\newpage
