\documentclass{article}
\usepackage{amssymb}
\usepackage{amsmath}
%\usepackage{slide-article-tom}


\ifx\pdfoutput\undefined
     \usepackage[dvips]{graphicx}
\else
     \usepackage[pdftex]{graphicx}
     \pdfcompresslevel9
\fi

\usepackage{hyperref}

%\definecolor{Emerald}{cmyk}{1,0,0.50,0}
\hypersetup{colorlinks,
            linkcolor=blue,
            %pdfpagemode=FullScreen
            pdfpagemode=None
            }



%\usepackage{hyper}
%\usepackage{hthtml}
%\def\hyperref#1#2#3#4{\hturl{#1}}


%\def\@linkcolor{red}
%\def\@linkcolor{blue}
%\def\@anchorcolor{black}
%\def\@citecolor{green}
%\def\@filecolor{cyan}
%\def\@urlcolor{magenta}
%\def\@menucolor{red}
%\def\@pagecolor{red}


\def\pagedone{\newpage}

\def\tthdump#1{#1}

\tthdump{\def\sectionhead#1{\begin{center}{\LARGE\hypertarget{#1}
     {#1}\hyperlink{Our general topics:}{\hfil$\leftarrow$}}\end{center}}}
     
%%tth:\def\sectionhead#1{{\LARGE#1\hypertarget{#1}{#1}}
%%tth:     \special{html: <A NAME="#1"></A><a href="\#Top of file">       Top</a>}}

\tthdump{\def\exercises#1{\begin{center}{\LARGE Exercises: \hypertarget{#1.ex}
     {#1}\hyperlink{Our general topics:}{\hfil$\leftarrow$}}\end{center}}}
     
%%tth:\def\exercises#1{{\LARGE Exercises: #1\hypertarget{#1.ex}{#1.ex}}
%%tth:     \special{html: <A NAME="#1.ex"></A><a href="\#Top of file">       Top</a>}}

\tthdump{\def\quotesection#1{\begin{center}{\LARGE\hypertarget{#1}
     {#1}\hyperlink{The quotes}{\hfil$\twoheadleftarrow$}}\end{center}}}
     
%%tth:\def\quotesection#1{{\LARGE#1\hypertarget{#1}{#1}}
%%tth:     \special{html: <A NAME="#1"></A><a href="\#The quotes">       <-</a>}}

%%tth:\def\makehyperlink#1{\special{html: <a href="\##1">}{\large#1}\special{html: </a>}}

%%tth:\def\makehyperlinkex#1{\ \ \special{html: <a href="\##1">}{\large (ex)}\special{html: </a>}}

%%tth:\def\binom#1#2{\left(\begin{array}{c}#1\\#2\end{array}\right)}

\tthdump{\def\httimes{\times}}
%%tth:\def\httimes{x\ }

\tthdump{\def\mybigtriangledown{\bigtriangledown}}
%%tth:\def\mybigtriangledown{\mathrm{Del}}

\tthdump{\def\htvec#1{\vec{#1}}}
%%tth:\def\htvec#1{{\bf #1}}

\def\myspan{\mathrm{span}}
\def\matrix{\mathrm{matrix}}

\tthdump{\def\myreal{\mathbb{R}}}
%%tth:\def\myreal{\mathbf{R}}
%%\def\myreal{\Re}
\tthdump{\def\mycomplex{\mathbb{C}}}
%%tth:\def\mycomplex{\mathbf{C}}
\tthdump{\def\myfield{\mathbb{F}}}
%%tth:\def\myfield{\mathbf{F}}
\tthdump{\def\myquaternions{\mathbb{H}}}
%%tth:\def\myquaternions{\mathbf{H}}
\tthdump{\def\myrationals{\mathbb{Q}}}
%%tth:\def\myrationals{\mathbf{Q}}
\tthdump{\def\myintegers{\mathbb{Z}}}
%%tth:\def\myintegers{\mathbf{Z}}

%\def\dim\mathrm{dim}
\def\im{\mathrm{im}}
\def\deriv{\mathrm{D}}
\def\trace{\mathrm{trace}}
\def\detm{\mathrm{det}_m}
\def\detv{\mathrm{det}_v}
\def\perm{\mathrm{perm}}
\def\sign{\mathrm{sign}}

% kets and bras etc.
\def\ket#1{|{#1}\rangle}
\def\bra#1{\langle{#1}|}
\def\braket#1#2{\langle{#1}|{#2}\rangle}
\newcommand {\ra} {\rangle}
\newcommand {\la} {\langle}

%\def\sectionhead#1{\begin{center}{\LARGE #1}\end{center}}
%\def\sectionhead#1{\section{#1}}

% defines a 2 element column vector.
\def\col#1#2{\left(\begin{array}{c}#1\\#2\end{array}\right)}
\def\tcol#1#2{(#1, #2)^T}



%Symbols
\def\hilbert{\mathit{H}}
\def\complex{\mathbb{C}}
\def\tensor{\otimes}
\def\lnanda{\land\negthickspace\negthickspace\negthinspace\sim}
\def\lnandb{\land\negthickspace\negthickspace\negmedspace\negthinspace\sim}
\def\lnandc{\land\negthickspace\negthickspace\negmedspace\negthinspace\negthinspace\sim}
\def\lnandd{\land\negthickspace\negthickspace\negmedspace\negthinspace\negthinspace\negthinspace\sim}

	% macros to typeset quantum gates
% \def\Qcontrol{
% \begin{picture}(4,1.5)(0,0.5)
% \put(0,0.75){\line(1,0){4}}
% \put(2,0.75){\circle{0.3}}
% \put(2,0.6){\line(0,-1){1.85}}
% \end{picture}}
% 
% \def\Rcontrol{
% \begin{picture}(4,1.5)(0,0.5)
% \put(0,0.75){\line(1,0){4}}
% \put(2,0.75){\circle{0.3}}
% \end{picture}}
% 
% \def\Rtoggle{
% \begin{picture}(4,1.5)(0,0.5)
% \put(0,0.75){\line(1,0){4}}
% \put(2,0.75){\makebox(0,0){$\times$}}
% \put(2,0.91){\line(0,-1){0.16}}
% \end{picture}}
% 
% \def\Qtoggle{
% \begin{picture}(4,1.5)(0,0.5)
% \put(0,0.75){\line(1,0){4}}
% \put(2,0.75){\makebox(0,0){$\times$}}
% \put(2,0.75){\line(0,-1){2.0}}
% \end{picture}}
% 
% \def\Qpass{
% \begin{picture}(4,1.5)(0,0.5)
% \put(0,0.75){\line(1,0){4}}
% \end{picture}}
% 
% \def\Qcross{
% \begin{picture}(4,1.5)(0,0.5)
% \put(0,0.75){\line(1,0){4}}
% \put(2,0.9){\line(0,-1){2.15}}
% \end{picture}}

\def\Qcontrol{
\begin{picture}(40,15)(0,5)
\put(0,07.5){\line(10,0){40}}
\put(20,07.5){\circle{05}}
\put(20,06){\line(0,-10){18.5}}
\end{picture}}

\def\Rcontrol{
\begin{picture}(40,15)(0,05)
\put(0,07.5){\line(10,0){40}}
\put(20,07.5){\circle{05}}
\end{picture}}

\def\Rtoggle{
\begin{picture}(40,15)(0,05)
\put(0,07.5){\line(10,0){40}}
\put(20,07.5){\makebox(0,0){$\times$}}
\put(20,09.1){\line(0,-10){01.6}}
\end{picture}}

\def\Qtoggle{
\begin{picture}(40,15)(0,05)
\put(0,07.5){\line(10,0){40}}
\put(20,07.5){\makebox(0,0){$\times$}}
\put(20,07.5){\line(0,-10){20}}
\end{picture}}

\def\Qpass{
\begin{picture}(40,15)(0,05)
\put(0,07.5){\line(10,0){40}}
\end{picture}}

\def\Qcross{
\begin{picture}(40,15)(0,05)
\put(0,07.5){\line(10,0){40}}
\put(20,09){\line(0,-10){21.5}}
\end{picture}}


\begin{document}
\raggedright
%%tth:\special{html: <A NAME="Top of file"></A>}

%\pagestyle{myfooters}
%\pagestyle{plain}

\thispagestyle{empty}

%%tth:\special{html:<title> A brief overview of quantum computing</title>}

%Slide 1
\title{{\LARGE\bf A brief overview of quantum computing} \\ or, \\ Can we compute faster in a multiverse?}
\author{Tom Carter
\newline
\newline
\newline
\tthdump{\href
{https://csustan.csustan.edu/\~tom/Lecture-Notes/Quantum-Computing/qc-article.pdf}{https://csustan.csustan.edu/\~\ tom/Lecture-Notes/Quantum-Computing/qc-article.pdf}}
%%tth:\href{https://csustan.csustan.edu/\~tom/Lecture-Notes/Quantum-Computing/qc-article.pdf}{https://csustan.csustan.edu/\~tom/Lecture-Notes/Quantum-Computing/qc-article.pdf}
}

\date{July 16, 2005}

\maketitle

%Slide 2
\sectionhead{Our general topics:}

%%tth:\begin{itemize}
%%tth:\item
\tthdump{\hyperlink{Hilbert space and quantum mechanics}
               {\ \ $\circledcirc$ Hilbert space and quantum mechanics\newline}}
%%tth:\makehyperlink{Hilbert space and quantum mechanics}
%%tth:\item
\tthdump{\hyperlink{Tensor products}
               {\ $\circledcirc$ Tensor products\newline}}
%%tth:\makehyperlink{Tensor products}
%%tth:\item
\tthdump{\hyperlink{Quantum bits (qubits)}
               {\ $\circledcirc$ Quantum bits (qubits)\newline}}
%%tth:\makehyperlink{Quantum bits (qubits)}
%%tth:\item
\tthdump{\hyperlink{Entangled quantum states}
               {\ $\circledcirc$ Entangled quantum states\newline}}
%%tth:\makehyperlink{Entangled quantum states}
%%tth:\item
\tthdump{\hyperlink{Quantum computing}
               {\ $\circledcirc$ Quantum computing\newline}}
%%tth:\makehyperlink{Quantum computing}
%%tth:\item
\tthdump{\hyperlink{Simple quantum gates}
               {\ $\circledcirc$ Simple quantum gates\newline}}
%%tth:\makehyperlink{Simple quantum gates}
%%tth:\item
\tthdump{\hyperlink{Tractability of computation}
               {\ $\circledcirc$ Tractability of computation\newline}}
%%tth:\makehyperlink{Tractability of computation}
%%tth:\item
\tthdump{\hyperlink{Factoring}
               {\ $\circledcirc$ Factoring\newline}}
%%tth:\makehyperlink{Factoring}
%%tth:\item
\tthdump{\hyperlink{Notes on factoring}
               {\ $\circledcirc$ Notes on factoring\newline}}
%%tth:\makehyperlink{Notes on factoring}
%%tth:\item
\tthdump{\hyperlink{Quantum algorithms for satisfiability}
               {\ $\circledcirc$ Quantum algorithms for satisfiability\newline}}
%%tth:\makehyperlink{Quantum algorithms for satisfiability}
%%tth:\item
\tthdump{\hyperlink{Possibilities for physical implementation}
               {\ $\circledcirc$ Possibilities for physical implementation\newline}}
%%tth:\makehyperlink{Possibilities for physical implementation}
%%tth:\item
\tthdump{\hyperlink{Decoherence and error correction}
               {\ $\circledcirc$ Decoherence and error correction\newline}}
%%tth:\makehyperlink{Decoherence and error correction}
%%tth:\item
\tthdump{\hyperlink{Prospects}
               {\ $\circledcirc$ Prospects\newline}}
%%tth:\makehyperlink{Prospects}
%%tth:\item
\tthdump{\hyperlink{References}
               {\ $\circledcirc$ References\newline}}
%%tth:\makehyperlink{References}
%%tth:\item
\tthdump{\hyperlink{On-line references}
               {\ $\circledcirc$ On-line references\newline}}
%%tth:\makehyperlink{On-line references}
%%tth:\end{itemize}

\pagedone

%%% \begin{itemize}
%%% 	\item Hilbert spaces and quantum mechanics
%%% 	\item Tensor products and entangled quantum states
%%% 	\item Quantum bits (qubits), the physics of computation, elements of quantum computing
%%% 	\item Tractability of computation (e.g., factoring and boolean satisfiability)
%%% 	\item Models for quantum computing
%%% 	\item Suggestions for practical implementations of quantum computers
%%% 	\item Problems and prospects
%%% \end{itemize}
%%% %\thepage
%%% \pagedone

\quotesection{The quotes}
%%tth:\begin{itemize}

%%tth:\item
\tthdump{\hyperlink{Twelve men}
               {\ $\circledcirc$ Twelve men\newline}}
%%tth:\makehyperlink{Twelve men}
%%tth:\item
\tthdump{\hyperlink{Magic}
	        {$\circledcirc$ Magic\newline}}
%%tth:\makehyperlink{Magic}
%%tth:\item
\tthdump{\hyperlink{Shocking}
	        {$\circledcirc$ Shocking\newline}}
%%tth:\makehyperlink{Shocking}
%%tth:\item
\tthdump{\hyperlink{Finis}
	        {$\circledcirc$ Finis\newline}}
%%tth:\makehyperlink{Finis}
%%tth:\end{itemize}

\tthdump{\hyperlink{Our general topics:}{\hfil To top $\leftarrow$}}
%%tth:{\special{html: <a href="\#Top of file">       Back to top of file</a>}}

\pagedone



%Slide 3
\quotesection{Twelve men}
%%tth:\begin{quote}
"There was a time when the newspapers said that only 12 men understood the
theory of relativity. I do not believe there ever was such a time. There
might have been a time when only 1 man did, because he was the only guy who
caught on, before he wrote his paper. But after people read the paper a lot
of people understood the theory of relativity in some way or other,
certainly more than 12. On the other hand, I think I can safely say that
nobody understands quantum mechanics"

-Richard Feynman
%%tth:\end{quote}

\pagedone

\sectionhead{Hilbert space and quantum mechanics}

\begin{itemize}
	\item A Hilbert space $\hilbert$ is a complete normed vector space over $\complex$ :
	\begin{enumerate}
		\item $\hilbert$ is a vector space over $\complex$
		\item There is an inner product \newline
			$\braket \cdot \cdot$ : $\hilbert$ x $\hilbert \rightarrow \complex$
			\newline
			which is conjugate linear: \newline
			$\braket v w = \overline{\braket w v} $  \newline
			$\braket {\alpha v} {w} = \alpha \braket v w $
				for $\alpha \in \complex$ \newline
			$\braket {v+w} z = \braket v z + \braket w z $ \newline
			$\braket vv \ge 0$ and $\braket vv = 0$ iff $v = 0$
		\item From the inner product, as usual, we define the norm of a vector: \newline
			$ \Vert v \Vert ^ 2 = \braket v v $
		\item $\hilbert$ is complete with respect to the norm.
	\end{enumerate}
	
\end{itemize}

\pagedone

%Slide 4
\pagedone
\begin{itemize}
	\item We will typically use the bra/ket notation: \newline
		$ \ket v $ is a vector in $\hilbert$, and \newline
		$ \bra v $ is the covector which is the conjugate transpose of v.
	\item This notation also allows us to represent the outer product of a vector and
		covector as $\ket v \bra w$, which, for example, acts on a vector $\ket z$
		as $\ket v \braket w z$.
		 For example, if \{$v_1$,$v_2$\} is an orthonormal basis for a two-dimensional
		  Hilbert space, $\ket {v_1}\bra {v_2}$ is the transformation
			that maps $\ket {v_2}$ to $\ket {v_1}$ and $\ket {v_1}$ to $\tcol 00$ since
			$$\begin{array}{l}
			\ket {v_1}\bra {v_2}\ket {v_2} = \ket {v_1}\braket {v_2}{v_2} = \ket {v_1}\\
			\ket {v_1}\bra {v_2}\ket {v_1} = \ket {v_1}\braket {v_2}{v_1} = 0 \ket {v_1} = \col 00.\\
  			\end{array}$$
			Equivalently, $\ket {v_1}\bra {v_2}$ can be written in matrix form where
			$\ket {v_1} = \tcol 10$, $\bra {v_1} = (1, 0)$, $\ket {v_2} = \tcol 01$,
			 and $\bra {v_2} = (0, 1)$.
			Then 
			$$\ket {v_1}\bra {v_2} = \col 10 (0, 1) = 
					\left(\begin{array}{cc}0&1\\0&0\end{array}\right).$$
	\end{itemize}


%Slide 5
\pagedone
	\begin{itemize}		
	\item A unitary operator $ U : \hilbert \to \hilbert $ is a linear mapping
		whose conjugate transpose is its inverse:  $ U^\dag = U^{-1} $
	\item Unitary operators are norm preserving: \newline
		$ \Vert Uv \Vert ^2 = \bra v U^\dag U \ket v = \braket v v = \Vert v \Vert ^2 $
	\item We will think of a quantum state as a (normalized) vector $ \ket v \in \hilbert $.
		For math folks, we are in effect working in Complex projective space, normalizing
		to 1 so that the probabilities make sense.
	\item The dynamical evolution of a quantum system is expressed as a unitary operator acting on
		the quantum state.

	\item Eigenvalues of a unitary matrix are of the form $ e ^ {i\omega} $ where $\omega$ is a
		real-valued angle.  A unitary operator is in effect a rotation.
\pagedone
	\item Just for reference, a typical expression of Schr\"odinger's equation looks like
		$$\left[-\frac{\hbar^2}{2m_e}\bigtriangledown^2+V(x,y,z)\right]\Psi =
				i\hbar\frac{\partial}{\partial t}
		\Psi$$
		with general solution
		$$\Psi(x,y,z,t)=\sum_{n=0}^\infty c_n\Psi_n(x,y,z)\exp\left(\frac{-iE_nt}{\hbar}\right)$$
		where $\Psi_n(x,y,z)$ is an eigenfunction solution of the time independent Schr\"odinger
		equation with $E_n$ the corresponding eigenvalue.  The inner product, giving a time 
		dependent probability, looks like
		$$P(t) = \int\overline\Psi\Psi dv.$$
%\pagedone
		\item Another way to think of this is that we have to find the Hamiltonian ${\cal H}$
 which generates evolution according to:
         $$i\hbar\frac{\partial}{\partial t}\ket{\Psi(t)}={\cal H }\ket{\Psi(t)}.$$
In our context, we will have to solve for ${\cal H}$ given a desired $U$:
$$
\ket{\Psi_f}=\exp\left(-\frac{i}{\hbar}\int{\cal{H}}dt\right)\ket{\Psi_0}=
 U\ket{\Psi_0}
$$
A solution for 	 ${\cal H}$ always exists,
 as long as the linear operator $U$ is unitary.
 
	\item A measurement consists of applying an operator $O$ to a quantum state $v$.  To
		correspond to a classical observable, $O$ must be {\em Hermitian}, $O^\dag = O$, so
		that all its eigenvalues are real.  If one of its eigenvalues $\lambda$ is associated with
		a single eigenvector $u_\lambda$, then we observe the value $\lambda$ with probability
		$\vert \braket v {u_\lambda} \vert ^ 2$ (i.e., the square of the length of
		the projection along $u_\lambda$).
\end{itemize}
	
		
%Slide 7
\pagedone

\begin{itemize}
	\item In general, if there is more than one eigenvector $u_\lambda$ associated with the
		eigenvalue $\lambda$, we let $P_\lambda$ be the projection operator onto the subspace
		spanned by the eigenvectors, and the probability of observing $\lambda$ when the
		system is in state $v$ is $\Vert P_\lambda v\Vert ^ 2 $.
		
	\item Most projection operators do not commute with each other, and are not invertible.
	  	Therefore, we can expect that the order in which we do measurements will matter, and that
	   	doing a measurement will irreversibly change the state of the quantum system.

\end{itemize}



%Slide 8
\pagedone
\sectionhead{Tensor products}
\begin{itemize}
	\item We can form tensor products of a wide variety of objects.  For example: 
	\begin{enumerate}
		\item The tensor product of an $n$ dimensional vector $u$ and an $m$ dimensional vector $v$ is an $nm$ dimensional vector $u \tensor v$.
		\item If $A$ and $B$ are operators on $n$ and $m$ dimensional vectors, respectively, then $A \tensor B$ is an operator on $nm$ dimensional vectors.
		\item if $\hilbert_1$ and $\hilbert_2$ are Hilbert spaces, then $\hilbert_1 \tensor \hilbert_2$ is also a Hilbert space.  If $\hilbert_1$ and $\hilbert_2$ are finite dimensional with bases $\{u_1, u_2, \ldots u_n\}$ and $\{v_1, v_2, \ldots v_m\}$ respectively, then $\hilbert_1 \tensor \hilbert_2$ has dimension $nm$ with basis $\{u_i \tensor v_j | 1 \le i \le n, 1 \le j \le m\}$.
\end{enumerate}
\end{itemize}


%Slide 9
\pagedone
\begin{itemize}
		\item Tensor products obey a number of nice rules.
		For matrices $A$, $B$, $C$, $D$, $U$, vectors $u$, $v$, $w$, and scalars $a$, $b$, $c$, $d$ the following hold:
\begin{eqnarray*}
(A \tensor B) (C \tensor D) &=& AC\tensor BD\\
(A \tensor B) (u \tensor v) &=& Au\tensor Bv\\
(u+v)\tensor w&=& u\tensor w + v\tensor w\\
u\tensor(v+w)&=& u\tensor v + u\tensor w\\
au\tensor bv &=& ab(u\tensor v)
\end{eqnarray*}

Thus for matrices,
$$\left(\begin{array}{cc}A & B\\C & D\end{array}\right) \tensor U = 
\left(\begin{array}{cc}A \tensor U & B \tensor U\\C \tensor U & D \tensor U\end{array}\right),$$

which specializes for scalars to
$$\left(\begin{array}{cc}a & b\\c & d\end{array}\right) \tensor U = 
\left(\begin{array}{cc}a U & b U\\c U & d U\end{array}\right).$$

\end{itemize}


%Slide 10
\pagedone
\begin{itemize}

	\item The conjugate transpose distributes over tensor products:
$$(A\tensor B)^\dag= A^\dag\tensor B^\dag.$$

	\item The tensor product of several matrices is unitary if and only if each one of the
matrices is unitary up to a constant.  Let $U = A_1\tensor \dots \tensor A_n$.  Then
$U$ is unitary if $A_i^\dag A_i = k_i I$ and $\prod_ik_i = 1$.
\begin{eqnarray*}U^\dag U &=& (A_1^\dag\tensor \dots \tensor A_n^\dag )(A_1\tensor \dots \tensor A_n)\\
&=& A_1^\dag A_1\tensor \dots \tensor A_n^\dag A_n\\
&=& k_1I\tensor \dots \tensor k_nI\\
&=& I\\
\end{eqnarray*}

\end{itemize}


%Slide 11
\pagedone
\begin{itemize}
	\item Note that $\braket {u \tensor v} {w \tensor z} = \braket uw \braket vz$.  
	This implies that $\braket {0 \tensor u} {0 \tensor u} = 0$, and therefore $0 \tensor u$ must
	be the zero vector of the tensor product Hilbert space.
	
	This in turn implies (reminds us?) that the tensor product space is actually the equivalence
	classes in a quotient space.
	
	In particular, if $A$ and $B$ are vector spaces, $F$ is the free abelian group on $A\times B$,
	and $K$ is the subgroup of $F$ generated by all elements of the following forms (where \newline	
	$a, a_1, a_2\in A, b, b_1, b_2\in B, \alpha$ a scalar):
	\begin{enumerate}
		\item $(a_1 + a_2,b) - (a_1,b) - (a_2,b)$
		\item $(a,b_1 + b_2) - (a,b_1) - (a,b_2)$
		\item $(\alpha a,b) - (a,\alpha b)$
	\end{enumerate}
	then $A\tensor B$ is the quotient space $F/K$.
	 
\end{itemize}



%Slide 12
\pagedone


\sectionhead{Quantum bits (qubits)}

\begin{itemize}
	\item A quantum bit, or qubit\index{qubit}, is a unit vector in a two dimensional
complex vector space for which a particular orthonormal basis, denoted by
$\{\ket 0, \ket 1\}$, 
has been fixed. It is important to notice that the basis vector $\ket 0$ is NOT the zero vector of the vector space.
\item For example, the basis
$\ket 0$ and $\ket 1$ may correspond to the $\ket{\uparrow}$ and 
$\ket{\to}$ polarizations of a photon respectively, or to the polarizations
$\ket{\nearrow}$ and $\ket{\nwarrow}$. Or $\ket 0$ and $\ket 1$ could 
correspond to the spin-up and spin-down states ($\ket{\uparrow}$ and $\ket{\downarrow}$) of an electron.
\end{itemize}



%Slide 13
\pagedone
\begin{itemize}

\item For the purposes of quantum computing, the basis states $\ket 0$ and $\ket 1$ 
are taken to encode the classical bit values
$0$ and $1$ respectively. 
Unlike classical bits however, qubits can be in a superposition of
$\ket 0$ and $\ket 1$ such as $a\ket 0 + b\ket 1$
where $a$ and $b$ are complex numbers such that 
$\vert a\vert^2 + \vert b\vert^2 = 1$. If such a superposition is measured with
respect to the basis $\{\ket 0,\ket 1\}$, the probability that the 
measured value is $\ket 0$ is $\vert a\vert ^2$ and the probability that the
measured value is $\ket 1$ is  $\vert b\vert ^2$.

\end{itemize}



%Slide 14
\pagedone
\begin{itemize}

\item Key properties of quantum bits:
 \begin{enumerate}
 \item A qubit can be in a superposition\index{superposition} state of $0$ and $1$.  
 \item Measurement of a qubit in a superposition state will yield
 probabilistic results.
 \item Measurement of a qubit changes the state to the one measured.
 \item There is no transformation which exactly copies all qubits.  This is known as the `no cloning' principle.  Interestingly, it is nonetheless possible to `teleport' a quantum state, but in the process, the original quantum state is destroyed \ldots
 \end{enumerate}

\end{itemize}

\pagedone

\quotesection{Magic}

%%tth:\begin{quote}
"The Universe is full of magical things patiently waiting for our wits to
grow sharper."

-Eden Phillpotts
%%tth:\end{quote}

%%tth:\begin{quote}
"Any sufficiently advanced technology is indistinguishable from magic."

-Arthur C. Clarke
%%tth:\end{quote}


%Slide 15
\pagedone
\sectionhead{Entangled quantum states}
\begin{itemize}
	\item If we have available more than one (physical) qubit, we may be able to {\em entangle} them.  The tensor product of the Hilbert spaces for the individual qubits is the appropriate model for these entangled systems.

	\item For example, if we have two qubits with bases $\{\ket 0_1,\ket 1_1\}$ and
	$\{ \ket 0_2,\ket 1_2\}$ respectively, the tensor product space has the basis 
	$$\{\ket 0_1\tensor\ket 0_2,\ket 0_1\tensor\ket 1_2,\ket 1_1\tensor\ket 0_2,\ket 1_1\tensor\ket 1_2\}.$$  We can (conveniently) denote this basis as
$$\{\ket{00},\ket{01},\ket{10},\ket{11}\}.$$
\end{itemize}



%Slide 16
\pagedone

\begin{itemize}

	\item More generally, if we have $n$ qubits to which we can apply common measurements, we will be working in the $2^n$-dimensional Hilbert space with basis
$$\{\ket{00\ldots00},\ket{00\ldots01},\ldots,\ket{11\ldots10},\ket{11\ldots11}\}$$
\item A typical quantum state for an $n$-qubit system is
$$\sum_{i = 0}^{2^n-1}a_i\ket i$$
where $a_i\in\complex$, $\sum\vert a_i\vert^2=1$, and $\{\ket i\}$ is the basis, with (in our notation) $i$ written as an $n$-bit binary number.

\end{itemize}



%Slide 17
\pagedone

\begin{itemize}
\item A classical (macroscopic) physical 
object broken into pieces can be described and measured as separate components.
An $n$-particle quantum system cannot always be
described in terms of the states of its component pieces. For instance, 
the state \newline$\ket{00}+\ket{11}$ cannot be decomposed into separate states
of each of the two qubits in the form
$$(a_1\ket 0 + b_1\ket 1)\tensor (a_2\ket 0 + b_2\ket 1).$$
This is because 
$$(a_1\ket 0 + b_1\ket 1)\tensor (a_2\ket 0 + b_2\ket 1) = $$
$$  a_1a_2\ket{00} + a_1b_2\ket{01} + b_1a_2\ket{10} + b_1b_2\ket{11}$$ and
$a_1b_2 = 0$ implies that either $a_1a_2 = 0$ or $b_1b_2 = 0$.
States which cannot be decomposed in this way are called entangled states.
These are states that don't have classical counterparts, and
for which our intuition is likely to fail.

\end{itemize}



%Slide 18
\pagedone

\begin{itemize}
\item Particles are entangled if a measurement of one
affects a measurement of the other. For example, the state 
$\frac{1}{\sqrt{2}}(\ket{00}+\ket{11})$ is entangled since the 
probability of measuring the first bit as $\ket 0$ is $1/2$ 
if the second bit has not been measured. However, if the second bit
has been measured, the probability that the first bit is 
measured as $\ket 0$ is either $1$ or $0$, depending on whether the
second bit was measured as $\ket 0$ or $\ket 1$, respectively. On the other hand, the state
$\frac{1}{\sqrt{2}}(\ket{00}+\ket{01})$ is not entangled. Since 
$\frac{1}{\sqrt{2}}(\ket{00}+\ket{01}) = \ket 0\tensor \frac{1}{\sqrt{2}}(\ket{0}+\ket{1})$, any 
measurement of the first bit will yield $\ket 0$ regardless of
measurements of the second bit.  Similarly, the second bit has a 
fifty-fifty chance of being measured as $\ket 0$ regardless of 
measurements of the first bit. Note that entanglement in terms of particle measurement dependence is equivalent to the definition of entangled
states as states that cannot be written as a tensor product of individual
states.
\end{itemize}

\pagedone

\quotesection{Shocking}
%%tth:\begin{quote}
``Anyone who is not shocked by quantum theory has not understood it.''

--Neils Bohr
%%tth:\end{quote}

%%tth:\begin{quote}
``One is led to a new notion of unbroken wholeness which denies the classical
analyzability of the world into separately and independently existing parts.
The inseparable quantum interconnectedness of the whole universe is the
fundamental reality.''

--David Bohm
%%tth:\end{quote}

%%tth:\begin{quote}
``I don't like it, and I'm sorry I ever had anything to do with it.''

--Erwin Schrodinger
%%tth:\end{quote}

%Slide 19
\pagedone
\sectionhead{Quantum computing}
\begin{itemize}

	\item This exponential growth in number of states, together with the ability to subject the entire space to transformations (either unitary dynamical evolution of the system, or a measurement projection into an eigenvector subspace), provides the foundation for quantum computing.
	\item An interesting (apparent) dilemma is the energetic costs/irreversability of classical computing.  Since unitary transformations are invertible, quantum computations (except measurements) will all be reversible.  Most classical boolean operations such as $b_1\land b_2$, $b_1\lor b_2$, and $b_1\lnandb b_2$ are irreversible, and therefore cannot directly be used as basic operations for quantum computers.

\end{itemize}



%Slide 20
\pagedone

\begin{itemize}

	\item The logical nand-gate ($b_1\lnandc b_2$) is sufficient to generate all the traditional boolean functions (e.g., $\sim\negmedspace b \equiv b\ \lnandb b$).  We are likely to end up looking for simple quantum gates that are similarly generic for quantum operations.

	\item In general, if we had enough time, we could simulate any quantum computation with a classical computer.  The real potential value of quantum computers lies in speeding up computations.  The critical questions are:

	\begin{enumerate}
		\item How much can we speed up particular computations?
		\item Can we develop a practical implementation of a particular quantum computation?
		\item Can we build a physical implementation of a quantum computer?  
		\item Does the implementation allow us to carry out useful computations before decoherence interactions with the environment disturb the system too much?
		\item Given the ``no cloning'' principle, can we develop quantum error detection/correction systems? In particular, we can't just take measurements for error control since measurements have irreversible effects on quantum systems.	  
	\end{enumerate}


\end{itemize}


%Slide 21
\pagedone

\sectionhead{Simple quantum gates}

\begin{itemize}
	\item  These are some examples of useful single-qubit quantum state transformations.
Because of linearity, the transformations are fully specified by
their effect on the basis vectors. 
The associated matrix is also shown.
$$\begin{array}{ll}
\begin{array}{lrcl}
I\ :& \ket{0} & \to & \ket{0}\\
  & \ket{1} & \to & \ket{1}\\
\end{array} &
\left(\begin{array}{cc}1 & 0\\ 0 & 1\end{array}\right)\\
\begin{array}{lrcl}
\sigma_x:& \ket{0} & \to & \ket{1}\\
  & \ket{1} & \to & \ket{0}\\
\end{array} &
\left(\begin{array}{cc}0 & 1\\ 1 & 0\end{array}\right)\\
\begin{array}{lrcl}
\sigma_y:& \ket{0} & \to & \ket{1}\\
  & \ket{1} & \to &-\ket{0}\\
\end{array} &
\left(\begin{array}{cc}0 & -1\\ 1 & 0\end{array}\right)\\
\begin{array}{lrcl}
\sigma_z:& \ket{0} & \to & \ket{0}\\
  & \ket{1} & \to & -\ket{1}\\
\end{array} &
\left(\begin{array}{cc}1 & 0\\ 0 & -1\end{array}\right)\\
\end{array}$$ 
$I$ is the identity transformation, $\sigma_x$ is negation, $\sigma_z$ is
a phase shift operation, and $\sigma_y = \sigma_z\sigma_x$ is a combination of both.  
All these gates are unitary.  For example
$$\sigma_y\sigma_y^\dag = \left(\begin{array}{cc}0 & -1\\ 1 & 0\end{array}\right) 
	 \left(\begin{array}{cc}0 & 1\\ -1 & 0\end{array}\right) = I.$$


\end{itemize}



%Slide 22
\pagedone

\begin{itemize}

\item Another important single-bit transformation is the 
Hadamard transformation defined by
$$\begin{array}{lrcl}
H:& \ket{0} & \to & \frac{1}{\sqrt 2}(\ket 0 + \ket 1)\\
  & \ket{1} & \to & \frac{1}{\sqrt 2}(\ket 0 - \ket 1).\\
\end{array}$$
 
Applied to $n$ bits each in the $\ket 0$ state, the transformation generates a superposition of all $2^n$ 
possible states. 
\begin{eqnarray*}
& &(H\otimes H \otimes \dots \otimes H)\ket{00\dots 0}\\
&=&\frac{1}{\sqrt {2^n}}\left((\ket 0+\ket 1)\otimes\dots\otimes(\ket 0+\ket 1
)\right)\\ 
&=&\frac{1}{\sqrt {2^n}}\sum_{x=0}^{2^n-1}\ket x.
\end{eqnarray*}
The transformation acting on $n$ bits is called the 
Walsh or Walsh-Hadamard  
transformation $W$.

\pagedone

\item An important example of a two qubit gate is the controlled-{\sc not} gate, $C_{not}$, which complements the second 
bit if the first bit is $1$ and leaves the bit unchanged otherwise.
$$\begin{array}{ll}\begin{array}{lrcl}
C_{not}:& \ket{00} & \to & \ket{00}\\
        & \ket{01} & \to & \ket{01}\\
        & \ket{10} & \to & \ket{11}\\
        & \ket{11} & \to & \ket{10}\\
\end{array} & \left(\begin{array}{cccc}1 & 0 & 0 & 0\\ 0 & 1 & 0 & 0\\
				       0 & 0 & 0 & 1\\ 0 & 0 & 1 & 0\end{array}\right)\\
\end{array}$$
The transformation $C_{not}$ is unitary since $C_{not}^\dag=C_{not}$ and
$C_{not}C_{not}= I$. 
The $C_{not}$ gate cannot
be decomposed into a tensor product of two 
single-bit transformations.

%pictures of gates
\item It is useful to have graphical representations of quantum state
transformations, especially when several transformations are
combined.
The controlled-{\sc not} gate $C_{not}$ is typically represented by a circuit of the
form
$$\begin{array}{c}\Qcontrol\\ \Rtoggle\\\end{array}$$
The open circle indicates the control bit, and the $\times$ indicates the conditional
negation of the subject bit.  In general there can be multiple control bits.  Some authors use
a solid circle to indicate negative control, in which the subject bit is toggled
when the control bit is $0$.

Similarly, the controlled-controlled-{\sc not}, which  negates the last bit
of three if and only if the first two are both $1$, has the following 
graphical representation.
$$\begin{array}{c}\Qcontrol\\ \Qcontrol\\ \Rtoggle\\\end{array}$$

Single bit operations are graphically represented by 
appropriately labelled boxes as shown.

% \begin{center}
% \begin{picture}(10,5)(0,0)
% \put(4,0){\framebox(2,2){$Z$}}
% \put(1,1){\line(1,0){3}}
% \put(6,1){\line(1,0){3}}
% \put(4,3){\framebox(2,2){$Y$}}
% \put(1,4){\line(1,0){3}}
% \put(6,4){\line(1,0){3}}
% \end{picture}
% \end{center}

\begin{center}
\begin{picture}(100,50)(0,0)
\put(40,0){\framebox(20,20){$Z$}}
\put(10,10){\line(10,0){30}}
\put(60,10){\line(10,0){30}}
\put(40,30){\framebox(20,20){$H$}}
\put(10,40){\line(10,0){30}}
\put(60,40){\line(10,0){30}}
\end{picture}
\end{center}

\item 
The bra/ket notation is useful in defining other
unitary operations. Given two arbitrary 
unitary transformations $U_1$ and $U_2$, the ``conditional'' 
transformation $\ket 0\bra 0 \otimes U_1 + \ket 1\bra 1\otimes U_2$ is
also unitary.  For example,
the controlled-{\sc not} gate can defined by
$$C_{not} = \ket 0\bra 0 \otimes I + \ket 1\bra 1\otimes X.$$

\pagedone

\item
The three-bit controlled-controlled-{\sc not}\index{controlled controlled not} 
gate or Toffoli gate is also an 
instance of this conditional definition:
$$T = \ket 0\bra 0\otimes I \otimes I + \ket 1\bra 1 \otimes C_{not}.$$
 $$\begin{array}{lrcl}
 T:	& \ket{000} & \to & \ket{000}\\
         & \ket{001} & \to & \ket{001}\\
         & \ket{010} & \to & \ket{010}\\
         & \ket{011} & \to & \ket{011}\\
 	& \ket{100} & \to & \ket{100}\\
         & \ket{101} & \to & \ket{101}\\
         & \ket{110} & \to & \ket{111}\\
         & \ket{111} & \to & \ket{110}\\
 \end{array}$$
 
$T$ can be used to construct a complete set of the classical boolean connectives and thus general combinatory circuits since
it can be used to construct the $not$ and $and$ operators in the
following way:
\begin{eqnarray*}
T\ket{1, 1, x} & = & \ket{1, 1, \sim x}\\
T\ket{x, y, 0} & = & \ket{x, y, x \wedge y}\\
\end{eqnarray*}

\end{itemize}

\pagedone


\sectionhead{Tractability of computation}

\begin{itemize}
	\item We can generally categorize computational algorithms according to how the resources needed for execution of the algorithm increase as we increase the size of the input.  Typical resources are time and (storage) space.  In different contexts, we may be interested in worst-case or average-case performance of the algorithm.  For theoretical purposes, we will typically be interested in large input sets \ldots
	\item The hope of quantum computing is that problems that are difficult or impossible for classical computers to solve can be handled by quantum computers.
\pagedone
	\item A standard mechanism for comparing the growth of functions with domain $\mathbb{N}$ is ``big-Oh.''  One way of defining this notion is to associate each function with a set of functions.  We can then compare algorithms by looking at their ``big-Oh'' categories.  
	\item Given a function $f$, we define $O(f)$ by:
$$ g \in O(f) \iff $$
\centerline{there exist $c > 0$ and $N \ge 0$ such that}
\centerline{$\vert g(n) \vert \le c\vert f(n) \vert $ for all $n \ge N$.}
	\item We further define $\theta(f)$ by:
	
\centerline{$g \in \theta(f)$ iff $g \in O(f)$ and $f \in O(g)$.}
\pagedone
	\item In general we will consider the run-time of algorithms in terms of the growth of the number of elementary computer operations as a function of the number of bits in the (encoded) input.  Some important categories -- an algorithm's run-time $f$ is:
	
	\begin{enumerate}
		\item Logarithmic if $f \in \theta(\log(n))$.	
		\item Linear if $f \in \theta(n)$.	
		\item Quadratic if $f \in \theta(n^2)$.	
		\item Polynomial if $f \in \theta(P(n))$ for some polynomial $P(n)$.	
		\item Exponential if $f \in \theta(b^n)$ for some constant $b > 1$.	
		\item Factorial if $f \in \theta(n!)$.
	
	\end{enumerate}
\pagedone
	\item Typically we say that a problem is {\em tractable} if (we know) there exists an algorithm whose run-time is (at worst) polynomial that solves the problem.  Otherwise, we call the problem {\em intractable}.

	\item There are many problems which have the interesting property that if someone (an oracle?) provides you with a solution to the problem, you can tell in polynomial time whether what they provided you actually is a solution.  Problems with this property are called Non-deterministically Polynomial, or NP, problems.  One way to think about this property is to imagine that we have arbitrarily many machines available.  We let each machine work on one possible solution, and whichever machine finds the (a) solution lets us know.
	\item There are some even more interesting NP problems which are universal for the class of NP problems.  These are called NP-complete problems.  A problem $S$ is NP-complete if $S$ is NP and, there exists a polynomial time algorithm that allows us to translate any NP problem into an instance of $S$.  If we could find a polynomial time algorithm to solve a single NP-complete problem, we would then have a polynomial time solution for each NP problem.
\pagedone
	\item Some examples:
	
	\begin{enumerate}
		\item Factoring a number is NP.  First, we recognize that if $M$ is the number we want to factor, then the input size $m$ is approximately $\log(M)$ (that is, the input size is the number of digits in the number).  The elementary school algorithm (try dividing by each number less than $\sqrt M$) has run-time approximately $10^{\frac{m}{2}}$, which is exponential in the number of digits.  On the other hand, if someone hands you two numbers they claim are factors of $M$, you can check by multiplying, which takes on the order of $m^2$ operations.

It is worth noting that there is a polynomial time algorithm to determine whether or not a number is prime, but for composite numbers, this algorithm does not provide a factorization.  Factoring is a particularly important example because various encryption algorithms such as RSA (used in the PGP software) depend for their security on the difficulty of factoring numbers with several hundred digits.
\pagedone
		\item Satisfiability of a boolean expression is NP-complete.  Suppose we have $n$ boolean variables $\{b_1,b_2,\ldots,b_n\}$ (each with the possible values 0 and 1).  We can form a general boolean expression from these variables and their negations:
$$ f(b_1,b_2,\ldots,b_n) = \bigwedge_k(\bigvee_{i,j\le n}(b_i,\sim b_j)).$$
A solution to such a problem is an assignment of values 0 or 1 to each of the $b_i$ such that $f(b_1,b_2,\ldots,b_n) = $1.  There are $2^n$ possible assignments of values.  We can check an individual possible solution in polynomial time, but there are exponentially many possibilities to check.  If we could develop a feasible quantum computation for this problem, we would in some sense resolve the traditional P$\overset{?}=$NP problem \ldots
\pagedone
	\item The discrete Fourier transform of a sequence $\overset{\rightharpoonup}a=\langle a_j\rangle_{j=0}^{q-1}$ is the sequence $\overset{\rightharpoonup}A=\langle A_k\rangle_{k=0}^{q-1}$ where
$$A_k=\frac{1}{\sqrt{q}}\sum_{j=0}^{q-1}a_je^{\frac{2\pi ijk}{q}}$$
One way to think about this is that $\overset{\rightharpoonup}A=F\overset{\rightharpoonup}a$ where the linear transformation $F$ is given by:
$$[F]_{j,k}=\frac{1}{\sqrt{q}}e^{\frac{2\pi ijk}{q}}$$
Note that the inverse of $F$ is $F^\dag$ -- that is,
$$[F^{-1}]_{k,j}=\frac{1}{\sqrt{q}}e^{-\frac{2\pi ijk}{q}}.$$
Suggestively, this says that the discrete Fourier transform is a unitary operation.

The action of this transformation on a vector of dimension $q$ looks as though it would take the $q^2$ operations of matrix multiplication, but there is enough structure that the classical fast Fourier transform algorithm can be done in $q\log(q)$ operations.

The corresponding quantum Fourier transform $U_{QFT}$ with base $2^n$ is defined by
$$U_{QFT}: \ket{x} \mapsto \frac{1}{\sqrt{2^n}}\sum_{c=0}^{2^n-1}e^{\frac{2\pi icx}
{2^n}}\ket{c}.$$
We will see that this can be accomplished in approximately $n^2$ operations rather than $n2^n$.  This is an exponential speed-up of the process.

	\end{enumerate}
\pagedone
\end{itemize}

\sectionhead{Factoring}
\begin{itemize}
	\item The quantum algorithm which has probably done the most for popularizing quantum computation is Shor's factoring algorithm.  As noted above, a fast algorithm for factoring numbers with several hundred digits would invalidate some of the most widely used encryption systems.  Shor's algorithm provides theoretical evidence for such an algorithm, waiting only for a practical physical realization.
	\item The general approach used by Shor is based on a classical probabilistic method for factoring.  The classical algorithm is exponential in the number of digits -- Shor's is (quantum) polynomial.
\pagedone
	\item Outline of Shor's algorithm for factoring a number $M$:
\begin{enumerate}
\item Choose an integer $1 < y < M$ arbitrarily. If $y$ is not relatively
prime to $M$, we've found a factor of $M$. Otherwise apply the
rest of the algorithm.  

	\item Let $n$ be such that $M^2 \leq 2^n < 2M^2$.
We begin with $n$ qubits, each in state $\ket 0$.  We now apply the Walsh transformation $W$ to superpose all states:
$$\sum_{a=0}^{2^n-1}\ket 0 \overset{W}\longmapsto \frac{1}{\sqrt{2^n}}\sum_{a=0}^{2^n-1}\ket a.$$

	\item Apply a transformation which implements raising to powers $\pmod{M}$:
	$$\frac{1}{\sqrt{2^n}}\sum_{a=0}^{2^n-1}\ket a \mapsto \frac{1}{\sqrt{2^n}}\sum_{a=0}^{2^n-1}\ket {a, f(a)}$$
where $f(a) = y^a\pmod{M}$.
 
\item Measure to find a state whose amplitude has the same period as $f$.

%We measure the qubits of the state obtained in the previous step that encode $f(a)$. 
%A random value $u$ is obtained. We don't actually use the
%value $u$; only the
%effect the measurement has on our set of superpositions is of interest. 
%This measurement projects the state space onto the
%subspace compatible with the measured value, so the state after
%measurement is 
%$$C\sum_{a}g(a)\ket{a,u},$$
%for some scale factor $C$ where 
%$$g(a) = \left\{ \begin{array}{ll}
%                1  & \mbox{if $f(a)=u$} \\
%                0  & \mbox{otherwise}
%                \end{array}
%        \right.$$
%Note that the $a$'s that actually appear in the sum, those with $g(a)\ne 0$,
%differ from each other by multiples of the period, and thus $g(a)$ is the
%function we are looking for. If we could just measure
%two successive $a$'s in the sum, we would have the period. 
%Unfortunately the quantum world permits only one measurement.  Also, since we won't be using its %value, we will drop the $\ket u$ part.

\item Apply a quantum Fourier transform to invert the frequency.

%Apply the quantum Fourier transform to the state obtained by the measurement.
%$$\sum_{a}g(a)\ket{a} \overset{QFT}\longmapsto \sum_{c}G(c)\ket{c}$$
%Standard Fourier analysis tells us that when the
%period $r$ of $g(a)$ is a power of two, the result of the quantum Fourier
%transform is $$C'\sum_{j}\rho_j\ket{j\frac{2^n}{ r}}$$
%where $|\rho_j|=1$.
%When the period $r$ does not divide $2^n$, the transform approximates
%the exact case so most of the amplitude
%is attached to integers close to multiples of $\frac{2^n}{r}$.

\item Extract the period, which we expect to be the order of $y\pmod M$.

%Measure the state in the standard basis for quantum computation, 
%and call the result $v$.  In the case 
%where the period happens to be a power of $2$ so that
%the quantum Fourier transform gives exactly multiples of the scaled 
%frequency, the period is easy to extract. In this case,
%$v=j\frac{2^n}{r}$ for some $j$. Most of the time $j$ and $r$ will
%be relatively prime, in which case reducing the fraction $\frac{v}{2^n}$
%to its lowest terms will yield a fraction whose denominator $q$
%is the period $r$. The fact that in general the quantum Fourier
%transform only gives approximately multiples of the scaled
%frequency complicates the extraction of the period from the
%measurement. When the period is not a power of $2$, a good guess for
%the period can be obtained using the continued fraction expansion of
%$\frac{v}{2^n}$.

\item Find a factor of $M$.

When our estimate for the period, $q$, is even, we use the 
Euclidean algorithm
to efficiently check whether either $y^{q/2}+1$ or $y^{q/2}-1$ has 
a non-trivial common factor with $M$.

\item Repeat the algorithm, if necessary.

%Various things could have gone wrong so that this process does 
%not yield a factor of $M$:
%\begin{enumerate}
%\item The value $v$ was not close enough to a multiple of $\frac{2^n}{r}$.
%\item The period $r$ and the multiplier $j$ could have had a common factor
%so that the denominator $q$ was actually a factor of the period, rather than the 
%period itself.
%\item We find $M$ as $M$'s factor.
%\item The period of $f(a) = y^a \pmod M$ is odd.
%\end{enumerate}
%A few repetitions of this algorithm yields a factor of $M$ with
%high probability.
\end{enumerate}

\pagedone

\item Here's another version of the outline of Shor's algorithm for factoring

We begin with 2 $n$-qubit registers.  Apply the Walsh transformation on the first to give a uniform superposition of states: 
$$\ket{\stackrel{\rightarrow}{0}}\tensor\ket{\stackrel{\rightarrow}{0}}
\Rightarrow\frac{1}{\sqrt{Q}}\sum_{l=0}^{Q-1} |l\ra\otimes |\stackrel{\rightarrow}{0}\ra$$
Apply a transformation which computes $y^l~ mod ~N$:
$$\frac{1}{\sqrt{Q}}\sum_{l=0}^{Q-1} |l\ra\otimes|y^l mod N\ra $$
Measure the second register:
$$\frac{1}{\sqrt{A}}\sum_{l=0|y^l=y^{l_0}}^{Q-1} |l\ra\otimes|y^{l_0}\ra=$$
$$\frac{1}{\sqrt{A}}\sum_{j=0}^{A-1} |jr+l_0\ra\otimes|y^{l_0}\ra$$
\pagedone
Apply the quantum Fourier transform over $Z_Q$ on the first register:
$$\frac{1}{\sqrt{Q}}\sum_{k=0}^{Q-1}\left(\frac{1}{\sqrt{A}} 
\sum_{j=0}^{A-1} e^{2\pi i (jr+l_0)k/Q}\right)
|k\ra\otimes|y^{l_0}\ra$$
Measure the first register.  Let $k_1$ be the outcome. Approximate the fraction $\frac{k_1}{Q}$ by a fraction with denominator smaller than $N$.
If the denominator $d$ doesn't satisfy $y^d=1\ \mathrm{mod} ~N$, throw it away, else call the denominator $r_1$.\\
Repeat all previous steps $\rm{poly}(\rm{log}(N))$ times to get $r_1$, $r_2$, \ldots \\
Output the minimal $r$. 

\end{itemize}
\pagedone

\sectionhead{Notes on factoring}
\begin{itemize}
	\item To factor a number $M$, we choose a number $y < M$ with $gcd(y,M) = 1$.  We then find $r$, the order of $y$ in the multiplicative group$\pmod M$.  If $r$ is even, then $(y^{r/2}+1)(y^{r/2}-1)=$ $(y^r-1)\equiv 0\pmod M$.  Then $gcd(y^r-1,M)$ is a non-trivial factor of $M$ except when $r$ is odd or $y^{r/2}\equiv -1\pmod M$.  This procedure produces a non-trivial factor of $M$ with probability at least $1 - 1/2^{k-1}$, where $k$ is the number of distinct odd prime factors of $M$.  If we don't get a factor, we can choose a new $y$ and repeat the process.  By repeating the process, we can make our likelihood of success as close to one as we like. Note that if $M$ is even, finding a factor is easy; if $M$ is a power of a prime, there are other fast classical methods of factoring which we can use on $M$ before we start this process.

\item We want to find the period of the function $f(a)=y^a\pmod M$.  We do that by measuring to find a state whose amplitude has the same period as $f$.

We measure the qubits of the state obtained from encoding $f(a)$. 
A random value $u$ is obtained. We don't actually use the
value $u$; only the
effect the measurement has on our set of superpositions is of interest. 
This measurement projects the state space onto the
subspace compatible with the measured value, so the state after
measurement is 
$$C\sum_{a}g(a)\ket{a,u},$$
for some scale factor $C$ where 
$$g(a) = \left\{ \begin{array}{ll}
                1  & \mbox{if $f(a)=u$} \\
                0  & \mbox{otherwise}
                \end{array}
        \right.$$
Note that the $a$'s that actually appear in the sum, those with $g(a)\ne 0$,
differ from each other by multiples of the period, and thus $g(a)$ is the
function we are looking for. If we could just measure
two successive $a$'s in the sum, we would have the period. 
Unfortunately the quantum world permits only one measurement.

	\item Shor's method uses a quantum version of the Fourier transform to find the period of the function $y^a\pmod M$.
We apply the quantum Fourier transform to the state obtained by the measurement.
$$\sum_{a}g(a)\ket{a} \overset{QFT}\longmapsto \sum_{c}G(c)\ket{c}$$
Standard Fourier analysis tells us that when the
period $r$ of $g(a)$ is a power of two, the result of the quantum Fourier
transform is $$C'\sum_{j}\rho_j\ket{j\frac{2^n}{ r}}$$
where $|\rho_j|=1$.
When the period $r$ does not divide $2^n$, the transform approximates
the exact case so most of the amplitude
is attached to integers close to multiples of $\frac{2^n}{r}$.

%The quantum Fourier transform operates on the amplitude of the 
%quantum state, by sending
%$$\sum_{a}f(a)\ket{a} \to \sum_{c}F(c)\ket{c}$$ where $F(c)$ is
%the discrete Fourier transform of $f(a)$, and $a$ and $c$ both range
%over the binary representations for the integers between $0$ and $N-1$. 
%If the state
%were measured after the Fourier transform was performed, 
%the probability that the result was
%$\ket c$ would be $|F(c)|^2$.

%Fourier transforms in general map
%from the time domain to the frequency domain.
%In our case, the Fourier transform maps a function of period
%$r$ to a function which has non-zero values only at multiples
%of the frequency
%$\frac{1}{r}$. 
%Thus, when the state is measured, the result would be a multiple
%of $\frac{N}{r}$, say $j\frac{N}{r}$.

	\item In order for Shor's factoring algorithm to be a polynomial algorithm, the quantum
Fourier transform must be efficiently computable. Shor
developed a quantum Fourier transform construction with base $2^n$ using
only $\frac{n(n+1)}{2}$ gates. The construction makes use of two types of
gates. One is a gate to perform the Hadamard transformation $H$. 
We will denote by $H_j$ the Hadamard transformation applied to the $j$th
bit. The other type of gate performs transformations of the form
$$S_{j,k} = \left(\begin{array}{cccc}1&0&0&0\\0&1&0&0\\0&0&1&0\\0&0&0&e^{i\theta_{k-j}}\end{array}\right)$$
where $\theta_{k-j}={\pi}/{2^{k-j}}$, which acts on the $k$th element,
depending on the value of the $j$th element.  Think of this as acting on the basis $\{\ket{00},\ket{01},\ket{10},\ket{11}\}$ \ldots

The quantum Fourier transform is given by 
$$H_0S_{0,1}\dots S_{0,n-1}H_1\dots$$
$$H_{n-3}S_{n-3,n-2}S_{n-3,n-1}H_{n-2}S_{n-2,n-1}H_{n-1}.$$
This actually produces the reverse of the Fourier transform, so it typically will be followed by a bit reversal transformation.
\pagedone
	\item There is a second piece of Shor's algorithm which must be accomplished in polynomial time.  We need to extract (using the QFT) the period of the function $a \mapsto (y^a)\pmod M$.
%  We therefore need to first form a uniform superposition of all basis states (which can be done via %the Walsh transform):
%$$\sum_{a=0}^{2^n-1}\ket 0 \overset W\mapsto \frac{1}{2^{n/2}}\sum_{a=0}^{2^n-1}\ket a$$
We must transform as:
$$\frac{1}{\sqrt{2^n}}\sum_{a=0}^{2^n-1}\ket a \mapsto \frac{1}{\sqrt{2^n}}\sum_{a=0}^{2^n-1}\ket{a,y^a(mod M)}$$

We want to develop a transformation which computes the function $f_{y,M}(a)=y^a \pmod M$.  First,
we write $y^a$ as $y^a=y^{2^0a_0}\cdot y^{2^1 a_1} \cdot
\ldots y^{2^{m-1} a_{m-1}}$, where $m$ is the number of digits in the binary expansion of $M$.  Then, modular exponentiation can be
computed by initializing the result register to $\ket 1$, and
successively effecting $m$ multiplications by $y^{2^i}\pmod M$,
depending on the value of the qubit $\ket{a_i}$.
\pagedone
If $a_i=1$, we want
the operation
$$\ket{y^{2^0a_0+\ldots 2^{i-1}a_{i-1}},0} \mapsto $$
$$\ket{y^{2^0a_0+\ldots 2^{i-1}a_{i-1}},y^{2^0a_0+\ldots
    2^{i-1}a_{i-1}}\cdot y^{2^i}}$$
to be performed; otherwise, when
$a_i=0$ we just require 
$$\ket{y^{2^0a_0+\ldots 2^{i-1}a_{i-1}},0} \mapsto $$
$$\ket{y^{2^0a_0+\ldots 2^{i-1}a_{i-1}},y^{2^0a_0+\ldots
    2^{i-1}a_{i-1}}}.$$
Note that in both cases the result can be written as
$\ket{y^{2^0a_0+\ldots 2^{i-1}a_{i-1}},y^{2^0a_0+\ldots 2^ia_i}}$.
%To avoid an accumulation of intermediate data in the memory of the
%quantum computer, a particular care should be taken to erase the
%partial information generated. This is done by running backwards a controlled multiplication
%network with the value $y^{-2^i} \bmod N$. This quantity can be
%efficiently precomputed in a classical way. shows the network for a complete modular
%exponentiation. It is made out of $m$ stages; each stage performs the
%following sequence of operations:
%\begin{equation}
%\begin{array}{ll}
% |y^{2^0a_0+\ldots 2^{i-1}a_{i-1}},0\rangle 
%\rightarrow &\hspace{2cm} \mbox{\small (multiplication)}\\ \nonumber
%\hspace{2cm}
% |y^{2^0a_0+\ldots 2^{i-1}a_{i-1}}, y^{2^0a_0+\ldots 2^ia_i}\rangle 
%\rightarrow &\hspace{2cm}  \mbox{\small (swapping)}\\ \nonumber
%\hspace{2cm} 
%| y^{2^0a_0+\ldots 2^ia_i},y^{2^0a_0+\ldots 2^{i-1}a_{i-1}}\rangle 
%\rightarrow &\hspace{2cm}  \mbox{\small (resetting)}\\ \nonumber
%\hspace{2cm}
%| y^{2^0a_0+\ldots 2^ia_i},0\rangle
%\end{array}
%\end{equation}

\pagedone

\item To extract the period, we measure the state in the standard basis for quantum computation, 
and call the result $v$.  In the case 
where the period happens to be a power of $2$ so that
the quantum Fourier transform gives exactly multiples of the scaled 
frequency, the period is easy to extract. In this case,
$v=j\frac{2^n}{r}$ for some $j$. Most of the time $j$ and $r$ will
be relatively prime, in which case reducing the fraction $\frac{v}{2^n}$
to its lowest terms will yield a fraction whose denominator $q$
is the period $r$. The fact that in general the quantum Fourier
transform only gives approximately multiples of the scaled
frequency complicates the extraction of the period from the
measurement. When the period is not a power of $2$, a good guess for
the period can be obtained using the continued fraction expansion of
$\frac{v}{2^n}$.
 \item Various things could have gone wrong so that this process does 
not yield a factor of $M$:
\begin{enumerate}
\item The value $v$ was not close enough to a multiple of $\frac{2^n}{r}$.
\item The period $r$ and the multiplier $j$ could have had a common factor
so that the denominator $q$ was actually a factor of the period, rather than the 
period itself.
\item We find $M$ as $M$'s factor.
\item The period of $f(a) = y^a \pmod M$ is odd.
\end{enumerate}
A few repetitions of this algorithm yields a factor of $M$ with
high probability.

\end{itemize}
\pagedone


\sectionhead{Quantum algorithms for satisfiability}
\begin{itemize}
\item Various approaches have been developed which provide hope that the NP-complete boolean satisfiability problem can be solved in polynomial time.  It is not clear that any of the published techniques will be effective.  Some of the methods seem to require either exponential space/hardware (e.g., bulk spin resonance via NMR) or exponential measurement precision.  This is a very active area of current research.

One algorithm which has been well analyzed is Grover's search algorithm.  It gives quadratic speedup of solving satisfiability, but in its general form can do no better than that, and hence does not give the exponential speedup needed to get $P=NP$.

\item Following is an outline of Grover's general search algorithm.  If $P(x)$ is a boolean function for $0 \le x < N$, classical search algorithms take on the order of $\frac{N}{2}$ operations to find an item $x_0$ for which $P(x_0) = 1$.  Grover's algorithm takes on the order of $\sqrt{N}$ operations.  Grover's algorithm has been shown to be optimal for the general search problem.  This is not an exponential speedup, but it is an improvement over the classical algorithms.  However, problems such as satisfiability have additional structure which can make them easier to solve.

\item Grover's algorithm consists of the following steps:
\begin{enumerate}
\item Let $n$ be such that $2^n \geq N$, and prepare a register containing a superposition of all $x_i\in [0\dots 2^n-1]$.
\item Apply a unitary transformation that computes $P(x_i)$ on this register:
$$U_P: \frac{1}{\sqrt{2^n}}\sum_{x=0}^{n-1}\ket{x, 0} \to 
	\frac{1}{\sqrt{2^n}}\sum_{x=0}^{n-1}\ket{x, P(x)}.$$
	For any $x_0$ such that $P(x_0)$ is true, $\ket{x_0, 1}$ will be part of the 
resulting superposition, but since its 
amplitude is $\frac{1}{\sqrt{2^n}}$, 
the probability that a measurement produces $x_0$ is only $2^{-n}$. 
\item Change amplitude $a_j$ to $-a_j$ for all $x_j$ such that $P(x_j)=1$.
\item Apply inversion about the average to increase 
amplitude of $x_j$ with $P(x_j)=1$ and decrease other amplitudes. 
\item Repeat steps 2 through 4 $\frac{\pi}{4}\sqrt{2^n}$ times.
\item Measure the last qubit of the quantum state, representing $P(x)$.  
Because of the amplitude 
change, there is a high probability that the result will be $1$.  If this is the
case, the measurement has projected the state
onto the subspace $\frac{1}{\sqrt{2^k}}\sum_{i=1}^{k}\ket{x_i, 1}$ where $k$ is the 
number of solutions.  Further measurement of the remaining bits will provide one of these solutions.
\end{enumerate}
\item An interesting feature of this algorithm is that repeating steps 2 through 4 a total of $\frac{\pi}{4}\sqrt{2^n}$ times is optimal.  In particular, if the process is repeated more times, the probability of a successful measurement decreases back toward zero \ldots
\pagedone

\item An alternative approach builds the unitary transformation for the boolean expression, applies the transformation to molecules in solution, then uses bulk spin resonance analysis via NMR to measure the expected values of the spins, and thus solves the satisfiability problem.  However, realistic implementations seem to require an exponentially large NMR sample. 
\item The general estimate is that if $n$ is the number of qubits, and $M$ is the number of molecules in the sample, then $n2^n < M$.  For a typical sample, $M \approx 10^{24} \approx 2^{80}$ and so $n < 74$.  For an upper limit, a reasonable estimate of the number of elementary particles in the accessible universe is $\approx 10^{80} \approx 2^{265}$ which corresponds with $\approx 256$ qubits \ldots
\end{itemize}
\pagedone
%

\sectionhead{Possibilities for physical implementation}
\begin{itemize}
\item Implementations of quantum computers will be a difficult experimental challenge. Quantum computer equipment must satisfy a variety of constraints: (1) the qubits must interact
very weakly with their environment to minimize decoherence and preserve their superpositions, (2) the qubits must interact very strongly with one another for the logic gates and information transfer to be effective, and (3) the initialization and readout of states must be efficient. Not many known physical systems can satisfy these requirements, although there are some possibilities.
\pagedone
\item A collection of charged ions held in an electromagnetic trap is one possibility. Each atom stores a qubit of information in a pair of internal electron levels. Each atom's levels are protected from environmental influences. Scaling to larger numbers of qubits should be able to be done by adding more atoms to the collection.  When appropriate laser radiation is applied to the atoms, only one of the two internal states fluoresces.  This allows detection of the state of each qubit. The atoms are coupled by virtue of their mutual Coulomb repulsion.  Experimental development of trapped ion quantum computation is at the level of single-ion and two-ion qubit systems. Extensions to larger
numbers of trapped ions has been difficult, but there do not seem to be impossible theoretical limits to scaling. 
\pagedone
\item Another system which could be developed into a quantum computer is a single molecule, in which nuclear spins of individual atoms represent qubits. This is the basis of the NMR technique mentioned above. The spins can be manipulated, initialized, and measured. For example, the carbon and hydrogen nuclei in a chloroform
molecule can be used to represent two qubits. Applying a radio-frequency pulse to the hydrogen nucleus addresses that qubit and causes it to rotate from a $\ket 0$ state to a superposition $\frac{1}{\sqrt{2}}(\ket 0 + \ket 1)$ state. Interactions
through chemical bonds allow multiple-qubit logic to be performed.  However, it is difficult to find molecules with more than 10 spins in them and with a large coupling constant between every pair of spins \ldots
\end{itemize}

\pagedone

\sectionhead{Decoherence and error correction}
\begin{itemize}
\item Decoherence in general arises from interactions with the environment, which typically has the effect of measuring the system and thus collapsing a quantum computation.  In addition, we have to be careful about leaving temporary qubits floating around.  We can expect them to be entangled with the rest of the system, and thus an observation of the ``dust'' left behind by intermediate computations could effect a measurement of the system, invalidating later stages.  Thus, one emphasis in research on quantum computation has been on how to efficiently avoid leaving any garbage floating about.

\pagedone

\item As noted above, error detection/correction is difficult in the quantum environment since we cannot reliably clone an arbitrary qubit.  Further, any intermediate measurement of the system for error control is likely to invalidate our computation.  There are, however, approaches using polarization encoding schemes for error control.
\end{itemize}
\pagedone

\sectionhead{Prospects}
\begin{itemize}

\item The history of quantum mechanical algorithms is very brief.  There are two main approaches that have resulted in descriptions of efficient quantum computational algorithms: the first is estimates of periodicity that resulted in the factorization algorithm, and the second is amplitude amplification
that has led to Grover's quantum search and related algorithms.
\item Over the past 70 or 80 years, physicists have observed various quantum mechanical phenomena that lead to puzzling and even
apparently paradoxical results. Most of these still remain to be investigated from a quantum computing perspective.
\pagedone
\item One interesting question is how slight difference in the laws of quantum mechanics might affect these issues. Some interesting work by Abrams et al. shows that if there was even the slightest amount of nonlinearity in quantum mechanics, it would be possible to modify the amplitude amplification scheme of Grover's quantum search algorithm to obtain an efficient algorithm solving the NP-complete satisfiability problem. However, most people
believe that such nonlinearity probably does not exist because it would also lead to faster-than-light communication, noncausality, and other violations of fundamental physical principles \ldots


\end{itemize}
\pagedone

\quotesection{Finis}
%%tth:\begin{quote}
``Nature uses only the longest threads to weave her patterns, so that each
small piece of her fabric reveals the organization of the entire tapestry.''

-- Richard Feynman
%%tth:\end{quote}
\pagedone
\pagedone

\footnotesize
\bibliographystyle{plain}
 
\tthdump{\hypertarget{References}{}\hyperlink{Our general topics:}{\hfil To top $\leftarrow$}}
%%tth:{\special{html: <A NAME="References"></A><a href="\#Top of file">       Top</a>}}
 
\begin{thebibliography}{12}

%%tth:{\special{html: <font size="+0">}}

% \bibitem{abrams}
% Abrams D S and Lloyd S, 
% Simulation of Many-Body Fermi Systems on a Universal Quantum
% Computer, 
% {\em Phys.Rev.Lett.} {\bf 79}  2586--2589, 1997


\bibitem{abrams2}
Abrams D S and Lloyd S, 
Non-Linear Quantum Mechanics implies Polynomial Time 
solution for NP-complete and $\#$P problems,
%in {\it LANL e-print} quant-ph/9801041, http://xxx.lanl.gov (1998)
\hyperref{http://xxx.lanl.gov/abs/quant-ph/9801041}{}{}
%\hyperURL{http}{xxx.lanl.gov/abs/quant-ph}{9801041}
{http://xxx.lanl.gov/abs/quant-ph/9801041}


% \bibitem{adleman}
% Adleman L, Demarrais J and Huang M-D,
% Quantum Computability,
% {\em SIAM Journal of Computation}
% {\bf 26} 5 pp 1524--1540 October, 1997  
% 
% \bibitem{adleman2}
% Adleman L, 
% Molecular computation of solutions to combinatorial problems, 
% {\em Science}, 266, 1021--1024, Nov. 11, 1994
% 
% \bibitem{aharonov1}
% Aharonov D and Ben-Or M,
%  Fault-Tolerant Quantum Computation with Constant Error,
% {\em Proc. of the 29th Annual ACM Symposium on Theory of Computing (STOC)}
%  1997 
% 
% 
% \bibitem{aharonov2}
% Aharonov D and Ben-Or M,
% Polynomial Simulations of Decohered Quantum Computers
% {\em 37th Annual Symposium on Foundations of Computer Science}
%  (FOCS) pp 46--55, 1996
% 
% \bibitem{aharonov3}
% Aharonov D, Kitaev A Yu and  Nisan N,
% Quantum Circuits with Mixed States,
% {\em Proc. of the 30th Annual ACM Symposium on 
% Theory of Computing (STOC)} 1998
% 
% 
% \bibitem{aharonov4}
% Aharonov D, Ben-Or M, Impagliazo R and Nisan N,
% Limitations of Noisy Reversible Computation,
% in {\it LANL e-print} quant-ph/9611028, http://xxx.lanl.gov (1996)
% 
% 
\bibitem{aharonov5}
Aharonov D, Beckman D, Chuang I and  Nielsen M,
What Makes Quantum Computers Powerful? 
%http://wwwcas.phys.unm.edu/~mnielsen/science.html
\hyperref{http://wwwcas.phys.unm.edu/\~mnielsen/science.html}{}{}
%\hyperURL{http}{wwwcas.phys.unm.edu/~mnielsen}{science.html}
{http://wwwcas.phys.unm.edu/\~mnielsen/science.html}
% 
% 
% %\bibitem{anderson}
% %Anderson P W, More Is Different,
% 		%{\em Science},
% 		%{\bf 177},
% 		%4047,
% 		%393--396,1972
% 

\bibitem{aharonov}
Aharonov, D., Quantum Computation,
Annual Reviews of Computational Physics VI,
Edited by Dietrich Stauffer, World Scientific, 1998

% \bibitem{aspect1}
% Aspect A, Dalibard J and Roger G, 
% Experimental test of Bell's inequalities using time-varying analyzers,
% {\em Phys. Rev. Lett.} {\bf 49}, 1804--1807, 1982
% 
% \bibitem{aspect2}
% Aspect A, 
% Testing Bell's inequalities,
% {\em Europhys. News.} {\bf 22}, 73--75, 1991
% 
% \bibitem{bara}
%  Barahona F, in {\it J. Phys. A} vol. 15, (1982) 3241 

\bibitem{barenco1}
Barenco A 
A universal two-bit gate for quantum computation,
{\em Proc. R. Soc. Lond. A} {\bf 449} 679--683, 1995


% \bibitem{barenco2}
% Barenco A and Ekert A K 
% Dense coding based on quantum entanglement,
% {\em J. Mod. Opt.} {\bf 42} 1253--1259, 1995


\bibitem{barenco3}
Barenco A, Deutsch D, Ekert E and Jozsa R, 
Conditional quantum dynamics and quantum gates,
{\em Phys. Rev. Lett.} {\bf 74} 4083--4086, 1995


\bibitem{barenco4}
Barenco A, Bennett C H, Cleve R, DiVincenzo D P, Margolus N,
Shor P, Sleator T, Smolin J A and Weinfurter H, 
Elementary gates for quantum computation,
{\em Phys. Rev. A} {\bf 52}, 3457--3467, 1995


%\bibitem{barenco5}
%Barenco A 1996
%Quantum physics and computers,
%Contemp. Phys. {\bf 37} 375--389 1996


% \bibitem{barenco6}
% Barenco A, Ekert A, Suominen K A and Torma P, 
% Approximate quantum Fourier transform and decoherence,
% {\em Phys. Rev. A} {\bf 54}, 139--146, 1996
% 
% \bibitem{sym}
% Barenco A, Berthiaume A, Deutsch D, Ekert A, Jozsa R, and  Macchiavello C,
% Stabilization of Quantum Computations by Symmetrization,
% {\em  SIAM J. Comp.}{\bf 26},5, 1541--1557, 1997
% 
% 
% \bibitem{barenco7}
% Barenco A, Brun T A, Schak R and Spiller T P, 
% Effects of noise on quantum error correction algorithms,
% {\em Phys. Rev. A} {\bf 56} 1177--1188, 1997
% 
% 
% %\bibitem{barnum1}
% %Barnum H, Fuchs C A, Jozsa R and Schumacher B 1996
% %A general fidelity limit for quantum channels,
% %Phys. Rev. A {\bf 54} 4707-4711
% 
% \bibitem{barnum2}
% Barnum H, Caves C,
%  Fuchs C A, Jozsa R and Schumacher B,
% Non commuting mixed states cannot be broadcast,
% {\em Phys. Rev. Lett.}{\bf 76} 2818--2822,  1996 
% 
% 
% \bibitem{barnum3}
% Barnum H, Nielsen M and Schumacher B,
%  in {\it  Phys. Rev. A}, {\bf 57},6, 1998, pp. 4153--4175
% 
% \bibitem{beals1}
% Beals R, 
% Quantum computation of Fourier transform over symmetric groups
% {\em Proc. of the 29th Annual ACM Symposium on Theory of Computing (STOC)}
%  1997 
% 
% 
% \bibitem{beals2}
% Beals R, Buhrman H, Cleve R, Mosca M and de Wolf R, 
% Quantum Lower Bounds by Polynomials,
% in {\em 39th Annual Symposium on Foundations of Computer Science(FOCS)}, (1998)
% 
% 
% \bibitem{beckman1}
% Beckman D, Chari A, Devabhaktuni S and Preskill J 
% Efficient networks for quantum factoring,
% {\em Phys. Rev. A} {\bf 54}, 1034--1063, 1996
% 

\bibitem{bell}
Bell J S
On the Einstein-Podolsky-Rosen paradox,
{\em Physics} {\bf 1} 195--200, 1964

\bibitem{bell1}
Bell J S 
On the problem of hidden variables in quantum theory,
Rev. Mod. Phys. {\bf 38} 447-52, 1966
{\em Speakable and unspeakable in quantum mechanics} 1987
(Cambridge University Press) 


% \bibitem{benioff1}
% Benioff P,
% The Computer as a Physical Systems: A Microscopic
% Quantum Mechanical Hamiltonian Model of
%  Computers as Represented by Turing
% Machines, 
% {\em 
% J. Stat. Phys.} {\bf 22} 563--591, 1980
% 
% 
% \bibitem{benioff2}
% Benioff P 
% Quantum mechanical Hamiltonian models of Turing machines,
% {\em J. Stat. Phys.} {\bf 29} 515-546 1982
% 
% %Quantum mechanical models of Turing machines that
% %dissipate no energy,
% %{\em Phys. Rev. Lett.} {\bf 48} 1581--1585 1982
% 
% 
% \bibitem{bennett1}
% Bennett C H, 
% Logical reversibility of computation,
% {\em IBM J. Res. Develop.} {\bf 17} 525--532, 1973
% 
% \bibitem{bennett2}
% Bennett C H, 
% The Thermodynamics of Computation - a Review, 
%  {\em International Journal of Theoretical Physics,} 
% {\bf 21}, No. 12, p 905, 1982
% 
% %\bibitem{bennett3}
% %Bennett C H and Landauer R, 
% %The fundamental physical limits of computation,
% %{\em Scientific American,} July 38--46, 1985 
% 
% \bibitem{bennett4}
% Bennett C H, 
% Demons, engines and the second law,
% {\em Scientific American} {\bf 257} no. 5 (November) pp 88--96,
% 1987
% 
% %\bibitem{bennett5}
% %Bennett C H, Brassard G, Briedbart S and Wiesner S, 
% %Quantum cryptography, or unforgeable subway tokens,
% % {\em Advances in Cryptology: Proceedings of Crypto '82}
% %(Plenum, New York) pp 267--275, 1982
% 
% %\bibitem{bennett6}
% %Bennett C H and Brassard G, 
% %Quantum cryptography: public key distribution and coin tossing,
% %in {\em Proc. IEEE Conf. on Computers, Syst. and Signal Process.}
% %pp 175--179, 1984
% 
% %\bibitem{bennett7}
% %Bennett C H and Wiesner S J,
% %Communication via one- and two-particle operations on
% %Einstein-Podolsky-Rosen states,
% %{\em Phys. Rev. Lett.} {\bf 69}, 2881--2884, 1992
% 
% %\bibitem{bennett8}
% %Bennett C H,
% %Quantum information and computation,
% %{\em Phys. Today} {\bf 48 10} 24--30, 1995
% 
% 
% 
% \bibitem{bennett9}
% C.H. Bennett, 
% Time/Space Trade-offs for Reversible Computation, 
% {\em SIAM Journal of Computation}, {\bf 18}, 4, pp 766--776, 1989
% 
% %\bibitem{bennett10}
% %Bennett C H and Brassard G 1989,
% %SIGACT News {\bf 20}, 78-82
% 
% 
% 
% \bibitem{bennett11}
% Bennett C H, Bessette F, Brassard G, Savail L and Smolin J 
% Experimental quantum cryptography,
% {\em J. Cryptology} {\bf 5}, pp 3--28, 1992
% 
% %\bibitem{bennett12}
% %Bennett C H
% %Quantum Computers: Certainty from Uncertainty, 
% %Nature, 362, 1993 
% 

\bibitem{bennett13}
Bennett C H, Brassard G, Cr\'epeau C, Jozsa R, Peres A and
Wootters W K 
Teleporting an unknown quantum state via dual classical and
Einstein-Podolsky-Rosen channels,
{\em Phys. Rev. Lett.} {\bf 70} 1895--1898, 1993

%\bibitem{bennettdiv}
%Bennett C H,
%Progress towards quantum computation
%1995


%\bibitem{bennett12} 
%Bennett C H, Brassard G, Popescu S,
%Schumacher B, Smolin J A and Wootters W K 1996a
%Purification of noisy entanglement and faithful
%teleportation via noisy channels,
%Phys. Rev. Lett. {\bf 76} 722-725

\bibitem{bennett14} 
Bennett C H, DiVincenzo D P, Smolin J A
and Wootters W K 
Mixed state entanglement and quantum error correction, 
{\em Phys. Rev. A} {\bf 54} 3825, 1996 

\bibitem{bbbv}
Bennett C H, Bernstein E, Brassard G and Vazirani U 
Strengths and Weaknesses of quantum computing,
{\em SIAM Journal of Computation}
{\bf 26} 5 pp 1510--1523 October, 1997  

% \bibitem{berman1}
% Berman G P, Doolen G D, Holm D D, Tsifrinovich V I 
% Quantum computer on a class of one-dimensional Ising
% systems,
% {\em Phys. Lett.} {\bf 193} 444-450 1994
% 
% 
% \bibitem{bv}
% Bernstein E and Vazirani U, 1993,
% Quantum complexity theory,
% {\em SIAM Journal of Computation}
% {\bf 26} 5 pp 1411--1473 October, 1997 
% 
% 
% \bibitem{bert1}
% Berthiaume A, Deutsch D and Jozsa R, 
% The stabilization of quantum computation, in
% {\em Proceedings of the Workshop on Physics and
% Computation, PhysComp 94} 60-62 Los Alamitos: IEEE
% Computer Society Press, 1994
% 
% \bibitem{bert2}
% Berthiaume A and Brassard G, 
% The quantum challenge to structural complexity theory,
% in {\em Proc. of the Seventh Annual Structure in Complexity Theory
% Conference} 1992
% (IEEE Computer Society Press, Los Alamitos, CA) 132--137 
% 
% \bibitem{bert3}
% Berthiaume A and Brassard G,
% Oracle quantum computing,
% in {\em Proc. of the Workshop on Physics of Computation: PhysComp '92}
% (IEEE Computer Society Press, Los Alamitos, CA) 60--62, 1992
% 
% 
% 
% \bibitem{bogosian}
% Boghosian B M and Taylor W,
% Simulating quantum mechanics on a quantum computer
% in {\it Physica D}, 120 (1998) pp. 30-42
% 
% %\bibitem{bohm1}
% %Bohm D 1951
% %{\em Quantum Theory}
% %(Englewood Cliffs, N. J.)
% 
% %\bibitem{bohm2}
% %Bohm D and Aharonov Y 1957
% %Phys. Rev. {\bf 108} 1070
% 
% %\bibitem{bollinger}
% %Bollinger J J, Heinzen D J, Itano W M, Gilbert S L and Wineland D J, 
% %{\em Phys Rev Lett} 63 1031, 1989
% 
% %\bibitem{boneh}
% %D. Boneh, R. Lipton,
% %Quantum Cryptoanalysis Of Hidden Linear Functions, 
% %{\em Proceedings of CRYPTO'95}
% 
% 
% \bibitem{bouwmeester}
%  Bouwmeester D, Pan J-W, Mattle K, Weinfurter H, Zeilinger A,
% Experimental quantum teleportation
% {\em Nature} {\bf 390}, 575-579 1997
% 
% 

\bibitem{boyer1}
Boyer M, Brassard G, Hoyer P and Tapp A,
Tight bounds on quantum searching,
in {\it Fortsch.Phys.} 46, (1998) pp. 493-506 


\bibitem{brassard1}
Brassard G,
Searching a quantum phone book, 
{\bf Science} {\bf 275} 627-628 1997

% \bibitem{brassard2}
% Brassard G and Crepeau C,
% {\em SIGACT News} {\bf 27} 13-24 1996
% 
% \bibitem{brassard3}
% Brassard G, Hoyer P and Tapp A,
% Quantum Algorithm for the Collision Problem
% in {\it LANL e-print} quant-ph/9705002, http://xxx.lanl.gov (1997)
% 
% 
% \bibitem{brassard4}
% Brassard G and Hoyer P, 
% An Exact Quantum-Polynomail Algorithm for Simon's Problem
% {\em Proceedings of the 5th Israeli Symposium on Theory 
% of Computing and Systems (ISTCS)}, 1997 
% 
% 
% 
% \bibitem{brassard5}
% Brassard G,
% Teleportation as Quantum Computation,
% in {\it Physica D}, 120 (1998) 43-47
% 
% 
% \bibitem{brassard6}
% Brassard G,
% The dawn of a new era for quantum cryptography: {T}he
% experimental
%          prototype is working!,
% Sigact News,
% {\bf 20}(4), 78--82, 1989
% 
% %\bibitem{braunstein1}
% %Braunstein S L, Mann A and Revzen M 1992
% %Maximal violation of Bell inequalities for mixed states,
% %Phys. Rev. Lett. {\bf 68}, 3259-3261
% 
% %\bibitem{braunstein2}
% %Braunstein S L and Mann A 1995
% %Measurement of the Bell operator and quantum teleportation,
% %Phys. Rev. A {\bf 51}, R1727-R1730
% 
% 
% %\bibitem{brillouin}
% %Brillouin L 1956,
% %{\em Science and information theory}
% %(Academic Press, New York)
% 
% 
% %\bibitem{brune}
% %Brune M, Nussenzveig P, Schmidt-Kaler F, Bernardot F, Maali A,
% %Raimond J M and Haroche S,
% %From Lamb shift to light shifts: vacuum and subphoton cavity fields
% %measured by atomic phase sensitive detection,
% %{\em Phys. Rev. Lett.} {\bf 72}, 3339-3342 1994
% 
% 
% \bibitem{cleve3}
% Buhrman H, Cleve R and van Dam W,
% Quantum Entanglement and Communication Complexity,
% in {\it LANL e-print} quant-ph/9705033,  http://xxx.lanl.gov (1997)
% 
% 
% \bibitem{buhrman}
% Buhrman H, Cleve R and Wigderson A,
% Quantum vs. Classical Communication and Computation, in
% {\em Proc. of the 30th Annual ACM Symposium on 
% Theory of Computing (STOC)} (1998)
% 


\bibitem{calshor}
Calderbank A R and Shor P W,
Good quantum error-correcting codes exist, 
{\em Phys. Rev. A} {\bf 54} 1098-1105,  1996


% \bibitem{calderbank3}
% Calderbank A R, Rains E M, Shor P W and Sloane N J A 
% Quantum error correction and orthogonal geometry,
% {\em Phys. Rev. Lett.} {\bf 78} 405--408, 1997
% 
%  
% \bibitem{gf4}
% Calderbank A R, Rains E M, Shor P W and Sloane N J A,
% Quantum error correction via codes over $GF(4)$
% in {\it LANL e-print} quant-ph/9608006,  http://xxx.lanl.gov (1996), 
% To appear in {\it IEEE Transactions on Information Theory.}
% 
% 
% %\bibitem{caves1}
% %Caves C M 1990
% %Quantitative limits on the ability of a Maxwell Demon
% %to extract work from heat,
% %Phys. Rev. Lett. {\bf 64} 2111-2114
% 
% %\bibitem{caves2}
% %Caves C M, Unruh W G and Zurek W H 1990
% %comment,
% %Phys. Rev. Lett. {\bf 65} 1387
% 
% \bibitem{chernoff}
% Chernoff. 
% See 
% Feller W, An Introduction to Probability Theory and Its Applications, Wiley, New York,1957

\bibitem{decoherence}
Chuang I L, Laflamme R, Shor P W and Zurek W H,
Quantum computers, factoring, and decoherence,
{\em Science} {\bf 270} 1633--1635, 1995

\bibitem{decoherence2}
 Chuang I L, Laflamme R and Paz J P, 
Effects of Loss and Decoherence on a Simple Quantum Computer,
%in {\it LANL e-print} quant-ph/9602018,  http://xxx.lanl.gov (1996)
\hyperref{http://xxx.lanl.gov/abs/quant-ph/9602018}{}{}
%\hyperURL{http}{xxx.lanl.gov/abs/quant-ph}{9602018}
{http://xxx.lanl.gov/abs/quant-ph/9602018}



% \bibitem{chuang1}
% I.Chuang and W.C.D. Leung and Y. Yamamoto, 
% Bosonic Quantum Codes for Amplitude Damping,
% in {\it Phys. Rev. A}, {\bf 56}, 2, (1997) pp. 1114-1125
% 
% \bibitem{chuang2}
% Chuang I L and Yamamoto 
% Creation of a persistent qubit using error correction
% Phys. Rev. A {\bf 55}, 114--127, 1997
% 
% 
% \bibitem{chuang3}
% Chuang I L ,Vandersypen L M K, Zhou X, Leung D W and Lloyd S, 
% Experimental realization of a quantum algorithm, 
% in {\it Nature}, 393, 143-146 (1998)
% 
% \bibitem{church}
% Church A 
% An unsolvable problem of elementary number theory,
% {\em Amer. J. Math.} {\bf 58} 345--363, 1936
% 
% %\bibitem{church2}
% %modern church thesis
% 
% \bibitem{cirac1}
% Cirac I J and Zoller P
% \newblock Quantum computations with cold trapped ions,
% {\em Phys. Rev. Let.}, 74: 4091-4094, 1995.
% 
% \bibitem{cirac2}
% Cirac J I, Pellizari T and Zoller P,
% Enforcing coherent evolution in dissipative quantum dynamics,
% {\em Science} {\bf 273}, 1207, 1996
% 
% \bibitem{cirac3}
% Cirac J I, Zoller P, Kimble H J and Mabuchi H 
% Quantum state transfer and entanglement distribution among distant
% nodes of a quantum network,
% {\em Phys. Rev. Lett.} {\bf 78}, 3221, 1997
% 
% 
% \bibitem{tsirelson}
% Cirel'son (Tsirelson) B, 
% Reliable storage of information in a system of unreliable 
% components with local interactions.
% {\em Lecture notes in Mathematics}{\bf 653} 15--30 ,1978
% 

\bibitem{clausen}
Clausen M, 
Fast Generalized Fourier transforms, 
{\em Theoret. Comput. Sci.} {\bf 56} 55--63 1989 

% \bibitem{clauser}
% Clauser J F, Holt R A, Horne M A and Shimony A 
% Proposed experiment to test local hidden-variable theories,
% {\em Phys. Rev. Lett.} {\bf 23} 880--884, 1969
% 
% %\bibitem{clauser1}
% %Clauser J F and Shimony A,
% %Bell's theorem: experimental tests and implications,
% %{\em Rep. Prog. Phys.} {\bf 41} 1881--1927, 1978
% 
% 
% %\bibitem{cleve}
% %Cleve R and DiVincenzo D P,
% %Schumacher's quantum data compression as a quantum computation,
% %{\em Phys. Rev. A} {\bf 54} 2636, 1996
% 
% 
% \bibitem{cleve2}
% Cleve R and Buhrman H,
% Substituting Quantum Entanglement for Communication
% in{\em Phys rev A,}{\bf 56} 2, (1997) pp. 1201-1204
% 
% 
% 
% \bibitem{cleve4}
% Cleve R, van Dam W, Nielsen M and Tapp A, 
%  Quantum Entanglement and the Communication Complexity of
%  the Inner Product Function,
% in {\it LANL e-print} quant-ph/9708019,  http://xxx.lanl.gov (1997)
% 
% 
% \bibitem{cohentan}
% C. Cohen-Tanoudji, 
% {\em Quantum Mechanics}, 
% Wiley press, New York (1977)

\bibitem{coppersmith}
Coppersmith D, 
An approximate Fourier transform useful in quantum factoring,
IBM Research Report RC 19642, 1994


\bibitem{fft}
Cormen T, Leiserson C and Rivest R, 
{\em Introduction to Algorithms}, 
 (pp 776--800 for FFT, 837--844 for primality test, 812 for extended Euclid 
algorithm, 834--836 for RSA cryptosystem) MIT press, 1990 


\bibitem{cory4} Cory D G, Fahmy A F, and  Havel T F,
Nuclear magnetic resonance spectroscopy: an experimentally
accessible paradigm for quantum computing, in {\em Proc. of the 4th
Workshop on Physics and Computation} (Complex Systems Institute,
Boston, New England) 1996

%\bibitem{cory1}
%D.~G. Cory, M.~P. Price, and  Havel T F,
%\newblock {\em Physica D}, 1998.
%\newblock In press (quant-ph/9709001).

% \bibitem{cory2}
%  Cory D. G, Fahmy A F, and  Havel T F
% \newblock {\em Proc. Nat. Acad. of Sciences of the U. S.},
%  94:1634--1639, 1997.
% 
% 
% 
% \bibitem{cory3}
%  Cory D G,  Mass W, Price M, Knill E, Laflamme R,  Zurek W H, and
%  Havel T F and Somaroo S S,
% Experimental quantum error correction
% in {\it Phys. Rev. Let.}{\bf 81} 10, (1998)
% pp. 2152-2155
% 
% 
% %
% %\bibitem{coding}
% %Cover T M and Thomas J A, 
% %{\em Elements of Information Theory,}
% %John Wiley and Sons, New York, 1991
% 
% 
% %\bibitem{crandall}
% %R.~E.~Crandall 
% %\newblock The challenge of large numbers,
% %In {\em Scientific American} February 59-62, 1997
% 
% %\bibitem{chupp}
% %Chupp T.E and Hoare R.J,
% %{\em  Phys Rev Lett} 64 2261 1990
% 
% 
% \bibitem{vandam2}
% van Dam W, 
% A Universal Quantum Cellular Automaton, 
% {\em Proceedings of the Fourth Workshop on Physics and Computation}, 1996
% 

\bibitem{deutsch}
Deutsch, D.,
The Fabric of Reality,
Penguin Books Ltd, Harmondsworth, Middlesex, England, 1997

\bibitem{deutsch1}
Deutsch D,
Quantum theory, the Church-Turing principle and the
universal quantum computer,
In {\em Proc. Roy. Soc. Lond.} A {\bf 400} 97-117, 1985

\bibitem{deutsch2}
Deutsch D,
Quantum computational networks,
In {\em Proc. Roy. Soc. Lond.} A {\bf 425} 73-90, 1989

\bibitem{deutsch3}
Deutsch D and Jozsa R,
Rapid solution of problems by quantum computation,
In {\em Proc. Roy. Soc. Lond} A {\bf 439} 553-558, 1992


\bibitem{deutsch4}
Deutsch D, Barenco A and Ekert A,  
Universality in quantum computation,
In {\em  Proc. R. Soc. Lond.} A {\bf 449} 669-677, 1995

%\bibitem{deutsch5}
%D.~Deutsch D and A.~Ekert and R.~Jozsa and C.~Macchiavello 
%and S.~Popescu and A.~Sanpera 
%Quantum privacy amplification and the security of
%quantum cryptography over noisy channels,
%In {\em Phys. Rev. Lett.} {\bf 77} 2818, 1996

% \bibitem{diaconis}
% Diaconis P and Rockmore D, 
% Efficient Computation of the Fourier transform on finite groups,
% {\em J. AMS} {\bf 3} No. 2, 297--332, 1990
% 
% \bibitem{dieks}
% Dieks D,
% Communication by {EPR} Devices,
% Phys Lett A, 
% 92(6) 271--272 1982
% 
% %\bibitem{cooling}
% %Diedrich F, Bergquist J C, Itano W M and.
% %Wineland D J 1989
% %Laser cooling to the zero-point energy of motion,
% %Phys. Rev. Lett. {\bf 62} 403
% 
% 
% %\bibitem{dieks}
% %Dieks D 1982
% %Communication by electron-paramagnetic-resonance devices,
% %Phys. Lett. A {\bf 92} 271
% 

\bibitem{twobit}
DiVincenzo D P,
Two-bit gates are universal for quantum computation,
{\em Phys. Rev. A} {\bf 51} 1015-1022 1995

\bibitem{div-rev}
DiVincenzo D P, 
Quantum computation,
{\em Science} {\bf 270} 255-261 1995

%\bibitem{divincenzo1}
%DiVincenzo D P and Shor P W,
%Fault-tolerant error correction with efficient quantum codes,
%{\em Phys. Rev. Lett.} {\bf 77} 3260-3263,  1996


% \bibitem{divincenzo2}
% DiVincenzo D P,
% Quantum Gates and Circuits,
% in {\it 
% Proceedings of the ITP Conference
%      on Quantum Coherence and Decoherence}, December, (1996), Proc. R. Soc. London A
% 
% 
% 
% 
% 
% \bibitem{durr}
% Durr C and Hoyer P, 
% A Quantum Algorithm for Finding the Minimum,
% in {\it LANL e-print} quant-ph/9607014,  http://xxx.lanl.gov (1996)
% 
% \bibitem{durr1}
% Durr C, LeThanh H and Santha M, 
% A decision procedure for well formed linear quantum 
% cellular automata, 
% {\em Proceeding of the 37th IEEE Symposium on Foundations of Computer Science},
%        38--45, 1996, and 
% {\em Random Structures $\&$ Algorithms}, 1997
% 

\bibitem{epr}
Einstein A, Rosen N and Podolsky B,
{\em Phys. Rev.} {\bf 47}, 777 1935


%\bibitem{ekert1}
%Ekert A,
%Quantum cryptography based on Bell's theorem
%{\em Phys. Rev. Lett.} {\bf 67}, 661--663,  1996

\bibitem{ekert2}
Ekert A and Jozsa R 
Quantum computation and Shor's factoring algorithm,
{\em Rev. Mod. Phys.} {\bf 68} 733 1996

% \bibitem{ekert3}
% Ekert A and Macchiavello C 1996
% Quantum error correction for communication,
% Phys. Rev. Lett. {\bf 77} 2585-2588
% 
% %\bibitem{ekert4}
% %Ekert A 1997
% %From quantum code-making to quantum code-breaking,
% %(preprint quant-ph/9703035)
% 
% 
% %\bibitem{vanenk}
% %van Enk S J, Cirac J I and Zoller P 1997
% %Ideal communication over noisy channels: a quantum optical implementation,
% %Phys. Rev. Lett. {\bf 78}, 4293-4296
% 
% \bibitem{farhi}
% Another proof of the  parity lower bound, using interesting techniques,  
% was found recently by 
% E. Farhi, J. Goldstone, S. Gutmann, M. Sipser 
% A Limit on the Speed of Quantum Computation in Determining Parity
% in {\it LANL e-print} quant-ph/9802045,  http://xxx.lanl.gov (1998)
% 
% 


\bibitem{feynman1}
Feynman R P
Simulating physics with computers,
In {\em Int. J. Theor. Phys.} {\bf 21} 467-488, 1982

\bibitem{feynman2}
 Feynman R P,
\newblock Quantum mechanical computers,
In {\em Found. of Phys.} {\bf 16} 507-531, 1986
see also Optics News February 1985, 11-20.

\bibitem{Feynman-96}
R. Feynman,
\newblock Feynman lectures on computation, 1996.


% \bibitem{fortnow}
% Fortnow L and Rogers J,
% Complexity Limitations on quantum computation
% Technical report 97-003, DePaul University, School of Computer science,
% 1997
% 
% \bibitem{fredkin}
% Fredkin E and Toffoli T 1982
% Conservative logic,
% Int. J. Theor. Phys. {\bf 21} 219-253
% 
% \bibitem{freedman}
% Freedman M, 
% Logic, P/NP and the quantum field computer, 
% preprint, 1997
% 
% 
% 
% \bibitem{gacs}
% P. Ga'cs,
% Self Correcting Two Dimensional Arrays,
% in {\em Randomness and Computation},1989,
% edited by S. Micali, vol 5,
% in series ``Advances in Computing Research",
%   pages 240-241,246-248,
%  series editor: F.P.Preparata
% 
% \bibitem{gacs1}
% P. Ga'cs,
% one dimensional self correcting array.
% 
% \bibitem{gardiner}
%  Gardiner C W, Quantum Noise, Springer-Verlag,
% Berlin, 1991

\bibitem{grey}
Garey M R and Johnson D S,
Computers and Intractability, 
published by Freeman and Company,
		New York,
		1979

% \bibitem{gaudi}
% A. Gaud\'{i}. The set of ropes is presented in {\em la sagrada familia}
% in Barcelona, Spain. Pictures of {\em la sagrada familia} can be found in:  
% http://futures.wharton.upenn.edu/~jonath22/Gaudi/eltemple.html
% 
% 
% 
% \bibitem{geroch}
% R. Geroch and G. Hartle
%  Computability and Physical theories
% in {\em Between Quantum and Cosmos}
% edited by Zurek and Van der Merwe and Miller,
% Princeton University Press, 1988, 549-566

\bibitem{gershenfeld}
Gershenfeld N A and  Chuang I L
\newblock Bulk spin-resonance quantum computation,
{\em Science}, 275:350--356, 1997.


%\bibitem{glauber}
%Glauber R J 1986, 
%in {\em Frontiers in Quantum Optics}, Pike E R and Sarker S, eds
%(Adam Hilger, Bristol)


%\bibitem{golay}
%Golay M J E 1949 
%Notes on digital coding,
%Proc. IEEE {\bf 37} 657


% \bibitem{gottesman1}
% Gottesman D 1996
% Class of quantum error-correcting codes saturating the quantum Hamming
% bound,
% Phys. Rev. A {\bf 54}, 1862-1868
% 
% 
% \bibitem{gottesman2}
% Gottesman D 
% A theory of fault-tolerant quantum computation, 
% in {\it Phys. Rev. A},{\bf 57} 127--137
% 
% \bibitem{gottesman5}
% Gottesman D, Evslin J, Kakade D and Preskill J,  preprint (1996)
% 
% 
% 
% 
% \bibitem{ghz}
% Greenberger D M, Horne M A and Zeilinger A 1989
% Going beyond Bell's theorem,
% in {\em Bell's theorem, quantum theory and conceptions of the universe},
% Kafatos M, ed, (Kluwer Academic, Dordrecht) 73-76
% 
% %\bibitem{greenberger2}
% %Greenberger D M, Horne M A, Shimony A and Zeilinger A 1990
% %Bell's theorem without inequalities,
% %Am. J. Phys. {\bf 58}, 1131-1143
% 
% \bibitem{griffiths}
% R. B. Griffiths and C. S. Niu 1996
% Semi classical Fourier transform for quantum computation
% in {\em Phys. Rev. Lett.} ,76, pp. 3228--3231
% 

\bibitem{grover1}
 Grover L K,
Quantum mechanics helps in searching for a needle in a haystack, 
Phys. Rev. Lett. {\bf 79}, 325-328  1997
and the original STOC paper:
A fast quantum mechanical algorithm for database search
{\em Proc. of the 28th Annual ACM Symposium on 
Theory of Computing (STOC)} 212--221, 1996


\bibitem{grover2}
 Grover L K, 
A framework for fast quantum mechanical algorithms,
%in {\it LANL e-print} quant-ph/9711043,  http://xxx.lanl.gov (1997)
\hyperref{http://xxx.lanl.gov/abs/quant-ph/9711043}{}{}
%\hyperURL{http}{xxx.lanl.gov/abs/quant-ph}{9711043}
{http://xxx.lanl.gov/abs/quant-ph/9711043}



\bibitem{grover3}
 Grover L K,
Quantum computers can search arbitrarily large databases by a single query
in {\it Phys. Rev. Let.} {\bf 79} 23, 4709--4712, 1997


\bibitem{grover4}
 Grover L K, 
A fast quantum mechanical algorithm for estimating the median,
%in {\it LANL e-print} quant-ph/9607024,  http://xxx.lanl.gov (1997)
\hyperref{http://xxx.lanl.gov/abs/quant-ph/9607024}{}{}
%\hyperURL{http}{xxx.lanl.gov/abs/quant-ph}{9607024}
{http://xxx.lanl.gov/abs/quant-ph/9607024}

\bibitem{hagley}
Hagley E et. al, 
Generation of Einstein Podolsky Rosen pairs of atoms, 
{\em Phys. Rev. Lett,} {\bf 79}, 1--5, 1997


%\bibitem{hamming1}
%Hamming R W 1950 
%Error detecting and error correcting codes,
%Bell Syst. Tech. J. {\bf 29} 147

\bibitem{hamming2}
Hamming R W 1986
{\em Coding and information theory}, 2nd ed,
(Prentice-Hall, Englewood Cliffs)


\bibitem{hardy}
Hardy G H and Wright E M 1979
{\em An introduction to the theory of numbers}
(Clarendon Press, Oxford)


\bibitem{haroche} 
Haroche S and Raimond J-M 1996
Quantum computing: dream or nightmare?
Phys. Today August 51-52


%\bibitem{hellman}
%Hellman M E 1979
%The mathematics of public-key cryptography,
%Scientific American {\bf 241} August 130-139


%\bibitem{hill} ??????
%Hill R 1986
%{\em A first course in coding theory}
%(Clarendon Press, Oxford)


\bibitem{hodges}
Hodges A 1983
{\em Alan Turing: the enigma}
(Vintage, London)


% \bibitem{hughes}
% Hughes R J, Alde D M, Dyer P, Luther G G, Morgan G L ans Schauer M 1995
% Quantum cryptography,
% Contemp. Phys. {\bf 36} 149-163

\bibitem{hungerford}
Hungerford T W, 1974
{\em Algebra}
(Springer-Verlag, New York)
 
% %\bibitem{jones}
% %Jones D S 1979
% %{\em Elementary information theory}
% %(Clarendon Press, Oxford)


\bibitem{jones2}
A. J. Jones, M. Mosca and R. H. Hansen, 
Implementation of a Quantum Search Algorithm on a Nuclear Magnetic
     Resonance Quantum Computer, in 
{\it Nature} 393 (1998) 344-346, and see also A. J. Jones and M. Mosca, 
Implementation of a Quantum Algorithm to Solve Deutsch's Problem on a
     Nuclear Magnetic Resonance Quantum Computer, 
in {\it J. Chem. Phys.} 109 (1998) 1648-1653


%\bibitem{jozsa1}
%Jozsa R and Schumacher B 1994
%A new proof of the quantum noiseless coding theorem,
%J. Mod. Optics {\bf 41} 2343

% \bibitem{jozsa2}
% Jozsa R 1997
% Entanglement and quantum computation,
% appearing in {\em Geometric issues in the foundations of science},
% Huggett S {\em et. al.}, eds, (Oxford University Press)
% 
% %\bibitem{jozsa3}
% %Jozsa R 1997
% %Quantum algorithms and the Fourier transform,
% %submitted to {\em Proc. Santa Barbara conference on quantum coherence
% %and decoherence} (preprint quant-ph/9707033)
% 
% \bibitem{tsirelson1}
% Khalfin L A and Tsirelson B S,
% Quantum/Classical Correspondence in the 
% Light of Bell's Inequalities. 
% {\em Foundations of physics,}{\bf 22} No. 7, 879--948
% July 1992
% 
% \bibitem{kelvin}
% Lord Kelvin,  differential analyzer (1870), presented in the Science Museum of Aarhus, 
% Denmark (1998)
% 
% %\bibitem{keyes1}
% %Keyes R W and Landauer R 1970
% %IBM J. Res. Develop. {\bf 14}, 152
% 
% \bibitem{keyes2}
% Keyes R W 
% Science {\bf 168}, 796, 1970
% 
% 
% %\bibitem{kholevo}
% %Kholevo A S 1973
% %Probl. Peredachi Inf {\bf 9}, 3; Probl. Inf. Transm. (USSR) {\bf 9}, 177
% 
% 
% 
% \bibitem{kitaev0}
% Kitaev A Yu, Quantum Computations: Algorithms and Error Corrections, 
% in {\it Russian Math. Serveys}, {\bf 52}:6, 1191-1249
% 
% \bibitem{kitaev1}
% Kitaev A Yu, 
% Quantum measurements and the Abelian stablizer problem,
% in {\it LANL e-print} quant-ph/9511026,  http://xxx.lanl.gov (1995)
% 
% 
% \bibitem{kitaev2}
%  Kitaev. A. Yu
%  Quantum error correction with imperfect gates,
% {\em Quantum Communication, Computing, and 
% Measurement},
% 		eds: Hirota, Holevo and Caves,
% 		181--188,
% 		Plenum Press,
% 		New York, 1997.
% 
% \bibitem{kitaev3}
% Kitaev A Yu 1997
% Fault-tolerant quantum computation by anyons,
% in {\it LANL e-print} quant-ph/9707021,  http://xxx.lanl.gov (1997)
% 
% 
% \bibitem{kitaevNP}
% A. Yu. Kitaev, 
% private communication
% 
% 
% \bibitem{knill1}
% Knill E and Laflamme R, 
% Concatenated quantum codes,
% in {\it LANL e-print} quant-ph/9608012,  http://xxx.lanl.gov (1996)
% 
% \bibitem{knill2}
% Knill E, Laflamme R, and Zurek W,
% \newblock Resilient quantum computation,
%  {\em Science}, vol 279, p.342, 1998. 

\bibitem{knill3}
Knill E and Laflamme R 1997
A theory of quantum error-correcting codes,
Phys. Rev. A {\bf 55} 900-911

\bibitem{knill4}
Knill E, Laflamme R and Zurek W H 1997
Resilient quantum computation: error models and thresholds
%in {\it LANL e-print} quant-ph/9702058,  http://xxx.lanl.gov (1997)
\hyperref{http://xxx.lanl.gov/abs/quant-ph/9702058}{}{}
%\hyperURL{http}{xxx.lanl.gov/abs/quant-ph}{9702058}
{http://xxx.lanl.gov/abs/quant-ph/9702058}



% \bibitem{knill5}
% E. Knill,
% Non-Binary Unitary Error Bases and Quantum Codes,
% in {\it LANL e-print} quant-ph/9608048,  http://xxx.lanl.gov (1996)
% 
% 

\bibitem{knuth} 
Knuth D E 1981
{\em The Art of Computer Programming, Vol. 2: Seminumerical Algorithms},
2nd ed (Addison-Wesley).

% \bibitem{watrous}
% Kondacs A and Watrous J
% On the power of Quantum Finite State Automata
% {\em 38th Annal Symposium on Foundations of Computer Science},(FOCS)
% 1997
% 
% \bibitem{kwiat}
% Kwiat P G, Mattle K, Weinfurter H, Zeilinger A, Sergienko A and Shih Y 1995
% New high-intensity source of polarization-entangled photon pairs
% Phys. Rev. Lett. {\bf 75}, 4337-4341
% 
% 
% 
% 
% 
% %\bibitem{laflamme1}
% %R. Laflamme, E. Knill, W.H. Zurek, P. Catasti and S. Marathan.
% %\newblock quant-ph/9709025, 1997.
% 
% \bibitem{laflamme2}
% Laflamme R, Miquel C, Paz J P and Zurek W H 1996
% Perfect quantum error correcting code,
% Phys. Rev. Lett. {\bf 77}, 198-201
% 
% 
% \bibitem{landauer1} 
% Landauer R.
% \newblock Is quantum mechanics useful? 
% {\em Phil. Trans. Roy. Soc. of London},
%   353:367--376, 1995.
% 
% 
% \bibitem{landauer2}
% Landauer R 1961 IBM J. Res. Dev. {\bf 5} 183, 
% and 1970 IBM J. Res. Dev,  volume 14, page 152.
% 
% %\bibitem{landauer3} ??????
% %Landauer R 1991
% %Information is physical,
% %Phys. Today May 1991 23-29
% 
% %\bibitem{landauer4} ?????????????
% %Landauer R 1996
% %The physical nature of information,
% %Phys. Lett. A {\bf 217} 188
% 
% 
% \bibitem{lecerf}
% Lecerf Y 1963
% Machines de Turing r\'eversibles . R\'ecursive insolubilit\'e
% en $n \in N$ de l'equation $u=\theta^n u$, o\`u $\theta$ est un
% isomorphisme de codes,
% C. R. Acad. Francaise Sci. {\bf 257}, 2597-2600
% 
% \bibitem{leung}
% Leung D W, Nielsen M A, Cuang I L and Yamamuto Y, 
% Approximate quantum error correction can lead to better codes
% in {\it Phys. Rev. A},{\bf 56}, 1, 2567--2573, 1997
% 
% %\bibitem{levitin}
% %Levitin L B 1987
% %in {\em Information Complexity and Control in Quantum Physics},
% %Blaquieve A, Diner S, Lochak G, eds (Springer, New York) 15-47
% 
% %\bibitem{lidar}
% %Lidar D A and Biham O 1996
% %Simulating Ising spin glasses on a quantum computer
% %(preprint quant-ph/9611038)
% 
% \bibitem{lint}
% van Lint J H
% Coding Theory, Springer-Verlag, 1982

\bibitem{lipton}
Lipton R J, 
Using DNA to solve NP-complete problems. 
{\em Science}, {\bf 268} 542--545, Apr. 28, 1995

% \bibitem{lloyd2}
% Lloyd S,
% \newblock Universal quantum simulators, 
% {\em Science}, 273:1073--1078, 1996.
% 
% 
% \bibitem{lloyd3}
% Lloyd S 1993
% A potentially realisable quantum computer,
% Science {\bf 261} 1569; see also Science {\bf 263} 695 (1994).

\bibitem{lloyd4}
Lloyd S 1995
Almost any quantum logic gate is universal,
Phys. Rev. Lett. {\bf 75}, 346-349

% \bibitem{lloyd5}
% Lloyd S 1997
% The capacity of a noisy quantum channel,
% Phys. Rev. A {\bf 55} 1613-1622
% 
% %\bibitem{lo}
% %Lo H-K and Chau H F 1997
% %Is quantum bit commitment really possible?,
% %Phys. Rev. Lett. {\bf 78} 3410-3413
% 
% 
% 
% \bibitem{loss}
% Loss D and DiVincenzo D P 
% Quantum Computation with Quantum Dots,
% in {\it Phys. Rev. A},{\bf 57},1,  pp. 120-126, 1997
% 
% \bibitem{macwilliams}
% MacWilliams F J and Sloane N J A 1977
% {\em The theory of error correcting codes},
% (Elsevier Science, Amsterdam)
% 
% 
% %\bibitem{majumder} ??
% %Majumder et.al Phys. rev.lett. 65 2931 (1990)
% 
% \bibitem{mattle}
% Mattle K, Weinfurter H, Kwiat P G and Zeilinger A 1996
% Dense coding in experimental quantum communication,
% Phys. Rev. Lett. {\bf 76}, 4656-4659.
% 
% 
% %\bibitem{margolus1}
% %Margolus N 1986
% %Quantum computation,
% %Ann. New York Acad. Sci. {\bf 480} 487-497

\bibitem{margolus2}
Margolus N 1990 
Parallel Quantum Computation,
in {\em Complexity, Entropy and the Physics of
Information, Santa Fe Institute Studies in the Sciences
of Complexity,} vol VIII p. 273
ed Zurek W H (Addison-Wesley)


%\bibitem{maxwell}
%Maxwell J C 1871
%{\em Theory of heat} (Longmans, Green and Co, London)


%\bibitem{mayers1}
%Mayers D 1997
%Unconditionally secure quantum bit commitment is impossible,
%Phys. Rev. Lett. {\bf 78} 3414-3417

%\bibitem{mayers2}
%Mayers D 1997
% secure key distribution


%\bibitem{menezes}
%Menezes A J, van Oorschot P C and Vanstone S A 1997
%{\em Handbook of applied cryptography}
%(CRC Press, Boca Raton)

%\bibitem{mermin}
%Mermin N D 1990
%What's wrong with these elements of reality?
%Phys. Today (June) 9-11


%\bibitem{meyer}
%Meyer D A 1996
%Quantum mechanics of lattice gas automata I: one particle plane
%waves and potentials,
%(preprint quant-ph/9611005)


%\bibitem{minsky}
%Minsky M L 1967 
%{\em Computation: Finite and Infinite Machines}
%(Prentice-Hall, Inc., Englewood Cliffs, N. J.; also London 1972)


% \bibitem{miquel1}
% Miquel C, Paz J P and Perazzo 1996
% Factoring in a dissipative quantum computer
% Phys. Rev. A {\bf 54} 2605-2613
% 
% Miquel C, Paz J P and Zurek W H 1997
% Quantum computation with phase drift errors,
% Phys. Rev. Lett. {\bf 78} 3971-3974

%\bibitem{monroe1}
%Monroe C, Meekhof D M, King B E, Jefferts S R,
%Itano W M, Wineland D J and Gould P 1995a
%Resolved-sideband Raman cooling of a bound atom to the
%3D zero-pointenergy,
%Phys. Rev. Lett. {\bf 75} 4011-4014

% \bibitem{monroe2}
% Monroe C,  Meekhof D M, King B E,  Itano W M and  Wineland D J.
% \newblock Demonstration of a universal quantum logic gate,
% {\em Phys. Rev. Lett.}, 75:4714-4717, 1995.
% 
% 
% \bibitem{mott}
% N. F. Mott, 
% The wave Mechanics of $\alpha$-Ray Tracks, 
% in {\it Proc. Roy. Soc. London}, A126, 79-84, (1929), 
% and in  {\it Quantum Theory and Measurement}, edited by 
% Wheeler J A and Zurek W H, Princeton Univ. Press, Princeton, NJ (1983)
% 
% 
% \bibitem{mukamel}
% Mukamel D, private communication
% 
% 
% 
% %\bibitem{myers}
% %Myers J M 1997
% %Can a universal quantum computer be fully quantum?
% %Phys. Rev. Lett. {\bf 78}, 1823-1824
% 


\bibitem{neumann}
von Neumann, 
Probabilistic logic and the synthesis of reliable organisms
		  from unreliable components,
in {\em automata studies( Shanon,McCarthy eds)}, 1956 


% \bibitem{nisan}
% Nisan N and Szegedy M, 
% On the degree of Boolean functions as real polynomials, 
% {\em Proc. of the 24th Annual ACM Symposium on 
% Theory of Computing (STOC)} 1992
% 
% \bibitem{nielsen1}
% Nielsen M A and Chuang I L 1997
% Programmable quantum gate arrays,
% Phys. Rev. Lett. {\bf 79}, 321-324
% 
% \bibitem{palma}
% Palma G M, Suominen K-A \& Ekert A K 1996
% Quantum computers and dissipation,
% Proc. Roy. Soc. Lond. A {\bf 452} 567-584

\bibitem{papa}
Papadimitriou C H, 
Computational Complexity, 
Addison-Wesley,
1994

% \bibitem{optic}
% J. Mod. Opt. {\bf 41}, no 12 1994
% Special issue: quantum communication 
% 
% 
% 
% \bibitem{ozhi}
% See also in this context: 
% Y. Ozhigov, 
% Quantum computers cannot speed up iterated applications of a black box, 
% in {\it LANL e-print} quant-ph/9712051,  http://xxx.lanl.gov (1997)
% 
% 
% \bibitem{pellizzari}
% Pellizzari T, Gardiner S A, Cirac J I and Zoller P 1995
% Decoherence, continuous observation, and quantum
% computing: A cavity QED model,
% Phys. Rev. Lett. {\bf 75} 3788-3791

\bibitem{peres}
Peres A 1993
{\em Quantum theory: concepts and methods}
(Kluwer Academic Press, Dordrecht)


%\bibitem{phoenix}
%Phoenix S J D and Townsend P D 1995
%Quantum cryptography: how to beat the code breakers using quantum mechanics,
%Contemp. Phys. {\bf 36}, 165-195


%\bibitem{pippenger}, 
%Pippenger and Stamoulis and Tsitsiklis, 
%On a Lower Bound for the Redundancy of Reliable Networks with Noisy Gates, 
%in {\em IEEETIT: IEEE Transactions on Information Theory}, 
% volume 37, 1991 

%\bibitem{plenio}
%Plenio M B and Knight P L 1996
%Realisitic lower bounds for the factorization time of large numbers
%on a quantum computer,
%Phys. Rev. A {\bf 53}, 2986-2990.


%\bibitem{polkinghorne}
%Polkinghorne J 1994
%{\em Quarks, chaos and Christianity}
%(Triangle, London)

%\bibitem{preskill1}
%J.~Preskill.
%\newblock {\em Proc. Roy. Soc. of London A}, in press.


\bibitem{preskill2}
Preskill J 1997
Fault tolerant quantum computation,
%in {\it LANL e-print} quant-ph/9712048,  http://xxx.lanl.gov (1997), 
to appear in {\it Introduction to Quantum
Computation},  edited by H.-K. Lo, S. Popescu, and T. P. Spiller
\hyperref{http://xxx.lanl.gov/abs/quant-ph/9712048}{}{}
%\hyperURL{http}{xxx.lanl.gov/abs/quant-ph}{9712048}
{http://xxx.lanl.gov/abs/quant-ph/9712048}

\bibitem{preskill3}
Preskill J, Kitaev A, Course notes for Physics 229, Fall 1998, Caltech Univ.,
%http://www.theory.caltech.edu/ people/preskill/ph229
\hyperref{http://www.theory.caltech.edu/people/preskill/ph229}{}{}
%\hyperURL{http}{www.theory.caltech.edu/people/preskill}{ph229}
{http://www.theory.caltech.edu/people/preskill/ph229}


% 
% \bibitem{privman}
% Privman V, Vagner I D and Kventsel G 1997
% Quantum computation in quantum-Hall systems,
% in {\it Phys. Lett. A}, 239 (1998) 141-146
% 
% 
% \bibitem{rabin79} 
%  Rabin M O, 
%    Probabilistic Algorithms
% {\em    Algorithms and Complexity: New Directions 
% and Recent Results}, pp. 21-39,
%    Academic Press, 1976.
% 
% 
% 
% %\bibitem{rains1}
% %Rains E, Nonbinary quantum codes,
% %(quant-ph/9703048)
% 
% \bibitem{rains2}
% Rains E, Hardin R H, Shor P W and  Sloane N J A, 
% A non additive quantum code, 
% Phys.Rev.Lett. 79 953--954 1997

\bibitem{rieffel}
Rieffel E, Polak W
An Introduction to Quantum Computing for Non-Physicists
%{\it LANL e-print} quant-ph/9809016,  http://xxx.lanl.gov (1998),
\hyperref{http://xxx.lanl.gov/abs/quant-ph/9809016}{}{}
%\hyperURL{http}{xxx.lanl.gov/abs/quant-ph}{9809016}
{http://xxx.lanl.gov/abs/quant-ph/9809016}


\bibitem{rsa}
Rivest R, Shamir A and Adleman L 1979
On digital signatures and public-key cryptosystems,
MIT Laboratory for Computer Science, Technical Report, MIT/LCS/TR-212


\bibitem{sakurai}
J.J.Saqurai 
Modern Quantum Mechanics, revised edition.
Addison Wesley, 1994


%\bibitem{shro}
%Schroeder M R 1984
%{\em Number theory in science and communication}
%(Springer-Verlag, Berlin Heidelberg)


%\bibitem{schumacher1}
%Schumacher B 1995
%Quantum coding,
%Phys. Rev. A {\bf 51} 2738-2747


% \bibitem{umesh}
% L. J. Schulman and U. Vazirani
% in {\it LANL e-print} quant-ph/9804060,  http://xxx.lanl.gov (1998), 
% 
% 
% \bibitem{schumacher2}
% Schumacher B W and Nielsen M A 1996
% Quantum data processing and error correction
% Phys Rev A {\bf 54}, 2629
% 
% \bibitem{shamir}
% Shamir A 1979
% Factoring Numbers in O(log(n)) Arithmetic Steps,
% in {\it Information Processing Letters 8(1)}
% 28-31.
% 
% %\bibitem{shankar}
% %Shankar R 1980
% %{\em Principles of quantum mechanics}
% %(Plenum Press, New York)

\bibitem{shannon}
Shannon C E 1948 
A mathematical theory of communication
Bell Syst. Tech. J. {\bf 27} 379; also p. 623



\bibitem{shor1}
Shor P W,
\newblock Polynomial-time algorithms for prime factorization and
discrete logarithms on a quantum computer,
 {\em SIAM J. Comp.}, {\bf 26}, No. 5, pp 1484--1509, 
October 1997




% \bibitem{shor2}
% Shor P W,
% \newblock Scheme for reducing decoherence in quantum computer memory,
% {\em Phys. Rev. A}, 52: 2493-2496, 1995.
% 
% 
% \bibitem{shor3}
%  Shor P W,
% \newblock Fault tolerant quantum computation, 
% In {\em Proceedings of the 37th Symposium on the Foundations of Computer
%   Science}, pages 56--65, Los Alamitos, California, 1996. IEEE press.
% \newblock quant-ph/9605011.
% 
% \bibitem{shor4}
% Shor P W and Laflamme R 1997
% Quantum analog of the MacWilliams identities for classical coding theory,
% Phys. Rev. Lett. {\bf 78} 1600-1602
% 
% 
% 
% \bibitem{jsimon}
% Simon J
% On feasible numbers,
% in 
% {\em Proc. of the 9th Annual ACM Symposium on 
% Theory of Computing (STOC)}  195-207, 1977
% 
% \bibitem{simon}
% Simon D 1994
% On the power of quantum computation,
%  {\em SIAM J. Comp.}, {\bf 26}, No. 5, pp 1474--1483, 
% October 1997
% 
% \bibitem{simon2}
% Simon D 1998, private communication.
% 
% %\bibitem{slepian}
% %Slepian D 1974 ed,
% %{\em Key papers in the development of information theory}
% %(IEEE Press, New York)
% 
% \bibitem{solovay}
% R Solovay and A. C-C Yao,
% preprint, 1996
% 
% %\bibitem{spiller}
% %Spiller T P 1996
% %Quantum information processing: cryptography, 
% %computation and teleportation,
% %Proc. IEEE {\bf 84}, 1719-1746
% 
% 
% \bibitem{steane1}
% Steane A, 
% \newblock Multiple particle interference and quantum error correction,
%  {\em Proc. Roy. Soc. of London A}, 452:2551-2577, 1996.
% 

\bibitem{steane2}
Steane A M,
Error correcting codes in quantum theory,
Phys. Rev. Lett. {\bf 77} 793-797, 1996,
Simple quantum error-correcting codes,
Phys. Rev. A {\bf 54}, 4741-4751, 1996,
Quantum Reed-Muller codes,
submitted to IEEE Trans. Inf. Theory (preprint in {\it LANL e-print} quant-ph/9608026, 
 http://xxx.lanl.gov)
Active stabilization, quantum computation, and quantum state synthesis,
Phys. Rev. Lett. {\bf 78}, 2252-2255, 1997

\bibitem{Steane-97}
Steane A,
Quantum Computation, Reports on Progress in Physics 61 (1998) 117, preprint in 
%{\it LANL e-print} quant-ph/9708022, http://xxx.lanl.gov)
\hyperref{http://xxx.lanl.gov/abs/quant-ph/9708022}{}{}
%\hyperURL{http}{xxx.lanl.gov/abs/quant-ph}{9708022}
{http://xxx.lanl.gov/abs/quant-ph/9708022}

% \bibitem{steane6}
% Steane A M 
% The ion trap quantum information processor,
% Appl. Phys. B {\bf 64} 623-642 1997
% 
% 
% \bibitem{steane7}
% Steane A M 
% Space, time, parallelism and noise requirements for reliable
% quantum computing,
% in {\it Fortsch. Phys.} {\bf 46} (1998) 443-458
% 
% 
% \bibitem{stern1}
%  Stern A, Aharonov Y and Imry Y, "Phase uncertainty and loss of
% interference: a general picture" Phys. Rev. A {\bf 41}, 3436 (1990).
% and
%  "Dephasing of interference
% by a back reacting environment" in "Quantum coherence" ed. J. Anandan,
% World Scientific, 1990.
% 
% 
% %\bibitem{szilard}
% %Szilard L 1929 Z. Phys. {\bf 53} 840;
% %translated in Wheeler and Zurek (1983).
% 
% %\bibitem{teich}
% %Teich W G, Obermayer K and Mahler G 1988
% %Structural basis of multistationary quantum systems II. Effective
% %few-particle dynamics,
% %Phys. Rev. B {\bf 37} 8111-8121
% 
% \bibitem{tittel}
% W. Tittel, J. Brendel, B. Gisin, T. Herzog, H. Zbinden, N. Gisin
% Experimental demonstration of quantum-correlations over more than 10
%      kilometers, 
% in {\it Phys. Rev. A},{\bf  57}, 3229 (1998)

\bibitem{toffoli}
Toffoli T 1980
Reversible computing,
in {\em Automata, Languages and Programming}, Seventh Colloquium,
Lecture Notes in Computer Science, Vol. 84, de Bakker J W and
van Leeuwen J, eds, (Springer) 632-644



% \bibitem{turchette}
% Turchette Q A, Hood C J, Lange W, Mabushi H and Kimble H J 1995
% Measurement of conditional phase shifts for quantum logic,
% Phys. Rev. Lett. {\bf 75} 4710-4713


\bibitem{turing}
Turing A M 1936
On computable numbers, with an application to the
Entschneidungsproblem,
Proc. Lond. Math. Soc. Ser. 2 {\bf 42}, 230 ; see also
Proc. Lond. Math. Soc. Ser. 2 {\bf 43}, 544 


% \bibitem{unroh1}
%  Unruh W G,
% \newblock Maintaining coherence in quantum computers,
% {\em Phys. Rev. A}, 51:992--997, 1995.
% 
% \bibitem{valiant}
% Valiant, unpublished
% 
% \bibitem{valiant2}
% Valiant L. G, 
% Negation can be exponentially powerful.
% {\em Theoretical Computer Science,}
%        12(3):303-314, November 1980. 
% 
% \bibitem{valiant3}
% Valiant L G and  Vazirani V V.
%  NP is as easy as detecting unique solutions. 
% {\em Theoretical
%        Computer Science,} 47(1):85-93, 1986
% 
% \bibitem{vedral}
% Vedral V, Barenco A and Ekert A 1996 
% Quantum networks for elementary arithmetic operations,
% Phys. Rev. A {\bf 54} 147-153
% 
% 
% \bibitem{vergis} 
% Vergis A, Steiglitz K and Dickinson B , "The Complexity of Analog
% Computation", Math. Comput. Simulation 28, pp. 91-113. 1986
% 
% 
% %\bibitem{walsworth}
% %Walsworth et.al. Phys rev. lett. 64,2599, 1990.
% 
% \bibitem{warren1}
%  Warren W S,
% \newblock {\em Science}, 277:1688--1698, 1997.
% 
% 
% 
% 
% \bibitem{watrous2}
% Watrous J, 
% On one Dimensional quantum cellular automata, 
% {\em Complex Systems} {\bf 5} (1), pp 19--30, 1991
% 
% 
% 
% %\bibitem{weinfurter}
% %Weinfurter H 1994
% %Experimental Bell-state analysis,
% %Europhys. Lett. {\bf 25} 559-564
% 
% 
% 
% %\bibitem{weisner1}
% %Wiesner S 1983
% %Conjugate coding,
% %SIGACT News {\bf 15} 78-88
% 
% 
% \bibitem{weisner2}
% Wiesner S 1996
% Simulations of many-body quantum systems by a quantum computer
% in {\it LANL e-print} quant-ph/9603028, 
%  http://xxx.lanl.gov (1996)
% 

\bibitem{wheeler}
Wheeler J A and Zurek W H, eds, 1983
{\em Quantum theory and measurement}
(Princeton Univ. Press, Princeton, NJ)


% \bibitem{wigderson}
% A. Wigderson, private communication
% 
% %\bibitem{wineland}
% %Wineland D J, Monroe C, Itano W M, Leibfried D, King B, and
% %Meekhof D M 1997
% %Experimental issues in coherent quantum-state manipulation of trapped
% %atomic ions,
% %preprint, submitted to Rev. Mod. Phys.


\bibitem{wootters}
Wootters W K and Zurek W H 1982
A single quantum cannot be cloned,
Nature {\bf 299}, 802

% \bibitem{wootters1}
% Wootters W K
% A Measure of the Distinguishability of Quantum States
% {\em 
% Quantum Optics, General Relativity, and Measurement}
% eds: Marlan O. Scully and Pierre Meystre,
% 145--154, Plenum Press, New York, 1983
% 
% 
% 
% \bibitem{yao}
% Yao A C-C,
% Quantum circuit complexity,
% in {\em 33th Annual Symposium on Foundations of Computer Science(FOCS)}, (1993) pp. 352--361
% 
% 
% 
% 
% \bibitem{zalka1}
% Zalka C, 
% Efficient simulation of quantum systems by quantum computers
% {\em Proc. Roy. Soc. of London A}, in press,
% in {\it LANL e-print} quant-ph/9603026,
%  http://xxx.lanl.gov (1996)



\bibitem{zalka2}
Zalka C, 
Grover's quantum searching algorithm is optimal,
%in {\it LANL e-print} quant-ph/9711070 http://xxx.lanl.gov (1997)
\hyperref{http://xxx.lanl.gov/abs/quant-ph/9711070}{}{}
%\hyperURL{http}{xxx.lanl.gov/abs/quant-ph}{9711070}
{http://xxx.lanl.gov/abs/quant-ph/9711070}

%\bibitem{zbinden}
%Zbinden H, Gautier J D, Gisin N, Huttner B, Muller A, Tittle W 1997
%Interferometry with Faraday mirrors for quantum cryptography,
%Elect. Lett. {\bf 33}, 586-588

\bibitem{zurek1}
 Zurek W H,  
Decoherence and the transition from quantum to classical, 
Physics Today 44(10), October, 1991 36--44. 


%\bibitem{zurek2}
%Zurek W H 1989 
%Thermodynamic cost of computation, algorithmic complexity and the
%information metric,
%Nature {\bf 341} 119-124



\end{thebibliography}

\tthdump{\hyperlink{Our general topics:}{\hfil To top $\leftarrow$}}
%%tth:{\special{html: <a href="\#Top of file">       Back to top of file</a>}}

\pagedone

\sectionhead{On-line references}


Some of the references listed above are available on line.  They are listed again here for easy access:

%\bibitem{abrams2}
Abrams D S and Lloyd S, 
Non-Linear Quantum Mechanics implies Polynomial Time 
solution for NP-complete and $\#$P problems,
%in {\it LANL e-print} quant-ph/9801041, http://xxx.lanl.gov (1998)
\hyperref{http://xxx.lanl.gov/abs/quant-ph/9801041}{}{}
%\hyperURL{http}{xxx.lanl.gov/abs/quant-ph}{9801041}
{http://xxx.lanl.gov/abs/quant-ph/9801041}


%% %\bibitem{aharonov5}
%% Aharonov D, Beckman D, Chuang I and  Nielsen M,
%% What Makes Quantum Computers Powerful? 
%% \hyperref{http://wwwcas.phys.unm.edu/\~mnielsen/science.html}{}{}
%% %\hyperURL{http}{wwwcas.phys.unm.edu/~mnielsen}{science.html}
%% {http://wwwcas.phys.unm.edu/\~mnielsen/science.html}
%% % 
%% % 

%\bibitem{aharonov6}
Aharonov D, 
Quantum Computation,
%in {\it LANL e-print} quant-ph/9812037, http://xxx.lanl.gov (1998)
\hyperref{http://xxx.lanl.gov/abs/quant-ph/9812037}{}{}
%\hyperURL{http}{xxx.lanl.gov/abs/quant-ph}{9812037}
{http://xxx.lanl.gov/abs/quant-ph/9812037}

%\bibitem{decoherence2}
 Chuang I L, Laflamme R and Paz J P, 
Effects of Loss and Decoherence on a Simple Quantum Computer,
%in {\it LANL e-print} quant-ph/9602018,  http://xxx.lanl.gov (1996)
\hyperref{http://xxx.lanl.gov/abs/quant-ph/9602018}{}{}
%\hyperURL{http}{xxx.lanl.gov/abs/quant-ph}{9602018}
{http://xxx.lanl.gov/abs/quant-ph/9602018}

%\bibitem{grover2}
 Grover L K, 
A framework for fast quantum mechanical algorithms,
%in {\it LANL e-print} quant-ph/9711043,  http://xxx.lanl.gov (1997)
\hyperref{http://xxx.lanl.gov/abs/quant-ph/9711043}{}{}
%\hyperURL{http}{xxx.lanl.gov/abs/quant-ph}{9711043}
{http://xxx.lanl.gov/abs/quant-ph/9711043}

%\bibitem{grover4}
 Grover L K, 
A fast quantum mechanical algorithm for estimating the median,
%in {\it LANL e-print} quant-ph/9607024,  http://xxx.lanl.gov (1997)
\hyperref{http://xxx.lanl.gov/abs/quant-ph/9607024}{}{}
%\hyperURL{http}{xxx.lanl.gov/abs/quant-ph}{9607024}
{http://xxx.lanl.gov/abs/quant-ph/9607024}


%\bibitem{knill4}
Knill E, Laflamme R and Zurek W H 1997
Resilient quantum computation: error models and thresholds
%in {\it LANL e-print} quant-ph/9702058,  http://xxx.lanl.gov (1997)
\hyperref{http://xxx.lanl.gov/abs/quant-ph/9702058}{}{}
%\hyperURL{http}{xxx.lanl.gov/abs/quant-ph}{9702058}
{http://xxx.lanl.gov/abs/quant-ph/9702058}

\pagedone


%\bibitem{preskill2}
Preskill J 1997
Fault tolerant quantum computation,
%in {\it LANL e-print} quant-ph/9712048,  http://xxx.lanl.gov (1997), 
to appear in {\it Introduction to Quantum
Computation},  edited by H.-K. Lo, S. Popescu, and T. P. Spiller
\hyperref{http://xxx.lanl.gov/abs/quant-ph/9712048}{}{}
%\hyperURL{http}{xxx.lanl.gov/abs/quant-ph}{9712048}
{http://xxx.lanl.gov/abs/quant-ph/9712048}

%\bibitem{preskill3}
Preskill J, Kitaev A, Course notes for Physics 229, Fall 1998, Caltech Univ.,
\hyperref{http://www.theory.caltech.edu/people/preskill/ph229}{}{}
%\hyperURL{http}{www.theory.caltech.edu/people/preskill}{ph229}
{http://www.theory.caltech.edu/people/preskill/ph229}


%\bibitem{rieffel}
Rieffel E, Polak W
An Introduction to Quantum Computing for Non-Physicists
%{\it LANL e-print} quant-ph/9809016,  http://xxx.lanl.gov (1998),
\hyperref{http://xxx.lanl.gov/abs/quant-ph/9809016}{}{}
%\hyperURL{http}{xxx.lanl.gov/abs/quant-ph}{9809016}
{http://xxx.lanl.gov/abs/quant-ph/9809016}

%\bibitem{Steane-97}
Steane A,
Quantum Computation, Reports on Progress in Physics 61 (1998) 117,
%(preprint in {\it LANL e-print} quant-ph/9708022, http://xxx.lanl.gov)
\hyperref{http://xxx.lanl.gov/abs/quant-ph/9708022}{}{}
%\hyperURL{http}{xxx.lanl.gov/abs/quant-ph}{9708022}
{http://xxx.lanl.gov/abs/quant-ph/9708022}

%\bibitem{zalka2}
Zalka C, 
Grover's quantum searching algorithm is optimal,
%in {\it LANL e-print} quant-ph/9711070http://xxx.lanl.gov (1997)
\hyperref{http://xxx.lanl.gov/abs/quant-ph/9711070}{}{}
%\hyperURL{http}{xxx.lanl.gov/abs/quant-ph}{9711070}
{http://xxx.lanl.gov/abs/quant-ph/9711070}

\tthdump{\hyperlink{Our general topics:}{\hfil To top $\leftarrow$}}
%%tth:{\special{html: <a href="\#Top of file">       Back to top of file</a>}}


\end{document}
