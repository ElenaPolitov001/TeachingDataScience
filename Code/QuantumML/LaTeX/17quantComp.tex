\documentclass[table,serif,ignorenonframetext,xcolor=dvipsnames]{beamer} 

\usepackage{mathpazo} \usepackage{pifont}
\usepackage{amssymb}
\usepackage{amstext} 
\usepackage{amsmath}
\usepackage{latexsym}
\usepackage{xspace,multirow}
\usepackage{booktabs}%\usepackage{paralist}
%\usepackage{epic,autograph}

%\usecolortheme[named=ProcessBlue]{structure} 

%\usecolortheme[named=OliveGreen]{structure} 
%\usecolortheme[named=Periwinkle]{structure} 
%\setbeamercolor{frametitle}{fg=blue,bg=yellow}
%\usecolortheme[named=Plum]{structure} 
%\usetheme{AnnArbor}

%\usetheme{metropolis}
\usetheme{Madrid}\usecolortheme[named=Maroon]{structure} 
%\usecolortheme{beaver}

%\usetheme{Madrid}
%\usetheme{CambridgeUS}
%\usetheme{Singapore}
%\setbeamersize{sidebar width left=.2\paperwidth}
%\usetheme[height=7mm]{Rochester} 
%\usepackage{logicthemelive}
%\usetheme{Warsaw}

\setbeamertemplate{blocks}[rounded]
%%%%%%%%%%%%%%%%%%%%%%%%%%%%%%%%%%%%%%%%%%%%%%%%%%%%%%%%%%%%%%%%%%
%% OPTIONS
%%%%%%%%%%%%%%%%%%%%%%%%%%%%%%%%%%%%%%%%%%%%%%%%%%%%%%%%%%%%%%%%%%
%\usepackage{pgfpages}
%\pgfpagesuselayout{8 on 1}[letterpaper,border shrink=5mm]
%\nofiles

\usepackage{pgfpages}
\pgfpagesuselayout{4 on 1}[letterpaper,landscape,border shrink=5mm]
\nofiles

%\usepackage{pgfpages}
%\pgfpagesuselayout{2 on 1}[letterpaper,border shrink=5mm]
%\nofiles
%
\setbeameroption{show notes}
%%%%%%%%%%%%%%%%%%%%%%%%%%%%%%%%%%%%%%%%%%%%%%%%%%%%%%%%%%%%%%%%%%
%%%%%%%%%%%%%%%%%%%%%%%%%%%%%%%%%%%%%%%%%%%%%%%%%%%%%%%%%%%%%%%%%%
\setbeamertemplate{navigation symbols}{} 
%\setbeamercolor{math text}{fg=RawSienna}


%\setbeamertemplate{frametitle} 
%{ 
%\begin{centering} 
%\color{red} 
%\textbf{\insertframetitle} 
%\par 
%\end{centering} 
%} 


%%%%%%%%%%%%%%%%%%%%%%%%%%%%%%%%%%%%%%%%%%%%%%%%%%%%%%%%%%%%%%%%%%
%%%%%%%%%%%%%%%%%%%%%%%%%%%%%%%%%%%%%%%%%%%%%%%%%%%%%%%%%%%%%%%%%%


%\input{Crayola}
\definecolor{darkgreen}{cmyk}{.36,0,.42,.45} %1,0,.4,.5}
\definecolor{darkergreen}{cmyk}{.66,0,.72,.75} %1,0,.4,.5}
\definecolor{verygreen}{cmyk}{.41,.0,.21,.5} %1,0,.4,.5}
\definecolor{darkblue}{cmyk}{.41,.21,0,.49}
\definecolor{paleyellow}{cmyk}{0,0,0.25,0}
\definecolor{paleorg}{cmyk}{0,0.07,0.11,0}
\definecolor{greenish}{cmyk}{.65,0,.65,0}
\definecolor{gbish}{cmyk}{.7,0,.3,0}
\definecolor{gyish}{cmyk}{.4,0.25,0,0}
\definecolor{redish}{cmyk}{0,.82,.82,.06} %0,1,1,.3}
\definecolor{fog}{cmyk}{.43,.06,.00,.21}
\definecolor{seaweed}{cmyk}{.28,.0,.02,.62}
\definecolor{seablue}{cmyk}{.41,.21,.0,.49}
\definecolor{Gray}{cmyk}{0,0,0,0.50}



\newcommand{\marktrans}{\begin{center}
$\Diamond$ \quad $\Diamond$\quad $\Diamond$\quad $\Diamond$\quad $\Diamond$
\end{center}}     
\newcommand{\interesting}[1]{\colorbox{Red}{\color{White}$\mathbf{#1}$}}

\parindent 0pt

\setlength{\fboxrule}{1pt}



\newcommand{\CARD}{\textbf{Card}}
\newcommand{\IFF}{\textcolor{red}{iff}\xspace}
\newcommand{\eqdef}{%
   \ensuremath{\mathrel{\textcolor{red}{\stackrel{\rm def}{=}}}}\xspace}
\newcommand{\eqvdef}{%
   \ensuremath{\mathrel{\textcolor{red}{\stackrel{\rm def}{\iff}}}}\xspace}
\newcommand{\redflag}{\textcolor{Black}{\rule{2pt}{1cm}}%
   \textcolor{red}{\rule[.5cm]{.75cm}{.5cm}}}

\newcommand{\yellowbox}[2]{\par\colorbox{paleyellow}{\begin{minipage}{#1}
#2\end{minipage}}}

\newcommand{\bleu}[1]{\textbf{\textcolor{blue}{#1}}}
\newcommand{\vertt}[1]{{\textcolor{PineGreen}{#1}}}
\newcommand{\rouge}[1]{{\textcolor{red}{#1}}}
\newcommand{\brun}[1]{\textcolor{Brown}{#1}}
\newcommand{\orchid}[1]{\textcolor{Orchid}{#1}}

\newenvironment{Eqnarray}{\color{Brown}\begin{eqnarray*}}{\end{eqnarray*}\par}


\newcommand{\Math}[1]{\ensuremath{#1}}
\newcommand{\redtri}{\structure{\footnotesize$\blacktriangleright$}\xspace}
\newcommand{\Comment}[1]{\vertt{\texttt{(*} #1 \texttt{*)}}}

\newcommand{\displaybox}[1]{\fcolorbox{red}{White}{#1}}


\newcommand{\bluetri}{\textcolor{Blue}{$\blacktriangleright$}}
\newcommand{\bluebul}{\textcolor{Blue}{$\bullet$}}
\newcommand{\trsubseteq}{\mathrel{\sqsubseteq_{\rm tr}}}
\newcommand{\ints}{\ensuremath{\mathbb{Z}}\xspace}
\newcommand{\nat}{\ensuremath{\mathbb{N}}\xspace}
\newcommand{\intp}{\ensuremath{\mathbb{Z}^{+}}\xspace}
\newcommand{\defeq}{%
  {\rouge{\ensuremath{\mathbin{=_{\!\rm def}}}}}}
\newcommand{\defeqv}{%
  \rouge{\ensuremath{\mathbin{\equiv_{\!\rm def}}}}}
\newcommand{\defiff}{\rouge{\ensuremath{\iff_{\!\rm def}}}}
\renewcommand{\implies}{\rouge{\ensuremath{\Longrightarrow}}\xspace}

\newcommand{\pageheading}[1]{\begin{center}\huge%
   \textbf{\textcolor{red}{\uppercase{#1}}}\end{center}} 
   
   
\newcommand{\notdivides}{\mathbin{\not\mkern4mu\mid}}   
\newcommand{\why}{\Emph{(Why?)}}   
\newcommand{\thus}{\structure{\Large$\therefore$}\xspace}
\newcommand{\QED}{\hfill\rouge{QED}}
\newcommand{\prfb}{\hfill\orchid{proof on board}}

\newcommand{\mypmod}[1]{\;(\bmod\,#1)}
\newcommand{\ranin}{\stackrel{\rm ran}{\in}}
\newcommand{\Quad}[1]{\hspace*{#1em}}
\newcommand{\set}[1]{\{\,#1\,\}}
\newcommand{\Card}[1]{\Vert #1\Vert}
\newcommand{\of}{\colon}
\newcommand{\suchthat}{\mathrel{\,\stackrel{\rule{0.03em}{0.5ex}}%
{\rule[-.1ex]{0.03em}{0.5ex}}\,}}
\newcommand{\ff}{\mathbb{F}}
\renewcommand{\phi}{\varphi}
\newcommand{\msb}{\textrm{msb}}
\newcommand{\lsb}{\textrm{lsb}}
\newcommand{\concat}{\mathbin{||}}
\newcommand{\sig}{\mathrm{sig}}
\newcommand{\ver}{\mathrm{ver}}

\newcommand{\DES}{\mathrm{DES}}
\newcommand{\IP}{\mathrm{IP}}
\newcommand{\GSO}{\textrm{GSO}}
\newcommand{\PI}{\textrm{PI}}
\newcommand{\PO}{\textrm{PO}}
\newcommand{\DS}{\textrm{DS}}
\newcommand{\Auth}{\textsf{Auth}}
\newcommand{\suchThat}{\rouge{$\ni$}}
\newcommand{\floor}[1]{\lfloor#1\rfloor}
\newcommand{\ceiling}[1]{\lceil#1\rceil}
\newcommand{\sqdot}{\rule{0.5mm}{0.5mm}}
\newcommand{\lam}[1]{\lambda #1\,\sqdot\,}
\newcommand{\SC}{\mathcal{C}}
\newcommand{\SH}{\mathcal{H}}
\newcommand{\pair}[1]{\langle\,#1\,\rangle}

\newenvironment{alg}{%
  \begin{beamerboxesrounded}[upper=boo,lower=foo,shadow=true]{}\sf\small}{%
  \end{beamerboxesrounded}}
  
\newenvironment{examp}{%
  \begin{beamerboxesrounded}[upper=boo,lower=exmp,shadow=false]{}\sf\small}{%
  \end{beamerboxesrounded}}
  
  

\newenvironment{Alg}[1]{%
  \begin{beamerboxesrounded}[upper=boo,lower=foo,shadow=true]{\textsf{\textbf{#1}}}\sf\small}{%
  \end{beamerboxesrounded}}

%%%%%%%%%%%%%%%%%%%%%%%%%%%%%%%%%%%%%%%%%%%%%%%%%%%%%%%%%%%%%%%%%%
%%%%%%%%%%%%%%%%%%%%%%%%%%%%%%%%%%%%%%%%%%%%%%%%%%%%%%%%%%%%%%%%%%

\title{Quantum Computing}
\author[Royer]{Jim Royer}
\date{\today}
\institute[CIS 428/628]{\emph{CIS 428/628: Introduction to Cryptography}}
\author[Crypto]{Jim Royer}


%%%%%%%%%%%%%%%%%%%%%%%%%%%%%%%%%%%%%%%%%%%%%%%%%%%%%%%%%%%%%%%%%%
%%%%%%%%%%%%%%%%%%%%%%%%%%%%%%%%%%%%%%%%%%%%%%%%%%%%%%%%%%%%%%%%%%
%
% Document Type: LaTeX 2e
%\documentclass[12pt,fleqn,landscape]{article}

%\usepackage{amssymb}  \usepackage{latexsym}
%\usepackage{amstext} \usepackage{amsmath}
%\usepackage{mystile}  
%\usepackage{xspace}
%\usepackage{fancyheadings} 
%\usepackage{color}
%\usepackage{graphics} \usepackage{hyperref}
%\usepackage{graphicx} 

%\renewcommand{\baselinestretch}{1.1}

%  \topmargin	  -3cm  
%  \textwidth	  25cm		
%  \footskip	  1.5cm
%  \textheight	  18cm	
%  \oddsidemargin  -1cm	
%  \evensidemargin -1cm 
%  \columnseprule  0.4pt
%  \special{papersize=11in,8.5in}
%  \hfuzz 10pt  \vfuzz 10pt

%\pagestyle{fancy}

%\headrulewidth 0pt \footrulewidth 0pt

%\newcommand{\bsl}[1]{\textbf{\textsl{#1}}}

%\lfoot{}
%\cfoot{\textcolor{Sepia}{\Large\bsl{--- \thepage\/ ---}}}
%\rfoot{}

%\addtolength{\headheight}{10pt}

%
%\hfuzz=10pt \vfuzz=10pt

%\input{macs} 
%\input{crayola}

%
%\definecolor{darkgreen}{cmyk}{.36,0,.42,.45} %1,0,.4,.5}
%\definecolor{darkergreen}{cmyk}{.66,0,.72,.75} %1,0,.4,.5}
%\definecolor{verygreen}{cmyk}{.41,.0,.21,.5} %1,0,.4,.5}
%\definecolor{darkblue}{cmyk}{.41,.21,0,.49}
%\definecolor{paleyellow}{cmyk}{0,0,0.25,0}
%\definecolor{paleorg}{cmyk}{0,0.07,0.11,0}
%\definecolor{greenish}{cmyk}{.65,0,.65,0}
%\definecolor{gbish}{cmyk}{.7,0,.3,0}
%\definecolor{gyish}{cmyk}{.4,0.25,0,0}
%\definecolor{redish}{cmyk}{0,.82,.82,.06} %0,1,1,.3}
%\definecolor{fog}{cmyk}{.43,.06,.00,.21}
%\definecolor{seaweed}{cmyk}{.28,.0,.02,.62}
%\definecolor{seablue}{cmyk}{.41,.21,.0,.49}
%\definecolor{Gray}{cmyk}{0,0,0,0.50}

%\setlength{\unitlength}{1mm}\thicklines

%\newcommand{\marktrans}{\begin{center}
%$\Diamond$ \quad $\Diamond$\quad $\Diamond$\quad $\Diamond$\quad $\Diamond$
%\end{center}}     

%\renewcommand{\colon}{\mathpunct{:}}
%\parindent 0pt

%\setlength{\fboxrule}{1pt}

%

%\newcommand{\CARD}{\textbf{Card}}
%\newcommand{\IFF}{\textcolor{Red}{iff}\xspace}
%\newcommand{\eqdef}{%
%   \ensuremath{\mathrel{\textcolor{Red}{\stackrel{\rm def}{=}}}}\xspace}
%\newcommand{\eqvdef}{%
%   \ensuremath{\mathrel{\textcolor{Red}{\stackrel{\rm def}{\iff}}}}\xspace}
%\newcommand{\redflag}{\textcolor{Black}{\rule{2pt}{1cm}}%
%   \textcolor{Red}{\rule[.5cm]{.75cm}{.5cm}}}

%\newcommand{\yellowbox}[2]{\par\colorbox{paleyellow}{\begin{minipage}{#1}
%#2\end{minipage}}}

%\newcommand{\bleu}[1]{{\textcolor{blue}{#1}}}
%\newcommand{\vertt}[1]{\textcolor{PineGreen}{#1}}
%\newcommand{\rouge}[1]{{\textcolor{Red}{#1}}}
%\newcommand{\brun}[1]{\textcolor{Brown}{#1}}
%\newcommand{\orchid}[1]{\textcolor{Orchid}{#1}}

%\newenvironment{Eqnarray}{\color{Brown}\begin{eqnarray*}}{\end{eqnarray*}\par}

%
%\newcommand{\Math}[1]{{\color{Brown}{${#1}$}}}
%\newcommand{\redtri}{\textcolor{Red}{$\blacktriangleright$}}
%\renewcommand{\labelitemi}{\redtri}
%\renewcommand{\labelitemii}{\textcolor{Green}{$\bullet$}}

%\renewcommand{\descriptionlabel}[1]%
%	{\hspace{\labelsep}\bleu{#1}}
%\renewcommand{\Comment}[1]{\vertt{\texttt{(*} #1 \texttt{*)}}}

%\newcommand{\displaybox}[1]{\fcolorbox{Red}{White}{#1}}

%
%\newcommand{\bluetri}{\textcolor{Blue}{$\blacktriangleright$}}
%\newcommand{\bluebul}{\textcolor{Blue}{$\bullet$}}
%\newcommand{\trsubseteq}{\mathrel{\sqsubseteq_{\rm tr}}}
%\newcommand{\ints}{\ensuremath{\mathbb{Z}}\xspace}
%\newcommand{\nat}{\ensuremath{\mathbb{N}}\xspace}
%\newcommand{\intp}{\ensuremath{\mathbb{Z}^{+}}\xspace}
%\renewcommand{\defeq}{%
%  {\ensuremath{\mathbin{\rouge{=_{\!\rm def}}}}}}
%\newcommand{\defeqv}{%
%  \rouge{\ensuremath{\mathbin{\equiv_{\!\rm def}}}}}
%\renewcommand{\defiff}{\rouge{\ensuremath{\iff_{\!\rm def}}}}
%\renewcommand{\implies}{\rouge{\ensuremath{\Longrightarrow}}\xspace}

%\newcommand{\pageheading}[1]{\begin{center}\huge%
%   \textbf{\textcolor{Red}{\uppercase{#1}}}\end{center}} 
%\newenvironment{frame}[1]{%
%   \newpage\pageheading{#1}\begin{trivlist}\item[]}%
%   {\end{trivlist}}
%   
\newcommand{\True}{\mathrm{True}}
\newcommand{\Forged}{\mathrm{Forged}}
%\newcommand{\notdivides}{\mathbin{\not\mkern4mu\mid}}   
%\newcommand{\why}{\rouge{(Why?)}}   
%\newcommand{\thus}{\rouge{\Huge$\therefore$}\xspace}
%\newcommand{\QED}{\hfill\rouge{QED}}
%\newcommand{\prfb}{\hfill\orchid{proof on board}}

%\newcommand{\mypmod}[1]{\;(\bmod\,#1)}
%\newcommand{\ranin}{\stackrel{\rm \rouge{ran}}{\in}}
%\newcommand{\ff}{\mathbb{F}}
%\newcommand{\msb}{\textrm{msb}}
%\newcommand{\lsb}{\textrm{lsb}}
%\renewcommand{\concat}{\mathbin{||}}
%\newcommand{\sig}{\textrm{sig}}
%\newcommand{\ver}{\textrm{ver}}
%\newcommand{\suchThat}{\rouge{$\ni$}}

%\newcommand{\GSO}{\textrm{GSO}}
%\newcommand{\PI}{\textrm{PI}}
%\newcommand{\PO}{\textrm{PO}}
%\newcommand{\DS}{\textrm{DS}}
\newcommand{\Prob}{\mathrm{Prob}}
\newcommand{\prob}{\mathrm{Prob}}
\newcommand{\state}[1]{{\color{red}\mathopen{\vert}}#1{\color{red}\mathclose{\rangle}}}
\newcommand{\legendre}[2]{\genfrac{(}{)}{}{}{#1}{#2}}
%{#1 \overwithdelims () #2}}
\newcommand{\QR}{\mathbb{QR}}
\newcommand{\PQR}{\overline{\mathbb{QR}}}

%%%%%%%%%%%%%%%%%%%%%%%%%%%%%%%%%%%%%%%%%%%%%%%%%%%%%%%%%%%%%%%%%%
%%%%%%%%%%%%%%%%%%%%%%%%%%%%%%%%%%%%%%%%%%%%%%%%%%%%%%%%%%%%%%%%%%
\setbeamercolor{foo}{bg=Yellow!10!normal text.bg}
\setbeamercolor{boo}{bg=Yellow!30!normal text.bg}
%\begin{document}
%% Document Type: LaTeX 2e
%\documentclass[12pt,fleqn,landscape]{article}

%\usepackage{amssymb}  \usepackage{latexsym}
%\usepackage{amstext} \usepackage{amsmath}
%\usepackage{mystile}  \usepackage{xspace}
%\usepackage{fancyheadings} \usepackage{color}
%\usepackage{graphics} \usepackage{hyperref}
%\usepackage{graphicx} 

%\renewcommand{\baselinestretch}{1.1}

%  \topmargin	  -3cm  
%  \textwidth	  25cm		
%  \footskip	  1.5cm
%  \textheight	  18cm	
%  \oddsidemargin  -1cm	
%  \evensidemargin -1cm 
%  \columnseprule  0.4pt
%  \special{papersize=11in,8.5in}
%  \hfuzz 10pt  \vfuzz 10pt

%\pagestyle{fancy}

%\headrulewidth 0pt \footrulewidth 0pt

%\newcommand{\bsl}[1]{\textbf{\textsl{#1}}}

%\lfoot{}
%\cfoot{\textcolor{Sepia}{\Large\bsl{--- \thepage\/ ---}}}
%\rfoot{}

%\addtolength{\headheight}{10pt}

%
%\hfuzz=10pt \vfuzz=10pt

%\input<{macs} 
%\input{crayola}

%
%\definecolor{darkgreen}{cmyk}{.36,0,.42,.45} %1,0,.4,.5}
%\definecolor{darkergreen}{cmyk}{.66,0,.72,.75} %1,0,.4,.5}
%\definecolor{verygreen}{cmyk}{.41,.0,.21,.5} %1,0,.4,.5}
%\definecolor{darkblue}{cmyk}{.41,.21,0,.49}
%\definecolor{paleyellow}{cmyk}{0,0,0.25,0}
%\definecolor{paleorg}{cmyk}{0,0.07,0.11,0}
%\definecolor{greenish}{cmyk}{.65,0,.65,0}
%\definecolor{gbish}{cmyk}{.7,0,.3,0}
%\definecolor{gyish}{cmyk}{.4,0.25,0,0}
%\definecolor{redish}{cmyk}{0,.82,.82,.06} %0,1,1,.3}
%\definecolor{fog}{cmyk}{.43,.06,.00,.21}
%\definecolor{seaweed}{cmyk}{.28,.0,.02,.62}
%\definecolor{seablue}{cmyk}{.41,.21,.0,.49}
%\definecolor{Gray}{cmyk}{0,0,0,0.50}

%\setlength{\unitlength}{1mm}\thicklines

%\newcommand{\marktrans}{\begin{center}
%$\Diamond$ \quad $\Diamond$\quad $\Diamond$\quad $\Diamond$\quad $\Diamond$
%\end{center}}     

%\renewcommand{\colon}{\mathpunct{:}}
%\parindent 0pt

%\setlength{\fboxrule}{1pt}

%

%\newcommand{\CARD}{\textbf{Card}}
%\newcommand{\IFF}{\textcolor{Red}{iff}\xspace}
%\newcommand{\eqdef}{%
%   \ensuremath{\mathrel{\textcolor{Red}{\stackrel{\rm def}{=}}}}\xspace}
%\newcommand{\eqvdef}{%
%   \ensuremath{\mathrel{\textcolor{Red}{\stackrel{\rm def}{\iff}}}}\xspace}
%\newcommand{\redflag}{\textcolor{Black}{\rule{2pt}{1cm}}%
%   \textcolor{Red}{\rule[.5cm]{.75cm}{.5cm}}}

%\newcommand{\yellowbox}[2]{\par\colorbox{paleyellow}{\begin{minipage}{#1}
%#2\end{minipage}}}

%\newcommand{\bleu}[1]{\textbf{\textcolor{blue}{#1}}}
%\newcommand{\vertt}[1]{\textbf{\textcolor{PineGreen}{#1}}}
%\newcommand{\rouge}[1]{{\textcolor{Red}{#1}}}
%\newcommand{\brun}[1]{\textcolor{Brown}{#1}}
%\newcommand{\orchid}[1]{\textcolor{Orchid}{#1}}

%\newenvironment{Eqnarray}{\color{Brown}\begin{eqnarray*}}{\end{eqnarray*}\par}

%
%\newcommand{\Math}[1]{{\color{Brown}{${#1}$}}}
%\newcommand{\redtri}{\textcolor{Red}{$\blacktriangleright$}}
%\renewcommand{\labelitemi}{\redtri}
%\renewcommand{\labelitemii}{\textcolor{Green}{$\bullet$}}

%\renewcommand{\descriptionlabel}[1]%
%{\hspace{\labelsep}\bleu{#1}}
%\renewcommand{\Comment}[1]{\vertt{\texttt{(*} #1 \texttt{*)}}}

%\newcommand{\displaybox}[1]{\fcolorbox{Red}{White}{#1}}

%
%\newcommand{\bluetri}{\textcolor{Blue}{$\blacktriangleright$}}
%\newcommand{\bluebul}{\textcolor{Blue}{$\bullet$}}
%\newcommand{\trsubseteq}{\mathrel{\sqsubseteq_{\rm tr}}}
%\newcommand{\ints}{\ensuremath{\mathbb{Z}}\xspace}
%\newcommand{\nat}{\ensuremath{\mathbb{N}}\xspace}
%\newcommand{\intp}{\ensuremath{\mathbb{Z}^{+}}\xspace}
%\renewcommand{\defeq}{%
%  {\ensuremath{\mathbin{\rouge{=_{\!\rm def}}}}}}
%\newcommand{\defeqv}{%
%  \rouge{\ensuremath{\mathbin{\equiv_{\!\rm def}}}}}
%\renewcommand{\defiff}{\rouge{\ensuremath{\iff_{\!\rm def}}}}
%\renewcommand{\implies}{\rouge{\ensuremath{\Longrightarrow}}\xspace}

%\newcommand{\pageheading}[1]{\begin{center}\huge%
%   \textbf{\textcolor{Red}{\uppercase{#1}}}\end{center}} 
%\newenvironment{slide}[1]{%
%   \newpage\pageheading{#1}\begin{trivlist}\item[]}%
%   {\end{trivlist}}
%   
%   
%\newcommand{\notdivides}{\mathbin{\not\mkern4mu\mid}}   
%\newcommand{\why}{\rouge{(Why?)}}   
%\newcommand{\thus}{\rouge{\Huge$\therefore$}\xspace}
%\newcommand{\QED}{\hfill\rouge{QED}}
%\newcommand{\prfb}{\hfill\orchid{proof on board}}

%\newcommand{\mypmod}[1]{\;(\bmod\,#1)}
%\newcommand{\ranin}{\stackrel{\rm \rouge{ran}}{\in}}
%\newcommand{\ff}{\mathbb{F}}
%\newcommand{\msb}{\textrm{msb}}
%\newcommand{\lsb}{\textrm{lsb}}
%\renewcommand{\concat}{\mathbin{||}}
%\newcommand{\sig}{\textrm{sig}}
%\newcommand{\ver}{\textrm{ver}}
%\newcommand{\suchThat}{\rouge{$\ni$}}

%\newcommand{\GSO}{\textrm{GSO}}
%\newcommand{\PI}{\textrm{PI}}
%\newcommand{\PO}{\textrm{PO}}
%\newcommand{\DS}{\textrm{DS}}
%\newcommand{\Prob}{\mathrm{Prob}}
%\newcommand{\state}[1]{{\color{red}\mathopen{\vert}}#1{\color{red}\mathclose{\rangle}}}

%%%%%%%%%%%%%%%%%%%%%%%%%%%%%%%%%%%%%%%%%%%%%%%%%%%%%%%%%%%%%%%%%%%
%%%%%%%%%%%%%%%%%%%%%%%%%%%%%%%%%%%%%%%%%%%%%%%%%%%%%%%%%%%%%%%%%%%

%\begin{document}\sffamily\bfseries\boldmath
%\color{Black}\huge

%%%%%%%%%%%%%%%%%%%%%%%%%%%%%%%%%%%%%%%%%%%%%%%%%%%%%%%%%%%%%%%%%%%
%%%% Title

%\thispagestyle{empty}
%{\ }\vspace{2ex}
%\begin{center} \Huge 
%  \rule{18cm}{5pt} \\[1ex]
%\color{Maroon}
%   \uppercase{Quantum Cryptography \\ \& Computing} \\
%\color{Black}
%  \rule{18cm}{5pt} 
%\end{center}

%\vfill
%\begin{center}\huge 
%  \bleu{CIS 428/628 ---  Spring 2006} \\
%  \vertt{{Introduction to Cryptography}}
%\end{center}
%\vfill
%\begin{center}\LARGE
%  This is based on Chapter 19 of Trappe \& Washington and \\
%  \vertt{``A Physics-Free Introduction to the Quantum Computation
%  Model''} \\ by Stephen Fenner
%\end{center}  
%\setcounter{page}{0}

\newcommand{\Emph}[1]{\structure{\emph{#1}}}

\begin{document}
%%%%%%%%%%%%%%%%%%%%%%%%%%%%%%%%%%%%%%%%%%%%%%%%%%%%%%%%%%%%%%%%%%
\begin{frame}



\begin{columns}
\begin{column}{.48\textwidth}
\maketitle
\end{column}
\begin{column}{.48\textwidth}\centering

\vspace*{3ex}

%\includegraphics[height=6cm]{thetalk.png}

$\vdots$

\end{column}
\end{columns}

\end{frame}

\section{Quantum Computing: Preliminaries}
%%%%%%%%%%%%%%%%%%%%%%%%%%%%%%%%%%%%%%%%%%%%%%%%%%%%%%%%%%%%%%%%%%
\begin{frame}{References} \small
\begin{itemize}
  \item \Emph{A Physics-Free Introduction to the Quantum Computation Model}
  by Stephen A. Fenner. 
  \textcolor{RoyalBlue}{\footnotesize\url{https://arxiv.org/abs/cs/0304008}} \\
  \emph{(\dots more importantly, it is complex analysis free)}
\end{itemize}

\vspace*{-3ex}
\begin{columns}
\begin{column}{.6\textwidth}
\begin{itemize}
  \item 
  \Emph{The Talk} by Scott Aaronson and Zach Weinersmith, 
  \textcolor{RoyalBlue}{\footnotesize\url{http://www.smbc-comics.com/comic/the-talk-3}}
  
  \emph{(There is tons of misleading hype about quantum computing.
  This is a good, double-entendre-filled, dehyping.)}
\end{itemize}
\end{column}
\begin{column}{.35\textwidth}
%\includegraphics[height=3.4cm]{ewww.png}

\end{column}
\end{columns}
\begin{itemize}  
  \item \Emph{Quantum Computing Since Democritus} by Scott Aaronson
  \textcolor{RoyalBlue}{\footnotesize\url{https://www.scottaaronson.com/democritus/}}
  
  \emph{(This connects quantum computing to the wider intellectual world
  while being rather goofy.)}
\end{itemize}

\end{frame}


%%%%%%%%%%%%%%%%%%%%%%%%%%%%%%%%%%%%%%%%%%%%%%%%%%%%%%%%%%%%%%%%%%%
\begin{frame}{Quantum Computing and Cryptography}\small

\begin{itemize}
%  \item We have mentioned quantum computing several times during the 
%  course.  In particular:
%  \begin{itemize}   
    \item 
      Given RSA with key size $k$, \\[-0.5ex]
      \Quad1 it can be broken by a computer with
       quantum register size $\approx k$.\alert{\LARGE$^\star$}
  \bigskip
    \item 
      Similarly with discrete-log-based cryptosystems.
  \bigskip
      \item 
         There are latticed-based cryptosystems that 
         quantum computers seemingly cannot do better than classical computers 
         in breaking. 
              
%  \end{itemize}  
  \bigskip
  \item We will cover  enough about quantum computing 
        give you a \Emph{glimpse} of what is behind all the fuss. 

  \bigskip
  \item This is based on \Emph{A Physics-Free Introduction to the Quantum Computation Model}
  by Stephen A. Fenner. 
  \textcolor{RoyalBlue}{\footnotesize\url{https://arxiv.org/abs/cs/0304008}}. 
        
\end{itemize}
\bigskip
\hrule
\alert{\LARGE$^\star$}\Emph{Assuming that you can build a quantum 
      computer of that size.}
\end{frame}

\section{Circuits: Classical and Reversible}
%%%%%%%%%%%%%%%%%%%%%%%%%%%%%%%%%%%%%%%%%%%%%%%%%%%%%%%%%%%%%%%%%%
\begin{frame}{{Classical Boolean Circuits, I}}

We view them as naming maps
\Math{\set{0,1}^n\to\set{0,1}^n}

 \unitlength=1pt
%\begin{center}%\color{PineGreen} 
%\setstatediam{10}
%\begin{picture}(240,35)(0,5)
%%\put(-30,17){\footnotesize\Emph{inputs}}
%%\put(250,17){\footnotesize\Emph{outputs}}
%\put(10,28){$a$} \put(170,28){$a$ \Quad2 control}
%\put(10,8){$b$}  \put(170,8){$a\wedge b$\Quad1 target}
%\put(20,30){\line(1,0){140}}
%\put(20,10){\line(1,0){140}}
%\letstate A=(93,10) \drawstate(A){$\wedge$}
%%\put(89,7.5){$\wedge$}
%%\put(92,10){\circle{8}}
%\put(92.5,28){\line(0,-1){14}}
%\put(90,28){$\bullet$}
%\put(55,-5){\structure{$\Rrightarrow$ current flow $\Rrightarrow$}}
%\end{picture}\end{center}
\medskip

Consider
%\begin{center}\begin{picture}(240,50)(0,5) \setstatediam{10}
%\put(10,48){$a$} \put(170,48){$\neg a$}
%\put(10,28){$b$} \put(170,28){$(a\wedge b)\vee c$}
%\put(10,8){$c$}  \put(170,8){$c$}
%\put(20,50){\line(1,0){140}}
%\put(20,30){\line(1,0){140}}
%\put(20,10){\line(1,0){140}}
%\put(60,48){$\bullet$}
%\put(62.5,48){\line(0,-1){14}}
%\letstate C=(62.5,30) \drawstate(C){$\wedge$}
%%\put(59,27.5){$\wedge$}
%%\put(62,30){\circle{8}}
%\put(112.5,26){\line(0,-1){14}}
%\letstate A=(112,29) \drawstate(A){$\vee$} 
%%\put(109,27.5){$\vee$} %A
%%\put(112,30){\circle{8}}
%\put(110,8){$\bullet$}
%\letstate B=(112,50) \drawstate(B){$\neg$}
%%\put(109,48.5){$\neg$}
%%\put(112,50){\circle{8}}
%\end{picture}\end{center}
We can describe this by either of: 
\begin{itemize}
\item $b \gets a\wedge b;\;\; a\gets \neg a;\; \;b\gets b\vee c$
\hfill \Emph{\footnotesize $\state{x,y,z}$ = state vector}

\item $\state{a,b,c} \;\;\mapsto\;\; \state{a,a\wedge b,c}
\;\;\mapsto\;\; \state{\neg a, a\wedge b,c} 
\;\;\mapsto\;\; \state{\neg a, (a\wedge b)\vee c,c}$ 
\end{itemize}
\end{frame}


%%%%%%%%%%%%%%%%%%%%%%%%%%%%%%%%%%%%%%%%%%%%%%%%%%%%%%%%%%%%%%%%%%
\begin{frame}{Classical Boolean Circuits, II} \small

\Emph{Input/Output Conventions}

\begin{itemize}
  \item
    The first \Math{k} registers are input 
       \hfill \Math{0\leq k \leq n}\\
  \item
    The first \Math{\ell} registers are output 
       \hfill \Math{0\leq \ell \leq n}\\
  \item
    Each non-input register is assigned 0 or 1  \\ 
     \unitlength=1pt
\begin{picture}(240,40)(0,5)
\put(10,28){$a$} \put(170,28){$a$}
\put(10,8){$0$}  \put(170,8){$a$}
\put(20,30){\line(1,0){140}}
\put(20,10){\line(1,0){140}}
\put(89,6.7){$\vee$}
\put(92.5,10){\circle{8}}
\put(92.5,28){\line(0,-1){14}}
\put(90,28){$\bullet$}
\put(190,20){\color{blue}$a\mapsto (a,a)$}
\end{picture}
\end{itemize}

\end{frame}
%%%%%%%%%%%%%%%%%%%%%%%%%%%%%%%%%%%%%%%%%%%%%%%%%%%%%%%%%%%%%%%%%%
\begin{frame}{Uniform Computation}
\begin{itemize}
  \item
  A \Emph{circuit family}, \Math{\SC}, is a sequence of circuits
  \Math{C_0,\,C_1,\, C_2,\dots} \ \suchThat    \\
  \Quad2 for each \Math{i}, \ \Math{C_i} has \Math{i}-inputs and 1-output.
  
  \item 
  \Math{L(\SC) \defeq \set{w \suchthat |w|=n \;\&\; C_n(w)=1}}, 
  the \Emph{language defined by} \Math{\SC}.
  
  \item A circuit family is \Emph{ptime uniform} $\iff$ \\
  \Quad1 \Math{\exists} a poly-time alg \Math{D} \ \suchThat \\
  \Quad2 for all \Math{i}, \\
  \Quad3 
  \Math{D(\underbrace{1\dots 1}_{\text{$i$ many}}) =}
  a description of \Math{C_i}.
\end{itemize}  
\alert{FACT:}
\Math{P = } the languages accepted by ptime uniform circuit 
families.
   
\end{frame}

%%%%%%%%%%%%%%%%%%%%%%%%%%%%%%%%%%%%%%%%%%%%%%%%%%%%%%%%%%%%%%%%%%
\begin{frame}{Reversible Circuits, I}

Reversible circuits have inverses.
 \unitlength=1pt
\begin{description}

  \item[The controlled not gate (CNOT)] \ \\
\begin{picture}(240,40)(0,5)
\put(10,28){$a$} \put(170,28){$a$}
\put(10,8){$b$}  \put(170,8){$a \oplus b$}
\put(20,30){\line(1,0){140}}
\put(20,10){\line(1,0){140}}
\put(88,7.5){$\oplus$}
\put(92.5,28){\line(0,-1){15}}
\put(90,28){$\bullet$}
\end{picture}

  \item[Toffoli Gate]\  where \Math{\odot(x,y,z)=z\oplus(x\wedge y)}\\
\begin{picture}(240,60)(0,5)
\put(10,48){$a$} \put(170,48){$a$}
\put(10,28){$b$} \put(170,28){$b$}
\put(10,8){$c$}  \put(170,8){$c \oplus (a\wedge b)$}
\put(20,50){\line(1,0){140}}
\put(20,30){\line(1,0){140}}
\put(20,10){\line(1,0){140}}
\put(88,7.5){$\odot$}
\put(92.5,48){\line(0,-1){18}}
\put(92.5,28){\line(0,-1){15}}
\put(90,48){$\bullet$}
\put(90,28){$\bullet$}
\end{picture}

\end{description}

\vertt{Reversible circuits do not collapse states.} \quad \why
\end{frame}



%%%%%%%%%%%%%%%%%%%%%%%%%%%%%%%%%%%%%%%%%%%%%%%%%%%%%%%%%%%%%%%%%%%
\begin{frame}{Reversible Circuits, II}

\begin{gather*}
\left.\begin{array}{cc|cc}
\multicolumn{4}{c}{\hbox{CNOT Gate}} \\[1ex]
\multicolumn{2}{c|}{input}&\multicolumn{2}{c}{output} \\\hline 
0 & 0 & 0 & 0 \\\hline 
0 & 1 & 0 & 1 \\\hline 
1 & \interesting0 & 1 & \interesting1 \\\hline 
1 & \interesting1 & 1 & \interesting0
\end{array}\right.
\Quad5
\left.\begin{array}{ccc|ccc}
\multicolumn{6}{c}{\hbox{Toffoli Gate}} \\[1ex]
\multicolumn{3}{c|}{input}&\multicolumn{3}{c}{output} \\ \hline 
0 & 0 & 0 & 0 & 0 & 0 \\\hline 
0 & 0 & 1 & 0 & 0 & 1 \\\hline 
0 & 1 & 0 & 0 & 1 & 0 \\\hline 
0 & 1 & 1 & 0 & 1 & 1 \\\hline 
1 & 0 & 0 & 1 & 0 & 0 \\\hline 
1 & 0 & 1 & 1 & 0 & 1 \\\hline 
1 & 1 & \interesting0 & 1 & 1 & \interesting1 \\\hline 
1 & 1 & \interesting1 & 1 & 1 & \interesting0
\end{array}\right.
\end{gather*}
\centering

$\interesting0$ and $\interesting1$ are the \Emph{interesting} bits.

\end{frame}


\section{Circuits: Probabilistic and Majority}
%%%%%%%%%%%%%%%%%%%%%%%%%%%%%%%%%%%%%%%%%%%%%%%%%%%%%%%%%%%%%%%%%%
\begin{frame}{Probabilistic Circuits, I}

\Emph{The Biased Coin-Flip Gate} \Quad1 ---\fbox{$p,q$}---

\begin{center}\color{PineGreen}\begin{tabular}{c|ccc}
  input & \multicolumn{3}{c}{output} \\ \hline
  0     &   0:\Math{p}  && 1:\Math{(1-p)} \\
  1     &   0:\Math{q}  && 1:\Math{(1-q)}
\end{tabular}\end{center}  
\vfill
$\state{\vec{v}}$ : $2^n$ basis vectors
\hfill
$\SH$ : a $2^n$-dim.~real vector space
\\
\hfill \Emph{($\SH$ for Hilbert space)}
 \unitlength=1pt
\begin{center}
\begin{picture}(240,50)(0,5)
\put(10,48){$x_1$}
\put(12,37){\small$\vdots$}
\put(10,28){$x_i$}
\put(12,17){\small$\vdots$}
\put(10,8){$x_n$} 
\put(20,50){\line(1,0){140}}
\put(20,30){\line(1,0){140}}
\put(20,10){\line(1,0){140}}
\put(90,30){\fbox{$p,q$}}
\end{picture}\end{center}

\Math{\state{x_{1..i-1},\rouge{0},x_{i+1..n}} \;\;\mapsto\;\;
p\cdot \state{x_{1..i-1},\rouge{0},x_{i+1..n}} +
(1-p)\cdot \state{x_{1..i-1},\rouge{1},x_{i+1..n}}}

\Math{\state{x_{1..i-1},\rouge{1},x_{i+1..n}} \;\;\mapsto\;\;
q\cdot \state{x_{1..i-1},\rouge{0},x_{i+1..n}} +
(1-q)\cdot \state{x_{1..i-1},\rouge{1},x_{i+1..n}}}

\end{frame}



%%%%%%%%%%%%%%%%%%%%%%%%%%%%%%%%%%%%%%%%%%%%%%%%%%%%%%%%%%%%%%%%%%
\begin{frame}{Probabilistic Circuits, II}
\vspace*{2ex}
Consider the subspace spanned by 
\Math{\state{0}} and \Math{\state{1}}.

\vspace*{2ex}

 \unitlength=1pt
\begin{columns}[c]
\begin{column}{.4\textwidth}

\begin{picture}(80,80)(-20,-20)
\put(0,-10){\line(0,1){60}}
\put(-10,0){\line(1,0){60}}
\put(0,40){\line(1,-1){40}}
\put(0,0){\color{blue}\line(2,1){27}}
\put(13,30){\color{PineGreen}$q\state{0}+(1-q)\state{1}$}
\put(28,15){\color{blue}$p\state{0}+(1-p)\state{1}$}
\put(,){}
\put(0,0){\color{PineGreen}\line(1,2){13}}
\put(-15,37){$\state{1}$}
\put(37,-10){$\state{0}$}
\end{picture}
\end{column}
\begin{column}{.5\textwidth}
The gate \brun{\fbox{$p,q$}} always maps the 
line segment from (1,0) to (0,1) to itself.
\end{column}
\end{columns}



We can also represent the \brun{\fbox{$p,q$}} gate by the matrix:
\begin{Eqnarray}
&&\left[\begin{array}{ccc} p &\Quad{0.5}& q \\ 1-p && 1-q \end{array}\right]
\end{Eqnarray}
This is a \Emph {stochastic matrix}: all entries \Math{\geq0},
all columns sum to 1.
\end{frame}




%%%%%%%%%%%%%%%%%%%%%%%%%%%%%%%%%%%%%%%%%%%%%%%%%%%%%%%%%%%%%%%%%%
\begin{frame}{Probabilistic Circuits: \textsl{Gates as Linear Maps}} \small

\Emph{The irreversible AND gate is:}

\begin{columns}[c]
\begin{column}{.2\textwidth}
\begin{tabular}{cc|ccc}
$a$ & $b$ &   $a$ &  $a\wedge b$ \\ \hline
\vertt{0} &    \vertt{0}  &  \rouge{0}   & \rouge{0} \\
\vertt{0} &    \vertt{1}  &   \rouge{0}  & \rouge{0} \\
\vertt{1} &    \vertt{0}  &   \rouge{1}  & \rouge{0} \\
\vertt{1} &    \vertt{1}  &   \rouge{1}  & \rouge{1} 
\end{tabular}
\end{column}
\begin{column}{.35\textwidth}
\Quad1 $\begin{array}{c|cccc} a\, b& 
   \vertt{00} & \vertt{01} & \vertt{10} & \vertt{11} \\ \hline
\rouge{00} & 1 & 1 & 0 & 0 \\ 
\rouge{01} & 0 & 0 & 0 & 0 \\
\rouge{10} & 0 & 0 & 1 & 0 \\ 
\rouge{11} & 0 & 0 & 0 & 1\end{array}
$
\end{column}
\begin{column}{.27\textwidth}
\redtri \ All entries are 0--1 

\redtri \ One 1 in each col

\redtri \ \thus Stochastic
\end{column}
\end{columns}
\medskip

\Emph{Reversible gates are permutation matrices!} \hfill \why
\medskip

\begin{definition}
A \Emph{probabilistic circuit} is a circuit built from
Boolean \& \brun{\fbox{$p,q$}} gates, where
\begin{itemize}
  \item The input state is a \rouge{basis state}.
  \item The output state is of the form:
  \Math{\sum_{x\in\set{0,1}^n} p_x \state{x}} \quad \suchThat 
  
  \Quad1 
  (i) each \Math{p_x\geq 0} \quad 
  and \quad (ii) \Math{\sum |p_x| = 1}.
\end{itemize}
$p_{x}$ = the probability that the output will be $\state{x}$.
\end{definition}

%\orchid{``Majority coin flips'' circuit: on the board.}

\end{frame}

%%%%%%%%%%%%%%%%%%%%%%%%%%%%%%%%%%%%%%%%%%%%%%%%%%%%%%%%%%%%%%%%%%%
\begin{frame}{``Majority Coin Flips'' Circuit}

%\begin{center}\includegraphics[height=6cm]{majcoins.pdf}\end{center}
\centering
\fbox{$\frac12,\frac12$} = flip of a fair coin
\end{frame}

%%%%%%%%%%%%%%%%%%%%%%%%%%%%%%%%%%%%%%%%%%%%%%%%%%%%%%%%%%%%%%%%%%%
\begin{frame}{A Complexity-Theoretic Aside}

\begin{itemize}
  \item $\vec{C} = C_0$, $C_1$, $C_2,\dots $ : a ptime uniform probablistic 
  circuit family
  \item $(R,A)$ is an \Emph{acceptance criterion} when 
  $R,A\subset [0,1]$ with $R\cap A=\emptyset$.
  \Quad2 \emph{\Emph{(R for reject, A for accept)}}
  \item $\vec{C}$ \Emph{computes $L$ with acceptance
  	criterion $(R,A)$} when\\
	\Quad1  for each $n$ and each $x\in\set{\mathbf{0},\mathbf{1}}^n$:
	\begin{align*}
       x \in L &\quad\implies\quad Prob[C_n(x)=1]\in A\\
       x \notin L &\quad\implies\quad Prob[C_n(x)=1]\in R
	\end{align*}
\end{itemize}
\begin{center}
\begin{tabular}{c|c}
\hbox{Class} & \hbox{Acceptance Criterion} \\ \hline
 P 	& $(\set{0},\set{1})$ \\
NP  & $(\set{0},(0,1])$ \\
RP  & $(\set{0},(\frac12,1])$ \\
BPP  & $([0,q],[1-q,1])$ \lefteqn{\quad \hbox{where $0<q<\frac12$}}\\
PP  & $([0,\frac12],(\frac12,1])$ 
\end{tabular}
\end{center}

\end{frame}

%%%%%%%%%%%%%%%%%%%%%%%%%%%%%%%%%%%%%%%%%%%%%%%%%%%%%%%%%%%%%%%%%%
\begin{frame}{Quantum Circuits (\'a la Fenner), I} \small

\begin{itemize}
  \item states = vectors in \Math{\SH} \quad
        gates = matrices
  \item Now allow nonegative entries in matrices.
        (But all real numbers)
  \item Now require:  \Math{\Vert M v\Vert_2 = \Vert v\Vert_2 }
  	for all \Math{v}.
  \item \alert{Note:} \Math{\Vert \vec{a}\Vert_2 \; \defeq \;
        \sqrt{a_1^2 + \dots + a_n^2}}
  \item This forces the matrices to be \Emph{orthonormal}, \\
  \Quad2 i.e., its columns form an orthogonal basis of \Math{\SH}.
  \item Registers are now called \Emph{qubits} (quantum bits) 
  instead of bits.
  \item The \Emph{Hadamard gate}, --\fbox{$H$}--, has the matrix:
  $ \frac1{\sqrt2}\left[ \begin{array}{rr} 
      1 & 1 \\ 1 & -1\end{array}\right]$
      \quad \parbox{1.5cm}{\footnotesize\centering \orchid{See the \\ next slide}}
%      \Quad5 \hbox{\orchid{See picture on board}}
  ${H \state{0} = \frac1{\sqrt2}(\state0 + \state1)}$.
  \Quad{0.5}
  ${H \state{1} = \frac1{\sqrt2}(\state0 - \state1)}$.
  \Quad{0.5}
  \bleu{Note:} ${H^2 = I}$.
  
  \item \alert{Fact:}$\set{H, \hbox{Toffoli gates}}$ are a
  \Emph{universal} collection of quantum gates.
  \item The \fbox{$p,q$} gates now correspond to 
  \Emph{measurements}.
\end{itemize}

\end{frame}

%%%%% %%%%% %%%%% %%%%% 
%%%%%%%%%%%%%%%%%%%%%%%%%%%%%%%%%%%%%%%%%%%%%%%%%%%%%%%%%%%%%%%%%%%
\begin{frame}{Hadamard Gate Geometrically}

\begin{columns}
\begin{column}{.5\textwidth}
%\includegraphics[height=6cm]{Hgate.pdf}
aa
\end{column}
\begin{column}{.32\textwidth}\small
\begin{enumerate}
 \item Transpose around the x-axis: \quad \Emph{$(x,y)\mapsto (x,-y)$}
 \item Then do a +45$^o$ rotation.
\end{enumerate} 


\end{column}
\end{columns}

\centering
${H \state{0} = \frac1{\sqrt2}(\state0 + \state1)}$.
  \Quad{3}
  ${H \state{1} = \frac1{\sqrt2}(\state0 - \state1)}$.


\end{frame}
%%%%% %%%%% %%%%% %%%%% 
\section{Quantum Circuits}
%%%%%%%%%%%%%%%%%%%%%%%%%%%%%%%%%%%%%%%%%%%%%%%%%%%%%%%%%%%%%%%%%%
\begin{frame}{Quantum Circuits (\'a la Fenner), II}%\LARGE

\begin{block}{QCF (Quantum Coin Flip)}
    This is a variation on Hadamard gate. 
      \begin{displaymath}
     \mathrm{QCF}\;=\;\frac1{\sqrt2}\left[ \begin{array}{rr} 
      1 & -1 \\ 1 & 1\end{array}\right]
  \end{displaymath}
  Note that \Math{(\mathrm{QCF})^2 
    = \left[\begin{array}{ll} 0 & 1 \\ 1 & 0 \end{array}\right]}
    = the not gate.
    
    So, QCF = \Math{\sqrt{\textrm{NOT}}},  \Emph{the square root of not}.
\end{block}

\begin{block}{Quantum I/O}
    \textbf{Input:} basis states\\
    \textbf{Output:} \Math{\sum_{x\in\set{0,1}^n} a_x\state{x}} \Quad2 
    \bleu{Note:} \Math{\sum a_x^2 = 1} \\
    \Math{a_x^2} = the probability associated with \Math{\state{x}}\\
    \Math{a_x} = the \Emph{probability amplitude} for \Math{\state{x}}
\end{block}
\end{frame}


%%%%%%%%%%%%%%%%%%%%%%%%%%%%%%%%%%%%%%%%%%%%%%%%%%%%%%%%%%%%%%%%%%%
\begin{frame}{Another Complexity-Theoretic Aside}\small

If we use quantum circuits, then 
\begin{center}\begin{tabular}{l|l|c}
{Class} & Description & {Acceptance Criterion } \\ \hline
EQP 
  & {\footnotesize Exact quantum polynomial time} 
  & $(\set{0},\set{1})$ 
  \\
C$_{\not=}$P
  & {\footnotesize Co-Exact-Counting Polynomial-Time} 
  & $(\set{0},(0,1])$ 
  \\
RQP
  & {\footnotesize One-sided Error Extension of EQP} 
  & $(\set{0},(\frac12,1])$
  \\
BQP 
  & {\footnotesize Bounded-Error Quantum Polynomial-Time} 
  & $([0,\frac1n),(\frac{n-1}n,1])$
  \\
PP 
  & {\footnotesize Probabilistic Polynomial-Time} 
  & $[0,\frac12],(\frac12,1])$
\end{tabular}\end{center}

\bigskip
\begin{center}
See: \textcolor{RoyalBlue}{\url{https://complexityzoo.uwaterloo.ca/Complexity_Zoo}}
\end{center}
\end{frame}
%%%%%%%%%%%%%%%%%%%%%%%%%%%%%%%%%%%%%%%%%%%%%%%%%%%%%%%%%%%%%%%%%%
\begin{frame}{``Traditional'' Quantum Circuits}


\begin{itemize}
  \item In place of vector spaces over \Math{\mathbb{R}},
  we use v.s.'s over \Math{\mathbb{C}}.
  \item In place of orthonormal matrices, we use
  \vertt{unitary} matrices. 
  \item Etc., etc. \Quad1 \orchid{See \S 6 of Fenner for details.}
\vfill
  \item Past this point, we shall be even sketchier than before.
  \item \rouge{\dots so}, we won't digress into complex
  linear algebra.
\vfill
\end{itemize}

\end{frame}

\section{Shor's Algorithms}

%%%%%%%%%%%%%%%%%%%%%%%%%%%%%%%%%%%%%%%%%%%%%%%%%%%%%%%%%%%%%%%%%%%
\begin{frame}{Towards Shor's Algorithm: \color{Maroon} Number Theory Facts, I}

\begin{center}
Suppose we want to factor $N$ (assuming $N$ isn't prime).
\end{center}

\begin{columns}
\begin{column}{.85\textwidth}


\begin{enumerate}[a]
  \item 
    \Emph{If} we find an $x\in\set{2,\dots,N-2}$ with
    $x^2\cong 1 \pmod N$

	\Emph{then} we can factor $N$.  \hfill \why
	\bigskip
  \item	
     \Emph{If} we can find an $a$ and an even $r$ with:
     \begin{enumerate}[i]
     \item $\gcd(a,N)=1$, 
     \item $a^r\cong 1 \pmod N$, and 
     \item $a^{r/2}\not\cong \pm 1 \pmod N$, 
    \end{enumerate}
	\Emph{then} we can factor $N$. \hfill \why     
\end{enumerate}

\end{column}
\end{columns}

\end{frame}

\note{\begin{enumerate}[a]
\item
 Suppose $1<x<N-1$ and $x^2\cong 1 \pmod N$. 

 Then $N | (x^2-1)$, i.e, $N| (x-1)(x+1)$.  

 Since $1<x<N-1$, neither $x-1=0$ nor $x+1=n$.

 So $\gcd(N,x-1)>1$ or $\gcd(N,x+1)>1$.
\bigskip
\item Use (a).
\end{enumerate}

}

%%%%%%%%%%%%%%%%%%%%%%%%%%%%%%%%%%%%%%%%%%%%%%%%%%%%%%%%%%%%%%%%%%%
\begin{frame}{Towards Shor's Algorithm: \color{Maroon} Number Theory Facts, II}\small


\begin{Alg}{Heuristic Procedure for Factoring}
\Quad1 Input $N$.

\Quad1 Pick $a\ranin\set{2,\dots,N-2}$.

\Quad1 If $\gcd(a,N)>1$, return $\gcd(a,N)$.
\hfill \Comment{It is a (nontrivial) factor}

\Quad1 \Comment{So, $\gcd(a,N)=1$}

\Quad1 Find the \Emph{smallest} $r>0$ with $a^r\cong 1 \pmod{N}$. 
\Comment{Expensive classically}

\Quad1 If $r$ is odd or $a^{r/2}\cong -1 \pmod{N}$, 

\Quad2 then: return \Emph{FAILURE}

\Quad2 else: use the trick of the previous page to compute
a factor of $N$

\Quad{4.2} return this factor.
\end{Alg}

\begin{itemize}
\item \Emph{\bfseries{FACT:}} If $N=p_1^{k_1}\dots p_s^{k_s}$ where $p_1,\dots,p_s$ are
distinct primes and $s>1$, then $Prob[\hbox{the procedure succeeds on }N]\geq  1-\frac1{2^{s-1}} \geq \frac12$.
\item So repeating the procedure on $N$ not too many times will find 
us a factor (with high probability). 
\item \Emph{BUT} the best know \Emph{classical} methods for finding $r$
are exponential time. 
\end{itemize}

\end{frame}

%%%%%%%%%%%%%%%%%%%%%%%%%%%%%%%%%%%%%%%%%%%%%%%%%%%%%%%%%%%%%%%%%%
\begin{frame}{Peter Shor's Clever Idea (One of Many)}%\LARGE

\begin{Alg}{Heuristic Procedure for Factoring}
\Quad1 Input $N$.

\Quad1 Pick $a\ranin\set{2,\dots,N-2}$.

\Quad1 If $\gcd(a,N)>1$, return $\gcd(a,N)$. 

\Quad1 Find the \Emph{smallest} $r>0$ with $a^r\cong 1 \pmod{N}$. 
\Comment{\Large PROBLEM}

\Quad1 If $r$ is odd or $a^{r/2}\cong -1 \pmod{N}$, 

\Quad2 then: return \Emph{FAILURE}

\Quad2 else:  compute a factor of $N$ and return it
\end{Alg}


Use QC to find  \Math{r}.  That is:
\begin{itemize}
      \item Consider \Math{1,a^1,a^2,a^3,\dots \mypmod{n}}.
      \item If \Math{a^r\equiv 1\mypmod{n}}, 
      then the sequence repeats every \Math{r} times.
      \item[\thus] \ Finding the period of the sequence, 
        finds \Math{r}.
      \item In signal processing, \Emph{Fourier transforms}
      are used to find periods.  

    \end{itemize}  




\end{frame}


%%%%%%%%%%%%%%%%%%%%%%%%%%%%%%%%%%%%%%%%%%%%%%%%%%%%%%%%%%%%%%%%%%
\begin{frame}{Quantum Fourier Transform}


\begin{align*}
  \mathrm{QFT}(\state{x}) \;\;\defeq\;\;
  \frac1{\sqrt{2^m}} \sum_{c\in\set{0,1}^m} e^{\frac{2\pi i x c}{2^m}} \state{c}
\end{align*}

\vfill

\begin{itemize}
  \item This can be realized as a quantum circuit.
  \item We'll come back to the properties of this thing shortly.
\end{itemize}
\vfill 
\end{frame}



%%%%%%%%%%%%%%%%%%%%%%%%%%%%%%%%%%%%%%%%%%%%%%%%%%%%%%%%%%%%%%%%%%
\begin{frame}{Shor's Factoring Algorithm, I}\small

\vspace*{-2ex}
\begin{align*}
   &\state{0\dots0,0\dots0} 
        \Quad1 \hbox{\color{PineGreen}$m+n$ long} \\
   & \Quad2 {\color{blue}\downarrow }\\
   & \frac1{\sqrt2} \left(\state{\rouge{0}0\dots 0,0\dots 0} 
      + \state{\rouge{1}0\dots 0,0\dots 0}\right) \\
   & \Quad2 {\color{blue}\downarrow} \\
   & \Quad2 \vdots \\
   & \Quad2 {\color{blue}\downarrow} \\
   & \frac1{\sqrt{2^m}}\textstyle \sum_{c\in\set{0,1}^m} \state{c,\vec{0}}
    \Quad1 \hbox{\color{PineGreen}superimposition of $2^m$ states} \\
   & \Quad2 {\color{blue}\downarrow} \\
   &  \frac1{\sqrt{2^m}}\textstyle  \sum_{c\in\set{0,1}^m}
    \state{c,a^c \bmod n \dots } \\
   & \Quad2 {\color{blue}\downarrow} \\
   & \mathrm{QFT}(\; \hbox{---} \; ) \Quad2 
     \Quad2 \hbox{\vertt{Now what???}}
\end{align*}


\end{frame}


%%%%%%%%%%%%%%%%%%%%%%%%%%%%%%%%%%%%%%%%%%%%%%%%%%%%%%%%%%%%%%%%%%
\begin{frame}{Shor's Factoring Algorithm, II}

\begin{columns}
\begin{column}{.9\textwidth}


\begin{itemize}
  \item When you measure \Math{\sum_i a_i \state{x_i}} \\
  \Quad2 
  you get state \Math{\state{x_i}} with probability \Math{a_i^2}.
  \bigskip
  
  \item Thanks to \rouge{QFT}, \\
  \Quad2 \vertt{states \vertt{near} the 
  period have pretty high probability.}
  \item \thus \ Measure, test, and refine. 
  
  See: {\footnotesize
  \Emph{Shor's Quantum Factoring Algorithm} by Samuel J. Lomonaco,
  \textcolor{RoyalBlue}{\url{https://arxiv.org/abs/quant-ph/0010034}}}

  \bigskip
    \item A similar trick (using QFT) can compute discrete logs.
\end{itemize}

\end{column}

\end{columns}


\end{frame}


%%%%%%%%%%%%%%%%%%%%%%%%%%%%%%%%%%%%%%%%%%%%%%%%%%%%%%%%%%%%%%%%%%
\begin{frame}{Quantum Algorithms Beyond Shor's}

\begin{block}{Grover's Algorithm}
\begin{itemize}
  \item Suppose that \Math{C:\set{0,1}^n\to\set{0,1}}
   is such that \\
   \Math{C(s)=1} for \vertt{only one} \Math{s\in\set{0,1}^n}.
  \item Classically, finding this \Math{s} 
        takes \Math{\Theta(2^m)} time. 
  \item Using QFT trickery, one can do this in 
         \Math{\Theta(\sqrt{2^m})} time.
  \item This is the best known quantum algorithm besides Shor's.
\end{itemize}
\end{block}
%\hrule

\begin{itemize}
\item 
  For other quantum algorithms, see:
  \begin{center}\footnotesize\color{RoyalBlue}
	\url{https://en.wikipedia.org/wiki/Quantum_algorithm}
  \end{center}
\item
  The take away is that quantum computers 
  \emph{are}\/ \textcolor{Purple}{magic bullets},

\Quad1 but only for some fairly special problems.
\item
  As factoring and discrete-log are among these special problems, 
  
  \Quad1 Cryptography must  respond, \Emph{e.g., lattice-based cryptosystems}.
\end{itemize}
\end{frame}
%%%%%%%%%%%%%%%%%%%%%%%%%%%%%%%%%%%%%%%%%%%%%%%%%%%%%%%%%%%%%%%%%%

%%%%%%%%%%%%%%%%%%%%%%%%%%%%%%%%%%%%%%%%%%%%%%%%%%%%%%%%%%%%%%%%%%%
%\begin{frame}[plain]{}  
%\centering\vfill\includegraphics[height=1cm]{SYR-ENG.pdf}\vfill
%\end{frame}


\end{document}
