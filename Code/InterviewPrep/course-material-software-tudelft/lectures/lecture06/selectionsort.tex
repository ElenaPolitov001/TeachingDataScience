\section{Selection Sort}
\label{sec:selection_sort}

\begin{frame}
	\frametitle{Selection Sort}
		\begin{block}{The smallest of variations...}
			Rather than shifting every element to it's correct place\dots\\
			\pause
			We instead repeatedly search for the smallest element and put it at the end of our sorted list/sorted part.
		\end{block}	
	\begin{center}
	\section{Selection Sort}
\label{sec:selection_sort}

\begin{frame}
	\frametitle{Selection Sort}
		\begin{block}{The smallest of variations...}
			Rather than shifting every element to it's correct place\dots\\
			\pause
			We instead repeatedly search for the smallest element and put it at the end of our sorted list/sorted part.
		\end{block}	
	\begin{center}
	\section{Selection Sort}
\label{sec:selection_sort}

\begin{frame}
	\frametitle{Selection Sort}
		\begin{block}{The smallest of variations...}
			Rather than shifting every element to it's correct place\dots\\
			\pause
			We instead repeatedly search for the smallest element and put it at the end of our sorted list/sorted part.
		\end{block}	
	\begin{center}
	\section{Selection Sort}
\label{sec:selection_sort}

\begin{frame}
	\frametitle{Selection Sort}
		\begin{block}{The smallest of variations...}
			Rather than shifting every element to it's correct place\dots\\
			\pause
			We instead repeatedly search for the smallest element and put it at the end of our sorted list/sorted part.
		\end{block}	
	\begin{center}
	\input{figures/tikz/selectionsort.tex}
	\end{center}
\end{frame}

\begin{frame}
	\frametitle{Implementation}
	\begin{exampleblock}{Time Complexity}
		Still $\Theta(n^2)$ as we need to find the minimum $O(n)$ times, which takes $O(n)$ time.
		\end{exampleblock}	
	\begin{alertblock}{See next week's homework}
		You will tell me next week ;)
	\end{alertblock}	
\end{frame}

\begin{frame}
	\frametitle{In-place sorting algorithms}
	\framesubtitle{I'm not going anywhere!}

	\begin{itemize}
		\item The sorting algorithms we have seen so-far are \alert{in-place} sorting algorithms.
			\pause
		\item They take a list and modify that list to become sorted.
			\pause
		\item Of course we can also make them not in-place, by first making a copy of the list and sorting that.
			\pause
		\item But after the break (when we discuss merge sort) we will see an example of an algorithm that is easier to make
			not in-place.
	\end{itemize}
\end{frame}



	\end{center}
\end{frame}

\begin{frame}
	\frametitle{Implementation}
	\begin{exampleblock}{Time Complexity}
		Still $\Theta(n^2)$ as we need to find the minimum $O(n)$ times, which takes $O(n)$ time.
		\end{exampleblock}	
	\begin{alertblock}{See next week's homework}
		You will tell me next week ;)
	\end{alertblock}	
\end{frame}

\begin{frame}
	\frametitle{In-place sorting algorithms}
	\framesubtitle{I'm not going anywhere!}

	\begin{itemize}
		\item The sorting algorithms we have seen so-far are \alert{in-place} sorting algorithms.
			\pause
		\item They take a list and modify that list to become sorted.
			\pause
		\item Of course we can also make them not in-place, by first making a copy of the list and sorting that.
			\pause
		\item But after the break (when we discuss merge sort) we will see an example of an algorithm that is easier to make
			not in-place.
	\end{itemize}
\end{frame}



	\end{center}
\end{frame}

\begin{frame}
	\frametitle{Implementation}
	\begin{exampleblock}{Time Complexity}
		Still $\Theta(n^2)$ as we need to find the minimum $O(n)$ times, which takes $O(n)$ time.
		\end{exampleblock}	
	\begin{alertblock}{See next week's homework}
		You will tell me next week ;)
	\end{alertblock}	
\end{frame}

\begin{frame}
	\frametitle{In-place sorting algorithms}
	\framesubtitle{I'm not going anywhere!}

	\begin{itemize}
		\item The sorting algorithms we have seen so-far are \alert{in-place} sorting algorithms.
			\pause
		\item They take a list and modify that list to become sorted.
			\pause
		\item Of course we can also make them not in-place, by first making a copy of the list and sorting that.
			\pause
		\item But after the break (when we discuss merge sort) we will see an example of an algorithm that is easier to make
			not in-place.
	\end{itemize}
\end{frame}



	\end{center}
\end{frame}

\begin{frame}
	\frametitle{Implementation}
	\begin{exampleblock}{Time Complexity}
		Still $\Theta(n^2)$ as we need to find the minimum $O(n)$ times, which takes $O(n)$ time.
		\end{exampleblock}	
	\begin{alertblock}{See next week's homework}
		You will tell me next week ;)
	\end{alertblock}	
\end{frame}

\begin{frame}
	\frametitle{In-place sorting algorithms}
	\framesubtitle{I'm not going anywhere!}

	\begin{itemize}
		\item The sorting algorithms we have seen so-far are \alert{in-place} sorting algorithms.
			\pause
		\item They take a list and modify that list to become sorted.
			\pause
		\item Of course we can also make them not in-place, by first making a copy of the list and sorting that.
			\pause
		\item But after the break (when we discuss merge sort) we will see an example of an algorithm that is easier to make
			not in-place.
	\end{itemize}
\end{frame}


