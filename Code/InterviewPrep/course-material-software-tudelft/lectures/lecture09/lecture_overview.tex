\begin{frame}
	\frametitle{You are here.}
	\begin{block}{The course so far}
		\begin{itemize}
			\item Time and Space complexity
			\item Lists, Stacks, and Queues
			\item Sorting
		\end{itemize}
	\end{block}
	\pause
	\begin{exampleblock}{Today's content}
		\begin{itemize}
			\item Trees!
			\item Properties and traversals
		\end{itemize}
	\end{exampleblock}
	\pause
	\begin{block}{The future}
		\begin{itemize}
			\item Heaps and Search Trees
			\item Maps, Sets, and Graphs
			\item P vs NP\dots
		\end{itemize}
	\end{block}
\end{frame}


\begin{frame}
	\frametitle{The midterms}
	Most of the results are in:
	\begin{itemize}
		\item Implementation exam: passrate of 92\%.
			\begin{itemize}
				\item Good job!
				\item Do keep in mind that with Graphs, implementation is going to become a bit trickier.
			\end{itemize}
		\item Multiple-choice part: passrate of 56\%.
			\begin{itemize}
				\item More in-line with what I expected.
				\item Two questions worth discussing.
			\end{itemize}
		\item Open questions part: passrate of ??\%.
			\begin{itemize}
				\item I hope to have the answer to this question soon!
			\end{itemize}
	\end{itemize}
\end{frame}

\begin{frame}
	\frametitle{Quick Feedback}
\begin{questionblock}{Written exam}
	The exam was: 
	\begin{multicols}{2}
	\begin{enumerate}[A.]
		\item Too easy
		\item Easy
		\item Fair
		\item Hard
		\item Too hard
		\item I didn't take the exam.
	\end{enumerate}
\end{multicols}
\end{questionblock}	

\begin{questionblock}{Written exam}
	The exam was: 
	\begin{multicols}{2}
	\begin{enumerate}[A.]
		\item Too short
		\item Short
		\item An appropriate length
		\item Long
		\item Too long
		\item I didn't take the exam.
	\end{enumerate}
\end{multicols}
\end{questionblock}	
\end{frame}

\begin{frame}
	\frametitle{Question 5}
	\lstinputlisting[basicstyle=\tiny\ttfamily]{code/mc-time-rec.py}
		\scriptsize
	Which of the following statements about the run time $T(n)$ of \texttt{recfunc} is \textbf{true}? In each of the
	answers $n$ represents the length of \texttt{xs}.
	\begin{multicols}{2}
	\begin{enumerate}[A.]
		\scriptsize
		\item $T(n) = \begin{cases}
			c_0 & \text{if } n < 2\\
			T(n/2) + c_1n + c_2 & \text{else}
		\end{cases}$\\ For some constants $c_0, c_1, c_2$.
	\item $T(n) = \begin{cases}
			c_0 & \text{if } n < 2\\
			2T(n/2) + c_1n + c_2 & \text{else}
		\end{cases}$\\ For some constants $c_0, c_1, c_2$.
	\item $T(n) = \begin{cases}
			c_0 & \text{if } n < 2\\
			2T(n/2) + c_1n^2 + c_2n + c_3 & \text{else}
		\end{cases}$\\ For some constants $c_0, c_1, c_2, c_3$.
	\item $T(n) = \begin{cases}
			c_0 & \text{if } n < 2\\
			2T(n-1) + c_1n^2 + c_2n + c_3 & \text{else}
		\end{cases}$\\ For some constants $c_0, c_1, c_2, c_3$.
	\end{enumerate}
\end{multicols}
\end{frame}

\begin{frame}
	\frametitle{Question 6}
	\lstinputlisting[basicstyle=\tiny\ttfamily]{code/mc-time-rec.py}
		\scriptsize
	Which of the following statements about the space $S(n)$ of \texttt{recfunc} is \textbf{true}? In each of the
	answers $n$ represents the length of \texttt{xs}.
	\begin{multicols}{2}
	\begin{enumerate}[A.]
		\scriptsize
		\item $S(n) = \begin{cases}
			c_0 & \text{if } n < 2\\
			S(n/2) + c_1n + c_2 & \text{else}
		\end{cases}$\\ For some constants $c_0, c_1, c_2$.
	\item $S(n) = \begin{cases}
			c_0 & \text{if } n < 2\\
			2S(n/2) + c_1n + c_2 & \text{else}
		\end{cases}$\\ For some constants $c_0, c_1, c_2$.
	\item $S(n) = \begin{cases}
			c_0 & \text{if } n < 2\\
			2S(n/2) + c_1n^2 + c_2n + c_3 & \text{else}
		\end{cases}$\\ For some constants $c_0, c_1, c_2, c_3$.
	\item $S(n) = \begin{cases}
			c_0 & \text{if } n < 2\\
			2S(n-1) + c_1n^2 + c_2n + c_3 & \text{else}
		\end{cases}$\\ For some constants $c_0, c_1, c_2, c_3$.
	\end{enumerate}
\end{multicols}
\end{frame}

	
