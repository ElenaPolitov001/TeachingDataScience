\section{Intermezzo}
\label{sec:intermezzo}

\begin{frame}
	\frametitle{Intermezzo: MapReduce}
	
	\begin{center}
	% Image based on: https://commons.wikimedia.org/wiki/File:Mapreduce.png
\begin{tikzpicture}[every node/.style={thick}]
  \colorlet{coul0}{orange!20} \colorlet{coul1}{blue!20} \colorlet{coul2}{red!20} \colorlet{coul3}{green!20}
  \tikzstyle{edge}=[->, very thick]
  \draw[thick, fill=violet!30] (-1, -2) rectangle node[rotate=90] {\textbf{Input data}} (0,2);
  \foreach \i in {0,1,2,3} {
    \node[draw, fill=coul\i] (data\i) at (1.5, 1.5 - \i) {Input};
    \node[ellipse, draw, fill=cyan!20] (map\i) at (3.5, 1.5 - \i) {\textsf{Map}};
    \draw[edge] (0,0) -- (data\i.west);
    \draw[edge] (data\i) -- (map\i);
  }
  \node[draw, minimum height=1cm, fill=purple!30] (resultat) at (10, 0) {\textbf{Results}};
  \foreach \i in {0,1,2} {
    \node[draw, fill=yellow!20] (paire\i) at (5.5, 1.5 - \i*1.5) {\begin{minipage}{1cm}Tuples \centering $\langle k,v \rangle$\end{minipage}};
    \node[ellipse, draw, fill=cyan!20] (reduce\i) at (7.5, 1.5 - \i*1.5) {\textsf{Reduce}};
    \draw[edge] (paire\i) -- (reduce\i);
    \draw[edge] (reduce\i.east) -- (resultat);
  }
          %paire
  \draw[edge] (map0.east) -- (paire0.west); \draw[edge] (map0.east) -- (paire1.west);
  \draw[edge] (map1.east) -- (paire0.west); \draw[edge] (map1.east) -- (paire2.west);
  \draw[edge] (map2.east) -- (paire1.west); \draw[edge] (map2.east) -- (paire0.west);
  \draw[edge] (map3.east) -- (paire1.west); \draw[edge] (map3.east) -- (paire2.west);
\end{tikzpicture}

	\hspace*{15pt}\hbox{\scriptsize Image Based on original image by :\thinspace{\itshape Clém IAGL}}
	\end{center}
\end{frame}

\begin{frame}
	\frametitle{What is MapReduce?}
	\begin{itemize}
		\item Original paper by Google: ``MapReduce: Simplified Data Processing on Large Clusters''.
		\item A technique they used to process their massive amounts of data streaming in.
			\pause
		\item The main idea?
			\begin{itemize}
				\item Split up the data.
				\item Apply some basic function to it (Map).
				\item Merge the results (Reduce).
			\end{itemize}
	\end{itemize}
\end{frame}

\begin{frame}
	\frametitle{Example}
		\begin{problemblock}{A word count}
			Imagine I have a several books and what to know how often certain words appear in the books.	\\
		\end{problemblock}	
		\pause
		\begin{answerblock}{MapReduce}
			The MapReduce strategy is to:
		\begin{enumerate}
			\item Split the books into pages.
			\item Pass every page to a different machine and ask it to count the words, returning a \texttt{dict} from a word
				to a count.
			\item Aggregate these results.
		\end{enumerate}
			
		\end{answerblock}
		\pause
			\begin{exampleblock}{Why is this nice?}
				We can easily \textit{parallelise} our problem over many machines (they do the heavy-lifting) and then simply
				aggregate the intermediate results (relatively quick/cheap operation).	
			\end{exampleblock}	
\end{frame}

