\section{Parsing math expressions Part 2}
\label{sec:parsing_math_expressions}

\begin{frame}
	\frametitle{A parse tree}
	\begin{columns}
		\column{0.555\textwidth}
  \begin{tikzpicture}[
    level distance = 2.5em,
    level 1/.style={sibling distance=11em},
    level 2/.style={sibling distance=4.5em},
    level 3/.style={sibling distance=2.25em},
  ]
  \node[ellipse] (t1) {/}
    child { node[ellipse] {$\times$}
      child { node[ellipse] {-} 
				child { node[ellipse] {8}} 
				child { node[ellipse] {$\sqrt{\invisible{x}}$}
					child { node[ellipse] {16}}}}
			child { node[ellipse] {2}}}
    child { node[ellipse] {+}
			child { node[ellipse] {-}
				child { node[ellipse] {4}}}
			child { node[ellipse] {3}}};
  \end{tikzpicture}\\
		\column{0.355\textwidth}
		\begin{questionblock}{What order?}
			In what order should we evaluate this parse tree to get the right answer?
			\begin{enumerate}[A.]
				\item Pre-order
				\item In-order
				\item Post-order
				\item Something else
			\end{enumerate}
		\end{questionblock}
	\end{columns}
	\pause
	\begin{answerblock}{Post-order}
		First find the two subresults, then combine them!
	\end{answerblock}
\end{frame}

\begin{frame}
	\frametitle{A basic implementation}
	
	\lstinputlisting{code/parsemath.py}
\end{frame}

