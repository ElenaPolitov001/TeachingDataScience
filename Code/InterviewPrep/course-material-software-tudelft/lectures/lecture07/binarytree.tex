\section{Binary Trees}
\label{sec:binary_trees}


\begin{frame}
	\frametitle{Binary Trees}
	\framesubtitle{Tikz taken from: \url{http://texample.net/tikz/examples/red-black-tree/}}
	\begin{center}
			\begin{tikzpicture}[->,>=stealth',level/.style={sibling distance = 4cm,
		  level distance = 1.5cm},
			  treenode/.style = {align=center, inner sep=0pt, text centered,
		    font=\sffamily},
		  arn_n/.style = {treenode, circle, white, font=\sffamily\bfseries, draw=black,
		    fill=black, text width=1.5em},% arbre rouge noir, noeud noir
		  arn_r/.style = {treenode, circle, red, draw=red,
		    text width=1.5em, very thick},% arbre rouge noir, noeud rouge
		  arn_x/.style = {treenode, rectangle, draw=black,
		    minimum width=0.5em, minimum height=0.5em}% arbre rouge noir, nil
			]
			\node [arn_n] {33}
    child{ node [arn_r] {15}
            child{ node [arn_n] {10}
            	child{ node [arn_r] {5} edge from parent node[above left]
                         {$x$}} %for a named pointer
							child{ node [arn_x] {}}
            }
            child{ node [arn_n] {20}
							child{ node [arn_r] {18}}
							child{ node [arn_x] {}}
            }
    }
    child{ node [arn_r] {47}
            child{ node [arn_n] {38}
							child{ node [arn_r] {36}}
							child{ node [arn_r] {39}}
            }
            child{ node [arn_n] {51}
							child{ node [arn_r] {49}}
							child{ node [arn_x] {}}
            }
		}
;
\end{tikzpicture}
	\end{center}
	
\end{frame}

\begin{frame}
	\frametitle{Binary Trees}
	\framesubtitle{All we need}

		\begin{block}{Binary Trees}
			\begin{itemize}
				\item Every node has at most 2 children.
					\pause
				\item One \textit{left} child and one \textit{right} child.
			\end{itemize}
		\pause
		A binary tree is \textit{proper} or \textit{full} if all nodes have either 0 or 2 children.
		\end{block}	

		\pause
		\begin{columns}
			\column{0.455\textwidth}
				
			\begin{tikzpicture}[
				level distance = 2.5em,
				level 1/.style={sibling distance=9em},
				level 2/.style={sibling distance=4.5em},
				level 3/.style={sibling distance=2.25em},
			]
			\node[ellipse] (t1) {root}
				child { node[ellipse]   {child 1}
					child { node[ellipse] {l1}}
					child { node[ellipse] {l2}}
				}
				child { node[ellipse]   {child 2}
					child { node[ellipse] {l3}}
					child { node[ellipse] {l4}
						child { node[ellipse] {l5}}
						child { node[ellipse] {l6}}
					}
				};
			\end{tikzpicture}
			\column{0.455\textwidth}
				
			\begin{questionblock}{Is it full?}
				Is this tree full?
				\begin{enumerate}[A.]
					\item Yes
					\item No
					\item I don't know
				\end{enumerate}
			\end{questionblock}
		\end{columns}
\end{frame}

\begin{frame}
	\frametitle{All kinds of fun properties!}
	
		\begin{block}{Many interesting properties}
			There are many interesting relations in binary trees, between the number of internal nodes vs leaves, height, etc.	
		\end{block}	
		\pause
		\begin{exampleblock}{Just one example}
			There is always one more leaf than there are internal nodes in a full binary tree.
		\end{exampleblock}	
\end{frame}
