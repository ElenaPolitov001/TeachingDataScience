\section{Multiway search tree}
\label{sec:multiway_search_tree}

\begin{frame}
	\frametitle{Generalising a bit}
		\begin{block}{Multiway search tree}
			\begin{itemize}
				\item Each node of $T$ has \textit{at least} 2 children (except for leaves of course).
				\item Each $d$-node (where $d$ is the amount of children the node has), stores $d-1$ values in increasing value.
				\item Between two keys $i$ and $i+1$ stored in the node, is rooted a tree that contains only keys $k$ such that
					$k_i \leq k \leq k_{i+1}$.
			\end{itemize}
			
		\end{block}	

		\begin{tikzpicture}[
			level distance = 2.5em,
			level 1/.style={sibling distance=14em},
			level 2/.style={sibling distance=4.5em},
			level 3/.style={sibling distance=2.25em},
			level 3/.style={sibling distance=4em},
			]
			\node[ellipse, draw] (t1) {22}
			child { node[ellipse, draw]   {5\quad10}
				child { node[ellipse, draw] {3\quad4}}
				child { node[ellipse, draw] {6\quad8}}
				child { node[ellipse, draw] {14}
					child { node[ellipse, draw] {11\quad13}}
					child { node[ellipse, draw] {17}}
				}
			}
			child { node[ellipse, draw]   {25}
				child { node[ellipse, draw] {23\quad24}}
				child { node[ellipse, draw] {27}}
			};
		\end{tikzpicture}
	
\end{frame}

\begin{frame}
	\frametitle{Practice?}
	
		\begin{block}{What should you know?}
			\begin{itemize}
				\item You should be able to search for items in a multiway search tree.
				\item And in the lab you will practice a bit with inserting nodes. Read the book carefully, it's not trivial!
				\item On the computer exam, we will not ask you to implement insertion or deletion in multiway search trees.
			\end{itemize}
		\end{block}	
\end{frame}
