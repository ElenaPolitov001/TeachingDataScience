\section{Final grade parts}
\label{sec:final_grade_parts}

The final grade of the course consists of several different components, each of which is discussed below.

\subsubsection*{Analysis exam Week 3 (10\%)}
\label{ssub:mc_exam_week_3_10}

This first exam will featur multiple choice and open questions answered on paper and focus mainly on
\cref{lo:notation,lo:runtime,lo:space}. The exam is to take 2 hours.

The score for this exam is denoted as $M_a$.

\subsubsection*{Implementation exam Week 6(10\%)}

This exam features several implementation questions that are completed on a computer. The exam focuses mainly on
\cref{lo:hash,lo:list,lo:stack,lo:queue,lo:sort} The exam is to take 2 hours.

The score for this exam is denoted as $M_i$.

\subsubsection*{Final implementation exam Week 10 (40\%)}

This exam features several implementation questions that are done on a computer. The exam is about all learning
objectives, with a larger emphasis on the material not covered by the midterm exam. The exam is to take 2 hours.  

The score for this exam is denoted as $F_i$.

\subsubsection*{Final analysis exam Week 10 (40\%)}

This exam features several multiple choice and open questions that are done on paper. The exam is about all learning
objectives, with a larger emphasis on the material not covered by the midterm exam. The exam is to take 2 hours.  

The score for this exam is denoted as $F_a$.

\subsubsection*{Final grade}
\label{ssub:final_grade}

The final grade for the course is computed as follows:
\begin{itemize}
	\item $G = 0.1M_a + 0.1M_i + 0.4F_i + 0.4F_a$.
	\item If $F_i < 5$, the final grade is a $\min(F_i, G)$.
	\item If $F_a < 5$, the final grade is a $\min(F_a, G)$.
	\item Else the final grade is $G$.
\end{itemize}

Both parts $F_a$ and $F_i$ can be retaken, in individual exam moments in the resit week.

