\subsection{Lectures}
\label{sub:lectures}

Table~\ref{tab:schedule} provides an overview of the lecturing schedule for the course, each lecture is discussed in
more detail below.

\begin{table}[htpb]
	\centering
	\caption{The course schedule, vertical lines indicate which material is tested in the exam in the week after that
	lecture. E.g. The line after lecture 8 indicates all material up to lecture 8 can be tested in the exam in week 5.}
	\label{tab:schedule}
	\begin{tabular}{c | r | l}
		Week & Lecture & Topic \\
		\hline
		1 & Lecture 1 & Course overview \& introduction to complexity\\
		1 & Lecture 2 & Complexity notation \\
		2 & Lecture 3 & Recursion \& Space complexity \\
		2 & Lecture 4 & Equality \& lists \\
		3 & Lecture 5 & Sorting \\
		3 & Lecture 6 & Stacks and Queues\\
		4 & Lecture 7 & Trees \\
		4 & Lecture 8 & Trees and heaps\\
		\hline
		6 & Lecture 9 & Search trees and priorityqueues\\
		6 & Lecture 10 & Hashmaps and hashsets\\
		7 & Lecture 11 & Introduction to graphs\\
		7 & Lecture 12 & Graph traversal and algorithms\\
		8 & Lecture 13 & Graph algorithms\\
		8 & Lecture 14 & Exam prep\\
		\hline
	\end{tabular}
\end{table}

\subsubsection*{Lecture 1: Counting operations}
\label{sub:lecture_1}

This lecture introduces the course material and presents a course overview. Furthermore we start with \cref{lo:runtime}
of the course, more specifically after this lecture the student is able to:
\begin{itemize}
	\item count primitive operations in iterative pieces of code.
	\item derive a total expression in terms of primitives for a piece of code.
\end{itemize}

\subsubsection*{Lecture 2: Complexity notation}
\label{sub:lecture_2}

This lecture concerns \cref{lo:notation,lo:runtime,lo:space} of the course, more specifically after this lecture the
student is able to:
\begin{itemize}
	\item describe the notion of $O$, $\Omega$, and $\Theta$ in complexity theory.
	\item order a set of functions based on their complexity.
	\item derive the right big-Oh notation for an iterative piece of code.
	\item derive the space complexity of algorithms.
\end{itemize}

\subsubsection*{Lecture 3: Recursion and Space complexity}
\label{sub:lecture_3}

This lecture concerns \cref{lo:runtime,lo:space} of the course, more specifically after this lecture the student is able
to:
\begin{itemize}
	\item derive a recurrence relation for time from a recursive piece of code.
	\item describe the use of stack frames in code and the impact on the space complexity of recursive code.
	\item derive a recurrence relation for space from a recursive piece of code.
	\item prove the run time complexity of a recursive piece of code using repeated unfolding.
	\item prove the run time complexity of a recursive piece of code using induction.
	\item describe the differences between space and run time complexity of recursive algorithms.
\end{itemize}

\subsubsection*{Lecture 4: Equality \& Hashing}
\label{sub:lecture_4}

This lecture concerns \cref{lo:hash,lo:list} of the course, more specifically after this lecture the student is able
to:
\begin{itemize}
	\item explain the need for equality and hash functions for a custom class.
	\item list the requirements a hash functions should fulfill.
	\item implement an equality and hash function for a custom python class.
	\item describe the differences between linked lists and position-based lists.
	\item describe the implementation of linked lists and arraylists.
	\item select the right type of list given a use case.
\end{itemize}

\subsubsection*{Lecture 5: Sorting}
\label{sub:lecture_5}

This lecture concerns \cref{lo:sort} of the course, more specifically after this lecture the student is able
to:
\begin{itemize}
	\item compare and implement bubble sort, selection sort, merge sort, bucket sort, and quicksort.
	\item combine different sorting strategies.
	\item select the right type of sorting given a use case.
\end{itemize}

\subsubsection*{Lecture 6: Stacks and Queues}
\label{ssub:lecture_6}

This lecture concerns \cref{lo:stack,lo:queue} of the course, more specifically after this lecture the student is able
to:
\begin{itemize}
	\item define the interface of a stack and of a queue.
	\item describe the differences between a stack and a queue.
	\item implement one interface in terms of the other (i.e. a stack from 2 queues or a queue from 2 stacks).
	\item select the right one (either a stack or a queue) given a use case.
\end{itemize}

\subsubsection*{Lecture 7: Trees}
\label{sub:lecture_7}

This lecture concerns \cref{lo:tree} of the course, more specifically after this lecture the student is able
to:
\begin{itemize}
	\item describe the interface of a tree.
	\item construct basic tree-traversal algorithms (like post-order).
	\item analyse the run time complexity of basic tree-traversal algorithms.
\end{itemize}

\subsubsection*{Lecture 8: Trees and heaps}
\label{sub:lecture_8}

This lecture concerns \cref{lo:tree,lo:heap} of the course, more specifically after this lecture the student is able
to:
\begin{itemize}
	\item describe the properties of a balanced tree.
	\item construct a binary search tree by repeatedly applying the insert operation.
	\item analyse the run time complexity of binary search tree algorithms.
\end{itemize}

\subsubsection*{Lecture 9 Search trees and Priorityqueues}
\label{sub:lecture_9}

This lecture concerns \cref{lo:tree,lo:pqueue} of the course, more specifically after this lecture the student is able
to:
\begin{itemize}
	\item describe the differences between binary search trees, red-black trees, and AVL trees.
	\item compare different implementations of a priorityqueue.
	\item construct a priorityqueue using a heap datastructure.
	\item using priorityqueues to sort lists.
\end{itemize}

Additionally the results from the midterm exam will be discussed, focusing on common mistakes.

\subsubsection*{Lecture 10: Hashmaps and hashsets}
\label{sub:lecture_10}

This lecture concerns \cref{lo:map,lo:set} of the course, more specifically after this lecture the student is able
to:
\begin{itemize}
	\item describe the operations of the \texttt{dict} datastructure in python.
	\item describe the interface of a map.
	\item describe the differences between a hashset and a hashmap.
	\item describe how hash conflicts are handled by these datastructures.
	\item list the requirements a hash functions should fulfill.
	\item implement an equality and hash function for a custom python class.
\end{itemize}


\subsubsection*{Lecture 11: Introduction to graphs}
\label{sub:lecture_11}

This lecture concerns \cref{lo:prove} of the course, more specifically after this lecture the student is able
to:
\begin{itemize}
	\item define basic graph components (vertices, edges, directedness).
	\item prove certain properties about graphs.
\end{itemize}

\subsubsection*{Lecture 12: Graph traversal}
\label{sub:lecture_12}

This lecture concerns \cref{lo:graph} of the course, more specifically after this lecture the student is able
to:
\begin{itemize}
	\item describe the operation of DFS and BFS.
	\item select the right traversal algorithm for a given use case.
	\item describe how DFS and BFS can be used to detect cycles in a graph.
\end{itemize}

\subsubsection*{Lecture 13: Graph traversal}
\label{sub:lecture_13}

This lecture concerns \cref{lo:graph} of the course, more specifically after this lecture the student is able
to:
\begin{itemize}
	\item apply and analyse Dijkstra's algorithm.
	\item prove the correctness of Dijkstra's algorithm.
	\item describe the properties of a Minimum Spanning Tree (MST).
	\item apply Prim's algorithm.
\end{itemize}

\subsection*{Lecture 14: Exam prep}
\label{sub:lecture_14}

This lecture covers no new material and instead focuses on exam preparation.


