\section{Learning Objectives}
\label{sec:learning_objectives}

The main learning objectives of the course are formulated as follows.

After this course, the student is able to:
\begin{enumerate}[label=\textbf{LO-\arabic*}]
	\item define big-Oh, Omega and Theta notation for complexity of code. \label{lo:notation}
	\item analyse the run time complexity of (recursive) (pseudo)code. \label{lo:runtime}
	\item analyse the space complexity of (recursive) (pseudo)code. \label{lo:space}
	\item define hash and equality functions for custom classes in python. \label{lo:hash}
	\item analyse and implement (doubly) linked lists, arraylists, stacks, and queues. \label{lo:list} \label{lo:queue}
		\label{lo:stack}
	\item describe and implement a variety of sorting algorithms (bubble sort, selection sort, merge sort, 
		quicksort). \label{lo:sort}
	\item analyse and implement search trees, such as binary search trees, AVL trees, and red-black trees. \label{lo:tree}
	\item analyse and implement priorityqueues and heaps. \label{lo:pqueue} \label{lo:heap}
	\item analyse and implement hash sets and maps. \label{lo:set} \label{lo:map}
	\item prove claims about trees and graphs and algorithms operating on them. \label{lo:prove}
	\item analyse and implement graph algorithms, such as: Breadth-First Search (BFS), Depth-First Search (DFS), Dijkstra, and
		Prim.\label{lo:graph}
	\item select the right datastructure for a given problem with certain run time and memory requirements. \label{lo:selecting}
\end{enumerate}

For every lecture we describe which of these we address and how we break this down in smaller learning objectives
specific to that lecture.

