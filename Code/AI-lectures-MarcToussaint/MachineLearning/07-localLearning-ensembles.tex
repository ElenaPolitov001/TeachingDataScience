\providecommand{\slides}{
  \newcommand{\slideshead}{
  \newcommand{\thepage}{\arabic{mypage}}
  %beamer
  \documentclass[t,hyperref={bookmarks=true}]{beamer}
%  \documentclass[t,hyperref={bookmarks=true},aspectratio=169]{beamer}
  \setbeamersize{text margin left=5mm}
  \setbeamersize{text margin right=5mm}
  \usetheme{default}
  \usefonttheme[onlymath]{serif}
  \setbeamertemplate{navigation symbols}{}
  \setbeamertemplate{itemize items}{{\color{black}$\bullet$}}

  \newwrite\keyfile

  %\usepackage{palatino}
  \stdpackages
  \usepackage{multimedia}

  %%% geometry/spacing issues
  %
  \definecolor{bluecol}{rgb}{0,0,.5}
  \definecolor{greencol}{rgb}{0,.6,0}
  %\renewcommand{\baselinestretch}{1.1}
  \renewcommand{\arraystretch}{1.2}
  \columnsep 0mm

  \columnseprule 0pt
  \parindent 0ex
  \parskip 0ex
  %\setlength{\itemparsep}{3ex}
  %\renewcommand{\labelitemi}{\rule[3pt]{10pt}{10pt}~}
  %\renewcommand{\labelenumi}{\textbf{(\arabic{enumi})}}
  \newcommand{\headerfont}{\helvetica{13}{1.5}{b}{n}}
  \newcommand{\slidefont} {\helvetica{10}{1.4}{m}{n}}
  \newcommand{\codefont} {\helvetica{8}{1.2}{m}{n}}
  \renewcommand{\small} {\helvetica{9}{1.4}{m}{n}}
  \renewcommand{\tiny} {\helvetica{8}{1.3}{m}{n}}
  \newcommand{\ttiny} {\helvetica{7}{1.3}{m}{n}}

  %%% count pages properly and put the page number in bottom right
  %
  \newcounter{mypage}
  \newcommand{\incpage}{\addtocounter{mypage}{1}\setcounter{page}{\arabic{mypage}}}
  \setcounter{mypage}{0}
  \resetcounteronoverlays{page}

  \pagestyle{fancy}
  %\setlength{\headsep}{10mm}
  %\addtolength{\footheight}{15mm}
  \renewcommand{\headrulewidth}{0pt} %1pt}
  \renewcommand{\footrulewidth}{0pt} %.5pt}
  \cfoot{}
  \rhead{}
  \lhead{}
  \lfoot{\vspace*{-3mm}\hspace*{-3mm}\helvetica{5}{1.3}{m}{n}{\texttt{github.com/MarcToussaint/AI-lectures}}}
%  \rfoot{{\tiny\textsf{AI -- \topic -- \subtopic -- \arabic{mypage}/\pageref{lastpage}}}}
%  \rfoot{\vspace*{-4.5mm}{\tiny\textsf{\topic\ -- \subtopic\ -- \arabic{mypage}/\pageref{lastpage}}}\hspace*{-4mm}}
  \rfoot{\vspace*{-4.5mm}{\tiny\textsf{\color{gray}\topic\ -- \subtopic\ -- \arabic{mypage}/\pageref{lastpage}}}\hspace*{-4mm}}
  %\lfoot{\raisebox{5mm}{\tiny\textsf{\slideauthor}}}
  %\rfoot{\raisebox{5mm}{\tiny\textsf{\slidevenue{} -- \arabic{mypage}/\pageref{lastpage}}}}
  %\rfoot{~\anchor{30,12}{\tiny\textsf{\thepage/\pageref{lastpage}}}}
  %\lfoot{\small\textsf{Marc Toussaint}}

  \definecolor{grey}{rgb}{.8,.8,.8}
  \definecolor{head}{rgb}{.85,.9,.9}
  \definecolor{blue}{rgb}{.0,.0,.5}
  \definecolor{green}{rgb}{.0,.5,.0}
  \definecolor{red}{rgb}{.8,.0,.0}
  \newcommand{\inverted}{
    \definecolor{main}{rgb}{1,1,1}
    \color{main}
    \pagecolor[rgb]{.3,.3,.3}
  }
  %auto-ignore
  \renewcommand{\a}{\alpha}
  \renewcommand{\b}{\beta}
  \renewcommand{\d}{\delta}
    \newcommand{\D}{\Delta}
    \newcommand{\e}{\epsilon}
    \newcommand{\g}{\gamma}
    \newcommand{\G}{\Gamma}
  \renewcommand{\l}{\lambda}
  \renewcommand{\L}{\Lambda}
    \newcommand{\m}{\mu}
    \newcommand{\n}{\nu}
    \newcommand{\N}{\nabla}
  \renewcommand{\k}{\kappa}
  \renewcommand{\o}{\omega}
  \renewcommand{\O}{\Omega}
    \newcommand{\p}{\phi}
    \newcommand{\ph}{\varphi}
  \renewcommand{\P}{\Phi}
  \renewcommand{\r}{\varrho}
    \newcommand{\s}{\sigma}
  \renewcommand{\S}{\Sigma}
  \renewcommand{\t}{\theta}
    \newcommand{\T}{\Theta}
  %\renewcommand{\v}{\vartheta}
    \newcommand{\x}{\xi}
    \newcommand{\X}{\Xi}
    \newcommand{\Y}{\Upsilon}
    \newcommand{\z}{\zeta}

  \renewcommand{\AA}{{\cal A}}
    \newcommand{\BB}{{\cal B}}
    \newcommand{\CC}{{\cal C}}
    \newcommand{\cc}{{\cal c}}
    \newcommand{\DD}{{\cal D}}
    \newcommand{\EE}{{\cal E}}
    \newcommand{\FF}{{\cal F}}
    \newcommand{\GG}{{\cal G}}
    \newcommand{\HH}{{\cal H}}
    \newcommand{\II}{{\cal I}}
    \newcommand{\KK}{{\cal K}}
    \newcommand{\LL}{{\cal L}}
    \newcommand{\MM}{{\cal M}}
    \newcommand{\NN}{{\cal N}}
    \newcommand{\oNN}{\overline\NN}
    \newcommand{\OO}{{\cal O}}
    \newcommand{\PP}{{\cal P}}
    \newcommand{\QQ}{{\cal Q}}
    \newcommand{\RR}{{\cal R}}
  \renewcommand{\SS}{{\cal S}}
    \newcommand{\TT}{{\cal T}}
    \newcommand{\uu}{{\cal u}}
    \newcommand{\UU}{{\cal U}}
    \newcommand{\VV}{{\cal V}}
    \newcommand{\XX}{{\cal X}}
    \newcommand{\xx}{\mathcal{x}}
    \newcommand{\YY}{{\cal Y}}
    \newcommand{\SOSO}{{\cal SO}}
    \newcommand{\GLGL}{{\cal GL}}

    \newcommand{\Ee}{{\rm E}}

  \newcommand{\NNN}{{\mathbb{N}}}
  \newcommand{\III}{{\mathbb{I}}}
  \newcommand{\ZZZ}{{\mathbb{Z}}}
  %\newcommand{\RRR}{{\mathrm{I\!R}}}
  \newcommand{\RRR}{{\mathbb{R}}}
  \newcommand{\SSS}{{\mathbb{S}}}
  \newcommand{\CCC}{{\mathbb{C}}}
  \newcommand{\DDD}{{\mathbb{D}}}
  \newcommand{\one}{{{\bf 1}}}
  \newcommand{\eee}{\text{e}}

  \newcommand{\NNNN}{{\overline{\cal N}}}

  \renewcommand{\[}{\Big[}
  \renewcommand{\]}{\Big]}
  \renewcommand{\(}{\Big(}
  \renewcommand{\)}{\Big)}
  \renewcommand{\|}{\,|\,}
  \renewcommand{\;}{\,;\,}
  \renewcommand{\=}{\!=\!}
    \newcommand{\<}{\left\langle}
  \renewcommand{\>}{\right\rangle}

  \newcommand{\na}[1][]{{\nabla_{\!\!#1}}}
  \newcommand{\he}[1][]{{\nabla_{\!\!#1}^2}}
  \newcommand{\Prob}{{\rm Prob}}
  \newcommand{\Dir}{{\rm Dir}}
  \newcommand{\Beta}{{\rm Beta}}
  \newcommand{\Bern}{{\rm Bern}}
  \newcommand{\Bin}{{\rm Bin}}
  \newcommand{\Mult}{{\rm Mult}}
  \newcommand{\Aut}{{\rm Aut}}
  \newcommand{\cor}{{\rm cor}}
  \newcommand{\corr}{{\rm corr}}
  \newcommand{\sd}{{\rm sd}}
  \newcommand{\tr}{{\rm tr}}
  \newcommand{\Tr}{{\rm Tr}}
  \newcommand{\rank}{{\rm rank}}
  \newcommand{\diag}{{\rm diag}}
  \newcommand{\dom}{{\rm dom}}
  \newcommand{\id}{{\rm id}}
  \newcommand{\Id}{{\rm\bf I}}
  \newcommand{\Gl}{{\rm Gl}}
  \renewcommand{\th}{\ensuremath{{}^\text{th}} }
  \newcommand{\lag}{\mathcal{L}}
  \newcommand{\inn}{\rfloor}
  \newcommand{\lie}{\pounds}
  \newcommand{\longto}{\longrightarrow}
  \newcommand{\speer}{\parbox{0.4ex}{\raisebox{0.8ex}{$\nearrow$}}}
  \renewcommand{\dag}{ {}^\dagger }
  \newcommand{\blbox}{\rule{1ex}{1ex}}
  \newcommand{\Ji}{J^\sharp}
  \newcommand{\h}{{}^\star}
  \newcommand{\w}{\wedge}
  \newcommand{\too}{\longrightarrow}
  \newcommand{\oot}{\longleftarrow}
  \newcommand{\To}{\Rightarrow}
  \newcommand{\oT}{\Leftarrow}
  \newcommand{\oTo}{\Leftrightarrow}
  \renewcommand{\iff}{~\Longleftrightarrow~}
  \newcommand{\Too}{\;\Longrightarrow\;}
  \newcommand{\oto}{\leftrightarrow}
  \newcommand{\ot}{\leftarrow}
  \newcommand{\ootoo}{\longleftrightarrow}
  \newcommand{\ow}{\stackrel{\circ}\wedge}
  \newcommand{\defeq}{\stackrel{\hspace{0.2ex}{}_\Delta}=}
%  \newcommand{\defeq}{{\overstack\Delta =}}
  \newcommand{\feed}{\nonumber \\}
  \newcommand{\comma}{~,\quad}
  \newcommand{\period}{~.\quad}
  \newcommand{\del}{\partial}
%  \newcommand{\quabla}{\Delta}
  \newcommand{\point}{$\bullet~~$}
  \newcommand{\doubletilde}{ ~ \raisebox{0.3ex}{$\widetilde {}$} \raisebox{0.6ex}{$\widetilde {}$} \!\! }
  \newcommand{\topcirc}{\parbox{0ex}{~\raisebox{2.5ex}{${}^\circ$}}}
  \newcommand{\topdot} {\parbox{0ex}{~\raisebox{2.5ex}{$\cdot$}}}
  \newcommand{\topddot} {\parbox{0ex}{~\raisebox{1.3ex}{$\ddot{~}$}}}
  \newcommand{\sym}{\topcirc}
  \newcommand{\tsum}{\textstyle\sum}
  \newcommand{\st}{\quad\text{s.t.}\quad}

  \newcommand{\half}{\ensuremath{\frac{1}{2}}}
  \newcommand{\third}{\ensuremath{\frac{1}{3}}}
  \newcommand{\fourth}{\ensuremath{\frac{1}{4}}}

  \newcommand{\ubar}{\underline}
  %\renewcommand{\vec}{\underline}
  \renewcommand{\vec}{\boldsymbol}
  %\renewcommand{\_}{\underset}
  %\renewcommand{\^}{\overset}
  %\renewcommand{\*}{{\rm\raisebox{-.6ex}{\text{*}}{}}}
  \renewcommand{\*}{\text{\footnotesize\raisebox{-.4ex}{*}{}}}

  \newcommand{\gto}{{\raisebox{.5ex}{${}_\rightarrow$}}}
  \newcommand{\gfrom}{{\raisebox{.5ex}{${}_\leftarrow$}}}
  \newcommand{\gnto}{{\raisebox{.5ex}{${}_\nrightarrow$}}}
  \newcommand{\gnfrom}{{\raisebox{.5ex}{${}_\nleftarrow$}}}

  %\newcommand{\RND}{{\SS}}
  %\newcommand{\IF}{\text{if }}
  %\newcommand{\AND}{\textsc{and }}
  %\newcommand{\OR}{\textsc{or }}
  %\newcommand{\XOR}{\textsc{xor }}
  %\newcommand{\NOT}{\textsc{not }}

  %\newcommand{\argmax}[1]{{\rm arg}\!\max_{#1}}
  %\newcommand{\argmin}[1]{{\rm arg}\!\min_{#1}}
  \DeclareMathOperator*{\argmax}{argmax}
  \DeclareMathOperator*{\argmin}{argmin}
  \DeclareMathOperator{\sign}{sign}
  \DeclareMathOperator{\acos}{acos}
  \DeclareMathOperator{\unifies}{unifies}
  \DeclareMathOperator{\Span}{span}
  \newcommand{\ortho}{\perp}
  %\newcommand{\argmax}[1]{\underset{~#1}{\text{argmax}}\;}
  %\newcommand{\argmin}[1]{\underset{~#1}{\text{argmin}}\;}
  \newcommand{\ee}[1]{\ensuremath{\cdot10^{#1}}}
  \newcommand{\sub}[1]{\ensuremath{_{\text{#1}}}}
  \newcommand{\up}[1]{\ensuremath{^{\text{#1}}}}
  \newcommand{\kld}[3][{}]{D_{#1}\big(#2\,\big|\!\big|\,#3\big)}
  %\newcommand{\kld}[2]{D\big(#1:#2\big)}
  \newcommand{\sprod}[2]{\big<#1\,,\,#2\big>}
  \newcommand{\End}{\text{End}}
  \newcommand{\txt}[1]{\quad\text{#1}\quad}
  \newcommand{\Over}[2]{\genfrac{}{}{0pt}{0}{#1}{#2}}
  %\newcommand{\mat}[1]{{\bf #1}}
  \newcommand{\arr}[2]{\hspace*{-.5ex}\begin{array}{#1}#2\end{array}\hspace*{-.5ex}}
  \newcommand{\mat}[3][.9]{
    \renewcommand{\arraystretch}{#1}{\scriptscriptstyle{\left(
      \hspace*{-1ex}\begin{array}{#2}#3\end{array}\hspace*{-1ex}
    \right)}}\renewcommand{\arraystretch}{1.2}
  }
  \newcommand{\Mat}[3][.9]{
    \renewcommand{\arraystretch}{#1}{\scriptscriptstyle{\left[
      \hspace*{-1ex}\begin{array}{#2}#3\end{array}\hspace*{-1ex}
    \right]}}\renewcommand{\arraystretch}{1.2}
  }
  \newcommand{\case}[2][ll]{\left\{\arr{#1}{#2}\right.}
  \newcommand{\seq}[1]{\textsf{\<#1\>}}
  \newcommand{\seqq}[1]{\textsf{#1}}
  \newcommand{\floor}[1]{\lfloor#1\rfloor}
  \newcommand{\Exp}[2][]{\text{E}_{#1}\{#2\}}
  \newcommand{\Var}[2][]{\text{Var}_{#1}\{#2\}}
  \newcommand{\cov}[2][]{\text{cov}_{#1}\{#2\}}

%\newcommand{\Exp}[2]{\left\langle{#2}\right\rangle_{#1}}
  \newcommand{\ex}{\setminus}

  \providecommand{\href}[2]{{\color{blue}USE PDFLATEX!}}
  \providecommand{\url}[2]{\href{#1}{{\color{blue}#2}}}
%  \newcommand{\link}[1]{\href{{\protect #1}}{\texttt{\protect #1}}}
  \newcommand{\anchor}[2]{\begin{picture}(0,0)\put(#1){#2}\end{picture}}
  \newcommand{\pagebox}{\begin{picture}(0,0)\put(-3,-23){
    \textcolor[rgb]{.5,1,.5}{\framebox[\textwidth]{\rule[-\textheight]{0pt}{0pt}}}}
    \end{picture}}

  \newcommand{\hide}[1]{
    \begin{list}{}{\leftmargin0ex \rightmargin0ex \topsep0ex \parsep0ex}
       \helvetica{5}{1}{m}{n}
       \renewcommand{\section}{\par SECTION: }
       \renewcommand{\subsection}{\par SUBSECTION: }
       \item[$~~\blacktriangleright$]
       #1%$\blacktriangleleft~~$
       \message{^^JHIDE--Warning!^^J}
    \end{list}
  }
  %\newcommand{\hide}[1]{{\tt[hide:~}{\footnotesize\sf #1}{\tt]}\message{^^JHIDE--Warning!^^J}}
  \newcommand{\Hide}{\renewcommand{\hide}[1]{\message{^^JHIDE--Warning (hidden)!^^J}}}
  \newcommand{\HIDE}{\renewcommand{\hide}[1]{}}
  \newcommand{\fullhide}[1]{\message{^^JHIDE--Warning (hidden)!^^J}}
  \newcommand{\todo}[1]{{\tt[TODO: #1]}\message{^^JTODO--Warning: #1^^J}}
  \newcommand{\Todo}{\renewcommand{\todo}[1]{\message{^^JTODO--Warning (hidden)!^^J}}}
  %\renewcommand{\title}[1]{\renewcommand{\thetitle}{#1}}
  \newcommand{\myauthor}[1]{\author{#1}\newcommand{\theauthor}{#1}}%\@author}
  \newcommand{\mytitle}[1]{\title{#1}\newcommand{\thetitle}{#1}}%\@title}
  \newcommand{\header}{\begin{document}\mytitle\cleardefs}
  \newcommand{\contents}{{\tableofcontents}\renewcommand{\contents}{}}
  \newcommand{\footer}{\small\bibliography{marc,bibs}\end{document}}
  \newcommand{\widepaper}{\usepackage{geometry}\geometry{a4paper,hdivide={25mm,*,25mm},vdivide={25mm,*,25mm}}}
  \newcommand{\moviex}[2]{\movie[externalviewer]{#1}{#2}} %\pdflatex\usepackage{multimedia}
  \newcommand{\rbox}[1]{\fboxrule2mm\fcolorbox[rgb]{1,.85,.85}{1,.85,.85}{#1}}
  \newcommand{\mpage}[2]{{\begin{minipage}{#1\columnwidth}#2\end{minipage}}}
  \newcommand{\redbox}[2]{\fboxrule1mm\fcolorbox[rgb]{1,.7,.7}{1,.7,.7}{\begin{minipage}{#1\columnwidth}\center#2\end{minipage}}}
  \newcommand{\onecol}[2]{
    \begin{minipage}[c]{#1\columnwidth}#2\end{minipage}}
  \newcommand{\twocol}[5][0]{
    \begin{minipage}[c]{#2\columnwidth}#4\end{minipage}\hspace*{#1\columnwidth}%
    \begin{minipage}[c]{#3\columnwidth}#5\end{minipage}}
  \newcommand{\threecol}[7][0]{%
    \begin{minipage}[c]{#2\columnwidth}#5\end{minipage}\hspace*{#1\columnwidth}%
    \begin{minipage}[c]{#3\columnwidth}#6\end{minipage}\hspace*{#1\columnwidth}%
    \begin{minipage}[c]{#4\columnwidth}#7\end{minipage}}
  \newcommand{\threecoltext}[7][c]{
    \begin{minipage}[#1]{#2\textwidth}#5\end{minipage}%
    \begin{minipage}[#1]{#3\textwidth}#6\end{minipage}%
    \begin{minipage}[#1]{#4\textwidth}#7\end{minipage}}
  \newcommand{\threecoltop}[7][0]{%
   \begin{minipage}[t]{#2\columnwidth}#5\end{minipage}\hspace*{#1\columnwidth}%
   \begin{minipage}[t]{#3\columnwidth}#6\end{minipage}\hspace*{#1\columnwidth}%
   \begin{minipage}[t]{#4\columnwidth}#7\end{minipage}}
  \newcommand{\fourcol}[9][0]{%
   \begin{minipage}[c]{#2\columnwidth}#6\end{minipage}\hspace*{#1\columnwidth}%
   \begin{minipage}[c]{#3\columnwidth}#7\end{minipage}\hspace*{#1\columnwidth}%
   \begin{minipage}[c]{#4\columnwidth}#8\end{minipage}\hspace*{#1\columnwidth}%
   \begin{minipage}[c]{#5\columnwidth}#9\end{minipage}}
  \newcommand{\helvetica}[4]{\setlength{\unitlength}{1pt}\fontsize{#1}{#1}\linespread{#2}\usefont{OT1}{phv}{#3}{#4}}
  \newcommand{\helve}[1]{\helvetica{#1}{1.5}{m}{n}}
  \newcommand{\german}{\usepackage[german]{babel}\usepackage[utf8]{inputenc}}

\newcommand{\norm}[1]{|\!|#1|\!|}
\newcommand{\expr}[1]{[\hspace{-.2ex}[#1]\hspace{-.2ex}]}

\newcommand{\Jwi}{J^\sharp_W}
\newcommand{\THi}{T^\sharp_H}
\newcommand{\Jci}{J^\natural_C}
\newcommand{\hJi}{{\bar J}^\sharp}
\renewcommand{\|}{\,|\,}
\renewcommand{\=}{\!=\!}
\newcommand{\myminus}{{\hspace*{-.0pt}\text{\rm -}\hspace*{-.5pt}}}
\newcommand{\myplus}{{\hspace*{-.0pt}\text{\rm +}\hspace*{-.5pt}}}
\newcommand{\1}{{\myminus1}}
\newcommand{\2}{{\myminus2}}
\newcommand{\3}{{\myminus3}}
\newcommand{\mT}{{\text{\rm -}\hspace*{-1pt}\top}}
\newcommand{\po}{{\myplus1}}
\newcommand{\pt}{{\myplus2}}
%\renewcommand{\-}{\myminus}
%\newcommand{\+}{\myplus}
\renewcommand{\T}{{\!\top\!}}
\newcommand{\xT}{{\underline x}}
\newcommand{\uT}{{\underline u}}
\newcommand{\zT}{{\underline z}}
\newcommand{\Sum}{\textstyle\sum}
\newcommand{\Int}{\textstyle\int}
\newcommand{\Prod}{\textstyle\prod}


\newenvironment{centy}{
\vspace{15mm}
\large
\hspace*{5mm}
\begin{minipage}{8cm}\it\color{blue}
}{
\end{minipage}
}

\newcommand{\old}{{\text{old}}}
\newcommand{\new}{{\text{new}}}
\newcommand{\MAP}{{\text{MAP}}}
\newcommand{\ML}{{\text{ML}}}

\newcommand{\redArrow}{\quad\anchor{0,-1}{\includegraphics[scale=.5]{figs/redArrow}}}
\newcommand{\pub}[1]{{\color{green}\helvetica{8}{1.3}{m}{n}#1\\}}
\DeclareMathOperator{\opKL}{KL}
\newcommand{\KL}[2]{\opKL\big(#1\,\big|\!\big|\,#2\big)} %\left(#1 |\!| #2\right)}

\renewcommand{\show}[2][.8]{\centerline{\includegraphics[width=#1\columnwidth]{#2}}}
\newcommand{\showh}[2][.8]{\includegraphics[width=#1\columnwidth]{#2}}
\newcommand{\shows}[2][.8]{\centerline{\includegraphics[scale=#1]{#2}}}
\newcommand{\showhs}[2][.8]{\includegraphics[scale=#1]{#2}}
\newcommand{\mov}[2]{\movie[externalviewer]{{\color{blue}\small #1}}{movies/#2}}
\newcommand{\movex}[2]{\movie[externalviewer]{#1}{#2}} %\pdflatex\usepackage{multimedia}
%\newcommand{\movgb}[1]{\hfill\movie[externalviewer]{\small[movie]}{/home/mtoussai/movies/10-goalDirectedBehavior/#1}}
\newcommand{\movh}[3][loop]{
\movie[#1]{\showh[#2]{movies/#3.png}}{movies/#3.avi}%
\movie[externalviewer]{$\circ$}{movies/#3.avi}
}
\newcommand{\movc}[3][loop]{\centerline{\movh[#1]{#2}{#3}}}
\newcommand{\cen}[1]{\centerline{#1}}

\newcommand{\citing}[1]{
{\color{citcol}\tiny#1\par}
}

\newcommand{\cit}[3]{
\par\smallskip
{\color{greencol}\tiny #1: \emph{#2}. #3 \par}
}

\newcommand{\citurl}[4]{
\par\smallskip
{\color{greencol}\tiny #1: \protect{\href{#4}{\color{blue}{#2.}}} #3 \par}
}

\newcommand{\cito}[3]{
\par\smallskip
{\color{bluecol}\tiny #1: \emph{#2}. #3 \par}
}

\newcommand{\redoMacrosInProof}{
  \renewcommand{\d}{\delta}
%  \renewcommand{\|}{\,|\,}
  \renewcommand{\=}{\!=\!}
}

%% \makeatletter
%% \newenvironment{code}{%
%%   \begin{lrbox}{\@tempboxa}\begin{minipage}{1\columnwidth}\codefont
%% }{
%%   \end{minipage}\end{lrbox}%
%%   \colorbox[rgb]{.95,.95,.95}{\usebox{\@tempboxa}}
%% }\makeatother

\newenvironment{code}{%
\codefont
\begin{shaded}
}{
\end{shaded}
}

%\newcommand{\refeq}[1]{(\ref{#1})}

\usepackage{algorithm}
\usepackage{algpseudocode}
\algrenewcommand{\algorithmicrequire}{\textbf{Input:~~}}
\algrenewcommand{\algorithmicensure}{\textbf{Output:}}
\algrenewcommand{\algorithmiccomment}[1]{\qquad\hfill~\hspace*{-5ex}\textit{// #1}}
\algrenewcommand{\alglinenumber}[1]{\helvetica{6}{1.3}{m}{n}#1:}

\newenvironment{algo}[1][8]{
\quad\begin{minipage}{.8\columnwidth}\helvetica{#1}{1.3}{m}{n}
\medskip\hrule\medskip
\begin{algorithmic}[1]
}{
\end{algorithmic}
\medskip\hrule\medskip
\end{minipage}
}

\usepackage{etoolbox}

%%%%%%%%%%%%%%%%%%%%%%%%%%%%%%%%%%%%%%%%%%%%%%%%%%%%%%%%%%%%%%%%%%%%%%%%%%%%%%%%

\usepackage{multirow}
\usepackage{colortbl}
%\setlength{\jot}{0pt}
%\setlength{\mathindent}{1ex}
\usepackage{empheq}

%%%%%%%%%%%%%%%%%%%%%%%%%%%%%%%%%%%%%%%%%%%%%%%%%%%%%%%%%%%%%%%%%%%%%%%%%%%%%%%

\newcommand{\mypause}{\pause}
%\newcommand{\dom}{{\text{dom}}}
\newcommand{\defi}[1]{\textbf{#1}}
\newcommand{\red}[1]{\emph{\color{red}#1}}
%\newcommand{\ul}{\underline}
\newcommand{\pos}{{\textsf{pos}}}
\newcommand{\eff}{{\textsf{eff}}}
\newcommand{\rot}{{\textsf{rot}}}
\newcommand{\veC}{{\textsf{vec}}}
\newcommand{\quat}{{\textsf{quat}}}
\newcommand{\col}{{\textsf{col}}}
\newcommand{\de}[2]{\frac{\partial #1}{\partial #2}}
\newcommand{\target}{{\text{target}}}
\newcommand{\near}{{\text{near}}}
\newcommand{\qfree}{Q_{\text{free}}}
\renewcommand{\vec}{\boldsymbol}
\newcommand{\lft}{\text{left}}
\newcommand{\rgh}{\text{right}}
\DeclareMathOperator{\real}{real}
\newcommand{\prev}{{\text{prev}}}
\newcommand{\TR}[2]{T_{{#1}\to{#2}}}
\newcommand{\RO}[2]{R_{{#1}\to{#2}}}
\newcommand{\liter}{\helvetica{8}{1.1}{m}{n}\parskip 1ex}
\newcommand{\Fc}{\color{green}F}
\newcommand{\muc}{\color{blue}\mu}
\newcommand{\Astar}{A$^*$}

%for AI course:
\newcommand{\defn}[1]{\textbf{#1}}

%for optimization course:
\newcommand{\adec}{\r_\a^-}
\newcommand{\ainc}{\r_\a^+}
\newcommand{\ldec}{\r_\l^-}
\newcommand{\linc}{\r_\l^+}
\newcommand{\minc}{\r_\m^+}
\newcommand{\mdec}{\r_\m^-}
\newcommand{\lsstop}{\r_{\text{ls}}}


\definecolor{boxcol}{rgb}{.85,.9,.92}
\newcommand{\eqbox}[1]{\centerline{\fboxrule0mm\fcolorbox{boxcol}{boxcol}{#1}}}
\newcommand{\movgb}[1]{\hfill\movie[externalviewer]{\small[movie]}{/home/mtoussai/movies/10-goalDirectedBehavior/#1}}
\newcommand{\demo}[1]{{{\color{blue}[\small #1]}}}

\graphicspath{{../pics-robotics/}{../pics-ML/}{../pics-all/}{../pics-all2/}{../pics-Optim/}}
\DeclareGraphicsExtensions{.pdf,.png,.jpg}

%\usepackage{pdfpages}
%\setbeamercolor{background canvas}{bg=}

\newcommand{\SUM}{\texttt{sum}}
\usepackage{float}

%% prevent pagebreaks before environment
\makeatletter
\newcommand{\NewParNoBreak}[1][\parskip]{\par\vspace*{-\parskip}\vspace*{#1}\nobreak\@afterheading}
\makeatother

%\newcommand{\idx}[2]{\label{IKgn}}

%%%%%%%%%%%%%%%%%%%%%%%%%%%%%%%%%%%%%%%%%%%%%%%%%%%%%%%%%%%%%%%%%%%%%%%%%%%%%%%%



%% \newwrite\tempfile
%% \immediate\openout\tempfile=z.keys.tex

%% \renewcommand{\key}[1]{
%% %%   \addtocounter{mypage}{1}
%% \makeatletter
%% \immediate\write\tempfile{\symbol{`\\}}
%% \makeatother
%%   \immediate\write\tempfile{hyperref[key:#1]{#1(\arabic{mypage})}}
%% %%  % \phantomsection\label{key:#1}
%% %%   %\index{#1@{\hyperref[key:#1]{#1 (\arabic{mysec}:\arabic{mypage})}}|phantom}
%% %%   \addtocounter{mypage}{-1}
%% }


  \graphicspath{{pics/}{../shared/pics/}}

  \title{\course \topic}
  \author{Marc Toussaint}
  \institute{Machine Learning \& Robotics Lab, U Stuttgart}

  \begin{document}

  \rfoot{\vspace*{-5mm}{\tiny
  \textsf{\arabic{mypage}/\pageref{lastpage}}}\hspace*{-4mm}}

  %% title slide!
  \slide{}{
    \thispagestyle{empty}

    \twocol{.27}{.6}{
      \vspace*{-5mm}\hspace*{-15mm}\includegraphics[width=4cm]{\coursepicture}
    }{\center

      \textbf{\fontsize{17}{20}\selectfont \course}

      ~

      %Lecture
      \topic\\

      \vspace{1cm}

      {\tiny~\emph{\keywords}~\\}

      \vspace{1cm}

      Marc Toussaint
      
      University of Stuttgart

      \coursedate

      ~

    }
  }
}

\newcommand{\slide}[2]{
  \slidefont
  \incpage\begin{frame}
  \addcontentsline{toc}{section}{#1}
  \vfill
  {\headerfont #1} \vspace*{-2ex}
  \begin{itemize}\item[]~\\
    #2
  \end{itemize}
  \vfill
  \end{frame}
}

\newenvironment{slidecore}[1]{
  \slidefont\incpage
  \addcontentsline{toc}{section}{#1}
  \vfill
  {\headerfont #1} \vspace*{-2ex}
  \begin{itemize}\item[]~\\
}{
  \end{itemize}
  \vfill
}


\providecommand{\key}[1]{
  \addtocounter{mypage}{1}
% \immediate\write\keyfile{#1}
  \addtocontents{toc}{\hyperref[key:#1]{#1 (\arabic{mypage})}}
%  \phantomsection\label{key:#1}
%  \index{#1@{\hyperref[key:#1]{#1 (\arabic{mysec}:\arabic{mypage})}}|phantom}
  \addtocounter{mypage}{-1}
}

\providecommand{\course}{}

\providecommand{\subtopic}{}

\providecommand{\sublecture}[2]{
  \renewcommand{\subtopic}{#1}
  \slide{#1}{#2}
}

\providecommand{\story}[1]{
~

Motivation: {\tiny #1}\clearpage
}

\newenvironment{items}[1][9]{
\par\setlength{\unitlength}{1pt}\fontsize{#1}{#1}\linespread{1.2}\selectfont
\begin{list}{--}{\leftmargin4ex \rightmargin0ex \labelsep1ex \labelwidth2ex
\topsep0pt \parsep0ex \itemsep3pt}
}{
\end{list}
}

\providecommand{\slidesfoot}{
  \end{document}
}


  \slideshead
}

\providecommand{\exercises}{
  \newcommand{\exerciseshead}{
  \documentclass[10pt,fleqn]{article}
  \stdpackages

  \definecolor{bluecol}{rgb}{0,0,.5}
  \definecolor{greencol}{rgb}{0,.4,0}
  \definecolor{shadecolor}{gray}{0.9}
  \usepackage[
    %    pdftex%,
    %%    letterpaper,
    %    bookmarks,
    %    bookmarksnumbered,
    colorlinks,
    urlcolor=bluecol,
    citecolor=black,
    linkcolor=bluecol,
    %    pagecolor=bluecol,
    pdfborder={0 0 0},
    %pdfborderstyle={/S/U/W 1},
    %%    backref,     %link from bibliography back to sections
    %%    pagebackref, %link from bibliography back to pages
    %%    pdfstartview=FitH, %fitwidth instead of fit window
    pdfpagemode=UseNone, %UseOutlines, %bookmarks are displayed by acrobat
    pdftitle={\course},
    pdfauthor={Marc Toussaint},
    pdfkeywords={}
  ]{hyperref}
  \DeclareGraphicsExtensions{.pdf,.png,.jpg,.eps}

  \renewcommand{\r}{\varrho}
  \renewcommand{\l}{\lambda}
  \renewcommand{\L}{\Lambda}
  \renewcommand{\b}{\beta}
  \renewcommand{\d}{\delta}
  \renewcommand{\k}{\kappa}
  \renewcommand{\t}{\theta}
  \renewcommand{\O}{\Omega}
  \renewcommand{\o}{\omega}
  \renewcommand{\SS}{{\cal S}}
  \renewcommand{\=}{\!=\!}
  %\renewcommand{\boldsymbol}{}
  %\renewcommand{\Chapter}{\chapter}
  %\renewcommand{\Subsection}{\subsection}

  \renewcommand{\baselinestretch}{1.1}
  \geometry{a4paper,headsep=7mm,hdivide={15mm,*,15mm},vdivide={20mm,*,15mm}}

  \fancyhead[L]{\thetitle, \textit{Marc Toussaint}---\coursedate}
  \fancyhead[R]{\thepage}
  \fancyhead[C]{}
  \fancyfoot{}
  \pagestyle{fancy}

  \parindent 0pt
  \parskip 0.5ex

  \newcommand{\codefont}{\helvetica{8}{1.2}{m}{n}}

  %auto-ignore
  \renewcommand{\a}{\alpha}
  \renewcommand{\b}{\beta}
  \renewcommand{\d}{\delta}
    \newcommand{\D}{\Delta}
    \newcommand{\e}{\epsilon}
    \newcommand{\g}{\gamma}
    \newcommand{\G}{\Gamma}
  \renewcommand{\l}{\lambda}
  \renewcommand{\L}{\Lambda}
    \newcommand{\m}{\mu}
    \newcommand{\n}{\nu}
    \newcommand{\N}{\nabla}
  \renewcommand{\k}{\kappa}
  \renewcommand{\o}{\omega}
  \renewcommand{\O}{\Omega}
    \newcommand{\p}{\phi}
    \newcommand{\ph}{\varphi}
  \renewcommand{\P}{\Phi}
  \renewcommand{\r}{\varrho}
    \newcommand{\s}{\sigma}
  \renewcommand{\S}{\Sigma}
  \renewcommand{\t}{\theta}
    \newcommand{\T}{\Theta}
  %\renewcommand{\v}{\vartheta}
    \newcommand{\x}{\xi}
    \newcommand{\X}{\Xi}
    \newcommand{\Y}{\Upsilon}
    \newcommand{\z}{\zeta}

  \renewcommand{\AA}{{\cal A}}
    \newcommand{\BB}{{\cal B}}
    \newcommand{\CC}{{\cal C}}
    \newcommand{\cc}{{\cal c}}
    \newcommand{\DD}{{\cal D}}
    \newcommand{\EE}{{\cal E}}
    \newcommand{\FF}{{\cal F}}
    \newcommand{\GG}{{\cal G}}
    \newcommand{\HH}{{\cal H}}
    \newcommand{\II}{{\cal I}}
    \newcommand{\KK}{{\cal K}}
    \newcommand{\LL}{{\cal L}}
    \newcommand{\MM}{{\cal M}}
    \newcommand{\NN}{{\cal N}}
    \newcommand{\oNN}{\overline\NN}
    \newcommand{\OO}{{\cal O}}
    \newcommand{\PP}{{\cal P}}
    \newcommand{\QQ}{{\cal Q}}
    \newcommand{\RR}{{\cal R}}
  \renewcommand{\SS}{{\cal S}}
    \newcommand{\TT}{{\cal T}}
    \newcommand{\uu}{{\cal u}}
    \newcommand{\UU}{{\cal U}}
    \newcommand{\VV}{{\cal V}}
    \newcommand{\XX}{{\cal X}}
    \newcommand{\xx}{\mathcal{x}}
    \newcommand{\YY}{{\cal Y}}
    \newcommand{\SOSO}{{\cal SO}}
    \newcommand{\GLGL}{{\cal GL}}

    \newcommand{\Ee}{{\rm E}}

  \newcommand{\NNN}{{\mathbb{N}}}
  \newcommand{\III}{{\mathbb{I}}}
  \newcommand{\ZZZ}{{\mathbb{Z}}}
  %\newcommand{\RRR}{{\mathrm{I\!R}}}
  \newcommand{\RRR}{{\mathbb{R}}}
  \newcommand{\SSS}{{\mathbb{S}}}
  \newcommand{\CCC}{{\mathbb{C}}}
  \newcommand{\DDD}{{\mathbb{D}}}
  \newcommand{\one}{{{\bf 1}}}
  \newcommand{\eee}{\text{e}}

  \newcommand{\NNNN}{{\overline{\cal N}}}

  \renewcommand{\[}{\Big[}
  \renewcommand{\]}{\Big]}
  \renewcommand{\(}{\Big(}
  \renewcommand{\)}{\Big)}
  \renewcommand{\|}{\,|\,}
  \renewcommand{\;}{\,;\,}
  \renewcommand{\=}{\!=\!}
    \newcommand{\<}{\left\langle}
  \renewcommand{\>}{\right\rangle}

  \newcommand{\na}[1][]{{\nabla_{\!\!#1}}}
  \newcommand{\he}[1][]{{\nabla_{\!\!#1}^2}}
  \newcommand{\Prob}{{\rm Prob}}
  \newcommand{\Dir}{{\rm Dir}}
  \newcommand{\Beta}{{\rm Beta}}
  \newcommand{\Bern}{{\rm Bern}}
  \newcommand{\Bin}{{\rm Bin}}
  \newcommand{\Mult}{{\rm Mult}}
  \newcommand{\Aut}{{\rm Aut}}
  \newcommand{\cor}{{\rm cor}}
  \newcommand{\corr}{{\rm corr}}
  \newcommand{\sd}{{\rm sd}}
  \newcommand{\tr}{{\rm tr}}
  \newcommand{\Tr}{{\rm Tr}}
  \newcommand{\rank}{{\rm rank}}
  \newcommand{\diag}{{\rm diag}}
  \newcommand{\dom}{{\rm dom}}
  \newcommand{\id}{{\rm id}}
  \newcommand{\Id}{{\rm\bf I}}
  \newcommand{\Gl}{{\rm Gl}}
  \renewcommand{\th}{\ensuremath{{}^\text{th}} }
  \newcommand{\lag}{\mathcal{L}}
  \newcommand{\inn}{\rfloor}
  \newcommand{\lie}{\pounds}
  \newcommand{\longto}{\longrightarrow}
  \newcommand{\speer}{\parbox{0.4ex}{\raisebox{0.8ex}{$\nearrow$}}}
  \renewcommand{\dag}{ {}^\dagger }
  \newcommand{\blbox}{\rule{1ex}{1ex}}
  \newcommand{\Ji}{J^\sharp}
  \newcommand{\h}{{}^\star}
  \newcommand{\w}{\wedge}
  \newcommand{\too}{\longrightarrow}
  \newcommand{\oot}{\longleftarrow}
  \newcommand{\To}{\Rightarrow}
  \newcommand{\oT}{\Leftarrow}
  \newcommand{\oTo}{\Leftrightarrow}
  \renewcommand{\iff}{~\Longleftrightarrow~}
  \newcommand{\Too}{\;\Longrightarrow\;}
  \newcommand{\oto}{\leftrightarrow}
  \newcommand{\ot}{\leftarrow}
  \newcommand{\ootoo}{\longleftrightarrow}
  \newcommand{\ow}{\stackrel{\circ}\wedge}
  \newcommand{\defeq}{\stackrel{\hspace{0.2ex}{}_\Delta}=}
%  \newcommand{\defeq}{{\overstack\Delta =}}
  \newcommand{\feed}{\nonumber \\}
  \newcommand{\comma}{~,\quad}
  \newcommand{\period}{~.\quad}
  \newcommand{\del}{\partial}
%  \newcommand{\quabla}{\Delta}
  \newcommand{\point}{$\bullet~~$}
  \newcommand{\doubletilde}{ ~ \raisebox{0.3ex}{$\widetilde {}$} \raisebox{0.6ex}{$\widetilde {}$} \!\! }
  \newcommand{\topcirc}{\parbox{0ex}{~\raisebox{2.5ex}{${}^\circ$}}}
  \newcommand{\topdot} {\parbox{0ex}{~\raisebox{2.5ex}{$\cdot$}}}
  \newcommand{\topddot} {\parbox{0ex}{~\raisebox{1.3ex}{$\ddot{~}$}}}
  \newcommand{\sym}{\topcirc}
  \newcommand{\tsum}{\textstyle\sum}
  \newcommand{\st}{\quad\text{s.t.}\quad}

  \newcommand{\half}{\ensuremath{\frac{1}{2}}}
  \newcommand{\third}{\ensuremath{\frac{1}{3}}}
  \newcommand{\fourth}{\ensuremath{\frac{1}{4}}}

  \newcommand{\ubar}{\underline}
  %\renewcommand{\vec}{\underline}
  \renewcommand{\vec}{\boldsymbol}
  %\renewcommand{\_}{\underset}
  %\renewcommand{\^}{\overset}
  %\renewcommand{\*}{{\rm\raisebox{-.6ex}{\text{*}}{}}}
  \renewcommand{\*}{\text{\footnotesize\raisebox{-.4ex}{*}{}}}

  \newcommand{\gto}{{\raisebox{.5ex}{${}_\rightarrow$}}}
  \newcommand{\gfrom}{{\raisebox{.5ex}{${}_\leftarrow$}}}
  \newcommand{\gnto}{{\raisebox{.5ex}{${}_\nrightarrow$}}}
  \newcommand{\gnfrom}{{\raisebox{.5ex}{${}_\nleftarrow$}}}

  %\newcommand{\RND}{{\SS}}
  %\newcommand{\IF}{\text{if }}
  %\newcommand{\AND}{\textsc{and }}
  %\newcommand{\OR}{\textsc{or }}
  %\newcommand{\XOR}{\textsc{xor }}
  %\newcommand{\NOT}{\textsc{not }}

  %\newcommand{\argmax}[1]{{\rm arg}\!\max_{#1}}
  %\newcommand{\argmin}[1]{{\rm arg}\!\min_{#1}}
  \DeclareMathOperator*{\argmax}{argmax}
  \DeclareMathOperator*{\argmin}{argmin}
  \DeclareMathOperator{\sign}{sign}
  \DeclareMathOperator{\acos}{acos}
  \DeclareMathOperator{\unifies}{unifies}
  \DeclareMathOperator{\Span}{span}
  \newcommand{\ortho}{\perp}
  %\newcommand{\argmax}[1]{\underset{~#1}{\text{argmax}}\;}
  %\newcommand{\argmin}[1]{\underset{~#1}{\text{argmin}}\;}
  \newcommand{\ee}[1]{\ensuremath{\cdot10^{#1}}}
  \newcommand{\sub}[1]{\ensuremath{_{\text{#1}}}}
  \newcommand{\up}[1]{\ensuremath{^{\text{#1}}}}
  \newcommand{\kld}[3][{}]{D_{#1}\big(#2\,\big|\!\big|\,#3\big)}
  %\newcommand{\kld}[2]{D\big(#1:#2\big)}
  \newcommand{\sprod}[2]{\big<#1\,,\,#2\big>}
  \newcommand{\End}{\text{End}}
  \newcommand{\txt}[1]{\quad\text{#1}\quad}
  \newcommand{\Over}[2]{\genfrac{}{}{0pt}{0}{#1}{#2}}
  %\newcommand{\mat}[1]{{\bf #1}}
  \newcommand{\arr}[2]{\hspace*{-.5ex}\begin{array}{#1}#2\end{array}\hspace*{-.5ex}}
  \newcommand{\mat}[3][.9]{
    \renewcommand{\arraystretch}{#1}{\scriptscriptstyle{\left(
      \hspace*{-1ex}\begin{array}{#2}#3\end{array}\hspace*{-1ex}
    \right)}}\renewcommand{\arraystretch}{1.2}
  }
  \newcommand{\Mat}[3][.9]{
    \renewcommand{\arraystretch}{#1}{\scriptscriptstyle{\left[
      \hspace*{-1ex}\begin{array}{#2}#3\end{array}\hspace*{-1ex}
    \right]}}\renewcommand{\arraystretch}{1.2}
  }
  \newcommand{\case}[2][ll]{\left\{\arr{#1}{#2}\right.}
  \newcommand{\seq}[1]{\textsf{\<#1\>}}
  \newcommand{\seqq}[1]{\textsf{#1}}
  \newcommand{\floor}[1]{\lfloor#1\rfloor}
  \newcommand{\Exp}[2][]{\text{E}_{#1}\{#2\}}
  \newcommand{\Var}[2][]{\text{Var}_{#1}\{#2\}}
  \newcommand{\cov}[2][]{\text{cov}_{#1}\{#2\}}

%\newcommand{\Exp}[2]{\left\langle{#2}\right\rangle_{#1}}
  \newcommand{\ex}{\setminus}

  \providecommand{\href}[2]{{\color{blue}USE PDFLATEX!}}
  \providecommand{\url}[2]{\href{#1}{{\color{blue}#2}}}
%  \newcommand{\link}[1]{\href{{\protect #1}}{\texttt{\protect #1}}}
  \newcommand{\anchor}[2]{\begin{picture}(0,0)\put(#1){#2}\end{picture}}
  \newcommand{\pagebox}{\begin{picture}(0,0)\put(-3,-23){
    \textcolor[rgb]{.5,1,.5}{\framebox[\textwidth]{\rule[-\textheight]{0pt}{0pt}}}}
    \end{picture}}

  \newcommand{\hide}[1]{
    \begin{list}{}{\leftmargin0ex \rightmargin0ex \topsep0ex \parsep0ex}
       \helvetica{5}{1}{m}{n}
       \renewcommand{\section}{\par SECTION: }
       \renewcommand{\subsection}{\par SUBSECTION: }
       \item[$~~\blacktriangleright$]
       #1%$\blacktriangleleft~~$
       \message{^^JHIDE--Warning!^^J}
    \end{list}
  }
  %\newcommand{\hide}[1]{{\tt[hide:~}{\footnotesize\sf #1}{\tt]}\message{^^JHIDE--Warning!^^J}}
  \newcommand{\Hide}{\renewcommand{\hide}[1]{\message{^^JHIDE--Warning (hidden)!^^J}}}
  \newcommand{\HIDE}{\renewcommand{\hide}[1]{}}
  \newcommand{\fullhide}[1]{\message{^^JHIDE--Warning (hidden)!^^J}}
  \newcommand{\todo}[1]{{\tt[TODO: #1]}\message{^^JTODO--Warning: #1^^J}}
  \newcommand{\Todo}{\renewcommand{\todo}[1]{\message{^^JTODO--Warning (hidden)!^^J}}}
  %\renewcommand{\title}[1]{\renewcommand{\thetitle}{#1}}
  \newcommand{\myauthor}[1]{\author{#1}\newcommand{\theauthor}{#1}}%\@author}
  \newcommand{\mytitle}[1]{\title{#1}\newcommand{\thetitle}{#1}}%\@title}
  \newcommand{\header}{\begin{document}\mytitle\cleardefs}
  \newcommand{\contents}{{\tableofcontents}\renewcommand{\contents}{}}
  \newcommand{\footer}{\small\bibliography{marc,bibs}\end{document}}
  \newcommand{\widepaper}{\usepackage{geometry}\geometry{a4paper,hdivide={25mm,*,25mm},vdivide={25mm,*,25mm}}}
  \newcommand{\moviex}[2]{\movie[externalviewer]{#1}{#2}} %\pdflatex\usepackage{multimedia}
  \newcommand{\rbox}[1]{\fboxrule2mm\fcolorbox[rgb]{1,.85,.85}{1,.85,.85}{#1}}
  \newcommand{\mpage}[2]{{\begin{minipage}{#1\columnwidth}#2\end{minipage}}}
  \newcommand{\redbox}[2]{\fboxrule1mm\fcolorbox[rgb]{1,.7,.7}{1,.7,.7}{\begin{minipage}{#1\columnwidth}\center#2\end{minipage}}}
  \newcommand{\onecol}[2]{
    \begin{minipage}[c]{#1\columnwidth}#2\end{minipage}}
  \newcommand{\twocol}[5][0]{
    \begin{minipage}[c]{#2\columnwidth}#4\end{minipage}\hspace*{#1\columnwidth}%
    \begin{minipage}[c]{#3\columnwidth}#5\end{minipage}}
  \newcommand{\threecol}[7][0]{%
    \begin{minipage}[c]{#2\columnwidth}#5\end{minipage}\hspace*{#1\columnwidth}%
    \begin{minipage}[c]{#3\columnwidth}#6\end{minipage}\hspace*{#1\columnwidth}%
    \begin{minipage}[c]{#4\columnwidth}#7\end{minipage}}
  \newcommand{\threecoltext}[7][c]{
    \begin{minipage}[#1]{#2\textwidth}#5\end{minipage}%
    \begin{minipage}[#1]{#3\textwidth}#6\end{minipage}%
    \begin{minipage}[#1]{#4\textwidth}#7\end{minipage}}
  \newcommand{\threecoltop}[7][0]{%
   \begin{minipage}[t]{#2\columnwidth}#5\end{minipage}\hspace*{#1\columnwidth}%
   \begin{minipage}[t]{#3\columnwidth}#6\end{minipage}\hspace*{#1\columnwidth}%
   \begin{minipage}[t]{#4\columnwidth}#7\end{minipage}}
  \newcommand{\fourcol}[9][0]{%
   \begin{minipage}[c]{#2\columnwidth}#6\end{minipage}\hspace*{#1\columnwidth}%
   \begin{minipage}[c]{#3\columnwidth}#7\end{minipage}\hspace*{#1\columnwidth}%
   \begin{minipage}[c]{#4\columnwidth}#8\end{minipage}\hspace*{#1\columnwidth}%
   \begin{minipage}[c]{#5\columnwidth}#9\end{minipage}}
  \newcommand{\helvetica}[4]{\setlength{\unitlength}{1pt}\fontsize{#1}{#1}\linespread{#2}\usefont{OT1}{phv}{#3}{#4}}
  \newcommand{\helve}[1]{\helvetica{#1}{1.5}{m}{n}}
  \newcommand{\german}{\usepackage[german]{babel}\usepackage[utf8]{inputenc}}

\newcommand{\norm}[1]{|\!|#1|\!|}
\newcommand{\expr}[1]{[\hspace{-.2ex}[#1]\hspace{-.2ex}]}

\newcommand{\Jwi}{J^\sharp_W}
\newcommand{\THi}{T^\sharp_H}
\newcommand{\Jci}{J^\natural_C}
\newcommand{\hJi}{{\bar J}^\sharp}
\renewcommand{\|}{\,|\,}
\renewcommand{\=}{\!=\!}
\newcommand{\myminus}{{\hspace*{-.0pt}\text{\rm -}\hspace*{-.5pt}}}
\newcommand{\myplus}{{\hspace*{-.0pt}\text{\rm +}\hspace*{-.5pt}}}
\newcommand{\1}{{\myminus1}}
\newcommand{\2}{{\myminus2}}
\newcommand{\3}{{\myminus3}}
\newcommand{\mT}{{\text{\rm -}\hspace*{-1pt}\top}}
\newcommand{\po}{{\myplus1}}
\newcommand{\pt}{{\myplus2}}
%\renewcommand{\-}{\myminus}
%\newcommand{\+}{\myplus}
\renewcommand{\T}{{\!\top\!}}
\newcommand{\xT}{{\underline x}}
\newcommand{\uT}{{\underline u}}
\newcommand{\zT}{{\underline z}}
\newcommand{\Sum}{\textstyle\sum}
\newcommand{\Int}{\textstyle\int}
\newcommand{\Prod}{\textstyle\prod}


\newenvironment{centy}{
\vspace{15mm}
\large
\hspace*{5mm}
\begin{minipage}{8cm}\it\color{blue}
}{
\end{minipage}
}

\newcommand{\old}{{\text{old}}}
\newcommand{\new}{{\text{new}}}
\newcommand{\MAP}{{\text{MAP}}}
\newcommand{\ML}{{\text{ML}}}

\newcommand{\redArrow}{\quad\anchor{0,-1}{\includegraphics[scale=.5]{figs/redArrow}}}
\newcommand{\pub}[1]{{\color{green}\helvetica{8}{1.3}{m}{n}#1\\}}
\DeclareMathOperator{\opKL}{KL}
\newcommand{\KL}[2]{\opKL\big(#1\,\big|\!\big|\,#2\big)} %\left(#1 |\!| #2\right)}

\renewcommand{\show}[2][.8]{\centerline{\includegraphics[width=#1\columnwidth]{#2}}}
\newcommand{\showh}[2][.8]{\includegraphics[width=#1\columnwidth]{#2}}
\newcommand{\shows}[2][.8]{\centerline{\includegraphics[scale=#1]{#2}}}
\newcommand{\showhs}[2][.8]{\includegraphics[scale=#1]{#2}}
\newcommand{\mov}[2]{\movie[externalviewer]{{\color{blue}\small #1}}{movies/#2}}
\newcommand{\movex}[2]{\movie[externalviewer]{#1}{#2}} %\pdflatex\usepackage{multimedia}
%\newcommand{\movgb}[1]{\hfill\movie[externalviewer]{\small[movie]}{/home/mtoussai/movies/10-goalDirectedBehavior/#1}}
\newcommand{\movh}[3][loop]{
\movie[#1]{\showh[#2]{movies/#3.png}}{movies/#3.avi}%
\movie[externalviewer]{$\circ$}{movies/#3.avi}
}
\newcommand{\movc}[3][loop]{\centerline{\movh[#1]{#2}{#3}}}
\newcommand{\cen}[1]{\centerline{#1}}

\newcommand{\citing}[1]{
{\color{citcol}\tiny#1\par}
}

\newcommand{\cit}[3]{
\par\smallskip
{\color{greencol}\tiny #1: \emph{#2}. #3 \par}
}

\newcommand{\citurl}[4]{
\par\smallskip
{\color{greencol}\tiny #1: \protect{\href{#4}{\color{blue}{#2.}}} #3 \par}
}

\newcommand{\cito}[3]{
\par\smallskip
{\color{bluecol}\tiny #1: \emph{#2}. #3 \par}
}

\newcommand{\redoMacrosInProof}{
  \renewcommand{\d}{\delta}
%  \renewcommand{\|}{\,|\,}
  \renewcommand{\=}{\!=\!}
}

%% \makeatletter
%% \newenvironment{code}{%
%%   \begin{lrbox}{\@tempboxa}\begin{minipage}{1\columnwidth}\codefont
%% }{
%%   \end{minipage}\end{lrbox}%
%%   \colorbox[rgb]{.95,.95,.95}{\usebox{\@tempboxa}}
%% }\makeatother

\newenvironment{code}{%
\codefont
\begin{shaded}
}{
\end{shaded}
}

%\newcommand{\refeq}[1]{(\ref{#1})}

\usepackage{algorithm}
\usepackage{algpseudocode}
\algrenewcommand{\algorithmicrequire}{\textbf{Input:~~}}
\algrenewcommand{\algorithmicensure}{\textbf{Output:}}
\algrenewcommand{\algorithmiccomment}[1]{\qquad\hfill~\hspace*{-5ex}\textit{// #1}}
\algrenewcommand{\alglinenumber}[1]{\helvetica{6}{1.3}{m}{n}#1:}

\newenvironment{algo}[1][8]{
\quad\begin{minipage}{.8\columnwidth}\helvetica{#1}{1.3}{m}{n}
\medskip\hrule\medskip
\begin{algorithmic}[1]
}{
\end{algorithmic}
\medskip\hrule\medskip
\end{minipage}
}

\usepackage{etoolbox}

%%%%%%%%%%%%%%%%%%%%%%%%%%%%%%%%%%%%%%%%%%%%%%%%%%%%%%%%%%%%%%%%%%%%%%%%%%%%%%%%

\usepackage{multirow}
\usepackage{colortbl}
%\setlength{\jot}{0pt}
%\setlength{\mathindent}{1ex}
\usepackage{empheq}

%%%%%%%%%%%%%%%%%%%%%%%%%%%%%%%%%%%%%%%%%%%%%%%%%%%%%%%%%%%%%%%%%%%%%%%%%%%%%%%

\newcommand{\mypause}{\pause}
%\newcommand{\dom}{{\text{dom}}}
\newcommand{\defi}[1]{\textbf{#1}}
\newcommand{\red}[1]{\emph{\color{red}#1}}
%\newcommand{\ul}{\underline}
\newcommand{\pos}{{\textsf{pos}}}
\newcommand{\eff}{{\textsf{eff}}}
\newcommand{\rot}{{\textsf{rot}}}
\newcommand{\veC}{{\textsf{vec}}}
\newcommand{\quat}{{\textsf{quat}}}
\newcommand{\col}{{\textsf{col}}}
\newcommand{\de}[2]{\frac{\partial #1}{\partial #2}}
\newcommand{\target}{{\text{target}}}
\newcommand{\near}{{\text{near}}}
\newcommand{\qfree}{Q_{\text{free}}}
\renewcommand{\vec}{\boldsymbol}
\newcommand{\lft}{\text{left}}
\newcommand{\rgh}{\text{right}}
\DeclareMathOperator{\real}{real}
\newcommand{\prev}{{\text{prev}}}
\newcommand{\TR}[2]{T_{{#1}\to{#2}}}
\newcommand{\RO}[2]{R_{{#1}\to{#2}}}
\newcommand{\liter}{\helvetica{8}{1.1}{m}{n}\parskip 1ex}
\newcommand{\Fc}{\color{green}F}
\newcommand{\muc}{\color{blue}\mu}
\newcommand{\Astar}{A$^*$}

%for AI course:
\newcommand{\defn}[1]{\textbf{#1}}

%for optimization course:
\newcommand{\adec}{\r_\a^-}
\newcommand{\ainc}{\r_\a^+}
\newcommand{\ldec}{\r_\l^-}
\newcommand{\linc}{\r_\l^+}
\newcommand{\minc}{\r_\m^+}
\newcommand{\mdec}{\r_\m^-}
\newcommand{\lsstop}{\r_{\text{ls}}}


\definecolor{boxcol}{rgb}{.85,.9,.92}
\newcommand{\eqbox}[1]{\centerline{\fboxrule0mm\fcolorbox{boxcol}{boxcol}{#1}}}
\newcommand{\movgb}[1]{\hfill\movie[externalviewer]{\small[movie]}{/home/mtoussai/movies/10-goalDirectedBehavior/#1}}
\newcommand{\demo}[1]{{{\color{blue}[\small #1]}}}

\graphicspath{{../pics-robotics/}{../pics-ML/}{../pics-all/}{../pics-all2/}{../pics-Optim/}}
\DeclareGraphicsExtensions{.pdf,.png,.jpg}

%\usepackage{pdfpages}
%\setbeamercolor{background canvas}{bg=}

\newcommand{\SUM}{\texttt{sum}}
\usepackage{float}

%% prevent pagebreaks before environment
\makeatletter
\newcommand{\NewParNoBreak}[1][\parskip]{\par\vspace*{-\parskip}\vspace*{#1}\nobreak\@afterheading}
\makeatother

%\newcommand{\idx}[2]{\label{IKgn}}

%%%%%%%%%%%%%%%%%%%%%%%%%%%%%%%%%%%%%%%%%%%%%%%%%%%%%%%%%%%%%%%%%%%%%%%%%%%%%%%%



%% \newwrite\tempfile
%% \immediate\openout\tempfile=z.keys.tex

%% \renewcommand{\key}[1]{
%% %%   \addtocounter{mypage}{1}
%% \makeatletter
%% \immediate\write\tempfile{\symbol{`\\}}
%% \makeatother
%%   \immediate\write\tempfile{hyperref[key:#1]{#1(\arabic{mypage})}}
%% %%  % \phantomsection\label{key:#1}
%% %%   %\index{#1@{\hyperref[key:#1]{#1 (\arabic{mysec}:\arabic{mypage})}}|phantom}
%% %%   \addtocounter{mypage}{-1}
%% }


  \DefineShortVerb{\@}

  \newcounter{solutions}
  \setcounter{solutions}{1}
  \newenvironment{solution}{
    \small
    \begin{shaded}
  }{
    \end{shaded}
  }
  
  \renewcommand{\hat}{\widehat}
  \newcommand{\bbg}{{\bar{\bar g}}}
  \graphicspath{{pics/}{../shared/pics/}}

  \renewcommand{\labelenumi}{{\alph{enumi})}}

  %%%%%%%%%%%%%%%%%%%%%%%%%%%%%%%%%%%%%%%%%%%%%%%%%%%%%%%%%%%%%%%%%%%%%%%%%%%%%%%%


  \mytitle{\course\\Exercise \exnum}
  \myauthor{Marc Toussaint\\ \small\addressUSTT}
  \date{\coursedate}
  
  \begin{document}
  \onecolumn
  \maketitle
}

\newcommand{\exsection}[1]{\section{#1}}

\newcommand{\exerfoot}{
  \end{document}
}

\newenvironment{items}[1][9]{
  \par\setlength{\unitlength}{1pt}\fontsize{#1}{#1}\linespread{1.2}\selectfont
  \begin{list}{--}{\leftmargin4ex \rightmargin0ex \labelsep1ex \labelwidth2ex
      \topsep0pt \parsep0ex \itemsep3pt}
}{
  \end{list}
}

  \exerciseshead
}

\providecommand{\script}{
  \newcommand{\scripthead}{
  \documentclass[9pt,twoside]{article}
  \stdpackages

  \usepackage{palatino}
  \usepackage[envcountsect]{beamerarticle}
  \usepackage{makeidx}
  \makeindex

  \definecolor{bluecol}{rgb}{0,0,.5}
  \definecolor{greencol}{rgb}{0,.4,0}
  \definecolor{shadecolor}{gray}{0.9}
  \usepackage[
    %    pdftex%,
    %%    letterpaper,
    %bookmarks,
    bookmarksnumbered,
    colorlinks,
    urlcolor=bluecol,
    citecolor=black,
    linkcolor=bluecol,
    %    pagecolor=bluecol,
    pdfborder={0 0 0},
    %pdfborderstyle={/S/U/W 1},
    %%    backref,     %link from bibliography back to sections
    %%    pagebackref, %link from bibliography back to pages
    %%    pdfstartview=FitH, %fitwidth instead of fit window
    pdfpagemode=UseOutlines, %bookmarks are displayed by acrobat
    pdftitle={\course},
    pdfauthor={Marc Toussaint},
    pdfkeywords={}
  ]{hyperref}
  \DeclareGraphicsExtensions{.pdf,.png,.jpg,.eps}

  \usepackage{multimedia}
  %\setbeamercolor{background canvas}{bg=}

  \renewcommand{\r}{\varrho}
  \renewcommand{\l}{\lambda}
  \renewcommand{\L}{\Lambda}
  \renewcommand{\b}{\beta}
  \renewcommand{\d}{\delta}
  \renewcommand{\k}{\kappa}
  \renewcommand{\t}{\theta}
  \renewcommand{\O}{\Omega}
  \renewcommand{\o}{\omega}
  \renewcommand{\SS}{{\cal S}}
  \renewcommand{\=}{\!=\!}
  %\renewcommand{\boldsymbol}{}
  %\renewcommand{\Chapter}{\chapter}
  %\renewcommand{\Subsection}{\subsection}

  \renewcommand{\baselinestretch}{1.0}
  \geometry{a5paper,headsep=6mm,hdivide={10mm,*,10mm},vdivide={13mm,*,7mm}}

  \fancyhead[OL,ER]{\course, \textit{Marc Toussaint}}
  \fancyhead[OR,EL]{\thepage}
  \fancyhead[C]{}
  \fancyfoot{}
  \pagestyle{fancy}

%  \setcounter{tocdepth}{3}
  \setcounter{tocdepth}{2}

   \columnsep 6ex
  %  \renewcommand{\familydefault}{\sfdefault}
  \newcommand{\headerfont}{\large}%helvetica{12}{1}{b}{n}}
  \newcommand{\slidefont} {}%\helvetica{9}{1.3}{m}{n}}
  \newcommand{\storyfont} {}
  %  \renewcommand{\small}   {\helvetica{8}{1.2}{m}{n}}
  \renewcommand{\tiny}    {\footnotesize}%helvetica{7}{1.1}{m}{n}}
  \newcommand{\ttiny} {\fontsize{5}{5}\selectfont}
  \newcommand{\codefont}{\fontsize{6}{6}\selectfont}%helvetica{8}{1.2}{m}{n}}

  %auto-ignore
  \renewcommand{\a}{\alpha}
  \renewcommand{\b}{\beta}
  \renewcommand{\d}{\delta}
    \newcommand{\D}{\Delta}
    \newcommand{\e}{\epsilon}
    \newcommand{\g}{\gamma}
    \newcommand{\G}{\Gamma}
  \renewcommand{\l}{\lambda}
  \renewcommand{\L}{\Lambda}
    \newcommand{\m}{\mu}
    \newcommand{\n}{\nu}
    \newcommand{\N}{\nabla}
  \renewcommand{\k}{\kappa}
  \renewcommand{\o}{\omega}
  \renewcommand{\O}{\Omega}
    \newcommand{\p}{\phi}
    \newcommand{\ph}{\varphi}
  \renewcommand{\P}{\Phi}
  \renewcommand{\r}{\varrho}
    \newcommand{\s}{\sigma}
  \renewcommand{\S}{\Sigma}
  \renewcommand{\t}{\theta}
    \newcommand{\T}{\Theta}
  %\renewcommand{\v}{\vartheta}
    \newcommand{\x}{\xi}
    \newcommand{\X}{\Xi}
    \newcommand{\Y}{\Upsilon}
    \newcommand{\z}{\zeta}

  \renewcommand{\AA}{{\cal A}}
    \newcommand{\BB}{{\cal B}}
    \newcommand{\CC}{{\cal C}}
    \newcommand{\cc}{{\cal c}}
    \newcommand{\DD}{{\cal D}}
    \newcommand{\EE}{{\cal E}}
    \newcommand{\FF}{{\cal F}}
    \newcommand{\GG}{{\cal G}}
    \newcommand{\HH}{{\cal H}}
    \newcommand{\II}{{\cal I}}
    \newcommand{\KK}{{\cal K}}
    \newcommand{\LL}{{\cal L}}
    \newcommand{\MM}{{\cal M}}
    \newcommand{\NN}{{\cal N}}
    \newcommand{\oNN}{\overline\NN}
    \newcommand{\OO}{{\cal O}}
    \newcommand{\PP}{{\cal P}}
    \newcommand{\QQ}{{\cal Q}}
    \newcommand{\RR}{{\cal R}}
  \renewcommand{\SS}{{\cal S}}
    \newcommand{\TT}{{\cal T}}
    \newcommand{\uu}{{\cal u}}
    \newcommand{\UU}{{\cal U}}
    \newcommand{\VV}{{\cal V}}
    \newcommand{\XX}{{\cal X}}
    \newcommand{\xx}{\mathcal{x}}
    \newcommand{\YY}{{\cal Y}}
    \newcommand{\SOSO}{{\cal SO}}
    \newcommand{\GLGL}{{\cal GL}}

    \newcommand{\Ee}{{\rm E}}

  \newcommand{\NNN}{{\mathbb{N}}}
  \newcommand{\III}{{\mathbb{I}}}
  \newcommand{\ZZZ}{{\mathbb{Z}}}
  %\newcommand{\RRR}{{\mathrm{I\!R}}}
  \newcommand{\RRR}{{\mathbb{R}}}
  \newcommand{\SSS}{{\mathbb{S}}}
  \newcommand{\CCC}{{\mathbb{C}}}
  \newcommand{\DDD}{{\mathbb{D}}}
  \newcommand{\one}{{{\bf 1}}}
  \newcommand{\eee}{\text{e}}

  \newcommand{\NNNN}{{\overline{\cal N}}}

  \renewcommand{\[}{\Big[}
  \renewcommand{\]}{\Big]}
  \renewcommand{\(}{\Big(}
  \renewcommand{\)}{\Big)}
  \renewcommand{\|}{\,|\,}
  \renewcommand{\;}{\,;\,}
  \renewcommand{\=}{\!=\!}
    \newcommand{\<}{\left\langle}
  \renewcommand{\>}{\right\rangle}

  \newcommand{\na}[1][]{{\nabla_{\!\!#1}}}
  \newcommand{\he}[1][]{{\nabla_{\!\!#1}^2}}
  \newcommand{\Prob}{{\rm Prob}}
  \newcommand{\Dir}{{\rm Dir}}
  \newcommand{\Beta}{{\rm Beta}}
  \newcommand{\Bern}{{\rm Bern}}
  \newcommand{\Bin}{{\rm Bin}}
  \newcommand{\Mult}{{\rm Mult}}
  \newcommand{\Aut}{{\rm Aut}}
  \newcommand{\cor}{{\rm cor}}
  \newcommand{\corr}{{\rm corr}}
  \newcommand{\sd}{{\rm sd}}
  \newcommand{\tr}{{\rm tr}}
  \newcommand{\Tr}{{\rm Tr}}
  \newcommand{\rank}{{\rm rank}}
  \newcommand{\diag}{{\rm diag}}
  \newcommand{\dom}{{\rm dom}}
  \newcommand{\id}{{\rm id}}
  \newcommand{\Id}{{\rm\bf I}}
  \newcommand{\Gl}{{\rm Gl}}
  \renewcommand{\th}{\ensuremath{{}^\text{th}} }
  \newcommand{\lag}{\mathcal{L}}
  \newcommand{\inn}{\rfloor}
  \newcommand{\lie}{\pounds}
  \newcommand{\longto}{\longrightarrow}
  \newcommand{\speer}{\parbox{0.4ex}{\raisebox{0.8ex}{$\nearrow$}}}
  \renewcommand{\dag}{ {}^\dagger }
  \newcommand{\blbox}{\rule{1ex}{1ex}}
  \newcommand{\Ji}{J^\sharp}
  \newcommand{\h}{{}^\star}
  \newcommand{\w}{\wedge}
  \newcommand{\too}{\longrightarrow}
  \newcommand{\oot}{\longleftarrow}
  \newcommand{\To}{\Rightarrow}
  \newcommand{\oT}{\Leftarrow}
  \newcommand{\oTo}{\Leftrightarrow}
  \renewcommand{\iff}{~\Longleftrightarrow~}
  \newcommand{\Too}{\;\Longrightarrow\;}
  \newcommand{\oto}{\leftrightarrow}
  \newcommand{\ot}{\leftarrow}
  \newcommand{\ootoo}{\longleftrightarrow}
  \newcommand{\ow}{\stackrel{\circ}\wedge}
  \newcommand{\defeq}{\stackrel{\hspace{0.2ex}{}_\Delta}=}
%  \newcommand{\defeq}{{\overstack\Delta =}}
  \newcommand{\feed}{\nonumber \\}
  \newcommand{\comma}{~,\quad}
  \newcommand{\period}{~.\quad}
  \newcommand{\del}{\partial}
%  \newcommand{\quabla}{\Delta}
  \newcommand{\point}{$\bullet~~$}
  \newcommand{\doubletilde}{ ~ \raisebox{0.3ex}{$\widetilde {}$} \raisebox{0.6ex}{$\widetilde {}$} \!\! }
  \newcommand{\topcirc}{\parbox{0ex}{~\raisebox{2.5ex}{${}^\circ$}}}
  \newcommand{\topdot} {\parbox{0ex}{~\raisebox{2.5ex}{$\cdot$}}}
  \newcommand{\topddot} {\parbox{0ex}{~\raisebox{1.3ex}{$\ddot{~}$}}}
  \newcommand{\sym}{\topcirc}
  \newcommand{\tsum}{\textstyle\sum}
  \newcommand{\st}{\quad\text{s.t.}\quad}

  \newcommand{\half}{\ensuremath{\frac{1}{2}}}
  \newcommand{\third}{\ensuremath{\frac{1}{3}}}
  \newcommand{\fourth}{\ensuremath{\frac{1}{4}}}

  \newcommand{\ubar}{\underline}
  %\renewcommand{\vec}{\underline}
  \renewcommand{\vec}{\boldsymbol}
  %\renewcommand{\_}{\underset}
  %\renewcommand{\^}{\overset}
  %\renewcommand{\*}{{\rm\raisebox{-.6ex}{\text{*}}{}}}
  \renewcommand{\*}{\text{\footnotesize\raisebox{-.4ex}{*}{}}}

  \newcommand{\gto}{{\raisebox{.5ex}{${}_\rightarrow$}}}
  \newcommand{\gfrom}{{\raisebox{.5ex}{${}_\leftarrow$}}}
  \newcommand{\gnto}{{\raisebox{.5ex}{${}_\nrightarrow$}}}
  \newcommand{\gnfrom}{{\raisebox{.5ex}{${}_\nleftarrow$}}}

  %\newcommand{\RND}{{\SS}}
  %\newcommand{\IF}{\text{if }}
  %\newcommand{\AND}{\textsc{and }}
  %\newcommand{\OR}{\textsc{or }}
  %\newcommand{\XOR}{\textsc{xor }}
  %\newcommand{\NOT}{\textsc{not }}

  %\newcommand{\argmax}[1]{{\rm arg}\!\max_{#1}}
  %\newcommand{\argmin}[1]{{\rm arg}\!\min_{#1}}
  \DeclareMathOperator*{\argmax}{argmax}
  \DeclareMathOperator*{\argmin}{argmin}
  \DeclareMathOperator{\sign}{sign}
  \DeclareMathOperator{\acos}{acos}
  \DeclareMathOperator{\unifies}{unifies}
  \DeclareMathOperator{\Span}{span}
  \newcommand{\ortho}{\perp}
  %\newcommand{\argmax}[1]{\underset{~#1}{\text{argmax}}\;}
  %\newcommand{\argmin}[1]{\underset{~#1}{\text{argmin}}\;}
  \newcommand{\ee}[1]{\ensuremath{\cdot10^{#1}}}
  \newcommand{\sub}[1]{\ensuremath{_{\text{#1}}}}
  \newcommand{\up}[1]{\ensuremath{^{\text{#1}}}}
  \newcommand{\kld}[3][{}]{D_{#1}\big(#2\,\big|\!\big|\,#3\big)}
  %\newcommand{\kld}[2]{D\big(#1:#2\big)}
  \newcommand{\sprod}[2]{\big<#1\,,\,#2\big>}
  \newcommand{\End}{\text{End}}
  \newcommand{\txt}[1]{\quad\text{#1}\quad}
  \newcommand{\Over}[2]{\genfrac{}{}{0pt}{0}{#1}{#2}}
  %\newcommand{\mat}[1]{{\bf #1}}
  \newcommand{\arr}[2]{\hspace*{-.5ex}\begin{array}{#1}#2\end{array}\hspace*{-.5ex}}
  \newcommand{\mat}[3][.9]{
    \renewcommand{\arraystretch}{#1}{\scriptscriptstyle{\left(
      \hspace*{-1ex}\begin{array}{#2}#3\end{array}\hspace*{-1ex}
    \right)}}\renewcommand{\arraystretch}{1.2}
  }
  \newcommand{\Mat}[3][.9]{
    \renewcommand{\arraystretch}{#1}{\scriptscriptstyle{\left[
      \hspace*{-1ex}\begin{array}{#2}#3\end{array}\hspace*{-1ex}
    \right]}}\renewcommand{\arraystretch}{1.2}
  }
  \newcommand{\case}[2][ll]{\left\{\arr{#1}{#2}\right.}
  \newcommand{\seq}[1]{\textsf{\<#1\>}}
  \newcommand{\seqq}[1]{\textsf{#1}}
  \newcommand{\floor}[1]{\lfloor#1\rfloor}
  \newcommand{\Exp}[2][]{\text{E}_{#1}\{#2\}}
  \newcommand{\Var}[2][]{\text{Var}_{#1}\{#2\}}
  \newcommand{\cov}[2][]{\text{cov}_{#1}\{#2\}}

%\newcommand{\Exp}[2]{\left\langle{#2}\right\rangle_{#1}}
  \newcommand{\ex}{\setminus}

  \providecommand{\href}[2]{{\color{blue}USE PDFLATEX!}}
  \providecommand{\url}[2]{\href{#1}{{\color{blue}#2}}}
%  \newcommand{\link}[1]{\href{{\protect #1}}{\texttt{\protect #1}}}
  \newcommand{\anchor}[2]{\begin{picture}(0,0)\put(#1){#2}\end{picture}}
  \newcommand{\pagebox}{\begin{picture}(0,0)\put(-3,-23){
    \textcolor[rgb]{.5,1,.5}{\framebox[\textwidth]{\rule[-\textheight]{0pt}{0pt}}}}
    \end{picture}}

  \newcommand{\hide}[1]{
    \begin{list}{}{\leftmargin0ex \rightmargin0ex \topsep0ex \parsep0ex}
       \helvetica{5}{1}{m}{n}
       \renewcommand{\section}{\par SECTION: }
       \renewcommand{\subsection}{\par SUBSECTION: }
       \item[$~~\blacktriangleright$]
       #1%$\blacktriangleleft~~$
       \message{^^JHIDE--Warning!^^J}
    \end{list}
  }
  %\newcommand{\hide}[1]{{\tt[hide:~}{\footnotesize\sf #1}{\tt]}\message{^^JHIDE--Warning!^^J}}
  \newcommand{\Hide}{\renewcommand{\hide}[1]{\message{^^JHIDE--Warning (hidden)!^^J}}}
  \newcommand{\HIDE}{\renewcommand{\hide}[1]{}}
  \newcommand{\fullhide}[1]{\message{^^JHIDE--Warning (hidden)!^^J}}
  \newcommand{\todo}[1]{{\tt[TODO: #1]}\message{^^JTODO--Warning: #1^^J}}
  \newcommand{\Todo}{\renewcommand{\todo}[1]{\message{^^JTODO--Warning (hidden)!^^J}}}
  %\renewcommand{\title}[1]{\renewcommand{\thetitle}{#1}}
  \newcommand{\myauthor}[1]{\author{#1}\newcommand{\theauthor}{#1}}%\@author}
  \newcommand{\mytitle}[1]{\title{#1}\newcommand{\thetitle}{#1}}%\@title}
  \newcommand{\header}{\begin{document}\mytitle\cleardefs}
  \newcommand{\contents}{{\tableofcontents}\renewcommand{\contents}{}}
  \newcommand{\footer}{\small\bibliography{marc,bibs}\end{document}}
  \newcommand{\widepaper}{\usepackage{geometry}\geometry{a4paper,hdivide={25mm,*,25mm},vdivide={25mm,*,25mm}}}
  \newcommand{\moviex}[2]{\movie[externalviewer]{#1}{#2}} %\pdflatex\usepackage{multimedia}
  \newcommand{\rbox}[1]{\fboxrule2mm\fcolorbox[rgb]{1,.85,.85}{1,.85,.85}{#1}}
  \newcommand{\mpage}[2]{{\begin{minipage}{#1\columnwidth}#2\end{minipage}}}
  \newcommand{\redbox}[2]{\fboxrule1mm\fcolorbox[rgb]{1,.7,.7}{1,.7,.7}{\begin{minipage}{#1\columnwidth}\center#2\end{minipage}}}
  \newcommand{\onecol}[2]{
    \begin{minipage}[c]{#1\columnwidth}#2\end{minipage}}
  \newcommand{\twocol}[5][0]{
    \begin{minipage}[c]{#2\columnwidth}#4\end{minipage}\hspace*{#1\columnwidth}%
    \begin{minipage}[c]{#3\columnwidth}#5\end{minipage}}
  \newcommand{\threecol}[7][0]{%
    \begin{minipage}[c]{#2\columnwidth}#5\end{minipage}\hspace*{#1\columnwidth}%
    \begin{minipage}[c]{#3\columnwidth}#6\end{minipage}\hspace*{#1\columnwidth}%
    \begin{minipage}[c]{#4\columnwidth}#7\end{minipage}}
  \newcommand{\threecoltext}[7][c]{
    \begin{minipage}[#1]{#2\textwidth}#5\end{minipage}%
    \begin{minipage}[#1]{#3\textwidth}#6\end{minipage}%
    \begin{minipage}[#1]{#4\textwidth}#7\end{minipage}}
  \newcommand{\threecoltop}[7][0]{%
   \begin{minipage}[t]{#2\columnwidth}#5\end{minipage}\hspace*{#1\columnwidth}%
   \begin{minipage}[t]{#3\columnwidth}#6\end{minipage}\hspace*{#1\columnwidth}%
   \begin{minipage}[t]{#4\columnwidth}#7\end{minipage}}
  \newcommand{\fourcol}[9][0]{%
   \begin{minipage}[c]{#2\columnwidth}#6\end{minipage}\hspace*{#1\columnwidth}%
   \begin{minipage}[c]{#3\columnwidth}#7\end{minipage}\hspace*{#1\columnwidth}%
   \begin{minipage}[c]{#4\columnwidth}#8\end{minipage}\hspace*{#1\columnwidth}%
   \begin{minipage}[c]{#5\columnwidth}#9\end{minipage}}
  \newcommand{\helvetica}[4]{\setlength{\unitlength}{1pt}\fontsize{#1}{#1}\linespread{#2}\usefont{OT1}{phv}{#3}{#4}}
  \newcommand{\helve}[1]{\helvetica{#1}{1.5}{m}{n}}
  \newcommand{\german}{\usepackage[german]{babel}\usepackage[utf8]{inputenc}}

\newcommand{\norm}[1]{|\!|#1|\!|}
\newcommand{\expr}[1]{[\hspace{-.2ex}[#1]\hspace{-.2ex}]}

\newcommand{\Jwi}{J^\sharp_W}
\newcommand{\THi}{T^\sharp_H}
\newcommand{\Jci}{J^\natural_C}
\newcommand{\hJi}{{\bar J}^\sharp}
\renewcommand{\|}{\,|\,}
\renewcommand{\=}{\!=\!}
\newcommand{\myminus}{{\hspace*{-.0pt}\text{\rm -}\hspace*{-.5pt}}}
\newcommand{\myplus}{{\hspace*{-.0pt}\text{\rm +}\hspace*{-.5pt}}}
\newcommand{\1}{{\myminus1}}
\newcommand{\2}{{\myminus2}}
\newcommand{\3}{{\myminus3}}
\newcommand{\mT}{{\text{\rm -}\hspace*{-1pt}\top}}
\newcommand{\po}{{\myplus1}}
\newcommand{\pt}{{\myplus2}}
%\renewcommand{\-}{\myminus}
%\newcommand{\+}{\myplus}
\renewcommand{\T}{{\!\top\!}}
\newcommand{\xT}{{\underline x}}
\newcommand{\uT}{{\underline u}}
\newcommand{\zT}{{\underline z}}
\newcommand{\Sum}{\textstyle\sum}
\newcommand{\Int}{\textstyle\int}
\newcommand{\Prod}{\textstyle\prod}


\newenvironment{centy}{
\vspace{15mm}
\large
\hspace*{5mm}
\begin{minipage}{8cm}\it\color{blue}
}{
\end{minipage}
}

\newcommand{\old}{{\text{old}}}
\newcommand{\new}{{\text{new}}}
\newcommand{\MAP}{{\text{MAP}}}
\newcommand{\ML}{{\text{ML}}}

\newcommand{\redArrow}{\quad\anchor{0,-1}{\includegraphics[scale=.5]{figs/redArrow}}}
\newcommand{\pub}[1]{{\color{green}\helvetica{8}{1.3}{m}{n}#1\\}}
\DeclareMathOperator{\opKL}{KL}
\newcommand{\KL}[2]{\opKL\big(#1\,\big|\!\big|\,#2\big)} %\left(#1 |\!| #2\right)}

\renewcommand{\show}[2][.8]{\centerline{\includegraphics[width=#1\columnwidth]{#2}}}
\newcommand{\showh}[2][.8]{\includegraphics[width=#1\columnwidth]{#2}}
\newcommand{\shows}[2][.8]{\centerline{\includegraphics[scale=#1]{#2}}}
\newcommand{\showhs}[2][.8]{\includegraphics[scale=#1]{#2}}
\newcommand{\mov}[2]{\movie[externalviewer]{{\color{blue}\small #1}}{movies/#2}}
\newcommand{\movex}[2]{\movie[externalviewer]{#1}{#2}} %\pdflatex\usepackage{multimedia}
%\newcommand{\movgb}[1]{\hfill\movie[externalviewer]{\small[movie]}{/home/mtoussai/movies/10-goalDirectedBehavior/#1}}
\newcommand{\movh}[3][loop]{
\movie[#1]{\showh[#2]{movies/#3.png}}{movies/#3.avi}%
\movie[externalviewer]{$\circ$}{movies/#3.avi}
}
\newcommand{\movc}[3][loop]{\centerline{\movh[#1]{#2}{#3}}}
\newcommand{\cen}[1]{\centerline{#1}}

\newcommand{\citing}[1]{
{\color{citcol}\tiny#1\par}
}

\newcommand{\cit}[3]{
\par\smallskip
{\color{greencol}\tiny #1: \emph{#2}. #3 \par}
}

\newcommand{\citurl}[4]{
\par\smallskip
{\color{greencol}\tiny #1: \protect{\href{#4}{\color{blue}{#2.}}} #3 \par}
}

\newcommand{\cito}[3]{
\par\smallskip
{\color{bluecol}\tiny #1: \emph{#2}. #3 \par}
}

\newcommand{\redoMacrosInProof}{
  \renewcommand{\d}{\delta}
%  \renewcommand{\|}{\,|\,}
  \renewcommand{\=}{\!=\!}
}

%% \makeatletter
%% \newenvironment{code}{%
%%   \begin{lrbox}{\@tempboxa}\begin{minipage}{1\columnwidth}\codefont
%% }{
%%   \end{minipage}\end{lrbox}%
%%   \colorbox[rgb]{.95,.95,.95}{\usebox{\@tempboxa}}
%% }\makeatother

\newenvironment{code}{%
\codefont
\begin{shaded}
}{
\end{shaded}
}

%\newcommand{\refeq}[1]{(\ref{#1})}

\usepackage{algorithm}
\usepackage{algpseudocode}
\algrenewcommand{\algorithmicrequire}{\textbf{Input:~~}}
\algrenewcommand{\algorithmicensure}{\textbf{Output:}}
\algrenewcommand{\algorithmiccomment}[1]{\qquad\hfill~\hspace*{-5ex}\textit{// #1}}
\algrenewcommand{\alglinenumber}[1]{\helvetica{6}{1.3}{m}{n}#1:}

\newenvironment{algo}[1][8]{
\quad\begin{minipage}{.8\columnwidth}\helvetica{#1}{1.3}{m}{n}
\medskip\hrule\medskip
\begin{algorithmic}[1]
}{
\end{algorithmic}
\medskip\hrule\medskip
\end{minipage}
}

\usepackage{etoolbox}

%%%%%%%%%%%%%%%%%%%%%%%%%%%%%%%%%%%%%%%%%%%%%%%%%%%%%%%%%%%%%%%%%%%%%%%%%%%%%%%%

\usepackage{multirow}
\usepackage{colortbl}
%\setlength{\jot}{0pt}
%\setlength{\mathindent}{1ex}
\usepackage{empheq}

%%%%%%%%%%%%%%%%%%%%%%%%%%%%%%%%%%%%%%%%%%%%%%%%%%%%%%%%%%%%%%%%%%%%%%%%%%%%%%%

\newcommand{\mypause}{\pause}
%\newcommand{\dom}{{\text{dom}}}
\newcommand{\defi}[1]{\textbf{#1}}
\newcommand{\red}[1]{\emph{\color{red}#1}}
%\newcommand{\ul}{\underline}
\newcommand{\pos}{{\textsf{pos}}}
\newcommand{\eff}{{\textsf{eff}}}
\newcommand{\rot}{{\textsf{rot}}}
\newcommand{\veC}{{\textsf{vec}}}
\newcommand{\quat}{{\textsf{quat}}}
\newcommand{\col}{{\textsf{col}}}
\newcommand{\de}[2]{\frac{\partial #1}{\partial #2}}
\newcommand{\target}{{\text{target}}}
\newcommand{\near}{{\text{near}}}
\newcommand{\qfree}{Q_{\text{free}}}
\renewcommand{\vec}{\boldsymbol}
\newcommand{\lft}{\text{left}}
\newcommand{\rgh}{\text{right}}
\DeclareMathOperator{\real}{real}
\newcommand{\prev}{{\text{prev}}}
\newcommand{\TR}[2]{T_{{#1}\to{#2}}}
\newcommand{\RO}[2]{R_{{#1}\to{#2}}}
\newcommand{\liter}{\helvetica{8}{1.1}{m}{n}\parskip 1ex}
\newcommand{\Fc}{\color{green}F}
\newcommand{\muc}{\color{blue}\mu}
\newcommand{\Astar}{A$^*$}

%for AI course:
\newcommand{\defn}[1]{\textbf{#1}}

%for optimization course:
\newcommand{\adec}{\r_\a^-}
\newcommand{\ainc}{\r_\a^+}
\newcommand{\ldec}{\r_\l^-}
\newcommand{\linc}{\r_\l^+}
\newcommand{\minc}{\r_\m^+}
\newcommand{\mdec}{\r_\m^-}
\newcommand{\lsstop}{\r_{\text{ls}}}


\definecolor{boxcol}{rgb}{.85,.9,.92}
\newcommand{\eqbox}[1]{\centerline{\fboxrule0mm\fcolorbox{boxcol}{boxcol}{#1}}}
\newcommand{\movgb}[1]{\hfill\movie[externalviewer]{\small[movie]}{/home/mtoussai/movies/10-goalDirectedBehavior/#1}}
\newcommand{\demo}[1]{{{\color{blue}[\small #1]}}}

\graphicspath{{../pics-robotics/}{../pics-ML/}{../pics-all/}{../pics-all2/}{../pics-Optim/}}
\DeclareGraphicsExtensions{.pdf,.png,.jpg}

%\usepackage{pdfpages}
%\setbeamercolor{background canvas}{bg=}

\newcommand{\SUM}{\texttt{sum}}
\usepackage{float}

%% prevent pagebreaks before environment
\makeatletter
\newcommand{\NewParNoBreak}[1][\parskip]{\par\vspace*{-\parskip}\vspace*{#1}\nobreak\@afterheading}
\makeatother

%\newcommand{\idx}[2]{\label{IKgn}}

%%%%%%%%%%%%%%%%%%%%%%%%%%%%%%%%%%%%%%%%%%%%%%%%%%%%%%%%%%%%%%%%%%%%%%%%%%%%%%%%



%% \newwrite\tempfile
%% \immediate\openout\tempfile=z.keys.tex

%% \renewcommand{\key}[1]{
%% %%   \addtocounter{mypage}{1}
%% \makeatletter
%% \immediate\write\tempfile{\symbol{`\\}}
%% \makeatother
%%   \immediate\write\tempfile{hyperref[key:#1]{#1(\arabic{mypage})}}
%% %%  % \phantomsection\label{key:#1}
%% %%   %\index{#1@{\hyperref[key:#1]{#1 (\arabic{mysec}:\arabic{mypage})}}|phantom}
%% %%   \addtocounter{mypage}{-1}
%% }


  \DefineShortVerb{\@}

  \newcounter{solutions}
  \setcounter{solutions}{1}
  \renewenvironment{solution}{
    \small
    \begin{shaded}
  }{
    \end{shaded}
  }

  \graphicspath{{pics/}{../shared/pics/}}

%%%%%%%%%%%%%%%%%%%%%%%%%%%%%%%%%%%%%%%%%%%%%%%%%%%%%%%%%%%%%%%%%%%%%%%%%%%%%%%%

  \mytitle{\course\\Lecture Script}
  \myauthor{Marc Toussaint}
  \date{\coursedate}

  \begin{document}

  %% \vspace*{2cm}

  \maketitle
  %\anchor{100,10}{\includegraphics[width=4cm]{optim}}

%  \vspace*{1cm}

  \emph{This is a direct concatenation and reformatting of all lecture
    slides and exercises from this course, including indexing to help
    prepare for exams.}

  {\tableofcontents}
}

%%%%%%%%%%%%%%%%%%%%%%%%%%%%%%%%%%%%%%%%%%%%%%%%%%%%%%%%%%%%%%%%%%%%%%%%%%%%%%%%

%% \renewcommand{\keywords}{}
%% \newcommand{\topic}{}
%% \renewcommand{\mypause}{}

  \newcounter{mypage}
  \setcounter{mypage}{0}
  \newcounter{mysec}
  \setcounter{mysec}{0}
  \newcommand{\incpage}{\addtocounter{mypage}{1}}
  \newcommand{\incsec} {\addtocounter{mysec}{1}}

\newcommand{\beginTocMinipage}{
  \addtocontents{toc}{\smallskip\noindent\hspace*{.036\columnwidth}}
  \addtocontents{toc}{\protect\begin{minipage}{.914\columnwidth}\small}
}
\newcommand{\closeTocMinipage}{
  \addtocontents{toc}{\protect\end{minipage}}
  \addtocontents{toc}{}
  \addtocontents{toc}{\medskip}
}

\renewcommand{\slides}[1][]{
  \clearpage
  \incsec
  \section{\topic}
  {\small #1}
  \beginTocMinipage
  \setcounter{mypage}{0}
  \smallskip\nopagebreak\hrule\medskip
}

\newcommand{\slidesfoot}{
  \closeTocMinipage
  \bigskip
}

\newcommand{\sublecture}[2]{
  \pagebreak[3]
  \incpage
  \closeTocMinipage
  \subsection{#1}
  {\storyfont #2}
  \beginTocMinipage
  {\hfill\tiny \textsf{\arabic{mysec}:\arabic{mypage}}}\nopagebreak%
  \smallskip\nopagebreak\hrule
}

\newcommand{\key}[1]{
  \pagebreak[2]
  \addtocounter{mypage}{1}
  \addtocontents{toc}{\hyperref[key:#1]{#1 (\arabic{mysec}:\arabic{mypage})}}
  \phantomsection\label{key:#1}
  \index{#1@{\hyperref[key:#1]{#1 (\arabic{mysec}:\arabic{mypage})}}|phantom}
  \addtocounter{mypage}{-1}
}

\newenvironment{slidecore}[1]{
  \incpage
  \subsubsection*{#1}%{\headerfont\noindent\textbf{#1}\\}%
  \vspace{-6ex}%
  \begin{list}{$\bullet$}{\leftmargin4ex \rightmargin0ex \labelsep1ex
    \labelwidth2ex \partopsep0ex \topsep0ex \parsep.5ex \parskip0ex \itemsep0pt}\item[]~\\\nopagebreak%
}{
  \end{list}\nopagebreak%
  {\hfill\tiny \textsf{\arabic{mysec}:\arabic{mypage}}}\nopagebreak%
  \smallskip\nopagebreak\hrule
}

\newcommand{\slide}[2]{
  \begin{slidecore}{#1}
    #2
  \end{slidecore}
}

\newcommand{\exsection}[1]{
  \subsubsection{#1}
}

\renewcommand{\exercises}{
  \subsection{Exercise \exnum}
}

\newcommand{\exerfoot}{
  \bigskip
}

\newcommand{\story}[1]{
  \subsection*{Motivation \& Outline}
  \addtocontents{toc}{\hyperref[mot\arabic{mysec}]{Motivation \& Outline}}
  \phantomsection\label{mot\arabic{mysec}}
  {\storyfont\sf #1}
  \medskip\nopagebreak\hrule
}

\newcounter{savedsection}
\newcommand{\subappendix}{\setcounter{savedsection}{\arabic{section}}\appendix}
\newcommand{\noappendix}{
  \setcounter{section}{\arabic{savedsection}}% restore section number
  \setcounter{subsection}{0}% reset section counter
%  \gdef\@chapapp{\sectionname}% reset section name
  \renewcommand{\thesection}{\arabic{section}}% make section numbers arabic
}

\newenvironment{items}[1][9]{
\par\setlength{\unitlength}{1pt}\fontsize{#1}{#1}\linespread{1.2}\selectfont
\begin{list}{--}{\leftmargin4ex \rightmargin0ex \labelsep1ex \labelwidth2ex
\topsep0pt \parsep0ex \itemsep3pt}
}{
\end{list}
}

  \scripthead
}

\providecommand{\course}{NO COURSE}
\providecommand{\coursepicture}{NO PICTURE}
\providecommand{\coursedate}{NO DATE}
\providecommand{\topic}{NO TOPIC}
\providecommand{\keywords}{NO KEYWORDS}
\providecommand{\exnum}{NO NUMBER}


\providecommand{\stdpackages}{
  \usepackage{amsmath}
  \usepackage{amssymb}
  \usepackage{amsfonts}
  \allowdisplaybreaks
  \usepackage{amsthm}
  \usepackage{eucal}
  \usepackage{graphicx}
  \usepackage{color}
  \usepackage{geometry}
  \usepackage{framed}
%  \usecolor{xcolor}
  \definecolor{shadecolor}{gray}{0.9}
  \setlength{\FrameSep}{3pt}
  \usepackage{fancyvrb}
  \fvset{numbers=left,xleftmargin=5ex}

  \usepackage{multicol} 
  \usepackage{fancyhdr}
}

\providecommand{\addressUSTT}{
  Machine~Learning~\&~Robotics~lab, U~Stuttgart\\\small
  Universit{\"a}tsstra{\ss}e 38, 70569~Stuttgart, Germany
}


\renewcommand{\course}{Machine Learning}
\renewcommand{\coursepicture}{course_ml}
\renewcommand{\coursedate}{Summer 2019}
\renewcommand{\topic}{Local Learning \& Emsemble Learning}
\renewcommand{\keywords}{}

\slides

%%%%%%%%%%%%%%%%%%%%%%%%%%%%%%%%%%%%%%%%%%%%%%%%%%%%%%%%%%%%%%%%%%%%%%%%%%%%%%%%

\key{Local learning and ensemble approaches}
\slide{Local learning \& Ensemble learning}{

\item ``Simpler is Better''
\pause
\begin{items}
\item We've learned about [kernel] ridge\|logistic regression
\item We've learned about high-capacity NN training
\item Sometimes one should consider also much simpler methods as baseline
\end{items}

~

\item Content:
\begin{items}
\item Local learners
\begin{items}[7]
\item local \& lazy learning, kNN, Smoothing Kernel, kd-trees
\end{items}

\item Combining weak or randomized learners
\begin{items}[7]
\item Bootstrap, bagging, and model averaging
\item Boosting
\item (Boosted) decision trees \& stumps, random forests
\end{items}
\end{items}

}

%%%%%%%%%%%%%%%%%%%%%%%%%%%%%%%%%%%%%%%%%%%%%%%%%%%%%%%%%%%%%%%%%%%%%%%%%%%%%%%%

\sublecture{Local \& lazy learning}{
}

%%%%%%%%%%%%%%%%%%%%%%%%%%%%%%%%%%%%%%%%%%%%%%%%%%%%%%%%%%%%%%%%%%%%%%%%%%%%%%%%

\key{Smoothing kernel}
\slide{Local \& lazy learning}{

\item Idea of local (or ``lazy'') learning:

Do not try to build one global model $f(x)$ from the data. Instead,
whenever we have a query point $x^*$, we build a specific local model
in the neighborhood of $x^*$.

~\mypause

\item Typical approach:

-- Given a query point $x^*$, find all $k$NN in the data
$D=\{(x_i,y_i)\}_{i=1}^N$

-- Fit a local model $f_{x^*}$ only to these $k$NNs, perhaps weighted

-- Use the local model $f_{x^*}$ to predict $x^* \mapsto \hat y_0$

~\mypause

\item \textbf{Weighted local least squares:}

\medskip
\eqbox{$L^{\text{local}}(\b,x^*) = \sum_{i=1}^n K(x^*,x_i) (y_i
- f(x_i))^2 + \l \norm{\b}^2$}
\medskip

where $K(x^*,x)$ is called \textbf{smoothing kernel}. The optimum is:
$$\hat \b = (X^\T W X + \l I)^\1 X^\T W y
  \comma W=\diag(K(x^*,x_{1:n}))$$

}

%%%%%%%%%%%%%%%%%%%%%%%%%%%%%%%%%%%%%%%%%%%%%%%%%%%%%%%%%%%%%%%%%%%%%%%%%%%%%%%%

\key{kNN smoothing kernel}
\key{Epanechnikov quadratic smoothing kernel}
\slide{Regression example}{

\small

kNN smoothing kernel: $K(x^*,x_i) = \begin{cases}
1 & \text{if } x_i \in \text{kNN}(x^*)\\
0 & \text{otherwise} \end{cases}$

Epanechnikov quadratic smoothing kernel:
$K_\l(x^*,x) = D(|x^*-x|/\l) \comma D(s) =
 \begin{cases} \frac{3}{4}(1-s^2) & \text{if }s\le 1\\
0 & \text{otherwise} \end{cases}$

~

~

\show[.45]{kernelSmoothing1}

\show[.45]{kernelSmoothing2}

\vspace*{-5mm}{\tiny\hfill(Hastie, Sec 6.3)}

}

%%%%%%%%%%%%%%%%%%%%%%%%%%%%%%%%%%%%%%%%%%%%%%%%%%%%%%%%%%%%%%%%%%%%%%%%%%%%%%%%

\slide{Smoothing Kernels}{

\show[.6]{SmoothingKernels}
{\tiny\hfill from Wikipedia}

}

%%%%%%%%%%%%%%%%%%%%%%%%%%%%%%%%%%%%%%%%%%%%%%%%%%%%%%%%%%%%%%%%%%%%%%%%%%%%%%%%

\slide{Which metric to use for NN?}{

\item This is \emph{the} crutial question? The fundamental question of generalization.
\begin{items}
\item Given a query $x^*$, which data points $x_i$ would you consider as being ``related'', so that the label of $x_i$ is correlated to the correct label of $x^*$?
\end{items}

~\pause

\item Possible answers beyond naive Euclidean distance $|x^*-x_i|$
\begin{items}
\item Some other kernel function $k(x^*, x_i)$
\item First encode $x$ into a ``meaningful'' latent representation $z$; then use Euclidean distance there
\pause
\item Take some off-the-shelf pretrained image NN, chop of the head, use this internal representation
\end{items}

}

%%%%%%%%%%%%%%%%%%%%%%%%%%%%%%%%%%%%%%%%%%%%%%%%%%%%%%%%%%%%%%%%%%%%%%%%%%%%%%%%

\key{kd-trees}
\slide{kd-trees}{

\item For local \& lazy learning it is essential to efficiently
retrieve the kNN

~

Problem: Given data $X$, a query $x^*$, identify the kNNs of
$x^*$ in $X$.

~

\item Linear time (stepping through all of $X$) is far too slow.

~

A kd-tree pre-structures the data into a binary tree, allowing $O(\log n)$
retrieval of kNNs.

}


%%%%%%%%%%%%%%%%%%%%%%%%%%%%%%%%%%%%%%%%%%%%%%%%%%%%%%%%%%%%%%%%%%%%%%%%%%%%%%%%

\slide{kd-trees}{

\show{kdt}
{\hfill\tiny(There are ``typos'' in this figure... Exercise to find them.)}

~

\item Every node plays two roles:

{\small

-- it defines a hyperplane that separates the data along \emph{one}
   coordinate

-- it hosts a data point, which lives exactly on the hyperplane
(defines the division)

}

}

%%%%%%%%%%%%%%%%%%%%%%%%%%%%%%%%%%%%%%%%%%%%%%%%%%%%%%%%%%%%%%%%%%%%%%%%%%%%%%%%

\slide{kd-trees}{

\item Simplest (non-efficient) way to construct a kd-tree:

{\small
-- hyperplanes divide alternatingly along 1st, 2nd, ... coordinate

-- choose random point, use it to define hyperplane, divide data,
   iterate

}

~

\item Nearest neighbor search:

{\small
-- descent to a leave node and take this as initial nearest point

-- ascent and check at each branching the possibility that a nearer
   point exists on the other side of the hyperplane

}

~

\item Approximate Nearest Neighbor ~ (libann on Debian..)

}

%%%%%%%%%%%%%%%%%%%%%%%%%%%%%%%%%%%%%%%%%%%%%%%%%%%%%%%%%%%%%%%%%%%%%%%%%%%%%%%%

\sublecture{Combining weak and randomized learners}{

%% \tiny
%% -- Bootstrap, bagging, and model averaging

%% -- Boosting

%% -- (Boosted) decision trees \& stumps

%% -- Random forests

}

%%%%%%%%%%%%%%%%%%%%%%%%%%%%%%%%%%%%%%%%%%%%%%%%%%%%%%%%%%%%%%%%%%%%%%%%%%%%%%%%

\slide{Combining learners}{

\item The general idea is:

-- Given data $D$, let us learn \emph{various models} $f_1,..,f_M$

-- Our prediction is then some combination of these, e.g.
$$f(x) = \sum_{m=1}^M \a_m f_m(x)$$

~

\item \emph{``Various models''} could be:

\medskip

\textbf{Model averaging:}~ Fully different types of models (using
different (e.g.\ limited) feature sets; neural nets; decision trees;
hyperparameters)

\medskip

\textbf{Bootstrap:}~ Models of same type, trained on randomized versions of $D$

\medskip

\textbf{Boosting:}~ Models of same type, trained on cleverly designed
modifications/reweightings of $D$

~

\item How can we choose the $\a_m$? ~ {\tiny (You should know that!)}

}

%%%%%%%%%%%%%%%%%%%%%%%%%%%%%%%%%%%%%%%%%%%%%%%%%%%%%%%%%%%%%%%%%%%%%%%%%%%%%%%%

\key{Bootstrapping}
\key{Bagging}
\slide{Bootstrap \& Bagging}{

\item \textbf{Bootstrap:}

-- Data set $D$ of size $n$

-- Generate $M$ data sets $D_m$ by resampling $D$ \emph{with
replacement}

-- Each $D_m$ is also of size $n$ ~ (some samples doubled or missing)

~

{\small

-- Distribution over data sets $\oto$ distribution over $\b$ (compare slide 02:13)

-- The ensemble $\{f_1,..,f_M\}$ is similar to cross-validation

-- Mean and variance of $\{f_1,..,f_M\}$ can be used for model
   assessment

}

~

~

\item \textbf{Bagging:} (``bootstrap aggregation'')
$$f(x) = \frac{1}{M} \sum_{m=1}^M f_m(x) $$

}

%%%%%%%%%%%%%%%%%%%%%%%%%%%%%%%%%%%%%%%%%%%%%%%%%%%%%%%%%%%%%%%%%%%%%%%%%%%%%%%%

\slide{}{

\item Bagging has similar effect to regularization:

~

\show[.5]{bagging}
{\tiny\hfill(Hastie, Sec 8.7)}

}

%%%%%%%%%%%%%%%%%%%%%%%%%%%%%%%%%%%%%%%%%%%%%%%%%%%%%%%%%%%%%%%%%%%%%%%%%%%%%%%%

\key{Model averaging}
\slide{Bayesian Model Averaging}{

\item If $f_1,..,f_M$ are very different models

-- Equal weighting would not be clever

-- More confident models {\tiny(less variance, less parameters,
   higher likelihood)} \\ ~~ $\to$ higher weight

~

\item Bayesian Averaging

$$P(y|x) = \sum_{m=1}^M P(y|x,f_m,D)~ P(f_m|D)$$

The term $P(f_m|D)$ is the weighting $\a_m$: it is high, when the
model is likely under the data ($\oto$ the data is likely under the
model \& the model has ``fewer parameters'').

}

%%%%%%%%%%%%%%%%%%%%%%%%%%%%%%%%%%%%%%%%%%%%%%%%%%%%%%%%%%%%%%%%%%%%%%%%%%%%%%%%

\key{Weak learners as features}
\slide{The basis function view:~ Models are features!}{

\item Compare model averaging $f(x) = \sum_{m=1}^M \a_m f_m(x)$ with
regression:
$$f(x) = \sum_{j=1}^k \phi_j(x)~ \b_j = \phi(x)^\T \b$$

~

\item We can think of the $M$ models $f_m$ as \textbf{features}
$\phi_j$ for linear regression!

-- We know how to find optimal parameters $\a$

-- But beware overfitting!

}

%%%%%%%%%%%%%%%%%%%%%%%%%%%%%%%%%%%%%%%%%%%%%%%%%%%%%%%%%%%%%%%%%%%%%%%%%%%%%%%%

\key{Boosting}
\slide{Boosting}{

\item In Bagging and Model Averaging, the models are trained on the
``same data'' (unbiased randomized versions of the same data)

~

\item Boosting tries to be cleverer:

-- It adapts the data for each learner

-- It assigns each learner a differently \emph{weighted} version of the data

~

\item With this, boosing can

-- \emph{Combine many ``weak'' classifiers to produce a powerful
  ``committee''}

-- A weak learner only needs to be somewhat better than random

}

%%%%%%%%%%%%%%%%%%%%%%%%%%%%%%%%%%%%%%%%%%%%%%%%%%%%%%%%%%%%%%%%%%%%%%%%%%%%%%%%

\key{AdaBoost**}
\slide{AdaBoost**}{

(Freund \& Schapire, 1997)

(classical Algo; use Gradient Boosting instead in practice)

~

\item Binary classification problem with data
$D=\{(x_i,y_i)\}_{i=1}^n$, $y_i\in\{-1,+1\}$

\item We know how to train discriminative functions $f(x)$; let
$$G(x) = \sign{f(x)} ~~ \in\{-1,+1\}$$ 

~

\item We will train a sequence of classificers $G_1,..,G_M$, each on
differently weighted data, to yield a classifier
$$G(x) = \sign{\sum_{m=1}^M \a_m G_m(x)}$$

}

%%%%%%%%%%%%%%%%%%%%%%%%%%%%%%%%%%%%%%%%%%%%%%%%%%%%%%%%%%%%%%%%%%%%%%%%%%%%%%%%

\slide{AdaBoost**}{ 

\show[.55]{adaboost}
{\tiny\hfill(Hastie, Sec 10.1)}

}

%%%%%%%%%%%%%%%%%%%%%%%%%%%%%%%%%%%%%%%%%%%%%%%%%%%%%%%%%%%%%%%%%%%%%%%%%%%%%%%%

\slide{AdaBoost**}{


\begin{algo}
\Require data $D=\{(x_i,y_i)\}_{i=1}^n$
\Ensure family of $M$ classifiers $G_m$ and weights $\a_m$
\State initialize $\forall_i:~ w_i = 1/n$
\For{$m= 1,..,M$}
\State Fit classifier $G_m$ to the training data weighted by $w_i$
\State $\text{err}_m = \frac{\sum_{i=1}^n w_i~ [y_i \not= G_m(x_i)]}{\sum_{i=1}^n w_i}$
\State $\a_m = \log[\frac{1-\text{err}_m}{\text{err}_m}]$
\State $\forall_i:~ w_i \gets w_i~ \exp\{\a_m~ [y_i \not= G_m(x_i)]\}$
\EndFor
\end{algo}

{\tiny\hfill(Hastie, sec 10.1)}

~

Weights unchanged for correctly classified points

Multiply weights with $\frac{1-\text{err}_m}{\text{err}_m}>1$ for mis-classified data points

~

\item \emph{Real AdaBoost:} A variant exists that combines
probabilistic classifiers $\s(f(x)) \in [0,1]$ instead of discrete
$G(x) \in \{-1,+1\}$

}

%%%%%%%%%%%%%%%%%%%%%%%%%%%%%%%%%%%%%%%%%%%%%%%%%%%%%%%%%%%%%%%%%%%%%%%%%%%%%%%%

\slide{The basis function view}{

\item In AdaBoost, each model $G_m$ depends on the data weights $w_m$

We could write this as
$$f(x) = \sum_{m=1}^M \a_m f_m(x,w_m)$$

The ``features'' $f_m(x,w_m)$ now have additional parameters $w_m$

We'd like to optimize
$$\min_{\a,w_1,..,w_M} L(f)$$
w.r.t.\ $\a$ and all the feature parameters $w_m$.

~

\item In general this is hard.

But assuming $\a_{\hat m}$ and $w_{\hat m}$ fixed, optimizing for $\a_m$ and $w_m$ is efficient.

~

\item AdaBoost does exactly this, choosing $w_m$ so that
the ``feature'' $f_m$ will best reduce the loss (cf.\ PLS)

{\tiny (Literally, AdaBoost uses exponential loss or
neg-log-likelihood; Hastie sec 10.4 \& 10.5)}

}

%%%%%%%%%%%%%%%%%%%%%%%%%%%%%%%%%%%%%%%%%%%%%%%%%%%%%%%%%%%%%%%%%%%%%%%%%%%%%%%%

\key{Gradient boosting}
\slide{Gradient Boosting}{

\item AdaBoost generates a series of basis functions by using
different data weightings $w_m$ depending on so-far classification
errors

\item We can also generate a series of basis functions
$f_m$ by fitting them to the gradient of the so-far loss

}

%%%%%%%%%%%%%%%%%%%%%%%%%%%%%%%%%%%%%%%%%%%%%%%%%%%%%%%%%%%%%%%%%%%%%%%%%%%%%%%%

\slide{Gradient Boosting}{

\item Assume we want to miminize some loss function
$$\min_f L(f) = \sum_{i=1}^n L(y_i, f(x_i))$$
We can solve this using gradient descent
$$f^*
 = f_0
 + \a_1 \underbrace{\frac{\del L(f_0)}{\del f}}_{\approx f_1}
 + \a_2 \underbrace{\frac{\del L(f_0+\a_1 f_1)}{\del f}}_{\approx f_2}
 + \a_3 \underbrace{\frac{\del L(f_0+\a_1 f_1+\a_2 f_2)}{\del f}}_{\approx f_3}
 + \cdots$$

-- Each $f_m$ aproximates the so-far loss gradient

-- We use linear regression to choose $\a_m$ (instead of line
   search)

\item More intuitively: $\frac{\del L(f)}{\del f}$ ``points into the
   direction of the error/redisual of $f$''. It shows how
   $f$ could be improved.

   Gradient boosting uses the next lerner $f_k \approx \frac{\del
   L(f_\text{so far})}{\del f}$ to approximate how $f$ can be improved.

   Optimizing $\a$'s does the improvement.

}

%%%%%%%%%%%%%%%%%%%%%%%%%%%%%%%%%%%%%%%%%%%%%%%%%%%%%%%%%%%%%%%%%%%%%%%%%%%%%%%%

\slide{Gradient Boosting}{

\begin{algo}
\Require function class $\FF$ (e.g., of discriminative functions),
data $D=\{(x_i,y_i)\}_{i=1}^n$, an arbitrary loss function $L(y,\hat y)$
\Ensure function $\hat f$ to minimize $\sum_{i=1}^n L(y_i,f(x_i))$
\State Initialize a constant $\hat f = f_0 = \argmin_{f\in\FF} \sum_{i=1}^n L(y_i,f(x_i))$
\For{$m=1:M$}
\State For each data point $i=1:n$ compute $r_{im}=-\frac{\del L(y_i,
f(x_i))}{\del f(x_i)}\big|_{f=\hat f}$
\State Fit a \textbf{regression} $f_m\in\FF$ to the
targets $r_{im}$, minimizing squared error\label{stepReg}
\State Find optimal coefficients (e.g., via feature logistic regression)\newline
 \cen{$\a = \argmin_\a \sum_i L(y_i,\sum_{j=0}^m \a_m f_m(x_i))$}
\mbox{}\hfill(often:~ fix $\a_{0:m\1}$ and only optimize over $\a_m$)
\label{stepFit}
\State Update $\hat f = \sum_{j=0}^m \a_m f_m$
%% \State Fit a \textbf{regression} tree to the targets $r_{im}$
%%  giving terminal regions $R_{jm}$, $j=1,2,..,J_m$
%% \State For $j=1,2,..,J_m$ compute terminal constants\newline
%% \cen{$\g_{jm} = \argmin_\g \sum_{i:x_i\in R_{jm}} L(y_i, f_{m−1}(x_i) + \g)$}
%% \State Update $f_m(x) = f_{m\1}(x) + \sum_{j=1}^{J_m} \g_{jm} [x\in R_{jm}]$
\EndFor
\end{algo}

~

\small
\item If $\FF$ is the set of regression/decision trees, then
step \ref{stepFit} usually re-optimizes the terminal constants of all
leave nodes of the regression tree $f_m$. (Step \ref{stepReg} only
determines the terminal regions.)

}

%%%%%%%%%%%%%%%%%%%%%%%%%%%%%%%%%%%%%%%%%%%%%%%%%%%%%%%%%%%%%%%%%%%%%%%%%%%%%%%%

\slide{Gradient boosting is the preferred method}{

\item Hastie's book quite ``likes'' gradient boosting
\begin{items}
\item Can be applied to any loss function
\item No matter if regression or classification
\item Very good performance
\item Simpler, more general, better than AdaBoost
\end{items}

}

%%%%%%%%%%%%%%%%%%%%%%%%%%%%%%%%%%%%%%%%%%%%%%%%%%%%%%%%%%%%%%%%%%%%%%%%%%%%%%%%

\slide{Classical examples for boosting}{
}

%%%%%%%%%%%%%%%%%%%%%%%%%%%%%%%%%%%%%%%%%%%%%%%%%%%%%%%%%%%%%%%%%%%%%%%%%%%%%%%%

\key{Decision trees**}
\slide{Decision Trees**}{

%% \item So far we always referred to our ``core learners'' (Part I of
%% this lecture)

\item Decision trees are particularly used in Bagging and Boosting contexts

~

\item Decision trees are ``linear in features'', but the features are
the terminal regions of a tree, which are constructed depending on the
data

~


\item We'll learn about

-- Boosted decision trees \& stumps

-- Random Forests

}

%%%%%%%%%%%%%%%%%%%%%%%%%%%%%%%%%%%%%%%%%%%%%%%%%%%%%%%%%%%%%%%%%%%%%%%%%%%%%%%%

\slide{Decision Trees}{

\item We describe CART (classification and regression tree)

\item Decision trees are linear in features:
$$f(x) = \sum_{j=1}^k c_j~ [x\in R_j]$$
where $R_j$ are disjoint rectangular regions and $c_j$ the
constant prediction in a region

\item The regions are defined by a binary decision tree

\cen{
\showh[.3]{decisionTree1}\qquad
\showh[.3]{decisionTree2}\qquad
\showh[.3]{decisionTree3}
}

}

%%%%%%%%%%%%%%%%%%%%%%%%%%%%%%%%%%%%%%%%%%%%%%%%%%%%%%%%%%%%%%%%%%%%%%%%%%%%%%%%

\slide{Growing the decision tree}{

\item The constants are the region averages $c_j =
 \frac{\sum_i y_i~ [x_i\in R_j]}{\sum_i [x_i \in R_j]}$

\item Each split $x_a > t$ is defined by a choice of
input dimension $a \in \{1,..,d\}$ and a threshold $t$

\item Given a yet unsplit region $R_j$, we split it by choosing
$$\min_{a,t} \[
 \min_{c_1} \sum_{i: x_i\in R_j \wedge x_a\le t} (y_i-c_1)^2
 +
 \min_{c_2} \sum_{i: x_i\in R_j \wedge x_a> t} (y_i-c_2)^2 \]$$
\begin{items}
\item Finding the threshold $t$ is really quick (slide along)
\item We do this for every input dimension $a$
\end{items}

}

%%%%%%%%%%%%%%%%%%%%%%%%%%%%%%%%%%%%%%%%%%%%%%%%%%%%%%%%%%%%%%%%%%%%%%%%%%%%%%%%

\slide{Deciding on the depth ~ (if not pre-fixed)}{

\item We first grow a very large tree (e.g.\ until at most 5 data
points live in each region)

~

\item Then we rank all nodes using ``weakest link pruning'':

Iteratively remove the node that least increases
$$\sum_{i=1}^n (y_i - f(x_i))^2$$

~

\item Use cross-validation to choose the eventual level of pruning

~

{\tiny

This is
equivalent to choosing a regularization parameter $\lambda$ for

\cen{$L(T) = \Sum_{i=1}^n (y_i - f(x_i))^2 ~+~ \lambda |T|$}
where the regularization $|T|$ is the tree size\\

}

}

%%%%%%%%%%%%%%%%%%%%%%%%%%%%%%%%%%%%%%%%%%%%%%%%%%%%%%%%%%%%%%%%%%%%%%%%%%%%%%%%

\slide{}{

\hspace*{-10mm}
\twocol{.35}{.7}{
\small
Example:

CART on the Spam data set

{\tiny(details: Hastie, p 320)}

~

\show[1]{decisionTree_SpamExample1}

~

\show[.7]{decisionTree_SpamExample3}

Test error rate: 8.7\%

}{
\show[.95]{decisionTree_SpamExample2}
}

}

%%%%%%%%%%%%%%%%%%%%%%%%%%%%%%%%%%%%%%%%%%%%%%%%%%%%%%%%%%%%%%%%%%%%%%%%%%%%%%%%

%% \slide{Interpretability}{

%% \item As for the ``weakest link pruning'', we can for each split
%% evaluate the increase in $\sum_{i=1}^n (y_i - f(x_i))^2$, if we would
%% remove it. Then

%% }

%%%%%%%%%%%%%%%%%%%%%%%%%%%%%%%%%%%%%%%%%%%%%%%%%%%%%%%%%%%%%%%%%%%%%%%%%%%%%%%%

\key{Boosting decision trees**}
\slide{Boosting trees \& stumps**}{

~

\item A \textbf{decision stump} is a decision tree with fixed depth 1
(just one split)

~

\item Gradient boosting of decision trees (of fixed depth $J$) and
stumps is very effective

~

Test error rates on Spam data set:
\center
\begin{tabular}{cc}
\hline
full decision tree & 8.7\% \\
boosted decision stumps & 4.7\% \\
boosted decision trees with $J=5$ & 4.5\%\\
\hline
\end{tabular}

}

%%%%%%%%%%%%%%%%%%%%%%%%%%%%%%%%%%%%%%%%%%%%%%%%%%%%%%%%%%%%%%%%%%%%%%%%%%%%%%%%

\key{Random forests**}
\slide{Random Forests:~ Bootstrapping \& randomized splits**}{

\item Recall that Bagging averages models $f_1,..,f_M$ where each
$f_m$ was trained on a bootstrap resample $D_m$ of the data $D$

This randomizes the models and avoids over-generalization

~

\item Random Forests do Bagging, but additionally randomize the
trees:
\begin{items}
\item When growing a new split, choose the input dimension $a$
only from a \emph{random subset} $m$ features

\item $m$ is often very small; even $m=1$ or $m=3$
\end{items}

~

\item Random Forests are the prime example for ``creating many randomized
weak learners from the same data $D$''

}

%%%%%%%%%%%%%%%%%%%%%%%%%%%%%%%%%%%%%%%%%%%%%%%%%%%%%%%%%%%%%%%%%%%%%%%%%%%%%%%%

\slide{Random Forests vs.\ gradient boosted trees}{\label{lastpage}

~

\show[.6]{decisionTree_SpamExample4}
{\tiny\hfill(Hastie, Fig 15.1)}

}

%%%%%%%%%%%%%%%%%%%%%%%%%%%%%%%%%%%%%%%%%%%%%%%%%%%%%%%%%%%%%%%%%%%%%%%%%%%%%%%%

\slidesfoot

